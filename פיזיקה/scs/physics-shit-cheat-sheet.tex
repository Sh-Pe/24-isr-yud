%! ~~~ Packages Setup ~~~ 
\documentclass[]{article}
\usepackage{lipsum}
\usepackage{rotating}


% Math packages
\usepackage[usenames]{color}
\usepackage{forest}
\usepackage{ifxetex,ifluatex,amssymb,amsmath,mathrsfs,amsthm,witharrows,mathtools,mathdots}
\usepackage{amsmath}
\WithArrowsOptions{displaystyle}
\renewcommand{\qedsymbol}{$\blacksquare$} % end proofs with \blacksquare. Overwrites the defualts. 
\usepackage{cancel,bm}
\usepackage[thinc]{esdiff}


% tikz
\usepackage{tikz}
\usetikzlibrary{graphs}
\newcommand\sqw{1}
\newcommand\squ[4][1]{\fill[#4] (#2*\sqw,#3*\sqw) rectangle +(#1*\sqw,#1*\sqw);}


% code 
\usepackage{listings}
\usepackage{xcolor}

\definecolor{codegreen}{rgb}{0,0.35,0}
\definecolor{codegray}{rgb}{0.5,0.5,0.5}
\definecolor{codenumber}{rgb}{0.1,0.3,0.5}
\definecolor{codeblue}{rgb}{0,0,0.5}
\definecolor{codered}{rgb}{0.5,0.03,0.02}
\definecolor{codegray}{rgb}{0.96,0.96,0.96}

\lstdefinestyle{pythonstylesheet}{
	language=Java,
	emphstyle=\color{deepred},
	backgroundcolor=\color{codegray},
	keywordstyle=\color{deepblue}\bfseries\itshape,
	numberstyle=\scriptsize\color{codenumber},
	basicstyle=\ttfamily\footnotesize,
	commentstyle=\color{codegreen}\itshape,
	breakatwhitespace=false, 
	breaklines=true, 
	captionpos=b, 
	keepspaces=true, 
	numbers=left, 
	numbersep=5pt, 
	showspaces=false,                
	showstringspaces=false,
	showtabs=false, 
	tabsize=4, 
	morekeywords={as,assert,nonlocal,with,yield,self,True,False,None,AssertionError,ValueError,in,else},              % Add keywords here
	keywordstyle=\color{codeblue},
	emph={var, List, Iterable, Iterator},          % Custom highlighting
	emphstyle=\color{codered},
	stringstyle=\color{codegreen},
	showstringspaces=false,
	abovecaptionskip=0pt,belowcaptionskip =0pt,
	framextopmargin=-\topsep, 
}
\newcommand\pythonstyle{\lstset{pythonstylesheet}}
\newcommand\pyl[1]     {{\lstinline!#1!}}
\lstset{style=pythonstylesheet}

\usepackage[style=1,skipbelow=\topskip,skipabove=\topskip,framemethod=TikZ]{mdframed}
\definecolor{bggray}{rgb}{0.85, 0.85, 0.85}
\mdfsetup{leftmargin=0pt,rightmargin=0pt,innerleftmargin=15pt,backgroundcolor=codegray,middlelinewidth=0.5pt,skipabove=5pt,skipbelow=0pt,middlelinecolor=black,roundcorner=5}
\BeforeBeginEnvironment{lstlisting}{\begin{mdframed}\vspace{-0.4em}}
	\AfterEndEnvironment{lstlisting}{\vspace{-0.8em}\end{mdframed}}


% Deisgn
\usepackage[labelfont=bf]{caption}
\usepackage[margin=0.6in]{geometry}
\usepackage{multicol}
\usepackage[skip=4pt, indent=0pt]{parskip}
\usepackage[normalem]{ulem}
\forestset{default}
\renewcommand\labelitemi{$\bullet$}
\usepackage{titlesec}
\titleformat{\section}[block]
{\fontsize{20}{40}}
{\large\dotfill \Large(\thesection)\large \dotfill}
{0em}
{\vspace{2pt} \newline \hfil \large \filleft \filright \MakeUppercase}
\usepackage{graphicx}
\graphicspath{ {./} }


\newcommand\ndoc  {\dotfill \\ \vfil {\begin{center}
            {\textbf{Shahar Perets, \textit{2025}} \\
                \scriptsize {Compiled With \LaTeX \, and Made Using Free Software}}
    \end{center}} \vfil	}

%! ~~~ Math shortcuts ~~~

% Letters shortcuts
\newcommand\N     {\mathbb{N}}
\newcommand\Z     {\mathbb{Z}}
\newcommand\R     {\mathbb{R}}
\newcommand\Q     {\mathbb{Q}}
\newcommand\C     {\mathbb{C}}
\newcommand\One   {\mathit{1}}

\newcommand\ml    {\ell}
\newcommand\mj    {\jmath}
\newcommand\mi    {\imath}

\newcommand\powerset {\mathcal{P}}
\newcommand\ps    {\mathcal{P}}
\newcommand\pc    {\mathcal{P}}
\newcommand\ac    {\mathcal{A}}
\newcommand\bc    {\mathcal{B}}
\newcommand\cc    {\mathcal{C}}
\newcommand\dc    {\mathcal{D}}
\newcommand\ec    {\mathcal{E}}
\newcommand\fc    {\mathcal{F}}
\newcommand\nc    {\mathcal{N}}
\newcommand\vc    {\mathcal{V}} % Vance
\newcommand\sca   {\mathcal{S}} % \sc is already definded
\newcommand\rca   {\mathcal{R}} % \rc is already definded

\newcommand\Si    {\Sigma}

% Set theory shortcuts
\newcommand\ra    {\rangle}
\newcommand\la    {\langle}

\newcommand\oto   {\leftarrow}
% Math A&B shortcuts
\newcommand\logn  {\log n}
\newcommand\logx  {\log x}
\newcommand\lnx   {\ln x}
\newcommand\cosx  {\cos x}
\newcommand\cost  {\cos \theta}
\newcommand\sinx  {\sin x}
\newcommand\sint  {\sin \theta}
\newcommand\tanx  {\tan x}
\newcommand\tant  {\tan \theta}
\newcommand\sex   {\sec x}
\newcommand\sect  {\sec^2}
\newcommand\cotx  {\cot x}
\newcommand\cscx  {\csc x}
\newcommand\sinhx {\sinh x}
\newcommand\coshx {\cosh x}
\newcommand\tanhx {\tanh x}

\newcommand\seq   {\overset{!}{=}}
\newcommand\peq   {\overset{*}{=}}
\newcommand\slh   {\overset{LH}{=}}
\newcommand\sle   {\overset{!}{\le}}
\newcommand\sge   {\overset{!}{\ge}}
\newcommand\sll   {\overset{!}{<}}
\newcommand\sgg   {\overset{!}{>}}

\DeclareMathOperator{\sech}   {sech}
\DeclareMathOperator{\csch}   {csch}
\DeclareMathOperator{\arcsec} {arcsec}
\DeclareMathOperator{\arccot} {arcCot}
\DeclareMathOperator{\arccsc} {arcCsc}
\DeclareMathOperator{\arccosh}{arccosh}
\DeclareMathOperator{\arcsinh}{arcsinh}
\DeclareMathOperator{\arctanh}{arctanh}
\DeclareMathOperator{\arcsech}{arcsech}
\DeclareMathOperator{\arccsch}{arccsch}
\DeclareMathOperator{\arccoth}{arccoth}
\DeclareMathOperator{\atant}  {atan2} 

\newcommand\dx    {\,\mathrm{d}x}
\newcommand\dt    {\,\mathrm{d}t}
\newcommand\ds    {\,\mathrm{d}s}
\newcommand\dtt   {\,\mathrm{d}\theta}
\newcommand\du    {\,\mathrm{d}u}
\newcommand\dv    {\,\mathrm{d}v}
\newcommand\df    {\mathrm{d}f}
\newcommand\dfdx  {\diff{f}{x}}
\newcommand\dit   {\limhz \frac{f(x + h) - f(x)}{h}}

\renewcommand\inf {\infty}
\newcommand  \ninf{-\inf}

% Greek Letters
\newcommand\ag        {\alpha}
\newcommand\bg        {\beta}
\newcommand\cg        {\gamma}
\newcommand\dg        {\delta}
\newcommand\eg        {\epsi}
\newcommand\zg        {\zeta}
\newcommand\hg        {\eta}
\newcommand\tg        {\theta}
\newcommand\ig        {\iota}
\newcommand\kg        {\keppa}
\renewcommand\lg      {\lambda}
\newcommand\og        {\omicron}
\newcommand\rg        {\rho}
\newcommand\sg        {\sigma}
\newcommand\yg        {\usilon}
\newcommand\wg        {\omega}

\newcommand\Ag        {\Alpha}
\newcommand\Bg        {\Beta}
\newcommand\Cg        {\Gamma}
\newcommand\Dg        {\Delta}
\newcommand\Eg        {\Epsi}
\newcommand\Zg        {\Zeta}
\newcommand\Hg        {\Eta}
\newcommand\Tg        {\Theta}
\newcommand\Ig        {\Iota}
\newcommand\Kg        {\Keppa}
\newcommand\Lg        {\Lambda}
\newcommand\Og        {\Omicron}
\newcommand\Rg        {\Rho}
\newcommand\Sg        {\Sigma}
\newcommand\Yg        {\Usilon}
\newcommand\Wg        {\Omega}

\newcommand\set   {\ell et \text{ }}
\newcommand\tl    {\tilde}
\newcommand\op    {^{-1}}

\newcommand\co        {\colon}
\newcommand\sof[1]    {\left | #1 \right |}
\newcommand\cl [1]    {\left ( #1 \right )}
\newcommand\csb[1]    {\left [ #1 \right ]}
\newcommand\ccb[1]    {\left \{ #1 \right \}}
\newcommand\norm[1]   {\left \vert \left \vert #1 \right \vert \right \vert}


%! ~~~ Document ~~~

\author{Shahar Perets}
\title{Physics Shit Cheat Sheet}
\begin{document}
	\begin{multicols}{3}
		\section{Mechanics}
		\[ v = \frac{\dx}{\dt} \]
		\[ a = \frac{\dv}{\dt} \]
        Where $\bar v$ is the average velocity:
		\[ \bar v = \frac{\Dg x}{\Dg t} \]
        In a constant acceleration:
		\begin{align*}
			v &= v_0 + at \\
			x &= x_0 + v_0t + \frac{1}{2}at^2 \\
			x &= x_0 + \frac{v_0 + v}{2}t \\
			\norm{v} &= \sqrt{v_0^2 + 2a(x - x_0)} 
		\end{align*}
		
		\section{Forces}
        \textbf{Newton's laws: }
        \begin{enumerate}
            \item A body remains at rest, or in motion at a constant speed in a straight line, unless it is acted upon by a force. 
            \item \hfil $ \sum \vec F = m \vec a $
            \item If two bodies exert forces $\vec F_1, \vec F_2$ on each other, then $\vec F_1 = -\vec F_2$.
        \end{enumerate}
		\[ F_g = mg \]
		\[ F_{sp} = k \, \Dg \ell \]
		\[ f_s \le \mu_s N \]
		\[ f_k = \mu_k N \]
        
		\section{Energy}
		\[ W = \int \vec F(s) \ds \]
		\begin{gather*}
			(\exists c\, \forall x \co |F(x)| = c) \\
			\implies W = F_x \cdot \Dg x = F\cost \Dg s
		\end{gather*}
		\[ E_k = \frac{1}{2}m\norm{v}^2 \]
		\[ U_g = mgh \]
		\[ U_{sp} = \frac{1}{2}k(\Dg \ell)^2 \]
		\[ {E_k}_1 + {U_g}_1 = {E_k}_2 + {U_g}_2 \]
        \[ \textstyle \forall i, j \co \, \cl{\sum E}_i = \cl{\sum E}_j \]
		\[ W_F = \Dg E = E_{\mathrm{final}} - E_{\mathrm{begining}} \]
        
        \columnbreak
		
		\section{Rotational Movement}
		\[ f = \frac{1}{T} \quad \csb{\mathrm{Hz}} \]
		\[ L = r \psi_{\mathrm{rad}} \]
		\[ \wg = 2\pi \. f = \frac{2\pi}{T} \]
		\[ v \peq \frac{2\pi r}{T} \]
		\[ \bar \wg = \frac{\Dg \theta}{\Dg t} \]
		\[ v = \wg r \]
		\[ a_R = \frac{v^2}{r} = \wg^2 r \]
		\[ P = 2\pi r \]
		Critical Velocity at max.: 
		\[ N = 0 \iff v = \sqrt{gr} \]
		\[ a_T = -g \sin \ag \]
		\[ \vec a = \vec a_T + \vec a_r \]
		\[ |a| = \sqrt{a_T^2 + a_R^2} \]
		\[ \tant = \frac{|a_T|}{|a_R|} \]
        
		\section{Gravity}
		\[ \cl{\frac{\bar r_1}{\bar r_2}} ^3= \cl{\frac{T_1}{T_2}}^2 \]
		\[ F_g = G \, \frac{Mm}{r^2} \]
		\[ U_G = - \frac{G Mm}{r} \]
		\[ E_k = \frac{GMm}{2r} = - \frac{U_G}{2} \]
        \begin{align*}
            \textstyle \sum E_{\text{mechanic}} \displaystyle &= E_k + U_G \\
            &=  -\frac{GMm}{2r} = -E_k
        \end{align*}
        \[ (W_g)_{A \to B} \peq GMm\cl{\frac{1}{r_A} - \frac{1}{r_B}} \]
        \[ T \peq \frac{2\pi r}{v} \peq 2\pi\sqrt{\frac{r^3}{GM}} \]
		\[ \rho = \frac{m}{v} \]
        
        \columnbreak
        
        \textbf{Kepler's laws of planetary motion: }
        \begin{enumerate}
            \item The orbit of a planet is an ellipse
            \item A line segment joining a planet and the Sun sweeps out equal areas during equal intervals of time.
            \item For a given gravitational system: 
            \[ \exists c \, \forall i \co \frac{T_i^2}{r_i^3} = c \peq \frac{4\pi^2}{GM} \]
        \end{enumerate}
		
		\section{Momentum}
		\[ \vec p = m \vec v \quad \csb{N \sec} \sim \csb{\frac{\mathrm{kg}\,m}{\sec}}  \]
		\[ \vec J = \int F \dt \seq \vec F \cdot \Dg \vec t = \sof{\vec F} \sof{\Dg \vec t} \cos\theta \]
		\[ \vec J_{\Sigma F} = \sum_{i = 1}^{n}\vec J_{F_i} = \Dg \vec p \]
		\[ \forall t_1, t_2 \in \R \co \sum_{i = 1}^{n} \vec p_i(t_1) = \sum_{i = 1}^{n}\vec p_i(t_2) \]
		In an inelastic collision: 
		\[ Q = \Dg E_k \]
		In an elastic collision, where $v_i$ before collision and $u_i$ after it: 
		\[ \vec v_1 - \vec v_2 = -(\vec u_1 - \vec u_2) \]
        
        Elastic Collision iff no loss of kinetic energy, Inelastic iff not ecstatic.  
        
        \section{Constants}
        \begin{center}
            \begin{tabular}{|c|c|c|}
                \hline $M \,[kg]$ & $R \,[m]$ & Obj \\
                \hline $5.974\cdot 10^{24}$ & $6.38\cdot 10^{6}$ &Earth \\
                \hline $1.99 \cdot 10^{30}$ & $6.96 \cdot 10^{8}$ & Sun \\
                \hline $7.35 \cdot 10^{22}$ & $1.74 \cdot 10^{6}$ & Moon \\
                \hline
            \end{tabular}
        \end{center}
        \[ G = 6.67 \cdot 10^{-11} \frac{\mathrm{N}m^2}{\mathrm{kg}} \]
	\end{multicols}
	\ndoc
	
\end{document}