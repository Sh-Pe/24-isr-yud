%! ~~~ Packages Setup ~~~ 
\documentclass[]{article}
\usepackage{lipsum}
\usepackage{rotating}


% Math packages
\usepackage[usenames]{color}

% Design
\usepackage[labelfont=bf]{caption}
\usepackage[margin=0.6in]{geometry}
\usepackage{multicol}
\usepackage[skip=4pt, indent=0pt]{parskip}
\usepackage[normalem]{ulem}

\usepackage{hyperref}
\hypersetup{
    pdftitle={Post Reading -- As I Grew Older},
}
\usepackage{yfonts}
\def\gothstart#1{\noindent\smash{\lower3ex\hbox{\llap{\Huge\gothfamily#1}}}
    \parshape=3 3.1em \dimexpr\hsize-3.4em 3.4em \dimexpr\hsize-3.4em 0pt \hsize}
\def\frakstart#1{\noindent\smash{\lower3ex\hbox{\llap{\Huge\frakfamily#1}}}
    \parshape=3 1.5em \dimexpr\hsize-1.5em 2em \dimexpr\hsize-2em 0pt \hsize}

\newcommand\ndoc  {\normalsize \dotfill \\ \vfil {\begin{center}
            {\textbf{\textit{Shahar Perets, 2025}} \\
                \scriptsize Compiled in {\LaTeX}\ and made using FOSS software}
    \end{center}} \vfil	}

%! ~~~ Document ~~~

\author{Shahar Perets}
\title{``As I Grew Older'' $\sim$ Post Reading $\sim$ Designing a Pre-Reading activity}
\date{June 9, 2025}
\begin{document}
    \maketitle
    
    \large \yinipar{I}n order to fully understand the poem ``As I Grew Older'', the reader should have rich knowledge and background about the writer, the time of writing, and much more.  To obtain success in understanding this complex background, I propose the following questions as a post-reading assignment:
    
    \rule{\linewidth}{0.5pt}
    {\sf \begin{enumerate}
        \item Read about Stylistic devices in online places, and summarize 5 ways that the writer can leverage to improve his poem. 
        \item Read about The Harlem Renaissance. Talk about the following points: 
        \begin{enumerate}
            \item What is The Harlem Renaissance (Where did it happen, in which years, etc.)? 
            \item Which event influenced and contributed to the establishment of The Harlem Renaissance? 
            \item Provide examples of non-poetic art that you like, from The Harlem Renaissance. 
        \end{enumerate}
        \item Read about Langston Hughes' (the writer of the poem ``As I Grew Older'') life. What events in his life made his life easier? Which made them harder? 
    \end{enumerate}}
    \vspace{-0.20cm}
    \rule{\linewidth}{0.5pt}
    
    The questions will prepare the students for the poem as a result of the following reasons [the numbering relates to the questions' numbering]: 
    \begin{enumerate}
        \item Questions No. 1 will help the students to understand details like repetitive words, breaking lines, metaphors, similes and more -- all of those, are in use in the poem. Also, it helps the student to learn how to search for reliable information online. 
        \item The poem follows closely events in this period;
        \begin{enumerate}
            \item Clause (a) will give the students general context on what The Harlem Renaissance is [note: we've had that question in class too]
            \item Clause (b) should give the students more historical context that I thought was lacking in our pre-reading. Events never occur in a vacuum, and there are other historical events (like The Great Migration, WWI and more) that led to The Harlem Renaissance. A better understating of The Harlem Renaissance will help the student to better understand the poem. 
            \item Lastly, this clause aims to give the students something from their life (like Jazz) that connects the study material to their personal lives and interests. It helps learning and memorizing in the long run. 
        \end{enumerate}
        \item The poem shows a transition in the life of Langston Hughes. A proper knowledge of his life will yield a better perception. 
    \end{enumerate}
    
    \ndoc
    
    
\end{document}