%  HISTORY 2024-2025 SUMMARY ~ by Shahar Perets ~ original .TeX
%  file ~ compile with LuaLaTeX
% 
%  Copyright (C) 2025  Shahar Perets
%  
%  This program is free software: you can redistribute it and/or modify
%  it under the terms of the GNU General Public License as published by
%  the Free Software Foundation, either version 3 of the License, or   
%  (at your option) any later version.                                 
%  
%  This program is distributed in the hope that it will be useful,     
%  but WITHOUT ANY WARRANTY; without even the implied warranty of      
%  MERCHANTABILITY or FITNESS FOR A PARTICULAR PURPOSE.  See the       
%  GNU General Public License for more details.                        
%  
%  You should have received a copy of the GNU General Public License     
%  along with this program.  If not, see https://www.gnu.org/licenses/.
%  

%! ~~~ Packages Setup ~~~ 
\documentclass[a4paper]{article}


% Design
	\usepackage[usenames]{color}
	% Layout
	\usepackage[margin=0.9in]{geometry}
	\usepackage{multicol}
	\usepackage[skip=5pt, indent=0pt]{parskip}
	\usepackage{setspace}
	\onehalfspacing
	\usepackage[normalem]{ulem}
	% Titles
	\usepackage{titlesec}
	\titleformat{\section}       [block]
		{\fontsize{15}{15}}
		{\huge פרק \sen \huge \bfseries\thesection \she}
		{0em}
		{\Large \,\,$\sim$\,\,\itshape}
		[\vspace{-13pt}\normalsize\dotfill]
	\titleformat{\subsection}    [block]
		{\large\itshape}
		{\normalfont\Large\bfseries\en{{\thesubsection}} \,\,$\sim$\,\,}
		{0em}
		{}
	\titleformat{\subsubsection} [block]
		{\normalsize\bfseries}
		{\normalfont\large\bfseries\en{{(\thesubsubsection)}}}
		{0.5em}
		{}
	\setcounter{secnumdepth}{3} % enable subsubsection
	\usepackage{etoolbox}
	\makeatletter
	\patchcmd{\chapter}{\if@openright\cleardoublepage\else\clearpage\fi}{}{}{}
	\makeatother
	\newcommand\npchapter[1] {\npage\chapter{#1}\thispagestyle{empty}\newpage}
	% Defaults
	\renewcommand\le{\leqslant}
	\renewcommand\ge{\geqslant}
	\renewcommand\labelitemi{$\bullet$}
	\newcommand\name{עבודת קיץ בהיסטוריה לקראת כיתה י''א}
	% Hyprref
	\usepackage{hyperref}
	\hypersetup{
		colorlinks,
		citecolor=black,
		filecolor=black,
		linkcolor=black,
		urlcolor=blue
	}
	% Headers
	\usepackage{calc}
	\usepackage{fancyhdr}
	\pagestyle{fancy}
	\fancyhead[C]{\textit{\textbf{\name}}}
	\fancyhead[L,R]{}
	\fancyfoot[R]{\textit{שחר פרץ, 2025}\, \en{{\Large({\thepage})}}}
	\fancyfoot[L]{\textit{\rightmark}}
	\fancyfoot[C]{}
	\renewcommand{\headrule}{\vspace{-7pt}\hfil\rule{200pt}{1pt}}
	% highlights
%	\newcommand\hl[1]   {\colorbox{yellow}{\!\!#1\!\!}}
	\newcommand\hl[1]   {#1}

% Hebrew initialzing
\usepackage[bidi=basic]{babel}
\PassOptionsToPackage{no-math}{fontspec}
\babelprovide[main, import, Alph=letters]{hebrew}
\babelprovide[import]{english}
\babelfont[hebrew]{rm}{David CLM}
\babelfont[hebrew]{sf}{David CLM}
\usepackage[shortlabels]{enumitem}
\newlist{hebenum}{enumerate}{1}

% Language Shortcuts
\newcommand\en[1] {\begin{otherlanguage}{english}#1\end{otherlanguage}}
\newcommand\he[1] {\begin{otherlanguage}{hebrew}#1\end{otherlanguage}}
\newcommand\sen   {\begin{otherlanguage}{english}}
	\newcommand\she   {\end{otherlanguage}}
\newcommand\del   {$ \!\! $}

\newcommand\npage {\vfil {\hfil \textbf{\textit{המשך בעמוד הבא}}} \hfil \vfil \pagebreak}
\newcommand\ndoc  {\dotfill \\ \vfil {\begin{center}
			{\textbf{\textit{שחר פרץ, 2025}} \\
				\scriptsize \textit{קומפל ב־}\en{\LaTeX}\,\textit{ ונוצר באמצעות תוכנה חופשית בלבד}}
	\end{center}} \vfil	}

\newcommand       {\rn}[1]{
	\textup{\uppercase\expandafter{\romannumeral#1}}
}

\newcommand\middleText[1] {\vfil {\hfil {#1}} \hfil \vfil \newpage}
\newcommand\envendproof{\vspace{-17pt}}
\newcommand\mathenvendproof{\par\vspace{-24pt}}

\makeatletter
\newcommand{\skipitems}[1]{
	\addtocounter{\@enumctr}{#1}
}
\makeatother

\author{שחר פרץ}
\title{עבודת קיץ בהיסטוריה $\sim$ תשפ''ה $\sim$ חופשת 2025}
\date{27 באוקטובר 2025}

\begin{document}
	\renewcommand{\footrule}{\rule{\linewidth-19pt}{0.25pt}\vspace{-5pt}}
	\thispagestyle{empty}
	\,\! % To stablize the next lines on smth
	
	{\vspace{0.5\textheight-2em} 
		{
			\begin{center} 
				{
					\textbf{{\name}
					} \\ 
					\textit{מאת שחר פרץ $\sim$ י'5 $\sim$ 27 באוקטובר, 2025}}
			\end{center}
		}
	}
	
	\newpage
	
	
	\section{לאומיות וציונות}
	\subsection{קטע מקור, פעילות איפלומטית, הצהרת בלפור ותוכנית באזל}
		\begin{enumerate}[A.]
			\item \textbf{שאלה: }\\
			הסבירו על פי קטע המקור ועל פי מה שלמדתם את חשיבותה של הצהרת בלפור עבור התנועה הציונית. 
			
			\textbf{תשובה: }\\
			במהלך המאה ה־19 התפתחה ברחבי אירופה והעולם תופעת ה\hl{לאומיות}, בה קבוצה של אנשים בעלי מאפיינים ועקרונות משותפים (הקרויים \hl{בני לאום}) מנסים לקדם הקמת \hl{מדינת לאום} בעבורם, הפועלת לפי עקורנותיהם. התנועה שמטרתה להקים מדינה זו, לשמור ולנהל אותה בהמשך, קרויה \hl{התנועה הלאומית} של הלאום. 
			
			לתנועה הלאומית של העם היהודי קוראים \hl{הציונות}, ומטרתה היא להקים מדינת לאום בארץ ישראל. התנועה התפתחה לקראת סוף המאה ה־19 וראשיתה באירופה. אחד ממנהגיה הבולטים של התנועה, שהקים את מוסדותיה המסודרים הראשונים, הוא \hl{בנימין זאב הרצל}. הרצל ועוד רבים אחרים תמכו בגישה הקרויה \hl{ציונות מדינית}, לפיה יש צורך לקבל אישור (צ'רטר) מהמדינה השולטת על א''י, ולא להיות ''גנבים בלילה``.
			
			אחד ממהנהגיה המאוחרים יותר של התנועה הציונית, חיים ויצמן (שהיה לימים נשיאה של מדינת הלאום שהוקמה), שהיה בעל קשרים עם \hl{המשטר הבריטי} עקב היותו ממציא האציטון והאתנול, הצליח לגרום למשטר הבריטי לפרסם הצהרה הקרויה \hl{הצהרת בלפור} שבה שר החוץ הבריטי טען שבריטניה קורה בעין יפה הקמת ''בית לאומי`` (אך \textit{לא} מדינת לאום) בארץ ישראל לעם היהודי, תוך הוספת הסתייגויות כמו ''בתנאי ברור שלא ייעשה שום דבר העלול לפגוע בזכויות האזרחיות והדתיות של עדות לא יהודיות בארץ ישראל``. 
			
			ההצהרה פורסמה תוך כדי התרחשות \hl{מלחמת העולם השנייה}, בה בריטניה נלחמה (בין היתר) באימפריה  העות'מנית ששלטה בא''י, והיה סיכוי ממשי (שאכן התממש) שבריטניה תהיה השליטה (היחידה או החלקית) בארץ בעתיד הקרוב. 
			
			בקטע המקור שקיבלנו, ד''ר חיים ויצמן שלח מכתב אל מנהיג ציוני אמריקאי בשם לואי ברנדייס. המכתב נפתח בדברי נימוס וממשיך בהבהרת חשיבות הצהרת בלפור לעם היהודי. אחת מהן, היא המטרה להשיג צ'רטר (אישור) כזה ממעצמה הנמצאת באיזור. חיים וייצמן מודע להתפתחויות הצבאיות בארץ ישראל, ולראיה כותב ''למסמך כזה יהיה ערך עצום בשעה זו, ויאפשר לנו לערוך את ההכנו תהדרושות למקרה שהתקדמות הבצא הבריטי אל תוך ארץ ישראל תתרחש בקרוב``. חיים וייצמן בהתאם לגישת הציונות המדינית מעוניין בקבלת צ'רטר מהמעצמה השולטת בא''י, ואכן הוא כותב ש''אנו חשים צורך שהמשא ומתן שלנו עם בריטניה [...] יובא לסיום על־ידי הצגת הצהרה מאת ממשלת בריטניה``. הצהרה זו קרויה ''הצהרת בלפור``. 
			
			יש חשיבות נוספת להצהרה, שנוסיף על פי מה שלמדנו. ההצהרה עוררה גל של שמחה ותקווה ברחבי העולם היהודי, ואף התקיימו טקסים ועצרות לציון המאורע. השמחה בקרב הגרעין הציוני בארץ שסבל מהשלטון העות'מני, היתה אף מרובה יותר. גלי שמחה אלו עזרו לאחד את התנועה הציונית, ולחזק את האמונה שלימים מדינת יהודית תקום. 
			
			לסיום, יש חשיבות רבה להצהרת בלפור בעבור התנועה הציונית, החל מהיקרבות להשגת הסדר ואישור ממשי מול הכוח שולט בא''י להקים מדינה לאומית, וכלה בחיזוק ואיחוד התנועה הציונית. 
			
			\item \textbf{שאלה: }\\
			הסבירו שני מניעים של בריטניה לפרסום הצהרת בלפור. בחרו קטע מקור מספר הלימוד והסבירו כיצד בא לידי ביטוי במקור שבחרתם \textit{אחד} מהמניעים שהסברתם. 
			
			\textbf{תשובה: }\\
			במהלך המאה ה־19 התרחשה באירופה ובעולם התחזקות של תופעת ה\hl{לאומיות}, בה קבוצה של אנשים בעלי עקרונות ומאפיינים משותפים מתאגדים ביחד על מנת להקים \hl{מדינה לאומית} על שטח שיש להם רגשות לאומיים ביחס אליו. המהדינה הלאומית תממש את עקורנותיה של ה\hl{תנועה הלאומית}, התנועה שאחראית להקים מדינת לאום זו. 
			
			בשנים 1914-1918 התרחשה מלחמה הקרויה \hl{מלחמת העולם הראשונה}. בין הצדדים הלוחמים נמנת האימפריה הבריטית, שנלחמה (בין היתר) באימפריה העות'מנית שבסופה של המלחמה תקרוס. לצד בריטניה לחמה גם צרפת, ונחתם הסכם בשם \hl{הסכם סייקס־פיקו} המסדיר את חלוקת השטח האימפריה העות'מנית בינהן לאחר המלחמה. במסגרת ההסכם ארץ ישראל נקבעה להיות תחת שלטון משותף בריטי־צרפתי. באמצע 1917 ארצות הברית שמרה על עמדה ניטרלית ולא הצטרפה לאף אחד מהצדדים הלוחמים. 
			
			לתנועה הלאומית היהודית קוראים \hl{הציונות}. הציונות התפתחה בסוף המאה ה־19, ובין מנהיגיה במאה ה־20 נמנה \hl{חיים ויצמן}, ממציא האתנול. הציונות שואפת להקים מדינה לאומית בארץ ישראל, שטח שהיה אז בשליטת האימפריה העות'מנית. חיים וייצמן היה במגעים עם בריטניה על מנת לפרסם הצהרת תמיכה לציונות, שבנובמבר 1917 נשלחה משר החוץ של בריטניה (ארתור ג'יימס בלפור) אל ללורד ווולטר רוטשילד (פעיל ציוני) ובה בריטניה הביאה את העבודה שהם ''רואים בעין יפה ייסוד בית לאומי לעם היהודי בארץ ישראל``. ההצהרה פורסמה והביאה לשמחה רבה ביישוב הציוני. 
			
			ישנם מניעים רבים לבריטניה בעקבותיהם היא פרסמה את הצהרת בלפור. חלק ממניעים אלו רגשיים, וחלקם מדיניים. בקטע המקור (מופיע בספר ''הלאומיות בישראל ובעמים``, הוצאת כנרת, 2014, עמוד 200 תחת השם ''ההסברים השונים לפרסום ההצרה``) ההיסטוריונית אביבה חלמיש כותבת על היחס בין מניעים אלו. 
			
			לפי קטע המקור, הגורמים המדיניים חזקים במיוחד. ''אין להבין את הטקע לפרסום הצהרת בלפור בלא גורמים שאינם לגמרי בתחום ההדיון הצרוף והשיקולים המדיניים הטהורים``, היא כותבת, בהתייחסות לרצונה של בריטניה להשפיע על משא ומתן עתידי על ארץ ישראל. בין אם המלחמה תגמר בוויעדת שלום שמטרתה להחליט למי תנתן השליטה בארץ, ובין אם חלוקה לפי הסכמי סייקס־פיקו, בריטניה רצתה לחזק את האהדה אליה ואחיזתה בארץ כדי לבסוף להיות השליטה בפועל. אביבה ממשיכה לדבר בעל המניעים להצהרה ואומרת של ''האסטרטגיה והכלכה שהם־הם שהכריעו``. 
			
			ישנם מניעים נוספים להצהרת בלפור. אחד מהם הוא הקשר הרגשי. היתה גישה אוהדת לציונות מצד כמה מראשי הממשל הבריטי, לדוגמה של החוץ וארתור ג'יימס בלפור וראש הממשלה לויד ג'ורג' שהיו פרוטנסטנטיים אדוקים שהושפעו מהתנ''ך וראו ערך בשיבת ציון. גם בני משפחת רוטשילד בענף הבריטי, נחום סוקולוב, וויצמן (שהיה מאוד מוכר בממשלה הבריטית עקב היותו ממציא האתנול) וקשריהם לממשל תרמו גם לניסוחה של ההצרה, ולהחלטה לפרסמה. 
			
			לסיכום, יש מניעים רגשיים דתיים לפרסום ההצהרה, לצד מניעים אסטרטגיים משמעותיים כמו כוונת בריטניה לבסס את נוכחותה באיזור. 
			
			
			\item \textbf{שאלה: }\\
			בחרו בתנועה לאומית שפעלה באחת מהארצות האלה: יוון, פולין, איטליה וגרמניה. \textbf{הציגו} את מטאות המאבק של התנועה שבחרתם, ו\textbf{הסבירו} כיצד סייעה הפעילות הדיפלומטית לקידום \textit{אחת} מהמטרות האלו. 
			
			\textbf{תשובה: }\\
			\hl{לאומיות} היא תהליך שהחל במאה ה־19 ונמשך עד ימנו, בו קבוצה של אנשים בעלי עקרונות, מאפיינים, ולרוב עבר משותף מנסים להקים \hl{מדינת לאום} על שטח אליו הם חשים חיבה (ה''מולדת``). התנועה האחראית להשגת מטרה זו ושמירת עקרוניהם קרויה \hl{התנועה הלאומית}. תנועות לאומיות רבות מנסות לפעול באמצעות כלים דיפלומטיים, כמו נסיון לגרום למדינה לקחת חסות על תנועה לאומית ולהלחם בשמה, או קבלת אישור מאימפריה להקים אוטונומיה על שטחה. 
			
			בתחילת המאה ה־19, גרמניה הייתה מחולקת למספר רב של שטחי שליטה ריבונות קטנים, שנמלכו או היו תחת חסות של ממלכת פרוסיה ואוסטריה. העם הגרמני החל להתממש ולהתחזק, והיה רצון להקים מדינת לאום מאוחדת לעם הגרמני. הגוף שניהל את המאבק להקמת אותה המדינה היה \hl{הפרלמנט הגרמני}. מטרת התנועה הלאומית הגרמנית הייתה בראש ובראשונה להקים מדינת לאום מאוחדת לעם הגרמני, שלא מפורקת למספר מדינות ומדינות חסות שונות. טרם זאת, התנועה הלאומית הגרמנית ניסתה (בהצלחה) להתשחרר משלטון נפוליאון. 
			
			התנועה הלאומית הגרמנית גם שקדה על איסוף, שימור ועיבוד התרבות הגרמנית, שלימים הפכו לבסיס לבניית הזהות הלאומית גרמנית. בין האנשים שתרמו לכך ניתן למנות את \hl{האחים גרים} שכתבו ופרסמו סיפורים עממיים רבים. 
			
			הפרלמנט הבין שאין ערך במאבקים לאומיים מבלי כוח צבאי, ולכן פנה הן אל אוסטריה והן אל פרוסיה, שסירבו. אך ב־1964 \hl{אוטו פון ביסמק} מונה לראש ממשלתו של ויליהם הראשון, מלך פרוסיה. בראי אביב העמים, ביסמק הבין את יכולת המאבק הלאומי, ובסדרה של מהלכים גרם לדנמרק, ואז לאוסטריה וצרפת, כל אחת מהן לפתוח במלחמה נגדו (שהוא בעצמו עורר באמצעים דיפלומטיים). תוך ניצול הרצון העז של העם לגיבוש מדינת לאום, והכנות חזקות ויציבות בהרבה מאלו של הצבאות היריבים, הוא ניצח במהרה בקרבות אלו. באמצעות המהלכים הגיפלומטיים הללו, התנועה הלאומית הגרמנית השיגה את המטרה הכי חשובה שלה – הקמת מדינת לאום גרמני מאוחדת. 
			
		\end{enumerate}
	\subsection{הגורמים לצמיחת התנועות הלאומיות והציונות}
		\begin{enumerate}[A.]
			\item \textbf{שאלה: }\\
			על צמיחת הלאומיות השפיעו גורמים משלושה תחומים: אירועים פוליטיים־היסטוריים, רעיונות (אידיאולוגיה), ושינויים במצב הכלילי־חברתי. \textbf{הציגו} גורמים מ\textit{שני} תחומים שונים (גורם אחד מכל תחום), ו\textbf{הסבירו} את ההשפעה של \textit{כל אחד} מהם על צמיחת הלאומיות באירופה. 
			
			\textbf{תשובה: }\\
			רעיון הלאומיות הוא רעיון שהתפתח במאה ה־19 ובו אנשים בעלי דעות ותחושות משותפות, מנסים להקים \hl{מדינת לאום} על שטח שהם בעלי רגש אליו הקרוי \hl{מולדת}. מקור הלאומיות באירופה, אך היא התפשטה בהדרגה למקומות נוספים בעולם. הקמת מדינת הלאום נעשית ע''י \hl{תנועה לאומית}, שמטרתה היא להקים את המדינה, וליצור ולתחזק \hl{זהות לאומית} שתחזיק את בני הלאום מאוחדים. התחזקות וצמיחת תנועות הלאומיות התרחשה בעקבות מספרים גורמים, חלקם אידיאולגיים, חלקם בעקבות שינויים במצב הכלכלי־חברתי, וחלקם בעקבות אירועים פוליטיים. 
			
			ראשית כל, נפרט על האירועים הפוליטיים שהובילו לחיזוק התנועות הלאומיות. הראשון מבין האירועים הללו הוא \hl{המהפכה האמריקאית}. האמריקאים במרכז־מערב אמריקה הצפונית הצליחו לגרש באמצעות מאבק מזוין את השלטון הבריטי, ולהקים ערים אמריקאיות, חוקה, בית משפט, צבא, ודמוקרטיה מתפקדת. בכך, הם הוכיחו לעולם שמהפכה ונצחון על הכוחות הגדולים הקיימים היא אפשרית. אמריקה חרטה על דגלה את ערכי הנאורות שיפורטו בפסקה הבאה, מה שעזר להפיצם בעולם. האירוע השני הוא \hl{המהפכה הצרפתית}, שהתחילה בנצחון על המלוכה והאריסטורקטיה, (והמשיכה בפירוק מנגוני המדינה, קריסה לאנרכיה, ומעבר למשטר טרור, אבל על זה אסור לדבר) ולאחריה \hl{כיבושי נפוליאון} ברחבי אירופה. מאידך הכיבושים הנרחבים גרמו להתנגדות רבה מצד הגרים באותם שטחים כבושים, אך הרצון להתנער מכיבושי נפוליאון יחדיו עם העובדה שנפוליאון הפיץ את רעיונות הלאומיות בכל אירופה, עזרו להפיץ ולחזק את התקווה לקיום מדינה לאומית. 
			
			עתה, נעסוק באידיאולוגיה. טרם ההתעוררות הלאומית באירופה, הסתובבו באירופה שני רעיונות (סותרים במידת מה) – ראשית, הנאורות, ולאחריה, הרומנטיקה. הנאורות ערערה על המוסדות הקיימים וקידשה את השכל והחשיבה העצמאית. שני ערכים אלו הכרחיים לקיום תנועות לאומיות – הסדר האריסטוקרטי והמלוכני הקיים בחברה לא ממש אפשר קיום מדינה לאומית, והצורך ביצירת זהות עצמאית לאומית לא מתאפשר ללא חשיבה עצמאית. ה\hl{רומנטיקה} שהגיעה כתגובה לנאורות עסקה באדם וברגשותיו, וכן בחקר העבר והטבע, עקורנות שעליהם התבססו הזהויות הלאומיות שהתפתחו לאחר מכן. בעקבות ההתרחקות והערעור על הכנסייה, החל תהליך ה\hl{חילון} בו אנשים מתרחקים מהדת כמוקד זהות עקרי. תהליך החילון גרר בעקבותו חיפוש אחר מוקד זהות חדש, שרבים מצאו בתנועות הלאומיות. 
			
			אחרונה, נדבר על השינויים במצב הכלכלי והחברתי. התרחשה \hl{המהפכה האגררית} (התייעלות יצירת המזון, שאפשרה עבודה של יותר אנשים בערים ומפעלים שלא עוסקים באוכל באופן ישיר) והחל \hl{תהליך התיעוש} ו\hl{המעבר אל העיר}. התהליכים גררו ריכוז רב של אנשים במקומות מטונפים וקרים, ולרוב גם התרחקות מהבית וממוקדי הזהות. התהליכים הללו גרמו לשתי תוצאות חיוביות בעבור הלאומיות – ראשית, החיפוש אחר מוקד זהות חדש הוביל אנשים רבים לפנות אל התנועות הלאומיות. הריכוז של האנשים רבים על שטח קטן עזר לתפוצה של הרעיונות הלאומיים בצורה מהירה יותר, תפוצה שהתחזקה אף יותר בזכות \hl{מהפכת הדפוס} שאפשרה לאדם הממוצע לקרוא ולעסוק בנושאים כגון אלו. ההתפשטות המהירה של רעיונות הלאומיות הובילו לחיזוק התנועות הלאומיות. 
			
			נתבונן בקטע המקור ''הד המהפכה באמריקה`` (בספר ''הלאומיות בישראל ובעמים``, הוצאת כנרת 2014, עמוד 26) העוסק בהשפעת המהפכה האמריקאית על התנועות הלאומיות. לפי קטע המקור, ''גבורתם של הרפובליקאים החדשים באמריקה רכשה לא הוקרה ברחבי אירופה וזכתה לאהדתם של כל שוחרי הצד וההומאניות, ובפרט של האנשים הצעירים`` – עקרונות הליברליזם, הנאורות, והערעור על מוסדות המלוכה שעליהם התבססה המהפכה המריקאית, נפוצו בקרב כל אירופה, ובמיוחד בדור המהפכני והחדש שגדל בה. אף בצרפת שהייתה שנים תחת שלטון מלוכני, ''קשה לתאר את ההתלהבות, בה נתקבלו בצרפת, בתוך הממלכה העתיקה, נציגיו של עם אשר התקומם נגד מלכו`` – העם הצרפתי המשולהב ראה את ההצלחה נגד בריטניה, ורצה לממשה גם בארצו, מה שאכן התרחש כעשור לאחר מכן. 
			
			\item \textbf{שאלה: }\\
			\textbf{הציגו} \textit{שניים} מן השלבים בהתפתחות הפעילות של תנועות לאומיות באירופה, ו\textbf{הסבירו} את התרומה של \textit{אחד} מהשלבים שהצגתם לעיצוב תודעה לאומית. 
			
			\textbf{תשובה: } \\
			\hl{בני לאום} הם אנשים בעלי אידיאולוגיה, מטרות, עקרונות ומאפיינים משותפים המעוניינים להקים \hl{מדינת לאום} (או \textit{מדינה לאומית}) בעבור בני אותו הלאום. \hl{תנועתם הלאומית} פועלת לשם השגת מטרה זו. התנועות הלאומיות החלו להתפתח במאה ה־19, באירופה ואז גם בשאר העולם. אחת ממטרות התנועה הלאומית היא פיתוח \hl{תודעה לאומית}, שמגדירה מוקד זהות וכוללת מאפיינים רבים (כמו שפה משותפת, עבר והיסטוריה משותף, וכמיהה למולדת). התודעה הלאומית המשותפת עוזרת לאחד את בני הלאום השונים ולהגדיר שאיפות משותפות. 
			
			הרעיונות הבסיסיים של התנועה הלאומית ופיתוח התודעה הלאומית לרוב מתבצע על־ידי קבוצה מצומצמת של משכילים שהושפעו מרעיונות הסובבים סביבם. אלו מגייסים אנשים, בעיקר ממעמד הבורגנות (מעמד הביניים, אנשים העוסקים בעתשייה, סחר ומקצועות חופשיים, אך אינם בעלי מעמד אצולה אריסטוקרטי) שמנסים להשיג את המטרות שהוצבו באמצעים שונים. רוח היוזמה והשינוי שאפיינה את הבורגנים, וכן השכלתם, אפשרו להם להשתלב בקלות להיות האנשים שיתחברו ראשונה אל התנועות הלאומיות. 
			
			בשלב הבא, אחרי שהתנועה הלאומית התבססה דיה, לרוב התנועה הלאומית יצאה ל\hl{פעילות פוליטית}, ולפיכך נשאו ונתנו עם השליטים המקומיים במטרה להקים מדינת לאום (בין אם באמצעות פירוק מעצמות והחלפתם במדינות לאום, ובין אם החלפת השלטון בשלטון לאומי). הפעילות הפוליטית התבטאה לעיתים בהפגנות ונסיונות הפיכה, ובעיתות אחרות באמצעות גיוס שליטים בעלי עוצמה צבאית ומדינית שתמכו ברעיון הלאומיות או שהתאים להם לתמוך ברעיון הלאומיות. 
			
			השלב הראשון, בו המשכילים ממציאים וחוקרים את ההיסטוריה של הלאום, הוא הכרחי ליצירת התודעה הלאומית, שמתפתחת ומופצת בשלב זה. 
			
			\item \textbf{שאלה: }\\
			''האמנציפציה שניתנה ליהודים במרכז ומערב אירופה הייתה חוקית, אך לא הוכיחה עצמה מבחינה חברתית``. \textbf{הסבירו} טענה זו, והביאו \textit{שני} נימוקים היסטוריים לביסוס דבריכם. 
			
			\textbf{תשובה: }\\
			במהלך בסוף המאה ה־19 והמאה ה־20, התרחש במערב אירופה (תחילה בצרפת ואז במקומות אחרים) תהליך ה\hl{אמנציפציה} – הכרה ביהודים כשווי זכויות (משפטית) כמו כל שאר האוכלוסיה באותן המדינות. תחילה, האמנציפציה ניתנה באופן קהילתי לריכוזי יהודים, ובהמשך ברמה האישית. האמנציפציה ביססה בחוק את שוויון הזכויות גם ליהודים. עם זאת, היא גרמה לתהליכים שהתנגדו אליהם, גם מחוץ לקהילות היהודיות אבל אף בתוכן. 
			
			\hl{אנטישמיות מודרנית} (או בקיצור ''אינטישמיות`` בשאלה הזו) היא שנאה כלפי יהודים, שלא נובעת ממניעים דתיים (דהיינו, הרג ישו, ורדיפת הכנסייה הקטולית את היהודים), אלא משנאת ה''גזע`` היהודי וכל מי שנולד כיהודי. לאנטישמיות מודרנית יש הבטים פוליטיים וחברתיים, כי לרוב אותם אנטישמיים יעדיפו למנוע מיהודים כל משרה ציבורית (ולעיתים כל משרה או חיים בכלל). 
			
			תהליך האמנציפציה גרם \hl{להתעבבות והתבוללות} היהודים בתוך החברה של המדינות בה הן נמצאו. היהודים הצליחו בכל אירופה באופן לא פורפורציונלי לשיעורם באוכלוסיה, מה שגרם לעלייתה והתחזקותה של האנטישמיות המודרנית. 
			
			תוך כדי כך, ההתבוללות בקרב הלא יהודים גרמה לרתיעה ופחד של הקהילות היהודיות, מחשש להתבוללות בגולה, ושהעם היהודי יאבד את ייחודו ויעלם תוך כדי התמזגות לדתות אחרות. 
			
			נתייחס למשפט ''האמנציפציה שניתנה להידים במרכז ובמערב אירופה הייתה חוקית, אך לא הוכיחה עצמה מבחינה חברתית``. אכן, האמנציפציה הייתה חוקית וכיבדה חוקים בסיסיים של שוויון זכויות (מתוקף היותה תהליך שמקדם זכויות זהות לאוכלוסיית מיעוט, היהודים). עם זאת, החברה סביב היהודים התקשתה לקבל אותם אליה והאנטישמיות התחזקה, וגם הקהילה היהודית לא הייתה מרוצה מהתוצאות שהובילה האמנציפציה ופחדה מפני התבוללות. בשני ההיבטים הללו, האמנציפציה לא הוכיחה את עצמה מבחינה חברתית. 
			
		\end{enumerate}
	\subsection{מאפייני ודפוסי פעילות תנועות לאומיות עד 1914}
		\begin{enumerate}[A.]
			\item \textbf{שאלה: }\\
			\textbf{הציגו} את פעילותו של אליעזר בן יהודה לחישוב השפה העברית, ו\textbf{הסבירו} \textit{על פי הקטע ועל פי מה שלמדתם}, מהי חשיבות השפה לליכוד בני הלאום. 
			
			\textbf{תשובה: }\\
			במהלך המאה ה־19 התפתחה באירופה וברחבי העולם תופעה ששמה \hl{הלאומיות}, המאגדת קבוצה של אנשים [=\hl{בני לאום}] בעלי מאפיינים משופים וטוענת כי לכל לאום יש זכות להקים \hl{מדינת לאום} הפועלת לפי עקרונותיו. \hl{התנועה הלאומית} של הלאום תנסה לפעול להקמת מדינה בעבור אותו הלאום. לאום מלוכד וחזק בסבירות גבוהה יותר יצליח במשימתו הלאומית. 
			
			ישנם מספר מרכיבים שעוזרים ללכד את בני הלאום. אחד מהם, הוא \hl{שפה משותפת}. שפה משותפת יוצרת תחושת שייכות בין בני הלאום, ומאפשרת להם לדבר בינהם בחופיות בצורה שמאפשרת ללאום לקדם את מטרותיו יחדיו. השפה מהווה חלק מהתרבות והתודעה הלאומית בני הלאום, ומאפשרת לו להעביר מסרים ולהפיץ את האידיאולוגיה שלו בין בני הלאום. 
			
			\hl{אליגזר בין־יהודה} היה יהודי שנולד ברוסיה ב־1858, והושפע מילדותו מרעיון הלאומיות. בפרט, הושפע רבות מ\hl{התנועה הציונית}, היא התנועה הלאומית היהודית. ב–1881 העברית שמשה כשפת קודש גרידא ולא היה בה שימוש בחיי היום־יום, והוא פעל לקדם את רעיון הפצת העבירת כשפת חולין מדוברת על הלאום העברי. 
			
			על אף התנגדות רבה, בן־יהודה המשיך לפעול. לשם כך, פרסם עיתונים ציוניים בעברית כמו ''הצבי``. הוא גם הקים חברות שקידמו את לימודי העברית במוסגות הלימוד הציוניים. ב־1908 בן־יהודה פרסם מילון עברי המתאר, בן היתר, מילים חדשות שמציא ע''מ לחדש את השפה התנ''כית ולהתאימה לחיי היום־יום המודרניים. 
			
			בכך, בן יהודה קידם ופיתח שפה משותפת, שתלכד את בני הלאום הציוני ותעזור להם לתקשר בינהם, במטרה להשיג את מטרותיה של הציונות. 
			
			נתבונן בקטע המצורף לשאלה זו המתאר את הדברים החשובים לאיחוד בני הלאום, מאת שמואל יבניאלי, ''ספר הציונות א'``. לפיו, ''אם חפץ ונחפוץ להשכיל ולחיות, עלינו לפתוח בהשכלת רוחנו [...] ובמה אם לא בשפה העברית?``. יבניאלי מציג שאלה קטורית, שמטרתה להמחיש את חשיבותה של העברית לתקשורת ביננו – בלעדיה, לא נוכל לתקשר, וכך ימנע מאיתנו להשכיל וללמוד וכן תפגע יכולתה של התנועה הלאומית לפעול. יבגני גם מדבר על חשיבותה של העברי תלאיחוד בני הלאום – ''נשים נא לפני בני נעורינו השכלה עברית [...] וחדלו להיות היהודים מהיות שני מחנות`` – דהיינו – אחת שנלמד ונתקשר בעברית, נפסיק להתנהג כאילו אנו מחנות שונים ונפעל ביחד. בכך יבגני מראה ששפה משותפת מאחדת את בני הלאום ומגבשת תודעה לאומית אחידה בעבור כולם, בצורה המאפשרת להם לפעול יחד בצורה אחידה. 
			
			\item \textbf{שאלה: }\\
			בחרו מקור מספר הלימוד המבטא תחום פעילות אחד של התנועה הציונית בארץ ישראל \textit{בין השנים 1981-1914}. \textbf{הציגו} את המקור שבחרתם ו\textbf{הסבירו} כיצד הוא מבטא תחום זה. 
			
			\textbf{תשובה: }\\ 
			במהלך המאה ה־19 התרחשה באירופה (תחילה במערב אירופה, ומשם התפתחה לשאר העולם) תופעת הלאומיות, תופעה שבה קבוצה של אנשים בעלי מאפיינים משותפים מנסים להקים מדינת לאום. אותם אנשים קרויים בני הלאום. תנועה לאומית לרוב מוקמת (אם בצורה מאורגנת ואם לאו) על־מנת להקים מדינת לאום זו, ולשמר על התודעה הלאומית – האוסף של העקרונות הזהות של בני הלאום. 
			
			התנועה הציונית היא התנועה הלאומית של העם היהודי. היא התפתחה בסוף המאה ה־19, ומטרתה להקים מדינה יהודית בארץ ישראל. אחת מהדרכים הללו הייתה כיבוש העבודה – רעיון לפיו אם הישראלים יוכלו ליצר אוכל לעצמם, לעבוד את שדותם בעצמם, ולמעשה, לבנות כלכלה שלמה ולא להיות תלוי בעובדי אדמה סביבך (כמו רוב היהודים באירופה), הם קיוו להיות קרובים יותר להקמת מדינה. פרט לכך, הקמת ישובים בארץ נעשתה במטרה להגדיל את האחיזה היהודית בה, כדי שביום מן הימים תקום בה מדינה. כיבוש העבודה והקמת הקבוצות שתרמו לכיבוש העבודה, תוך מימון חלקי מטעם התנועה הציונית, הייתה אבן דרך לקראת הקמת מדינת ישראל, בתחום העלייה וההתיישבות. 
			
			נתבונן בקטע המקור ''על הקמת הקבוצה בדגניה`` (עמוד 155 בספר ''הלאומיות בישראל ובעמים``, הוצאת כנרת 2014). קטע המקור הוא קטע מספר שכתב יוסף ברץ על הרעיון הציוני, והוא מדבר על האידיאוגיה סביב הקמת הקבוצה בדגניה. 
			
			לפי קטע המקור, ''כיבוש העבודה במושבות הקיימות לא סיפק את רוח החברים [...]`` – קטע המקור תומך בכיבוש העבודה, המהווה חלק חשוב בתחום העלייה וההתיישבות. המקור ממשיך ומפרט – ''עלינו איפה לאחוז [...] בהתיישבות המיוסדת על עבודה עצמית ועל תוצרת הבאה לספק את צרכי המקום`` – בהתאם לרעיון כיבוש העבודה, היהודים החדשים יבנו בעצמם את הנדרש לקיום הכלכלה של המדינה שעתידה לקום. 
			
			\item \textbf{שאלה: }\\
			\textbf{הציגו} שלושה מרכיבים נוספים (מלבד שפה) המלכדים את בני הלאום. בחרו קטע מקור מספר הלימוד ו\textbf{הסבירו} כיצד \textit{שניים} מהמאפיינים שהצגתם באים לידי ביטוי במקור זה. 
			
			\textbf{תשובה: }\\
			במהלך המאה ה־19 התפתחה באירופה ובשאר העולם תופעת ה\hl{לאומיות}, לפיה קבוצה מסויימת בעלת מאפיינים משותפים [=\hl{הלאום}] שאנשיה קרויים \hl{בני הלאום} יפעלו ויהיו זכאים למדינה משלהם. מדינה זו תקרא \hl{מדינת לאום}. התנועה שתפעל להקמת מדינת לאום זו תקרא \hl{התנועה הלאומית} (של הלאום). לאום מלוכד וחזק יצליח יותר בקלות להגשים את מטרותיו להקמת מדינה לאומית. אופן ליכוד בני הלאום מורכב ממספר מאפיינים, כגון היסטוריה משותפת, ארץ מולדת משותפת אליה רוצים להגיע, ותרבות משותפת. כל המאפיינים הללו ביחד עוזרים ליצור \hl{תודעה לאומית} אחידה המאחדת בין בני הלאום. הזהות הלאומית עוזרת למצוא בתנועה הלאומית מקור זהות בעבור אנשים המחפשים קבוצה להזדהות איתה. 
			
			\hl{תרבות משותפת} תעזור לבני הלאום להרגיש מחוברים אחד לשני, זאת באמצעות (בין היתר) שירה, ספרות, מוזיקה ועוד, המייחדת את אותו הלאום, וגורמת לו להרגיש כאילו יש משהו מיוחד בינהם. 
			
			\hl{מולדת} אליה בני הלאום רוצים להגיע (ולרוב רוצים להקים מדינה לאומית), עוזרת להציב שטח אדמה ברור שתנועה הלאומית לרצה להקים מדינה לאומית עליו, ובכך תאחד את בני הלאום לכדי מטרה אחת. נוסף על כך, היא תגרום להם להרגיש ביחד, שכן מטרתם ואהבתם היא המולדת המשותפת בינהם. בני הלאום מוכנים להקריב מעצמם בשביל המולדת, כלומר להלחם ולהסתכן בשביל להקים עליה מדינה לאומית. 
			
			מרכיב אחד שמלכד את בני הלאום הוא \hl{היסטוריה משותפת}. רבים הלאומים שמצאו בהיסטוריה שלהם דבר חיובי ומשותף שמאחד אותם יחדיו. ההיסטוריה המשותפת לרוב מהווה חלק ניכר בתודעה ובזהות הלאומית, שהכרחית הן בשביל לגייס אנשים ללאום, והן בשביל לשמור עליו מאוחד. 
			
			נתבונן בקטע ''מיהו איטלקי`` המופיע בספר ''הלאומיות בישראל ובעמים`` (הוצאת כנרת 2014, עמוד 18). בקטע כותב אדמנדו דה אמיציס שנולד ב־1846 ופעל למען התנועה הלאומית האיטלקית, מה מגדיר איטלקי, ולמעשה פורש את התודעה הלאומית שהוא תופש עצמו חלק ממנה. ברמה הסמנטית, הוא משפר מדוע הוא אוהב את איטליה. לפי דבריו, בענותו על השאלה הרטורית ''מדוע אני אוהב את איטליה``, השיג לעצמו ''מפני שאיטליה היא האדמה שבה קבורים אבותי`` וכן ''הטבע הסובב אותי וכל הנגלה לעיני [...] הוא איטלקי``. בכך, נבחין שאהבתו העזה לאדמת איטליה, המולדת המשותפת לעם האיטלקי, מהווה מבחינתו חלק ניכר מהזהות הלאומית האיטלקית. נוסף על כך, הוא אומר כי הוא אוהב את איטליה גם כי ''לשון דיבורי, הספרים שעליהם אני מתחנך – הם איטלקיים``. נסיק שגם הספרים והתרבות הם סיבה לרגשותיו הלאומיים העזים, וגם הם חלק משמעותי מהתודעה הלאומית שלו. 
			
			לסיום, תרבות, מולדת והיסטוריה משותפת עוזרים לליכוד בני הלאום ויצירת תודעה לאומית אחידה, כמו שראינו בטקסט ''מיהו איטלקי``. 
			
		\end{enumerate}
		
	
	\npage
	
	\section{גרמניה הנאצית, מלחמת העולם השנייה והשואה}
	\subsection{האידיאולוגיה הנאצית ובניית המשטר הנאצי (1)}
	\textbf{שאלה: }\\
	כרזה זו [תמונה על טופס השאלות] נוצרה בתקופת שלטונה של המפלגה הנאצית בגרמניה כחלק מהמאמץ התעמולתי של המשטר הנאצי. \textbf{הסבירו} איזה עקרון של האידיאולוגיה הנאצית בא לידי ביטוי בכרזה, \textbf{הסבירו} צעד או אירוע אחד מתחום הטרור במדיניות הנאצית בין השנים 1933-1939 שנועד להפוך את גרמניה לנאצית, \textbf{הסבירו} את חשיבותה של התעמולה בבניית המשטר הנאצי, ו\textbf{הסבירו} מדוע היה חשוב למשטר הנאצי לפעול גם בתחום החקיקה. 
	
	\textbf{תשובה: }\\
	בשנת 1933 בגרמניה, עלתה לשלטון ''המפלגה הנציונאל־סוציאליסטית`` או ''הנאצים``, ומנהיגה היה \hl{אדולף היטלר}. המפלגה עלתה לשלטון על רקע משברים מדיניים וכלכליים ברפובליקת ווינאמר הגרמנית, ומאוד מהר הפכה את גרמניה לדיקטטורה מוחלטת. לנאצים הייתה \hl{אידיאולוגיה מובהקת}, שפותחה על־ידי אדולף היטלר ונכתבה מפורשות בספרו מיין קאמפ (''המסע שלי``), הכוללת מאפיינים כמו תורת הגזע, מנהיג יחיד, אנטישמיות מודרנית, עקרון המנהיג, הלאום כערך עליון, שלילת הדמוקרטיה והקומוניזם, ומרחב מחיה. 
	
	תורת הגזע היא עקרון האידיאולוגיה הנאצית המבוסס על עיוות תורת הברירה הטבעית של צ'ארלס דרווין. \hl{עקרון הברירה הטבעית} אומר כי מבין המינים בטבע, המתאים ביותר לסביבתו שורד וממשיך להתפתח ולהעמיד צאצאיים. לפי היטלר, בני האדם לא נולדו שווים ודמם מכתיב לאיזה גזע הם משתייכים, ובין הגזעים יש מלחמה מתמדת בה החזק ביותר שורד. לפיו, זהו אופן פעולתו הטבעי של הטבע. היטלר חילק את הגזעים לשלושה גזעיים עיקריים – ה\hl{ארים} (ובפרט הגרמנים הטהורים), הסלאביים (גזע נחות שמטרתו להוות עבד ולשרת את הארים), וכאלו שאינם גזע ומשחיתים את התרבות (בינהם נמנים היהודים). 
	
	ה\hl{אנטישמיות} בתורת הגזע הנאצית היא אנטישמיות מודרנית, כלומר היא מובססת גזע ולא דת. לפיה, היהודים מנסים להשחית את התרבות, באמצעות השתלטות על התעשייה ובאמצעות זיבול הדם הארי הטהור (נישואי תערובת). האנטישמיות הנאצית היא עקרון בפני עצמו של האידיאולוגיה הנאצית. האידיאולוגיה גורסת שהיהודי הוא אויב העם הגרמני, וככזה, יש להוקיע אותנו מגרמניה. היטלר ניסה להשיג מטרה זו באמצעות גירוש ישיר מגרמניה, עידוד הגירה עקיף, פגיעה כלכלית ביהודים והדרתם מהחברה הגרמנית. 
	
	המשטר הנאצי הצליח להשיג את מטרותיו באמצעות טרור המפונה כלפי האזרחים, כלומר חקירה, עינוי, מאסר ורצח, שנועדו להפחיד מהתנגדות לשלטון. המשטר הנאצי גם פעל באמצעות תעמולה נרחבת, כלומר סרטים, כרזות, נאומים, שליטה בעיתונות ובתקשורת ועוד, באמצעותה הוא הפיץ את האידיאולוגיה הנאצית וחיזק את תפיסתה בקרב העם. 
	
	המשטר הנאצי תומך בגישה של מפלגה אחת, כלומר, המפלגה הנאצית היא יחידה, ולכן מנהלת ושולטת על כל אמצעי המדינה. בין היתר, נגרר מכך שלא ייתכנו בתי משפט עצמיים, שלא נתונים לשליטה ישירה של המפלגה הנאצית. כדי לשלוט בבתי המשפט וגם כדי לתת לגיטימציה חוקית לשלטון הנאצי, הנאצים פעלו באופן פעיל בתחום החקיקה. 
	
	התעמולה הכרחית על מנת להפיץ ולפזר את האידיאולוגיה הנאצית, ולוודא שכל קצוות החברה תומכים (ברצון או שלא ברצון) באידיאולוגיה הנאצית. 
	
	נבחין שביוני 1934, רהם, מפקד ה־SA (ארגון משטרתי לכאורה) הביע ביקורת על מדיניות הפנים שהנהיג היטלר. מאוחר יותר אף נודע להיטלר שרהם מתכנן הפיכה כנגדו. כיאה לטרור שהשליט היטלר, היטלר עצר את רהם וכ־1000 מפקידי ה־SA  והוציא אותם להורג. בכך, היטלר הבהיר מה יהיה דין מתנגדיו, והפך את ה־SA לארגון שולי שהוחלף בידי ה־SS הנאמן יותר. אירוע זה נקרא \hl{אירועי ליל הסכינים הארוכות}. 
	
	עתה, נתבונן בכרזה המתוארת בשאלה. בכרזה מוצגת אישה ארית צעירה, ומאחוריה רשת עכביש על עכביש בעל פנים עם מאפיינים יהודים לכאורה (אף ארוך ועקום, אוזניים גדולות וחיוך זדוני) עליו סמל של מגן דוד המתבונן בה. עקרון האנטישמיות בא לידי ביטוי בכרזה. היהודי המוצג כעכביש (חיה עם קונוטציה זדונית) מוצג כמי שמאיים על האישה, שמא יתמא את דמה בנישאי תערובת ויפגע בטוהר הדם הארי. 
	
	\subsection{האידיאולוגיה הנאצית ובניית המשטר הנאצי (2)}
		\begin{enumerate}[A.]
			\item \textbf{שאלה: }\\
			\textbf{הציגו} את העקרון מתוך האידיאולוגיה הנאצית שבא לידי ביטוי בקטע [הכרזה המופיעה על גבי דף השאלות]. בחרו במקור מילולי בספר הלימוד המבטא את אותו העקרון. הסבירו כיצד המקור שבחרתם מבטא עקרון זה. 
			
			\textbf{תשובה: }\\
			''\hl{הנאצים}`` (קיצור של הנציונל־סוציאליסטים) הייתה מפלגה שעלתה בראשות \hl{אדולף היטלר} בשנת 1933 לשלטון ברפובליקת וויאנמר הגרמנית. באותה התקופה היו משברים מדינתיים וכלכלים ברפובליקה, והיטלר, מנהיג המפלגה, ניצל אותם כדי להפוך את מפלגתו ליחידה בגרמניה במהלך השנים הבאות. לנאצים הייתה אידיאלוגיה ברורה שנכתה על־ידי היטלר בספרו מיין־קאמפ (''המסע שלי``). האידיאולוגיה כוללת מאפיינים כמו תורת הגזע, מנהיג יחיד, אנטישמיות מודרנית, עקרון המנהיג, הלאום כערך עליון, שלילת הדמוקרטיה והקומוניזם, ומרחב מחיה לארים. 
			
			כחלק מהמאפיין \hl{אנטישמיות} של האידיאולוגיה הנאצית, הנאצים האמינו שהיהודי הוא אוייב העם הגרמני, וככזה, יש להוקיע אותו מגרמניה. היהודים הם משחיתי תרבות, ומנסים להשתלט על התעשייה. השנאה ליהודים מבוססת דת ולא גזע, כלומר, מי שנולד להורים יהודים או שאף סביו יהודים, יחשב יהודי (או יהודי למחצה), ללא תלות בדתו או אמונתו, או רמת פטריוטיו לעם הגרמני. היהודי הוא טפיל של החברה, שלא תורם לה ומנצל אותה לצרכיה שלו. 
			
			בקטע המקור המופיע על דף השאלות, מוצגת כרזה נאצית עכביש עליו מצויר מגן דוד, ואנשים רבים התלויים ברשתו. הצגת היהודים כעכבישים מרושעים עוזרת לנאצים להמחיש שהיהודים אוייב האומה הגרמנית, מנסים לתפוס ולגרום רעה לאנשים, והם מייצגים את תמצית הרוע. 
			
			נתבונן במקור המילולי מספרו של היטלר ''מיין־קמאפ`` (עמוד 70, ''נאציזם, מלחמה ושואה``, הוצאת היי־סקול 2014), ספציפית בקטע החמישי. לפיו, ''היהודי מעולם לא היה נווד, אך מאז ומתמיד היה טפיל המשמין על חשבונם של אחרים``. היטלר פורש את דעתו האנטישמית וטוען שהיהודים לא תורמים לכלכלה ולחברה, ומנצלים אותה גרידא. נמשיך – ''תוצאות הלוואי של הנוכחות היהודית הם כתוצאות הלוואי של ערפד. בכל מקום בו היהודי מבסס את עצמו, נידונו מארחיו לדמם למוות במוקדם או במאוחר`` – גם כאן, היטלר טוען שהיהודי מנסה להשמיד את המקום בו הוא נמצא, והוא אויב החברה והעם, בהתאם למאפיין האנטישמיות באידיאולוגיה הנאצית (שהתבססה על ספר זה). 
			
			\item \textbf{שאלה: }\\
			''לאחר עליית הנאצים לשלטון פעל המשטר המאצי כדי להרוס את הדמוקרטיה באמצעים דמוקרטיים.``. \textbf{הסבירו} טענה זו. בססו את דבריכם על שתי דוגמאות מתחום החקיקה. בחרו מקור מספר הלימוד המבטא תחום פעולה אחר שבו השתמשו הנאצים לביסוס שלטונים בין השנים 1933-1935. הסבירו כיצד המקור שבחרתם מבטא תחום זה. 
			
			\textbf{תשובה: }\\
			בשנת 1933 עלתה לשלטון המפלגה הנציונל־סוציליסטית, או בקיצור, \textbf{הנאצית}, תחת השלטון של \hl{אדולף היטלר}, מנהיג המפלגה שהכתיב את האידיאולוגיה שלה בספרו מיין־קמאפ. האידיאלוגיה הנאצית שמותת על עקרונות תורת הגזע, אנטישמיות, אנטי־קומוניזם ואנטי־דמוקרטיה, מרחב מחיה, הלאום כערך עליון, ועקרון המנהיג. 
			
			הנאצים עלו באופן חוקתי לשלטונם בשנת 1933, ובהתאם לאידיאולוגיה האנטי־דמוקרטית השואפת לקיום מפלגה יחידה – המפלגה הנאצית – פתחו במאמצים להרוס כליל את הדמוקרטיה של רפובליקת ווינמאר. לשם כך, הנאציזם ניצלו את המערכת החוקתית של הרפובליקה כדי להשמיד אותה. לדוגמה, היטלר ב־27 בפברואר ביים את שריפת הרייכסטאג, מה שאפשר לו לקבל סמכות לחוקק באופן חופשי חוקים ללא אישור הרשויות האחרות. לדוגמה, היטלר פרסם את ''הצו נגד מעשי אלימות קומוניסטיים המסכנים את המדינה`` בו היטלר אוסר על קיום כל פעילות קומוניסטית, ומשמיד באופן חוקי את המפלגה הקומוניסטית ובכך פוגע אנושות בחופש הדיבור. בדוגמה נוספת, באמצעות צווים אלו היטלר השיג שליטה על הרייכסטאג (מעין ממשלה), וכך היטלר הצליח לחוקק באופן חוקי חוק שמסמיך את הממשלה (בשליטת היטלר) לחוקק חוקים, המנוגדים לחוקה (כלומר לא חוקיים). 
			
			שתי דוגמאות אלו מדגימות באופן נפלא את האמירה ''לאחר עליית הנאצים לשלטון פעל המשטר הנאצי כדי להרוס את הדמוקרטיה באמצעים דמוקרטיים`` – היטלר ניצל חורים בדמוקרטיה של רפובליקת וויאנמר, כדי להרוס את אותה הדמוקרטיה. 
			
			היטלר פעל בתחומים אחרים, פרט לתחום החקיקה, על מנת לבסס את שלטון הנאצים. אחד מהם, הוא טרור – הפעלת כוח מכוון כלפי מתנגדי משטר אמיתיים ומדומים, כדי ליצור תחושה של פחד מהשלטון ומניעה של התנגדות אמיתית. אחד מהאמצעים להפעיל טרור, היה יצירת מחנות ריכוז למתנגדי המשטר (המדומים והאמיתיים) בו עינה ורצח את מתנגדיו אלו. נבחין זאת בקטע מקור המופיע בספר ''נאציזם, מלחמה ושואה`` (תחתית עמוד 105, הוצאת היי־סקול 2014) המתאר את תקנות מחזה הריכוז דכאו, שנקבעו ע''י מפקד המחנה.
			
			לפי קטע המקור, ''העבריינים הבאים נחשבים כמסיתים ויש לתלותם: כל מי שעוסק בפוליטיקה, מסית בנאומים, יוצר אסיפות [...] אוסף מידע, נכון או כוזב, על המתרחש במחנה־הריכוז`` – כלומר, יצירת אופוזציה למשטר נענתה בתלייה, והשליטה על ידיעת לגבי מה קורה או לא קורה בתוך המחנה – נעשתה ע''י הנאצים, כדי לקבל שליטה על תחושת הפחד הנוצרת בקרב האוכלוסיה. קטע המקור ממשיך בסעיפים נוספים – ''העבריינים הבאים נחשבים למורדים ויש לירות בהם מיד, או לתלות אותם כעבור זמן: כל מי שתוקף תקיפה פיזית שומר או איש אס־אס, כל מי שמסרב לבצע תפקיד [...]``, דהיינו כל התנגדות פיזית לנאצים, דינה מוות. 
			
		\end{enumerate}
		
	\renewcommand{\footrule}{\rule{\linewidth-26pt}{0.25pt}\vspace{-5pt}}
		
	\subsection{משטר הגטאות בפולין בשנים 1939-1941}
		\begin{enumerate}[A.]
			\item \textbf{שאלה: }\\
			\textbf{הסבירו} כיצד השתמשו הגרמנים ביודנרט על מנת לפקח על חיי היהודים בגטאות ומה היו מטאות הגרמנים בהקמת יודנרטים אלה. בתשובתכם הביאו \textit{שלוש} דוגמאות לתפקידי היודנראט בגטאות. 
			
			\textbf{תשובה: }\\
			בשנת 1933, המפלגה הנציונל סוציאליסטית (או בקיצור, הנאצית) עלתה לשלטון ברפובליקת ווינמאר הגרמנית. המפלגה תומכת באידיאולוגיה שפותחה על־ידי מנהיגה אדולף היטלר, הדוגלת בין היתר בתורת הגזע ובאנטישמיות מודרנית. תוך זמן קצר המפלגה הפכה את הרפובליקה מדינה במשטר דיקטטורי נאצי. 
			
			אנטישמיות המודרנית היא תופעה המתארת שנאת יהודים על פי גזע, ולא על פי דת (בניגוד לאנטישמיות הקלאסית). האנטישמיות הנאצית היא אנטישמיות מודרנית המאופיינת בשנאה יוקדת לכל מי שנולד יהודי, ולפיה היהודי הוא טפיל על החברה שמנסה להשתמש בה לצרכיה שלו ולהשמיד אותה. היהודי מנסה לזהם את הדם הארי הטהור בדם יהודי, כדי להשמיד את הגזע הארי העדיף. 
			
			בשנת 1939 פתחה גרמניה במלחמה נגד פולין, שנכבשה מהר, ומשם המלחמה התרחבה לשאר אירופה. בפולין הוקמו גטאות – עיירות סגורות בפולין שהכילו יהודים בכמות לא פורפורציונלית לשטח המחייה בהן, במטרה לרכז את היהודים ולהרחיק אותם מהאוכלוסיה. הנאצים בחרו יודנראטים, אנשים יהודים, לנהל את החיים בגטו בהתאם להוראותיהם. הגטאות לקו במחלות, חוסר במזון, תברואה ירודה, צפיפות, ובעיות אחרות, עקב מעשי הנאצים. היודנראטים תפקדו כמו המשטרה הנאצית בגטאות. 
			
			מספר מטרות היו לנאצים בהקמת יודנראטים אלו. בינהן, יצירת מחלוקות בתוך החברה היהודית – במקום שהיהודים יפנו את זעמם לנאצים וימרדו, היהודים יתעסקו לשנוא את היודנראטים היהודים שתפקידם לנהל את הגטו ביום יום. בכך תגדל הפגיעה ביהודים ותקטן הפגיעה בנאצים. מטרה נוספת, הייתה חסיכה בכוח אדם נאצי. הנאצים לא נאלצו לנהל את הגטו בעצמם, אלא לתת הוראות ליודנראטים בלבד, ולדאוג שיבצעו את מטלות שנתנו עליהם. החסכון בכוח אדם אפשר לנאצים להתמקד במלחמה. המטרה האחרונה שאציין, היא הטעיה והסוואה. בכך שהנאצים הציגו את המצב כאילו חיי היהודים בגטו מנוהלים על ידי יהודים, הם התנערו מאשמה על מצבם הרע של היהודים (שנוצר בזכות פקודותיהם שלהם) ויצרו אשליה כמו והיהודי אכן ייצור נאלח ומלוכלך, בצורה ה''מוכיחה`` את האידיאולוגיה שלהם. כל המטרות לעיל משיגות את מימוש תורת הגזע האנטישמית. 
			
			נתבונן בקטע המקור (עמוד 189 בספר ''נאציזם, מלחמה ושואה`` הוצאת היי־סקול, 2014) הוא קטע מתוך יומנו של אדם בשם ח''א קפלן, המתאר את ההתייחסות ליודנראט בגטו וורשה. בקטע מתאר את השנאה כלפי היודנראט בגטו, בהתאם למטרה הראשונה של הנאצים שציינתי. 
			
			נצטט – ''היודנראט הוא תובעה בעיני יהודי גטו ורשה. כשמזכירים את היודנראט דמו של כל אחד מתחיל לרתוח`` – אכן, בוורשה נוצרה שנאה עזה מצד תשובי הגטו היהודים ליודנראט, שהייתה מופנת ליהודים במקום לנאצים. נמשיך – ''אמללא החחש מפני הנאצים היו מגיעים הדברים לידי שפיכת דם``. השנאה ליודנראטים היא כה רבה, עד כדי כך שיהודים היו מוכנים לרצוח את היודנראט – הנאצים, לא היו רוצים להחליף את תפקידו את היודנראט, ולסבול שנאה עזה זו. בינתיים, היהודים שונאים אחד את השני, במקום לפעול ביחד נגד הנאצים או בשביל לשפר את תנאי החיים בגטו, בהתאם לרצון האנטישמי של הנאצים. 
			\item \textbf{שאלה: }\\
			מתוך ספר הלימוד, \textbf{הביאו} \textit{שני} מקורות חזותייים המתארים את תנאי החיים בגטאות. \textbf{הסבירו} במה תורמים מקורות אלה להבנת תנאי החיים בגטאות, על פי מה שלמדתם בכיתה. 
			
			\textbf{תשובה: }\\
			בשנת 1933 עלתה לשלטון בגרמניה המפלגה הנאציונל סוציאליסטית בראשות אדולף היטלר. היטלר פיתח אידיאולוגיה נוקשה, אנטישמית, שתוארה בספרו ''מיין־קאמפ``. עד שנת 1939 היטלר פתח בסדרה של מהלכים במדינה, כדי לבסס את שלטונו עד לכדי היותו השליט היחיד של המדינה, וחיזוק הכוח הצבאי והמדיני של הרייך השלישי (גרמניה של היטלר). 
			
			האנטישמיות בה תמכה האידיאולוגיה האנצית גורסת שהיהודי משמיד את החברה שבה הוא נמצא, והוא טפיל שלה. היהודי מנסה לנצל את ה''מארח`` שלו, על מנת לחיות על גבי אותו המארח ולבסוף להשמיד אותו. על כן, יש להוקיע את היהודים מן החברה. 
			
			היטלר פתח במלחמה בשנת 1933 שהתחילה בכיבוש פולין. המלחמה התרחבה מהר לכל רחבי אירופה, וכדי לממש את האידיאולוגיה האנטישמית שלו, היטלר רצה לרכז יהודים במחנות כדי להרחיקם מהחברה ולטפל בהם מאוחר יותר. היטלר היה מעוניין להראות לעולם שהיהודים ייצור מטונף, ולא היה לו אינטרס לספק אי אילו תנאי חיים בסיסיים ליהודים שבמותם רצה ממילא. על כן, כמות המזון שסופקה לגטו הייתה מתחת למינימלית, והצפיפות הייתה רבה (כדי לחסוך במקום שמיש ללא יהודים מחוץ לגטו). מאפיינים אלו עודדו התפרצות מחלות. הגטאות שמשו כאמצעי הכחדה עקיף של היהודים. 
			
			נתבונן במקור חזותי (מופיע בתחתית בעמוד 191 בספר ''נאציזם, מלחמה ושואה`` הוצאת היי־סקול 2014). המקור מתאר חייל גרמני מוביל יהודים למעצר, לאחר שנתפסו בנסיון להבריח בבגדיהם מזון לתחומי הגטו. היהודים נראים צעירים ורזים, והחייל הנאצי הולך בתקיפות מאחוריהם. מהמקור נלמד שהרעב היה כה רע באותה התקופה, שילדים הסיכמו לסכן את חייהם, ובתנאי שיקבלו מזון. 
			
			עתה נתבונן במקור חזותי אחר (עמוד 192, באותו הספר). המקור החזותי מתאר ילד צעיר וקטן, כחוש, דומה על סף מוות (אך עם עינים פקוחות), מוטל ברחוב ועוברי אורח לא טורחים להביט בו. מכאן, נסיק שהרעב היה כה רע, עד כדי כך שילדים גססו (ומתו) ממנו. שכיחות הרעב, הייתה גבוהה במיוחד, שכן עוברי האורח לא טרחו להתבונן במראה שכן זהו מראה של מה בכך בתוך הגטו. 
			
			משני מקורות אלו נסיק שתנאי החיים בגטאות היו ירודים במיוחד, המזון שנכנס לא הספיק לקיים את תנאי החיים של כל תושבי הגטו, ולכן ילדים אף סיכנו את חייהם בנסיון להשיג מזון. מכאן שאמצעי ההכחדה העקיף של הנאצים, עבד. 
			
			\item \textbf{שאלה: }\\
			תארו את היחס ליהודים בפולין עד הכניסה לגטאות. הציגו הבדל \textit{אחד} בין המדיניות של הנאצים כלפי היהודים בפולין עד להקמת הגטאות, ובין המדיניות שלהם כלפי היהודים בתקופה שעד פרוץ המלחמה. \textbf{הסבירו} סיבה \textit{אחת} להבדל הזה. 
			
			\textbf{תשובה: }\\
			מפלגה הנציונל־סציאליסטית או בקיצור המפלגה ה\hl{נאצית} היא מפלגה ברפובליקת וויאנמר ברשות \hl{אדולף היטלר} ששלטה בגרמניה בשנים 1933-1945. למפלגה יש אידיאולוגיה שנתונה ע''י ספרו של היטלר ''מיין־קאמפ`` (בתרגום ישיר, המסע שלי), ועקרונותיהם הם שלילת הדמוקרטיה והקומוניזם, אנטישמיות מודרנית, עקרון המנהיג, עקרון מרחב המחייה, הלאום כערך עליון, ותורת הגזע. 
			
			האנטישמיות הנאצית היא אנטישמיות מודרנית, כלומר, היא מובססת גזע ולא דת. הנאצים מטילים על היהודים רבות מן בעיותיהן, ותולים בהן את מצבה העגום של גרמניה בשנת 1933 וכן את ההפסדים במלחמת העולם הראשונה. לפי עקרונות האנטישמיות הנאצים, היהודים מנסים לפגוע בחברה, לנצל אותה לצרכיהם, להשמיד את כל מה שנמצא סביבם, ולטמא את הדם הגרמני הטהור והטוב. 
			
			יחסם של הגרמנים ליהודים בגרמניה הורע בהדרגה, וב־1938 הגיע לשיאו לפני המלחמה. פוגרומים ביהודים היו נפוצים, מאות נהרגו, עשרות אלפים נכלאו, ורובם ככולם נושלו מרוב (או כל) הונם העצמי. אירועים אילו שירתו באופן ישיר את האידיאולוגיה הנאצית – הנאצים הקטינו את האחיזה של היהודים בכלכלה (שהם לכאורה הורסים) והרחיקו את היהודים מהחברה (שהם מאיימים להשמיד). הפוגרומים והכליאות לעיל, היו מאורגנים, ותוכננו בקפידה על ידי רשויות החוק הגרמניות – הגסטאפו (המשטרה החשאית) וה־SS. 
			
			הנאצים פתחו במלחמת העולם השנייה בשנת 1939, על מנת לממש את האידיאולוגיה שלהם ברחבי העולם. תחילה כבשו את פולין תוך זמן קצר. בפולין, המדיניות של הנאצים כלפי היהודים הייתה שונה – בגלל ריבוי היהודים בפולין, המחסור באיזורי מאסר, ואווירת המלחמה, מרבית הפגיעה הגרמנית ביהודים בוצעה באופן לא מנוהל, על ידי הצבא (הווארמכט). יהודים הוכו ללא סיבה ברחוב, הועמדו בפני מעשי השפלה אקראיים, נרצחו באקראי, ונכפו עליהם עבודות כפייה. 
			
			נבחין בהבדל משמעותי – בפולין, הנאצים איימו באופן לא מאורגן בכנסות, עונשי מאסר, עבודות כפייה ומוות, בניגוד לגרמניה, שבה הם נאסרו באופן שיטתי ו''מסודר``. אחת מהסיבות להבדל ביחס היא אווירת המלחמה. לעומת הסדר שהיה בתוך גרמניה, האווירה של המלחמה בקרב החיילים והיעדר רשויות מסודרות, הובילו להתפרעויות ביהודים שלא בהכרח משרתות באופן ישיר לצורכי האידיאולוגיה הנאצית. 
			
		\end{enumerate}
	
	\subsection{המעבר מדמוקרטיה לנאציזם}
		\begin{enumerate}[A.]
			\item \textbf{שאלה: }
			בתהליך המעבר של גרמניה ממשטר דמוקרטי לטוטאליטארי נאצי, בין השנים 1933-1939, ניתן להבחין במספר אירועים מכוננים: [רשימה] \textbf{בחרו} אירוע \textit{אחד} שלדעתכם היה המשמעותי ביותר בתהליך המעבר, \textbf{בססו} את תשובתכם על קטע מקור \textit{אחד} שמופיע בספר הלימוד. \textbf{הסבירו} איזה עקרון מהאידיאולוגיה הנאצית בא לידי ביטוי באירוע שבחרתם. 
			
			\textbf{תשובה: }\\
			בשנת 1933 עלתה לשלטון באופן חוקתי, ברפובליקת ווינמאר הגרמנית, המפלגה הנאציונל־סוציאליסטית, או בקיצור, הנאצים, בגרמניה. למפלגה אידיאולוגיה השוללת את הדמוקרטיה והקומוניזם, אנישמית, מבוססת על תורת הגזע ועקרון מרחב המחייה, וכן כוללת את עקרון המנהיג היחיד (הפיהרר). האידיאולוגיה פותחה על־ידי מנהיג המפלגה, אדולף היטלר, ונפרשה בספרו ''מיין־קמאפ`` (בתרגום חופשי: ''המסע שלי``). 
			
			המשטר הטוטטאליטרי הוא משטר בו יש אידיאולוגיה אחת תקינה, מפלגת המונים אחת שבראשה מנהיג אחד, המפקחת על האזרחים באמצעות משטרה חשאית. לרוב מופעל טרור כלפי מתנגדים אמיתיים ומדומים של המשטר. למשטר יש שליטה מוחלטת על אמצעי התקשורת, אמצעי הלוחמה, והכלכלה, והוא משתמש בהם לצרכיו שלו. 
			
			האידיאולוגיה הנאצית טוענת שלא כל האנשים נולדו שווים, ושהמנהיג התקין והטבעי היחיד של הרייך השלישי הוא היטלר. מתוקף רעיונות אלו, האידיאולוגיה שוללת את הדמוקרטיה. לכן, כאשר היטלר עלה לשלטון בשנת 1933, הוא ניסה להפוך את הרפובליקה עם שלטון טוטאליטיארי נאצי. 
			
			שריפת הרייכסטאג היה אירוע בו היטלר ביים את שריפה ברייכסטאג, בית הנבחרים של הרפובליקה. נתפש אדם קומוניסט שלכאורה ביצע את המעשה, והיטלר הכריז על מצב חירום. אירועים אלו אפשרו להיטלר לחוקק צווי חירום, צווים מיידים שנחקקים ללא אישור הרייכסטאג וללא צורך להתאים לחוקה. נוסף על כך, הוא אפשר להיטלר להטיל את האשמה בקומוניסטים, מתנגדיו הפוליטיים העקריים, ולעודד שנאה כנגדם ואף לחוקק חוקים השוללים את קומוניזם, ובפועל אוסרים ורוצחים את יריביו הפוליטיים העיקריים. 
			
			שריפת הרייכסטאג הייתה האירוע הראשון בהפיכת הרפובליקה מדמוקרטית לטוטאליטרית נאצית, שכן הוא חיזק את המפלגה הנאצית תוך שימוש במאפיינים של משטר טוטאליטרי – שלילית אידיאולוגיות אחרות, והפעלת טרור כלפי מתנגדי המשטר. 
			
			לדעתי, שריפת הרייכסטאג היא האירוע המשמעותי ביותר בהפיכת במעבר של גרמניה ממשטר דמוקרטי לטוטאליטרי נאצי. האירוע היה הפעם הראשונה שנחקקים חוקים אנטי־חוקתיים ברפובליקה, ומה שאפשר את ההדרדרות המהירה ממשטר דמוקרטי למשטר טוטאליטרי. 
			
%			נתבונן בקטע מקור ''
      %TODO
			
			\item \textbf{שאלה: }\\
			הסבירו מה היו המטרות של הנאצים במדיניותם נגד יהודי גרמניה בשנים 1933-1939. \textbf{הציגו} \textit{שתי} פעולות שנקטו הנאצים כדי לממש מטרות אלו. בססו את דבריכם על מקור מספר הלימוד. 
			
			\textbf{תשובה: }\\
			המפלגה הנציונל סוציאליסטית (או בקיצור, הנאצית) היא מפלגה שעלתה לשלטון בשנת 1933 באופן חוקתי ברפובליקת ווינמאר הדמוקרטית הגרמנית. המפלגה תומכת באידיאולוגיה שפותחה ע''י מנהיגה, אדולף היטלר, ובין היתר בתורת הגזע ובאנטישמיות מודרנית. תוך זמן קצר המפלגה הפכה את הרפובליקה מדינה במשטר דיקטטורי נאצי. 
			
			תורת הגזע, היא עקרון לפיהו לא כל בני האדם נולדו שווים, ויש חלוקה בינהן. החלוקה הזו מציבה את ה''ארים`` (בין היתר גרמנים) בראש הפרמידה של הגזעים השונים. לפי התורה, הגזעים נמצאים במלחמת טבע מתמדת, ובסופה החזקים (הארים) אמורים לנצח באופן טבעי. ארים הם יוצרי תרבות, בעוד גזעים אחרים אינם. ישנם גזעים שמשחיתים את התרבות, וגזעים שלא נחשבים כלל גזע – כמו היהודים, שנמצאים בתחתית הפרמידה. ההשתייכות לגזע נקבעת לפי ההורים. 
			
			אנטישמיות המודרנית היא תופעה בה שונאים יהודים על פי גזע, ולא על פי דת. האנטישמיות המודרנית הנאצית מאופיינת בשנאה יוקדת לכל מי שנולד יהודי, ולפיה היהודי הוא טפיל על החברה שמנסה להשתמש בה לצרכיה שלו ולהשמיד אותה. היהודי מנסה לזהם את הדם הארי הטהור בדם יהודי, כדי להשמיד את הגזע הארי העדיף. 
			
			מספר מטרות היו לנאצים בפעולותם נגד היהודים ב־1933-1939. האידיאולוגיה האנטישמית הייתה חזקה בחברה, ובפעולות אנטישמיות עזרו לחזק את מעמד הנאצים בקרב העם הגרמני באותה התקופה. נוסף על כך, הפעילות נגד היהודים אפשרה למשטר הנאצי לממש את שאיפותיו האידיאולוגיות כנגד היהודים. 
			
			בין הפעולות שהנאצים נקטו כדי לממש אידיאולוגיה זו הוא ליל הבדולח ב־1938, אירוע שבו לראשונה אסרו בצורה המונית יהודים וגירשו אותם למחנות ריכוז, בצורה מאורגנת, ונעשה פוגרום בבתי הכנסת, בתי העסק, ובתי היהודים השונים, נוסף על רצח המוני. כ־30000 יהודים נעצרו בצורה מאורגנת ורוכזו במחנות. בכך, הנאצים קיוו ''לטהר`` את גרמניה מיהודים, בהתאם לאידיאולוגיה שלהם, במטרה שהיהודים לא ישמידו את החברה הגרמנית. 
			
			דוגמה אחרת, היא חקיקת חוקי נירנברג. החוקים בין היתר כללו סעיפים האוסרים נישואים, יחסי מין, וכו', בין יהודים לגרמנים ארים. באופן דומה נאסק על יהודים להעסיק עוזרות בית שגילן מתחת ל־45, או להניף את דגל גרמניה. בכך, הנאצים הצליחו לממש את האידיאולוגיה שלהם התומכת בתורת הגזע, ולמנוע הכנסת ''דם יהודי`` ל''דם הארי הטהור`` ובכך לשמור על הגזע הארי מפני היהודים. 
			
			נתבונן בקטע המקור (מופיע בעמוד 109 בספר ''נאציזם, מלחמה ושואה``, הוצאת היי־סקול, 2014). קטע המקור מתאר הוראות סודיות שהבריק הגסטפו לגבי אופן ניהול ליל הבדולח. לפי הקטע, ''אפשר להחריב עסקים ודירות פרטיות של יהודים`` – טיהור הכלכלה הגרמנית מהיהודים, בהתאם לאידיאולוגיה הנאצית. ''יש לאסור יהודים רבים [...] כמספר המקומות בבתי הסוהר [...]`` – הרחקת היהודים מהחברה, העברתם למחנות הריכוז, בצורה כזו שתרחיק אותם מהארים הטהורים וכאורה תמנע מהם לפגוע בחברה. 
			
			
		\end{enumerate}
	\subsection{האידיאולוגיה הנאצית ומהלכי מלחמת העולם השנייה}
		\begin{enumerate}[A.]
			\item \textbf{שאלה: }\\
			הסבירו את עקרון מרחב המחייה באידיאולוגיה הנאצית. מתוך ספר הלימוד, הציגו מקור היסטורי אשר מבטא עקרון זה והסבירו במה הוא תורם להבנת העקרון. 
			
			\textbf{תשובה: }\\			
			בשנת 1933 עלתה בצורה חוקתית המפלגה הנציונל־סוציאליסטית (או בקיצור ''\hl{הנאצית}``). המפלגה הונהגה על ידי \hl{אדולף היטלר}, שהכתיב את האידיאולוגיה של המפלגה בספרו ''מיין־קאמפ``. האידיאולוגיה דוגלת באנטי־קומוניזם, אנטי־דמוקרטיה, אנטי־שמיות, אנטי־מנהיגים־אחרים (עקרון המנהיג), אנטי־מפלגות־אחרות, תורת הגזע (אנטי־סלאביים), עקרון מרחב המחייה (אנטי־שטחים־לסלאביים), והלאום כערך עליון. 
			
			תורת הגזע של היטלר היא עיוות של תיאוריית הברירה הטבעית של צ'ארלס דארווין. היא עקרון באידיאולוגיה הנאצית. תורת הגזע מחלקת את הגזעים (הנבדלים לפי דמם ומוצאם) לשלושה גזעים – יוצרי התרבות (הגזע הארי), שמאופיין ביופי חיצוני ואינטלקטואל, נושאי התרבות (הסלאביים), אלו שאינם יכולים ליצור תרבות אך מסוגלים לשרת את אלו הראשונים, ומהרסי תרבות (הנחות מבינהם הוא הגזע היהודי, שמוגדר כאנטי־גזע) שהורסים באופן אקטיבי את התרבות שבני הגזע הארי יצרו. מתוקף הארים גזע מושלם בעל דם טהור, יש לו צורך טבעי להתרבות. 
			
			בגלל הגידול המספרי שהיטלר רצה לראות בקרב הגזע הארי, ובפרט בעם הגרמני (הנעלה ביותר במדרג הארים), יהיה צורך של העם הגרמני להשתלט על מקורות מחיה נוספים. לשם כך, הגרמנים התפשטו מזרחה (לכיוון העמים הסלאביים הנחותים מהם), במטרה ליישב אותם בארים ולשעבד את הסלאביים. עקרון מרחב המחייה הוא חלק מהעקרונות של האידיאולוגיה הנאצית. 
			
			נתבונן בקטע המקור ''קטע מקור מספר 8`` ממיין־קאמפ (מתוך הספר ''נאציזם, מלחמה ושואה``, הוצאת היי־סקול 2014). קטע המקור מנסה להצדיק ולהסביר את עקרון מרחב המחייה. 
			
			היטלר כותב ש''רק מרחב מחייה גדול במידה מספקת יכול להבטיח לעם [הגרמני] את יכולו להמשיך ולהתקיים``, ו''באירופה [...] אנו יכולים להביא רק בחשבון את רוסיה ואת המדינות [הסלאביות] המתקיימות בשלויה ובכפוף לה [...]``. הציטוט הזה תורם להבנת העקרון בכך שהוא מבהיר את כוונתו של היטלר להתקדם לעבר המדינות הסלאביות ולנצל אותן כמרחב מחייה (ציטוט שני) בשביל שהעם הגרמני יוכל להמשיך ולגודל מתוקף תורת הגזע (ציטוט ראשון). 
			
			\item \textbf{שאלה: }\\
			הציגו מהו ''הסדר החדש`` ואת עקרון ''מרחב המחייה``. הסבירו את הקשר בין שנני המושגים תוך הבאת דוגמה אחת מתוך פעולות הנאצים באירופה הכבושה הממחישה קשר זה. מתוך ספר הלימוד, הביאו מקור היסטורי התורם ומוסיף להבנת הקשר בין מרחב המחייה ל''סדר החדש`` והסבירו במה הוא תורם להבנת קשר זה. 
			
			\textbf{תשובה: }\\
			המפלגה הנציונל־סציאליסטית או בקיצור המפלגה ה\hl{נאצית} היא מפלגה ברפובליקת וויאנמר ברשות \hl{אדולף היטלר} ששלטה בגרמניה בשנים 1933-1945. למפלגה יש אידיאולוגיה שנתונה ע''י ספרו של היטלר ''מיין־קאמפ`` (בתרגום ישיר, המסע שלי), ועקרונותיהם הם שלילת הדמוקרטיה והקומוניזם, אנטישמיות מודרנית, עקרון המנהיג, עקרון מרחב המחייה, הלאום כערך עליון, ותורת הגזע. 
			
			תורת הגזע היא תורה פסאדו־מדעית שגורסת שהעולם נחלק לשלושה גזעים עקריים – ארים, סלאביים, והורסי תרבות. הראשונים יוצרי תרבות, פאר היצירה, ונעדו לשלוט בעולם ביד רמה ולנהל את התרבות. השניים מסוגלים להבין את התרבות ולחיות בה, אך לא מסוגלים ליצור שום דבר חיובי, ועל כן הם יהיו משרתיהם של הארים, והאחרונים כוללים ערבים, צוענים, ואחרים, והן נחשבים אנשים הפוגעים בתרבות והורסים אותה. היהודים לא נחשבים כגזע כלל (אנטי־גזע) והם מוגדרים להיות הייצור הנחות ביותר במדרג הגזעים הנאצי. ישנם יוצאי דופן כמו האנגליים והצרפתיים שיושבים בין הסלאביים לארים. הגזעים נמצאים במלחמה תמידית, והגזע הטוב ביותר – הגזע הארי – נועד לנצח באופן טבעי. 
			
			עקרון מרחב המחייה אומר שעל הגזעים הארים להתרחב כדי לאפשר את הגידול באוכלוסיה שהיטלר היה מעוניין בו בקרב הארים. היטלר רצה להתרחב לשטחים ברוסיה ובמזרח אירופה, על מנת לאפשר לגרמנים להתרבות עליהם. 
			
			\hl{הסדר החדש} הוא רעיון שמנסה לממש את עקרונות תורת הגזע ומרחב המחייה. הסדר החדש מציג מצב בו שטחים שנכבשו (או בראייה הנאצית, חזרו למרותם הטבעית שכן תמיד היו שייכים עליהם) מקבלים יחס בהתאם לאוכלוסיה שם – אם האוכלוסיה ארית, אז היא תהפוך למדינת חסות עם פחות או יותר אוטונומיה פנימית, ואם המדינה סלאבית, היא תיכבש, תנוהל באמצעות משטר נאצי (כלומר תסופח), ואז תנוצל (כלומר, יתישבו בה וישעבדו את אוכלוסייתה). לגזעים הנחותים, הורסי התרבות, אין מקום בסדר החדש, והם מיועדים להשמדה וגירוש. 
			
			נתבונן בקטע מקור המתאר מידע שהיטלר מוסר למקורביו על מטרות הכיבושים במזרח, זמן קצר לאחר תחילת הפלישה לברית המועצות. היטלר מתאר כיצד ומדוע הוא רוצה לנצל ארצות אלו, לצרכי העם הגרמני. הוא פונה אל הארצות בכבושות בתור ה''עוגה``, שיש לחלק לפי הצרכים הנאצים (קטע המקור מופיע בעמוד 174, בספר ''נאציזם, מלחמה ושואה``, הוצאת היי־סקול, 2014). 
			
			לפי קטע המקור, ''עתה עלינו לחלוק את העוגה לפי צרכינו, כך שנוכל: ראשית, לשלוט בה; שנית, לנהל אותה; שלישית, לנצל אותה``. אכן, היטלר מתאר כיצד הוא מתכוון לממש את הסדר החדש – שליטה, ניהול, וניצול של המדינות הסאלביות. ''כל הבלטיות נועדו להיכלל בתוך גרמניה [...] קרים [חצי אי מיושב ברוסים] יפונה מכל הזרים [לא ארים] וייושב בגרמנים בלבד ויהיה לשטח של הרייך [...] יש למחוד את לנינגרד עד הייסוד ואחר כך למסור אותה לפינפלד [שתושביה ארים]`` – היטלר יודע במדויק כיצד הוא ירצה לנצל את המדינות הללו, ומטרת הניצול הזה היא במדויק שימושן למען הארים, בהתאם לתורת הגזע ולעקרון הסדר החדש. 
		\end{enumerate}
	
	
\end{document}