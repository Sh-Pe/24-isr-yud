%! ~~~ Packages Setup ~~~ 
\documentclass[]{article}
\usepackage{lipsum}
\usepackage{rotating}


% Math packages
\usepackage[usenames]{color}
\usepackage{forest}
\usepackage{ifxetex,ifluatex,amssymb,amsmath,mathrsfs,amsthm,witharrows,mathtools,mathdots}
\usepackage{amsmath}
\WithArrowsOptions{displaystyle}
\renewcommand{\qedsymbol}{$\blacksquare$} % end proofs with \blacksquare. Overwrites the defualts. 
\usepackage{cancel,bm}
\usepackage[thinc]{esdiff}


% tikz
\usepackage{tikz}
\usetikzlibrary{graphs}
\newcommand\sqw{1}
\newcommand\squ[4][1]{\fill[#4] (#2*\sqw,#3*\sqw) rectangle +(#1*\sqw,#1*\sqw);}


% code 
\usepackage{algorithm2e}
\usepackage{listings}
\usepackage{xcolor}

\definecolor{codegreen}{rgb}{0,0.35,0}
\definecolor{codegray}{rgb}{0.5,0.5,0.5}
\definecolor{codenumber}{rgb}{0.1,0.3,0.5}
\definecolor{codeblue}{rgb}{0,0,0.5}
\definecolor{codered}{rgb}{0.5,0.03,0.02}
\definecolor{codegray}{rgb}{0.96,0.96,0.96}

\lstdefinestyle{pythonstylesheet}{
    language=Java,
    emphstyle=\color{deepred},
    backgroundcolor=\color{codegray},
    keywordstyle=\color{deepblue}\bfseries\itshape,
    numberstyle=\scriptsize\color{codenumber},
    basicstyle=\ttfamily\footnotesize,
    commentstyle=\color{codegreen}\itshape,
    breakatwhitespace=false, 
    breaklines=true, 
    captionpos=b, 
    keepspaces=true, 
    numbers=left, 
    numbersep=5pt, 
    showspaces=false,                
    showstringspaces=false,
    showtabs=false, 
    tabsize=4, 
    morekeywords={as,assert,nonlocal,with,yield,self,True,False,None,AssertionError,ValueError,in,else},              % Add keywords here
    keywordstyle=\color{codeblue},
    emph={var, List, Iterable, Iterator},          % Custom highlighting
    emphstyle=\color{codered},
    stringstyle=\color{codegreen},
    showstringspaces=false,
    abovecaptionskip=0pt,belowcaptionskip =0pt,
    framextopmargin=-\topsep, 
}
\newcommand\pythonstyle{\lstset{pythonstylesheet}}
\newcommand\pyl[1]     {{\lstinline!#1!}}
\lstset{style=pythonstylesheet}

\usepackage[style=1,skipbelow=\topskip,skipabove=\topskip,framemethod=TikZ]{mdframed}
\definecolor{bggray}{rgb}{0.85, 0.85, 0.85}
\mdfsetup{leftmargin=0pt,rightmargin=0pt,innerleftmargin=15pt,backgroundcolor=codegray,middlelinewidth=0.5pt,skipabove=5pt,skipbelow=0pt,middlelinecolor=black,roundcorner=5}
\BeforeBeginEnvironment{lstlisting}{\begin{mdframed}\vspace{-0.4em}}
    \AfterEndEnvironment{lstlisting}{\vspace{-0.8em}\end{mdframed}}


% Design
\usepackage[labelfont=bf]{caption}
\usepackage[margin=0.6in]{geometry}
\usepackage{multicol}
\usepackage[skip=4pt, indent=0pt]{parskip}
\usepackage[normalem]{ulem}
\forestset{default}
\renewcommand\labelitemi{$\bullet$}
\usepackage{titlesec}
\titleformat{\section}[block]
{\fontsize{15}{15}}
{\sen \dotfill (\thesection)\dotfill\she}
{0em}
{\MakeUppercase}
\usepackage{graphicx}
\graphicspath{ {./} }

\usepackage[colorlinks]{hyperref}
\definecolor{mgreen}{RGB}{25, 160, 50}
\definecolor{mblue}{RGB}{30, 60, 200}
\usepackage{hyperref}
\hypersetup{
    colorlinks=true,
    citecolor=mgreen,
    linkcolor=black,
    urlcolor=mblue,
    pdftitle={Document by Shahar Perets},
    %	pdfpagemode=FullScreen,
}
\usepackage{yfonts}
\def\gothstart#1{\noindent\smash{\lower3ex\hbox{\llap{\Huge\gothfamily#1}}}
    \parshape=3 3.1em \dimexpr\hsize-3.4em 3.4em \dimexpr\hsize-3.4em 0pt \hsize}
\def\frakstart#1{\noindent\smash{\lower3ex\hbox{\llap{\Huge\frakfamily#1}}}
    \parshape=3 1.5em \dimexpr\hsize-1.5em 2em \dimexpr\hsize-2em 0pt \hsize}



% Hebrew initialzing
\usepackage[bidi=basic]{babel}
\PassOptionsToPackage{no-math}{fontspec}
\babelprovide[main, import, Alph=letters]{hebrew}
\babelprovide[import]{english}
\babelfont[hebrew]{rm}{David CLM}
\babelfont[hebrew]{sf}{David CLM}
%\babelfont[english]{tt}{Monaspace Xenon}
\usepackage[shortlabels]{enumitem}
\newlist{hebenum}{enumerate}{1}

% Language Shortcuts
\newcommand\en[1] {\begin{otherlanguage}{english}#1\end{otherlanguage}}
\newcommand\he[1] {\she#1\sen}
\newcommand\sen   {\begin{otherlanguage}{english}}
    \newcommand\she   {\end{otherlanguage}}
\newcommand\del   {$ \!\! $}

\newcommand\npage {\vfil {\hfil \textbf{\textit{המשך בעמוד הבא}}} \hfil \vfil \pagebreak}
\newcommand\ndoc  {\dotfill \\ \vfil {\begin{center}
            {\textbf{\textit{שחר פרץ, 2025}} \\
                \scriptsize \textit{קומפל ב־}\en{\LaTeX}\,\textit{ ונוצר באמצעות תוכנה חופשית בלבד}}
    \end{center}} \vfil	}

\newcommand{\rn}[1]{
    \textup{\uppercase\expandafter{\romannumeral#1}}
}

\makeatletter
\newcommand{\skipitems}[1]{
    \addtocounter{\@enumctr}{#1}
}
\makeatother


%! ~~~ Document ~~~

\author{שחר פרץ}
\title{\textit{היסטוריה 33}}
\begin{document}
    \maketitle
    
    שיעור הבא אחרון. במהלך השבוע הבא ``תנחת'' במשוב עבודה לקיץ (``מטמטם הדבר הזה'' – על הצרצר הכיתתי). היא איננה חובה ומקנה בונוס של 10 נקודות לציון המבחן הראשון. בין ספטמבר לינואר יש 2 מבחנים ומתכונת. הסיבה – כי יש בגרות חורף. אחרי כן, 4 שעות היסטוריה ירדו מהמערכת. אם יש דברים לא נכונים או לא קוהרנטיים בסיכום הזה, שרית (המורה) מאשימה את הצרצר. 
    
    \subsubsection*{קידוש החיים – המשך}
    הגדרה: להמשיך לדבוק בחיים בהיבטים הפיזיים והרוחניים (לזכור – לא רק פיזי, לא רק ההשרדות). 
    
    \subsection*{תנועות נוער}
    \textbf{הגדרה. }מקום חברתי, חינוכי, ופוליטי. היום לתנועות הנוער יש פחות ערך פוליטי, אך בעבר, העולם היה עולם אידיאולוגי, והשתייכותך לתנועת נוער מסויימת – אומר שההורים שלך היו מזדהים למפלגה פוליטית מסויימת. באופן כללי, אם היית שייך למפלגה מסויימת, למפלגה היה עיתון, וקופת חולים, ומערכת חינוך, וכו'. זה היה דבר הרבה יותר כולל והכל סבב סביב לאיזו אידיאולוגיה אתה משתייך אליה. המאה ה־20 קרויה ``המאה של האידיאולוגיות הגדולות''. 
    
    
    \subsubsection*{הבונד}
    \textbf{לא במיקוד}. 
    אנחנו נדבר בעיקר על תנועות הנוער הציוניות. הסיבה: כי אנחנו חיים במדינה ציונית. אז לצורך הנקודה לא נדבר על ``הבונד''. הבונד לא קיים אחרי מלחמת העולם השנייה כי רובם נרצחו בשואה. הסיבה: הדתיים והחרידים, אם לא היו שייכים למפלגות דתיות, היו שייכים לבונד. זו תנועה קומוניסטית ולכן לא ציונית, ולא מרחיב עליה. 
    
    קונטקסט: מנהיג תנועת הבונד בורשה מת לפני 16 שנים, והגיעו לשם מכל העולם. חוץ מישראל. הציונות והבונד זרים. 
    
    רוב האוכלוסיה היהודית בפולין הייתה שייכת לבונד, והציונים היו מיעוט, אך מי שלרוב הנהיג את המרידות היו הציונים והבונד לקח חלק (במקומות אחדים אף כנראה הנהיג). בגטאות הגדולים, בכולן, הציונים הנהיגו את המרדים. 
    
    \subsubsection*{תנועות הנוער}
    תנועות הנוער ידעו לפני כולם מה קורה במקומות אחרים. דיברנו על הקשריות (לא היו קשרים, כי גברים אפשר לזהות יהדות אם הוא נימול). הן היו עוברות בין הגטאות ומעבירות מידע. ככה הגיע המידע על ההשמדה לגטאות. תנועות הנוער הובילו בשלבים מאוחרים יותר את המרידות בגטאות. 
    
    עם עיבוש פולין ברחבה רוב הנהגת תנועת הנוער מזרחה. לאחר תקופה, חלק מהמדריכים חזרו לגטאות. בינהם – מרגכי אנילביץ' מ''השומר הצעיר'' שינהיב בעתיד את המרג בגטאות. תנועות הנוער באותה התקופה ניהלו מטבחים וכו', כארגון הררכי שיכול לגייס אנשים רבים. \textbf{למה הן היו רלוונטיות: }
    \begin{itemize}
        \item ערעור המסגרות החברתיות הקיימות
        \item ההנגה היהודית המבוגרת וההורים התקשו להתסגל למצב החדש. 
        \item חברי תנועות הנוער הצליחו להסתכל אל המצב החדש
        \item הקשר הקבוע של הצעירים עם תנועת התנוער הפכו את תנועת הנוער למקום שמקנה בטחון לצעירים מעורערים. 
        \item חברי תנועות הנוער לא היו אחריים על משפחה ולכן לקחו יותר סיכונים. 
    \end{itemize}
    
    מה הם עשו: 
    \begin{itemize}
        \item לרוב, מתחו חברי תנועות הנוער ביקורת על מדיניות היודנראט. בעיקר – בפתרון הסופי. 
        
        חברי תנועות הנוער הבינו יותר מהר מהאחרים, ש''להוריד את הראש'' ו''נתעלם וזה יעלם'' לא יעובד. בין היתר בגלל שיש להם מידע מוקדם יותר לאזורים האזרחיים, וידע שהתחילה ההשמדה. המבוגרים לא האמינו ברובם, ואלו שהאמינו, קיוו להפטר ממנה בשיטות הישנות. 
        \item אם לפני המלחמה תנועות הנוער הציוניות חינכו לעליה לא''י, כרגע מכחנים לטובת כלל הציבור. 
        \item הקמת מרכזי עזרה, חינוך ותרבות לתושבי הגטו (לא רק הילדים, אך בעיקר הילדים)
        \item יצירת קשרים עם גטאות אחרים – גטו ורשה שימש מרכז ממנו יצאו שליחים לגטאות אחרים כדי לפקח על קיום ופעילות תנועות הנוער ברים אחרות, ולהעביר מידע. הקשרים (ובעיקר קשריות) היו בעלי חזות ארית ולרוב זהות נוצרית שאולה. רבים מצאו את מותם. 
        \item פרסום עיתונות מחתרתית – חרף האיסורים. [שרית מסבירה: ``עיתון זה מן משהו כזה, שהיו מדפיסים על נייר, והיו בו כתבות, וטורים, ומגזינים'']. 
        
        למה העיתונות נעשתה ע''י תנועות הנוער? כי להן הייתה את המידע. העיתונים לרוב היו על נייר דק, או עמוד אחד. 
    \end{itemize}
    
    ``חשוב שתבינו את זה – ילד שנכנס לגטו הוא לא ילד''. ילדים בגטו במקרים רבי םדאגו למשפחתם, כלומר הביאו אוכל (נאמר דרך פרצות בגדרות ותעלות), דאגו לילדים בבית וכו'. לפעמים בידי ילד בגיל 8-9 הייתה האחריות בנוגע לאיזה אוכל יהיה למשפחה שלו. בגטו הורים לא יכלו לשמור על ילדים, ואף לפעמים המצב היה הפוך. זאת, בין היתר, כי רבים מההורים היו נשלחים לעבודות כפייה. את תפקיד ראש המשפחה תפסה פעמים רבות האישה. לא הייתה כמעט שום מסגרת לילדים. 
    
    \subsection*{דילמות היודנראטים}
    
    כאשר שואלים מה הדילמות של היודנראט – \textbf{הן שתיים}: 
    \begin{itemize}
        \item מציאת איזון בין חובת הציות לגרמנים לבין טובת הגטו: הדילמה אינה לבצע את הפקודות, אלא איך לבצע אותן. יש איזון דק בין איך לסייע ליהודים, לבין איך לגרום לך להשאר יודנראט ובחיים. עצם העובדה שהעלימו עין מאותם הילדים שהיבאו אוכל, זה גם משהו, במיוחד כאשר לדיד הנאצים הברחת אוכל דינה מוות. כמובן שהיהודים בגטו, שאינם היו יודנראטים, לא ראו את זה כטובה אלא כמובן מאליו (``ואנחנו לא שופטים אותם לגבי זה''). 
        \item כיצד להתמודד עם יחס הציבור היהודי ששנה ובז להם: זכרו, אחת ממטרות היודנראט זה להפנות את הזעם כלפיהם ולא כלפי הנאצים. לכן, ``הקהילה הידוית במקרה הטוב שונאת אותם''. 
    \end{itemize}
    הבהרה: לא הכריחו כמעט אף אחד להיות יודנראט. 
    
    קורים כל מני דברים לא הוגנים: יוהודים ספציפיים שניתנת להם האופציה ללכת לעבודות כפייה (ולקבל יותר כסף), שניתנתה פעמים רבות למקרים שלהם (רצו להציל קודם כל את המשפחה שלהם). נוסף על כן, לעיתים גם חלוקת האוכל הייתה לא שווה. 
    
    ``ויטמין P זה קיצור של פרוטקציה'', וגטאות בהרבה מקרים שנאו אותם על זה. היה מדובר בעניין של חיים ומוות, לא על כסף או בגדים. 
    
    \subsubsection{שתיקת הארכיון}
    
    ביום רביעי נראה את ``שתיקת הארכיון''. זהו סרט שצולם בגטו ורהש. קחו בחשבון ב־95\% ממה שצולם בגטאות, היו צילומים של הנאצים. הסרט מעולם לא הופק. שליטת הארכיון מנסה להסביר את הצילומים, ואיך הם רצו להציג את הגטו, היהודים ואת עצמם. 
    
    
    
    
    \ndoc
\end{document}