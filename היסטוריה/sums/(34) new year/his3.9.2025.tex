%! ~~~ Packages Setup ~~~ 
\documentclass[]{article}
\usepackage{lipsum}
\usepackage{rotating}


% Math packages
\usepackage[usenames]{color}
\usepackage{forest}
\usepackage{ifxetex,ifluatex,amssymb,amsmath,mathrsfs,amsthm,witharrows,mathtools,mathdots}
\usepackage{amsmath}
\WithArrowsOptions{displaystyle}
\renewcommand{\qedsymbol}{$\blacksquare$} % end proofs with \blacksquare. Overwrites the defualts. 
\usepackage{cancel,bm}
\usepackage[thinc]{esdiff}


% tikz
\usepackage{tikz}
\usetikzlibrary{graphs}
\newcommand\sqw{1}
\newcommand\squ[4][1]{\fill[#4] (#2*\sqw,#3*\sqw) rectangle +(#1*\sqw,#1*\sqw);}


% code 
\usepackage{algorithm2e}
\usepackage{listings}
\usepackage{xcolor}

\definecolor{codegreen}{rgb}{0,0.35,0}
\definecolor{codegray}{rgb}{0.5,0.5,0.5}
\definecolor{codenumber}{rgb}{0.1,0.3,0.5}
\definecolor{codeblue}{rgb}{0,0,0.5}
\definecolor{codered}{rgb}{0.5,0.03,0.02}
\definecolor{codegray}{rgb}{0.96,0.96,0.96}

\lstdefinestyle{pythonstylesheet}{
	language=Java,
	emphstyle=\color{deepred},
	backgroundcolor=\color{codegray},
	keywordstyle=\color{deepblue}\bfseries\itshape,
	numberstyle=\scriptsize\color{codenumber},
	basicstyle=\ttfamily\footnotesize,
	commentstyle=\color{codegreen}\itshape,
	breakatwhitespace=false, 
	breaklines=true, 
	captionpos=b, 
	keepspaces=true, 
	numbers=left, 
	numbersep=5pt, 
	showspaces=false,                
	showstringspaces=false,
	showtabs=false, 
	tabsize=4, 
	morekeywords={as,assert,nonlocal,with,yield,self,True,False,None,AssertionError,ValueError,in,else},              % Add keywords here
	keywordstyle=\color{codeblue},
	emph={var, List, Iterable, Iterator},          % Custom highlighting
	emphstyle=\color{codered},
	stringstyle=\color{codegreen},
	showstringspaces=false,
	abovecaptionskip=0pt,belowcaptionskip =0pt,
	framextopmargin=-\topsep, 
}
\newcommand\pythonstyle{\lstset{pythonstylesheet}}
\newcommand\pyl[1]     {{\lstinline!#1!}}
\lstset{style=pythonstylesheet}

\usepackage[style=1,skipbelow=\topskip,skipabove=\topskip,framemethod=TikZ]{mdframed}
\definecolor{bggray}{rgb}{0.85, 0.85, 0.85}
\mdfsetup{leftmargin=0pt,rightmargin=0pt,innerleftmargin=15pt,backgroundcolor=codegray,middlelinewidth=0.5pt,skipabove=5pt,skipbelow=0pt,middlelinecolor=black,roundcorner=5}
\BeforeBeginEnvironment{lstlisting}{\begin{mdframed}\vspace{-0.4em}}
	\AfterEndEnvironment{lstlisting}{\vspace{-0.8em}\end{mdframed}}


% Design
\usepackage[labelfont=bf]{caption}
\usepackage[margin=0.6in]{geometry}
\usepackage{multicol}
\usepackage[skip=4pt, indent=0pt]{parskip}
\usepackage[normalem]{ulem}
\forestset{default}
\renewcommand\labelitemi{$\bullet$}

\usepackage{graphicx}
\graphicspath{ {./} }

\usepackage[colorlinks]{hyperref}
\definecolor{mgreen}{RGB}{25, 160, 50}
\definecolor{mblue}{RGB}{30, 60, 200}
\usepackage{hyperref}
\hypersetup{
	colorlinks=true,
	citecolor=mgreen,
	linkcolor=black,
	urlcolor=mblue,
	pdftitle={Document by Shahar Perets},
	%	pdfpagemode=FullScreen,
}
\usepackage{yfonts}
\def\gothstart#1{\noindent\smash{\lower3ex\hbox{\llap{\Huge\gothfamily#1}}}
	\parshape=3 3.1em \dimexpr\hsize-3.4em 3.4em \dimexpr\hsize-3.4em 0pt \hsize}
\def\frakstart#1{\noindent\smash{\lower3ex\hbox{\llap{\Huge\frakfamily#1}}}
	\parshape=3 1.5em \dimexpr\hsize-1.5em 2em \dimexpr\hsize-2em 0pt \hsize}



% Hebrew initialzing
\usepackage[bidi=basic]{babel}
\PassOptionsToPackage{no-math}{fontspec}
\babelprovide[main, import, Alph=letters]{hebrew}
\babelprovide[import]{english}
\babelfont[hebrew]{rm}{David CLM}
\babelfont[hebrew]{sf}{David CLM}
%\babelfont[english]{tt}{Monaspace Xenon}
\usepackage[shortlabels]{enumitem}
\newlist{hebenum}{enumerate}{1}

% Language Shortcuts
\newcommand\en[1] {\begin{otherlanguage}{english}#1\end{otherlanguage}}
\newcommand\he[1] {\she#1\sen}
\newcommand\sen   {\begin{otherlanguage}{english}}
	\newcommand\she   {\end{otherlanguage}}
\newcommand\del   {$ \!\! $}

\newcommand\npage {\vfil {\hfil \textbf{\textit{המשך בעמוד הבא}}} \hfil \vfil \pagebreak}
\newcommand\ndoc  {\dotfill \\ \vfil {\begin{center}
			{\textbf{\textit{שחר פרץ, 2025}} \\
				\scriptsize \textit{קומפל ב־}\en{\LaTeX}\,\textit{ ונוצר באמצעות תוכנה חופשית בלבד}}
	\end{center}} \vfil	}

\newcommand{\rn}[1]{
	\textup{\uppercase\expandafter{\romannumeral#1}}
}

\makeatletter
\newcommand{\skipitems}[1]{
	\addtocounter{\@enumctr}{#1}
}
\makeatother

%! ~~~ Math shortcuts ~~~

% Letters shortcuts
\newcommand\N     {\mathbb{N}}
\newcommand\Z     {\mathbb{Z}}
\newcommand\R     {\mathbb{R}}
\newcommand\Q     {\mathbb{Q}}
\newcommand\C     {\mathbb{C}}
\newcommand\One   {\mathit{1}}

\newcommand\ml    {\ell}
\newcommand\mj    {\jmath}
\newcommand\mi    {\imath}

\newcommand\powerset {\mathcal{P}}
\newcommand\ps    {\mathcal{P}}
\newcommand\pc    {\mathcal{P}}
\newcommand\ac    {\mathcal{A}}
\newcommand\bc    {\mathcal{B}}
\newcommand\cc    {\mathcal{C}}
\newcommand\dc    {\mathcal{D}}
\newcommand\ec    {\mathcal{E}}
\newcommand\fc    {\mathcal{F}}
\newcommand\nc    {\mathcal{N}}
\newcommand\vc    {\mathcal{V}} % Vance
\newcommand\sca   {\mathcal{S}} % \sc is already definded
\newcommand\rca   {\mathcal{R}} % \rc is already definded
\newcommand\zc    {\mathcal{Z}}

\newcommand\prm   {\mathrm{p}}
\newcommand\arm   {\mathrm{a}} % x86
\newcommand\brm   {\mathrm{b}}
\newcommand\crm   {\mathrm{c}}
\newcommand\drm   {\mathrm{d}}
\newcommand\erm   {\mathrm{e}}
\newcommand\frm   {\mathrm{f}}
\newcommand\nrm   {\mathrm{n}}
\newcommand\vrm   {\mathrm{v}}
\newcommand\srm   {\mathrm{s}}
\newcommand\rrm   {\mathrm{r}}

\newcommand\Si    {\Sigma}

% Logic & sets shorcuts
\newcommand\siff  {\longleftrightarrow}
\newcommand\ssiff {\leftrightarrow}
\newcommand\so    {\longrightarrow}
\newcommand\sso   {\rightarrow}

\newcommand\epsi  {\epsilon}
\newcommand\vepsi {\varepsilon}
\newcommand\vphi  {\varphi}
\newcommand\Neven {\N_{\mathrm{even}}}
\newcommand\Nodd  {\N_{\mathrm{odd }}}
\newcommand\Zeven {\Z_{\mathrm{even}}}
\newcommand\Zodd  {\Z_{\mathrm{odd }}}
\newcommand\Np    {\N_+}

% Text Shortcuts
\newcommand\open  {\big(}
\newcommand\qopen {\quad\big(}
\newcommand\close {\big)}
\newcommand\also  {\mathrm{, }}
\newcommand\defis {\mathrm{ definitions}}
\newcommand\given {\mathrm{given }}
\newcommand\case  {\mathrm{if }}
\newcommand\syx   {\mathrm{ syntax}}
\newcommand\rle   {\mathrm{ rule}}
\newcommand\other {\mathrm{else}}
\newcommand\set   {\ell et \text{ }}
\newcommand\ans   {\mathscr{A}\!\mathit{nswer}}

% Set theory shortcuts
\newcommand\ra    {\rangle}
\newcommand\la    {\langle}

\newcommand\oto   {\leftarrow}

\newcommand\QED   {\quad\quad\mathscr{Q.E.D.}\;\;\blacksquare}
\newcommand\QEF   {\quad\quad\mathscr{Q.E.F.}}
\newcommand\eQED  {\mathscr{Q.E.D.}\;\;\blacksquare}
\newcommand\eQEF  {\mathscr{Q.E.F.}}
\newcommand\jQED  {\mathscr{Q.E.D.}}

\DeclareMathOperator\dom   {dom}
\DeclareMathOperator\Img   {Im}
\DeclareMathOperator\range {range}

\newcommand\trio  {\triangle}

\newcommand\rc    {\right\rceil}
\newcommand\lc    {\left\lceil}
\newcommand\rf    {\right\rfloor}
\newcommand\lf    {\left\lfloor}
\newcommand\ceil  [1] {\lc #1 \rc}
\newcommand\floor [1] {\lf #1 \rf}

\newcommand\lex   {<_{lex}}

\newcommand\az    {\aleph_0}
\newcommand\uaz   {^{\aleph_0}}
\newcommand\al    {\aleph}
\newcommand\ual   {^\aleph}
\newcommand\taz   {2^{\aleph_0}}
\newcommand\utaz  { ^{\left (2^{\aleph_0} \right )}}
\newcommand\tal   {2^{\aleph}}
\newcommand\utal  { ^{\left (2^{\aleph} \right )}}
\newcommand\ttaz  {2^{\left (2^{\aleph_0}\right )}}

\newcommand\n     {$n$־יה\ }

% Math A&B shortcuts
\newcommand\logn  {\log n}
\newcommand\logx  {\log x}
\newcommand\lnx   {\ln x}
\newcommand\cosx  {\cos x}
\newcommand\sinx  {\sin x}
\newcommand\sint  {\sin \theta}
\newcommand\tanx  {\tan x}
\newcommand\tant  {\tan \theta}
\newcommand\sex   {\sec x}
\newcommand\sect  {\sec^2}
\newcommand\cotx  {\cot x}
\newcommand\cscx  {\csc x}
\newcommand\sinhx {\sinh x}
\newcommand\coshx {\cosh x}
\newcommand\tanhx {\tanh x}

\newcommand\seq   {\overset{!}{=}}
\newcommand\slh   {\overset{LH}{=}}
\newcommand\sle   {\overset{!}{\le}}
\newcommand\sge   {\overset{!}{\ge}}
\newcommand\sll   {\overset{!}{<}}
\newcommand\sgg   {\overset{!}{>}}

\newcommand\h     {\hat}
\newcommand\ve    {\vec}
\newcommand\lv    {\overrightarrow}
\newcommand\ol    {\overline}

\newcommand\mlcm  {\mathrm{lcm}}

\DeclareMathOperator{\sech}   {sech}
\DeclareMathOperator{\csch}   {csch}
\DeclareMathOperator{\arcsec} {arcsec}
\DeclareMathOperator{\arccot} {arcCot}
\DeclareMathOperator{\arccsc} {arcCsc}
\DeclareMathOperator{\arccosh}{arccosh}
\DeclareMathOperator{\arcsinh}{arcsinh}
\DeclareMathOperator{\arctanh}{arctanh}
\DeclareMathOperator{\arcsech}{arcsech}
\DeclareMathOperator{\arccsch}{arccsch}
\DeclareMathOperator{\arccoth}{arccoth}
\DeclareMathOperator{\atant}  {atan2} 
\DeclareMathOperator{\Sp}     {span} 
\DeclareMathOperator{\sgn}    {sgn} 
\DeclareMathOperator{\row}    {Row} 
\DeclareMathOperator{\adj}    {adj} 
\DeclareMathOperator{\rk}     {rank} 
\DeclareMathOperator{\col}    {Col} 
\DeclareMathOperator{\tr}     {tr}

\newcommand\dx    {\,\mathrm{d}x}
\newcommand\dt    {\,\mathrm{d}t}
\newcommand\dtt   {\,\mathrm{d}\theta}
\newcommand\du    {\,\mathrm{d}u}
\newcommand\dv    {\,\mathrm{d}v}
\newcommand\df    {\mathrm{d}f}
\newcommand\dfdx  {\diff{f}{x}}
\newcommand\dit   {\limhz \frac{f(x + h) - f(x)}{h}}

\newcommand\nt[1] {\frac{#1}{#1}}

\newcommand\limz  {\lim_{x \to 0}}
\newcommand\limxz {\lim_{x \to x_0}}
\newcommand\limi  {\lim_{x \to \infty}}
\newcommand\limh  {\lim_{x \to 0}}
\newcommand\limni {\lim_{x \to - \infty}}
\newcommand\limpmi{\lim_{x \to \pm \infty}}

\newcommand\ta    {\theta}
\newcommand\ap    {\alpha}

\renewcommand\inf {\infty}
\newcommand  \ninf{-\inf}

% Combinatorics shortcuts
\newcommand\sumnk     {\sum_{k = 0}^{n}}
\newcommand\sumni     {\sum_{i = 0}^{n}}
\newcommand\sumnko    {\sum_{k = 1}^{n}}
\newcommand\sumnio    {\sum_{i = 1}^{n}}
\newcommand\sumai     {\sum_{i = 1}^{n} A_i}
\newcommand\nsum[2]   {\reflectbox{\displaystyle\sum_{\reflectbox{\scriptsize$#1$}}^{\reflectbox{\scriptsize$#2$}}}}

\newcommand\bink      {\binom{n}{k}}
\newcommand\setn      {\{a_i\}^{2n}_{i = 1}}
\newcommand\setc[1]   {\{a_i\}^{#1}_{i = 1}}

\newcommand\cupain    {\bigcup_{i = 1}^{n} A_i}
\newcommand\cupai[1]  {\bigcup_{i = 1}^{#1} A_i}
\newcommand\cupiiai   {\bigcup_{i \in I} A_i}
\newcommand\capain    {\bigcap_{i = 1}^{n} A_i}
\newcommand\capai[1]  {\bigcap_{i = 1}^{#1} A_i}
\newcommand\capiiai   {\bigcap_{i \in I} A_i}

\newcommand\xot       {x_{1, 2}}
\newcommand\ano       {a_{n - 1}}
\newcommand\ant       {a_{n - 2}}

% Linear Algebra
\DeclareMathOperator{\chr}     {char}
\DeclareMathOperator{\diag}    {diag}
\DeclareMathOperator{\Hom}     {Hom}
\DeclareMathOperator{\Sym}     {Sym}
\DeclareMathOperator{\Asym}    {ASym}

\newcommand\lra       {\leftrightarrow}
\newcommand\chrf      {\chr(\F)}
\newcommand\F         {\mathbb{F}}
\newcommand\co        {\colon}
\newcommand\tmat[2]   {\cl{\begin{matrix}
			#1
		\end{matrix}\, \middle\vert\, \begin{matrix}
			#2
\end{matrix}}}

\makeatletter
\newcommand\rrr[1]    {\xxrightarrow{1}{#1}}
\newcommand\rrt[2]    {\xxrightarrow{1}[#2]{#1}}
\newcommand\mat[2]    {M_{#1\times#2}}
\newcommand\gmat      {\mat{m}{n}(\F)}
\newcommand\tomat     {\, \dequad \longrightarrow}
\newcommand\pms[1]    {\begin{pmatrix}
		#1
\end{pmatrix}}

\newcommand\norm[1]   {\left \vert \left \vert #1 \right \vert \right \vert}
\newcommand\snorm     {\left \vert \left \vert \cdot \right \vert \right \vert}
\newcommand\smut      {\left \la \cdot \mid \cdot \right \ra}
\newcommand\mut[2]    {\left \la #1 \,\middle\vert\, #2 \right \ra}

% someone's code from the internet: https://tex.stackexchange.com/questions/27545/custom-length-arrows-text-over-and-under
\makeatletter
\newlength\min@xx
\newcommand*\xxrightarrow[1]{\begingroup
	\settowidth\min@xx{$\m@th\scriptstyle#1$}
	\@xxrightarrow}
\newcommand*\@xxrightarrow[2][]{
	\sbox8{$\m@th\scriptstyle#1$}  % subscript
	\ifdim\wd8>\min@xx \min@xx=\wd8 \fi
	\sbox8{$\m@th\scriptstyle#2$} % superscript
	\ifdim\wd8>\min@xx \min@xx=\wd8 \fi
	\xrightarrow[{\mathmakebox[\min@xx]{\scriptstyle#1}}]
	{\mathmakebox[\min@xx]{\scriptstyle#2}}
	\endgroup}
\makeatother


% Greek Letters
\newcommand\ag        {\alpha}
\newcommand\bg        {\beta}
\newcommand\cg        {\gamma}
\newcommand\dg        {\delta}
\newcommand\eg        {\epsi}
\newcommand\zg        {\zeta}
\newcommand\hg        {\eta}
\newcommand\tg        {\theta}
\newcommand\ig        {\iota}
\newcommand\kg        {\keppa}
\renewcommand\lg      {\lambda}
\newcommand\og        {\omicron}
\newcommand\rg        {\rho}
\newcommand\sg        {\sigma}
\newcommand\yg        {\usilon}
\newcommand\wg        {\omega}

\newcommand\Ag        {\Alpha}
\newcommand\Bg        {\Beta}
\newcommand\Cg        {\Gamma}
\newcommand\Dg        {\Delta}
\newcommand\Eg        {\Epsi}
\newcommand\Zg        {\Zeta}
\newcommand\Hg        {\Eta}
\newcommand\Tg        {\Theta}
\newcommand\Ig        {\Iota}
\newcommand\Kg        {\Keppa}
\newcommand\Lg        {\Lambda}
\newcommand\Og        {\Omicron}
\newcommand\Rg        {\Rho}
\newcommand\Sg        {\Sigma}
\newcommand\Yg        {\Usilon}
\newcommand\Wg        {\Omega}

% Other shortcuts
\newcommand\tl    {\tilde}
\newcommand\op    {^{-1}}

\newcommand\sof[1]    {\left | #1 \right |}
\newcommand\cl [1]    {\left ( #1 \right )}
\newcommand\csb[1]    {\left [ #1 \right ]}
\newcommand\ccb[1]    {\left \{ #1 \right \}}

\newcommand\bs        {\blacksquare}
\newcommand\dequad    {\!\!\!\!\!\!}
\newcommand\dequadd   {\dequad\duquad}

\renewcommand\phi     {\varphi}

\newtheorem{Theorem}{משפט}
\theoremstyle{definition}
\newtheorem{definition}{הגדרה}
\newtheorem{Lemma}{למה}
\newtheorem{Remark}{הערה}
\newtheorem{Notion}{סימון}


\newcommand\theo  [1] {\begin{Theorem}#1\end{Theorem}}
\newcommand\defi  [1] {\begin{definition}#1\end{definition}}
\newcommand\rmark [1] {\begin{Remark}#1\end{Remark}}
\newcommand\lem   [1] {\begin{Lemma}#1\end{Lemma}}
\newcommand\noti  [1] {\begin{Notion}#1\end{Notion}}

% DS


%! ~~~ Document ~~~

\author{שחר פרץ}
\title{\textit{היסטוריה 34}}
\begin{document}
	\maketitle
	\section{השואה בתקופת הפתרון הסופי}
	\textbf{תזכורת: }בשנים 1939-1941 מדובר על השואה \textit{עד} הפתרון הסופי. החל מיוני 41 מדובר על השואה \textit{בתקופת} הפתרון הסופי. מה היווה את המפנה? מבצע ברברוסה ב־1941. תחילה, ננסה להבין מה הקשר בין מבצע ברברוסה לבין הפתרון הסופי. 
	
	\textbf{תזכורת נוספת: }''מבצע ברברוסה'' הוא שם של הפלישה הגרמנית לברה''מ. 
	
	במה הפתרון הסופי שונה ממה שהיה קודם? זהו רצח שיטתי. מדובר על מבצע רצח מתוכנן. כאשר נדבר על שלבים בפתרון הסופי, נבחין שזה מתכונן, אך לא בדיוק. לקח זמן עד שזה הגיע לאושוויץ־בירקנהו. בגטו אחת המטרות הייתה תמותה עקיפה. הרוב לא מתו כי רצחו אותם באופן ישיר, אלא מתו מהתנאים. 
	
	\defi{\textit{הפתרון הסופי} הוא שם קוד למבצע להשמדת היהודים בעולם באופן שיטתי, המוני וישיר. }
	
	\subsection{הסיבות לפתרון הסופי}
	\begin{enumerate}
		\item \textbf{כמות היהודים} – הנאצים העריכו את כמות היהודים בברימה''ש בכחמישה מיליון. זה מספר לא נכון, אבל זה לא משנה – יש מיליוני יהודים בברית המועצות. 
		\item \textbf{יהודים \textit{קומוניסטים}} – יש פה מלחמה אידיאולוגית כפולה. חלק גדול ממנהיגי הקומוניזם בברית המועצות היו יהודים. 
		\item \textbf{אובדן בלמים מוסריים} – נבע מהריחוק מהבית, מרצף הנצחונות בשנתיים הראשונות, ומדעת הקהל הגרמנית. 
		\item \textbf{לכוון לרצונו של הפיהרר} – תשתדלו פחות לדבר על זה. הרעיון, הוא שאנשים בשטח רצו למממש את האידיאולוגיה של היטלר. 
	\end{enumerate}
	
	\subsection{שלבי הפתרון הסופי}
	\begin{multicols}{2}
		\begin{enumerate}
			\item בורות ירי
			\item משאיות גז
			\item חלמנו
			\item ועידת ואנזה (הנושא האחרון שילמד)
			\item מבצע ריינהרד
			\item בירקנאו (אושוויץ)
			\item צעדות המוות
		\end{enumerate}
	\end{multicols}
	יש כאן מדרג. ברור שהמדרג מתאר את הממדים של הרצח, אך לא רק זה. נבחין שככל שהמדרג מתקדם, באופן כללי, האינטראקציה עם הנאצים פוחתת – ``הם לא רואים אותם בלבן של העיניים''. נסכם: שני דברים משמעותיים קורים. שינוי בכמות (משתכלל אממ capacity יותר גדול) והגדלת הריחוק. 
	
	
	\subsubsection{בורות ירי}
	בורות הירי מתחילים ממש בהתחלה של כיבוש ברית המועצות. זמן קצר מבצע ברברוסה, תוך כדי ההתנהלות שלו, באיזורים שנכבשו. 
	
	כדי להבין את הנושא לאומק, נדבר את אייזנצגרופן (בתרגום מילולי: ``עוצבות המבצע''). אלו יחידות SS מובחרות שנלוות לכוחות של הורמכט. כאשר השטחים נכבשים, תפקידם לנהל אותו. הם קיבלו פקודה בשם \textit{פקודת הקומיסרים}, שבה היה כתוב ``להשמיד את אויבי המשטר'' (לא בהכרח יהודים), וכתוצאה מכך נבנו בורות הירי. 
	
	\textbf{תזכורת: }הורמכט הוא הצבע, ה־SS וה־SA אינם. 
	
	בורות ירי הוא דבר ``בסיסי''. לוקחים אנשים מהבית ליער הקרוב, נותנים להם לחפור בור / משתמשים במקומות בהם היו משוחות לחימה. אומרים להם להתפשט ויורים בהם לתוך הבור. בבורות ירי, במקרים רבים, אנשי ה־SS היו מיעוט ובפועל המקומיים השתתפו ברצח. כשמדובר בקהילות קטנות, הקהילה כולה נלקחת ליער. כמות הניצולים מבורות הירי היא אפסית. אלו שניצלו נכלאו בתוך כיסי אוייר ולא מתו מהירי, ובלילה יצאו וברחו. 
	
	הדבר החשוב: הוא הקרבה בין הרוצח לנרצח. דבר שני, הרצח מתקיים ליד הבית, ואין יכולת להסתיר את זה מהסביבה. אחד הדברים המרכזיים אצל הנאצים הוא \textit{הסוואה והטעייה}. זה לא מאפשר לא את זה ולא את זה. 
	
	גם ככל שנקדם בתהליך, \textbf{בורות ירי ממשיכים להתקיים} במלחמה. בכל פולין יש בורות ירי. זה המשיך לקרות לאורך כל המלחמה, בכל מיני מקומות. אבל כשיטה, זה התקדם. 
	
	\subsubsection{משאיות גז}
	``ואז החליטו לנסות משהו אחר''. מכניסים את האגזוז של המשאיות פנימה, ``נוסעים נוסעים נוסעים עד שהאנשים מתים''. גם זה התרחש במקומות המגורים של האנשים. התסובבו בערך 20 משאיות כאלו. 
	
	\subsubsection{חלמו}
	זהו מחנה ההשמדה הראשון. הוא לא ממש מוכר. המחנה נבצע באיזור לודג'. במחנה הזה רצחו בהתחלה באמצעות משאיות גז. אז מהוא ההבדל בין חלמו לבין משאיות הגז? זוהי הפעם הראשונה שישנו מקום יהודי שאליו מביאים אנשים כדי לרצוח אותם, והוא נבנה כדי לרצוח אנשים. זהו עדיין מחנה השמדה. 
	
	\textbf{תזכורת: }לודג' היה גטו צמוד לגבולות גרמניה. 
	
	יש כמה סיבות שלא המשיכו עם המתודות הללו: 
	\begin{itemize}
		\item \textbf{עלות כלכלית} – רצח בבורות ירי הוא רצח יקר. זה מבזבז כדורים. היו נסיונות להעמיד אנשים צמודים ולירות בהם עם כדור אחד כדי לא לבזבז תחמושת. 
		\item \textbf{איטי}
		\item \textbf{לא עמדו בזה מבחינה נפשית} – לא כל הנאצים הם פסיכופטים. הם אנשים רגילים. הם היו שיכורים כל הזמן כדי להיות מסוגלים לקחת חלק בזה, והם ``שתו את נפשם לדעת''. מדובר הן על האייזנצבורפן והן על משתפי הפעולה. 
	\end{itemize}
	
	\subsubsection{מבצע ריינהרד}
	במהלך המבצע הקימו שלושה מחנות השמדה: 
	\begin{itemize}
		\item בליז'ץ
		\item סוביבור
		\item טרבלינקה
	\end{itemize}
	המחנות הללו הוקמו בחלק המזרחי, ונועדו לרצח יהודי פולין (אם כי בהמשך הביאו אליהם אנשים ממקומות אחרים). אין קרמטוריומים (משרפות) במחנות האלו. הרגו את האנשים בתאי גזים ואז כיסו אותם בסיד כדי למנוע התפשטות מחלות. באיזור 95\% מהאנשים נרצחו עוד באותו היום, ולא הייתה כמעט סלקציה. בסוביבור למשל, המחנה היה מחולק לשניים, האיזור של הרצח והאיזור השני. 
	
	\textbf{המלצה של שרית לסרט. }הבריחה מסוביבור. 
	
	
	מובן שכאשר העלו אותם על רכבות למחנות השמדה, עשו הצגה שלמה של הסוואה והטעייה. אמרו להם לקחת מזוודות, כאשר הגיעו ביקשו מהם לכתוב מכתב למשפחה, להביא כמה שיותר דברים, ושיעבירו אותם להתיישבות במזרח. ביקשו מהם להתפשט בתואנה שמכניסים אותם למקלחת, ואז זרקו פחית ציקלון B לתוך החדר. זהו גז שבמגע עם האוויר הוא מתפשט, ואז אנשים נחנקים ומתים בתאי הגזים. 
	
	\subsubsection{אושוויץ־בירקנאו}
	אשוויץ הוא קונפקס של מחנות. הוא איננו מחנה אחד. ``מצד החיים'' שרואים בטלווזיה, זה בין אושוויץ אחד לאושוויץ בירקנאו. יש לנו 3 אושוויצים – אושוויץ 1 שהיה מחנה צבאי פולני עוד לפני הנאצים (עם השלט ``העבודה משחררת'', ויש בו בניינים מאבנים), אושוויץ 2 (בירקנאו, שם אין בניני אבנים, והוא הוקם כמחנה השמדה), ואשוויץ 3 (כרגע לא בחומר). בירקנאו מחנה ענק, והוא מהווה את פסגת הטכנולוגיה של הנאצים. במחנה הזה הנאצים מגייסים את הטכנולוגיה הכי מתקדמת של התקופה, כדי לבנות מפעל לרצח. רופאים ביצעו שם ניסויים רפואיים (לא רק מנגלה). משום שבירקנאו תוכן בצורה ``מושלמת'', הוא היה הרבה יותר יעיל. יש בו 4 קומפקסים של תאי גזים וקרמטורים ביחד. כלומר, אדם נכנס למלתחות להתקלח, ובאותו הקומפלקס מעבירים אותו למשרפות. אף אחד מהם לא נישאר שלם, ומי שהפציץ אותם היו הנאצים בבריחתם. זהו מפעל רצח, ``לא לצטט אותי אבל נכנס בן־אדם יוצא אפר''. סוביבור וטרבלינקה לא כל כך יעילים. בבריקנאו נרצחו 800K יהודים, רובם מפולין. 
	
	למה מיידנק לא כתוב כאן? כי הוא לא הוקם כמחנה השמדה. הוא הוקם כמחנה שבויים. מיידנק עבר הסבה למחנה השמדה ואז הקימו בו תאי גזים וקרמטורים. זה המקום היחיד במיידנק שבו הכל נשאר. המחנה הוקדם בתוך העיר לובלין (יכול להיות שטעיתי באיות), בניגוד למחנות אחרים שהוקמו בתוך יער. הבית של האנשים בעיר הוא 100 מטר מהמחנה, ואנשי לובלין ``לא יכולים להגיד שהם לא הבינו מה קרה''. שמעו את היריות, הריחו את שריפת הגופות. 
	
	
	כל מחנות ההשמדה הוקמו בתוך פולין, פרט לבירקנאו שהיה קצת על הגבול. הסיבה היא שהוא היה במיקום נוח ביחס לפסי רכבת, עם תשתית שהייתה קיימת עוד לפני הנאצים. המחנה הזה פעל עד סוף 44'. פסי הרכבת שנכנסים לתוך הגטו זהו ``שידרוג'' שהוכנס לגטו בעת השמדת יהודי הונגריה (כי היו המון יהודים). 
	
	באושוויץ־בירקנאו ישנה סלקציה. הסלקציה נעשית בבירקנאו וחלקם הולכים לאושויוץ 1, ומי שנשאר שם, נשאר כדי לעבוד. 
	
	\subsubsection{צעדות המוות}
	אף אחד לא באמת יודע למה הנאצים ביצעו את צעדות המוות. הרעיון: ברגע שבורחים מאשוויץ, הנאצים לקחו איתם את מי שמסוגל ללכת ואיינו על סף מוות, ומצעידים אותם לכיוון גרמניה (מערבה). הם הולכים בחורף (באותה השנה, היה נוראי) עשרות ואף מאות קילומטרים, ועוברים בישובים וכפרים. כל מי שנעצר, נורה. מי שנפל, ווידאו הריגה. יש כאלו שטוענים שהמטרה הייתה להעביר אותם לגרמניה לוהעביד אותם בכפייה. חלק לא קטן מבית שורדי אושוויץ נרצחו בצעדות המוות. במקומות מסויימים היו כמרים (בעיקר) שאספו את האנשים וקברו אותם קבורה ראויה בקבר אחים. על הקבר כתוב את המספר שעל היד (שכן הם לא ידעו את השם של האנשים). בשנים האחרונות זיהו את האנשים (אושוויץ היה מקום מסודר וידעו למי שייך כל מספר) וכתבו את השמות על הקברים. 
	
	
	אז, אם שואלים שאלה, האם הפתרון הסופי תוכנן מראש, התשובה היא לא. מכל שיטה למדנו והשתכללו כדי לבנות את אמצעי ההשמדה ההמונית הבא. הרעיון היה עוד במיין־קאמפף, אך הביצוע לא תוכנן וארך זמן. 
	
	
	
	
	
	
	\ndoc
\end{document}