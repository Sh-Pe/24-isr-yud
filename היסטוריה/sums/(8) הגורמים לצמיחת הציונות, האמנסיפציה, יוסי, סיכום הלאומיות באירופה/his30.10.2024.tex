%! ~~~ Packages Setup ~~~ 
\documentclass[]{article}


% Math packages
\usepackage[usenames]{color}
\usepackage{forest}
\usepackage{ifxetex,ifluatex,amsmath,amssymb,mathrsfs,amsthm,witharrows,mathtools}
\WithArrowsOptions{displaystyle}
\renewcommand{\qedsymbol}{$\blacksquare$} % end proofs with \blacksquare. Overwrites the defualts. 
\usepackage{cancel,bm}
\usepackage[thinc]{esdiff}


% tikz
\usepackage{tikz}
\newcommand\sqw{1}
\newcommand\squ[4][1]{\fill[#4] (#2*\sqw,#3*\sqw) rectangle +(#1*\sqw,#1*\sqw);}


% code 
\usepackage{listings}
\usepackage{xcolor}

\definecolor{codegreen}{rgb}{0,0.35,0}
\definecolor{codegray}{rgb}{0.5,0.5,0.5}
\definecolor{codenumber}{rgb}{0.1,0.3,0.5}
\definecolor{codeblue}{rgb}{0,0,0.5}
\definecolor{codered}{rgb}{0.5,0.03,0.02}
\definecolor{codegray}{rgb}{0.96,0.96,0.96}

\lstdefinestyle{pythonstylesheet}{
	language=Python,
	emphstyle=\color{deepred},
	backgroundcolor=\color{codegray},
	keywordstyle=\color{deepblue}\bfseries\itshape,
	numberstyle=\scriptsize\color{codenumber},
	basicstyle=\ttfamily\footnotesize,
	commentstyle=\color{codegreen}\itshape,
	breakatwhitespace=false, 
	breaklines=true, 
	captionpos=b, 
	keepspaces=true, 
	numbers=left, 
	numbersep=5pt, 
	showspaces=false,                
	showstringspaces=false,
	showtabs=false, 
	tabsize=4, 
	morekeywords={as,assert,nonlocal,with,yield,self,True,False,None,AssertionError,ValueError,in,else},              % Add keywords here
	keywordstyle=\color{codeblue},
	emph={object,type,isinstance,copy,deepcopy,zip,enumerate,reversed,list,set,len,dict,tuple,print,range,xrange,append,execfile,real,imag,reduce,str,repr,__init__,__add__,__mul__,__div__,__sub__,__call__,__getitem__,__setitem__,__eq__,__ne__,__nonzero__,__rmul__,__radd__,__repr__,__str__,__get__,__truediv__,__pow__,__name__,__future__,__all__,},          % Custom highlighting
	emphstyle=\color{codered},
	stringstyle=\color{codegreen},
	showstringspaces=false,
	abovecaptionskip=0pt,belowcaptionskip =0pt,
	framextopmargin=-\topsep, 
}
\newcommand\pythonstyle{\lstset{pythonstylesheet}}
\newcommand\pyl[1]     {{\lstinline!#1!}}
\lstset{style=pythonstylesheet}

\usepackage[style=1,skipbelow=\topskip,skipabove=\topskip,framemethod=TikZ]{mdframed}
\definecolor{bggray}{rgb}{0.85, 0.85, 0.85}
\mdfsetup{leftmargin=0pt,rightmargin=0pt,innerleftmargin=15pt,backgroundcolor=codegray,middlelinewidth=0.5pt,skipabove=5pt,skipbelow=0pt,middlelinecolor=black,roundcorner=5}
\BeforeBeginEnvironment{lstlisting}{\begin{mdframed}\vspace{-0.4em}}
	\AfterEndEnvironment{lstlisting}{\vspace{-0.8em}\end{mdframed}}


% Deisgn
\usepackage[labelfont=bf]{caption}
\usepackage[margin=0.6in]{geometry}
\usepackage{multicol}
\usepackage[skip=4pt, indent=0pt]{parskip}
\usepackage[normalem]{ulem}
\forestset{default}
\renewcommand\labelitemi{$\bullet$}
\usepackage{titlesec}
\usepackage{graphicx}
\graphicspath{ {./} }


% Hebrew initialzing
\usepackage[bidi=basic]{babel}
\PassOptionsToPackage{no-math}{fontspec}
\babelprovide[main, import, Alph=letters]{hebrew}
\babelprovide[import]{english}
\babelfont[hebrew]{rm}{David CLM}
\babelfont[hebrew]{sf}{David CLM}
\babelfont[english]{tt}{Monaspace Xenon}
\usepackage[shortlabels]{enumitem}
\newlist{hebenum}{enumerate}{1}

% Language Shortcuts
\newcommand\en[1] {\begin{otherlanguage}{english}#1\end{otherlanguage}}
\newcommand\he[1] {\begin{otherlanguage}{hebrew}#1\end{otherlanguage}}
\newcommand\sen   {\begin{otherlanguage}{english}}
	\newcommand\she   {\end{otherlanguage}}
\newcommand\del   {$ \!\! $}
\newcommand\ttt[1]{\en{\footnotesize\texttt{#1}\normalsize}}

\newcommand\npage {\vfil {\hfil \textbf{\textit{המשך בעמוד הבא}}} \hfil \vfil \pagebreak}
\newcommand\ndoc  {\dotfill \\ \vfil {\begin{center} {\textbf{\textit{שחר פרץ, 2024}} \\ \scriptsize \textit{נוצר באמצעות תוכנה חופשית בלבד}} \end{center}} \vfil	}

\newcommand{\rn}[1]{
	\textup{\uppercase\expandafter{\romannumeral#1}}
}

\makeatletter
\newcommand{\skipitems}[1]{
	\addtocounter{\@enumctr}{#1}
}
\makeatother

%! ~~~ Math shortcuts ~~~

% Letters shortcuts
\newcommand\N     {\mathbb{N}}
\newcommand\Z     {\mathbb{Z}}
\newcommand\R     {\mathbb{R}}
\newcommand\Q     {\mathbb{Q}}
\newcommand\C     {\mathbb{C}}

\newcommand\ml    {\ell}
\newcommand\mj    {\jmath}
\newcommand\mi    {\imath}

\newcommand\powerset {\mathcal{P}}
\newcommand\ps    {\mathcal{P}}
\newcommand\pc    {\mathcal{P}}
\newcommand\ac    {\mathcal{A}}
\newcommand\bc    {\mathcal{B}}
\newcommand\cc    {\mathcal{C}}
\newcommand\dc    {\mathcal{D}}
\newcommand\ec    {\mathcal{E}}
\newcommand\fc    {\mathcal{F}}
\newcommand\nc    {\mathcal{N}}
\newcommand\sca   {\mathcal{S}} % \sc is already definded
\newcommand\rca   {\mathcal{R}} % \rc is already definded

\newcommand\Si    {\Sigma}

% Logic & sets shorcuts
\newcommand\siff  {\longleftrightarrow}
\newcommand\ssiff {\leftrightarrow}
\newcommand\so    {\longrightarrow}
\newcommand\sso   {\rightarrow}

\newcommand\epsi  {\epsilon}
\newcommand\vepsi {\varepsilon}
\newcommand\vphi  {\varphi}
\newcommand\Neven {\N_{\mathrm{even}}}
\newcommand\Nodd  {\N_{\mathrm{odd }}}
\newcommand\Zeven {\Z_{\mathrm{even}}}
\newcommand\Zodd  {\Z_{\mathrm{odd }}}
\newcommand\Np    {\N_+}

% Text Shortcuts
\newcommand\open  {\big(}
\newcommand\qopen {\quad\big(}
\newcommand\close {\big)}
\newcommand\also  {\text{, }}
\newcommand\defi  {\text{ definition}}
\newcommand\defis {\text{ definitions}}
\newcommand\given {\text{given }}
\newcommand\case  {\text{if }}
\newcommand\syx   {\text{ syntax}}
\newcommand\rle   {\text{ rule}}
\newcommand\other {\text{else}}
\newcommand\set   {\ell et \text{ }}
\newcommand\ans   {\mathit{Ans.}}

% Set theory shortcuts
\newcommand\ra    {\rangle}
\newcommand\la    {\langle}

\newcommand\oto   {\leftarrow}

\newcommand\QED   {\quad\quad\mathscr{Q.E.D.}\;\;\blacksquare}
\newcommand\QEF   {\quad\quad\mathscr{Q.E.F.}}
\newcommand\eQED  {\mathscr{Q.E.D.}\;\;\blacksquare}
\newcommand\eQEF  {\mathscr{Q.E.F.}}
\newcommand\jQED  {\mathscr{Q.E.D.}}

\newcommand\dom   {\mathrm{dom}}
\newcommand\Img   {\mathrm{Im}}
\newcommand\range {\mathrm{range}}

\newcommand\trio  {\triangle}

\newcommand\rc    {\right\rceil}
\newcommand\lc    {\left\lceil}
\newcommand\rf    {\right\rfloor}
\newcommand\lf    {\left\lfloor}

\newcommand\lex   {<_{lex}}

\newcommand\az    {\aleph_0}
\newcommand\uaz   {^{\aleph_0}}
\newcommand\al    {\aleph}
\newcommand\ual   {^\aleph}
\newcommand\taz   {2^{\aleph_0}}
\newcommand\utaz  { ^{\left (2^{\aleph_0} \right )}}
\newcommand\tal   {2^{\aleph}}
\newcommand\utal  { ^{\left (2^{\aleph} \right )}}
\newcommand\ttaz  {2^{\left (2^{\aleph_0}\right )}}

\newcommand\n     {$n$־יה\ }

% Math A&B shortcuts
\newcommand\logn  {\log n}
\newcommand\logx  {\log x}
\newcommand\lnx   {\ln x}
\newcommand\cosx  {\cos x}
\newcommand\cost  {\cos \theta}
\newcommand\sinx  {\sin x}
\newcommand\sint  {\sin \theta}
\newcommand\tanx  {\tan x}
\newcommand\tant  {\tan \theta}
\newcommand\sex   {\sec x}
\newcommand\sect  {\sec^2}
\newcommand\cotx  {\cot x}
\newcommand\cscx  {\csc x}
\newcommand\sinhx {\sinh x}
\newcommand\coshx {\cosh x}
\newcommand\tanhx {\tanh x}

\newcommand\seq   {\overset{!}{=}}
\newcommand\slh   {\overset{LH}{=}}
\newcommand\sle   {\overset{!}{\le}}
\newcommand\sge   {\overset{!}{\ge}}
\newcommand\sll   {\overset{!}{<}}
\newcommand\sgg   {\overset{!}{>}}

\newcommand\h     {\hat}
\newcommand\ve    {\vec}
\newcommand\lv    {\overrightarrow}
\newcommand\ol    {\overline}

\newcommand\mlcm  {\mathrm{lcm}}

\DeclareMathOperator{\sech}   {sech}
\DeclareMathOperator{\csch}   {csch}
\DeclareMathOperator{\arcsec} {arcsec}
\DeclareMathOperator{\arccot} {arcCot}
\DeclareMathOperator{\arccsc} {arcCsc}
\DeclareMathOperator{\arccosh}{arccosh}
\DeclareMathOperator{\arcsinh}{arcsinh}
\DeclareMathOperator{\arctanh}{arctanh}
\DeclareMathOperator{\arcsech}{arcsech}
\DeclareMathOperator{\arccsch}{arccsch}
\DeclareMathOperator{\arccoth}{arccoth}
\DeclareMathOperator{\atant}  {atan2} 

\newcommand\dx    {\,\mathrm{d}x}
\newcommand\dt    {\,\mathrm{d}t}
\newcommand\dtt   {\,\mathrm{d}\theta}
\newcommand\du    {\,\mathrm{d}u}
\newcommand\dv    {\,\mathrm{d}v}
\newcommand\df    {\mathrm{d}f}
\newcommand\dfdx  {\diff{f}{x}}
\newcommand\dit   {\limhz \frac{f(x + h) - f(x)}{h}}

\newcommand\nt[1] {\frac{#1}{#1}}

\newcommand\limz  {\lim_{x \to 0}}
\newcommand\limxz {\lim_{x \to x_0}}
\newcommand\limi  {\lim_{x \to \infty}}
\newcommand\limh  {\lim_{x \to 0}}
\newcommand\limni {\lim_{x \to - \infty}}
\newcommand\limpmi{\lim_{x \to \pm \infty}}

\newcommand\ta    {\theta}
\newcommand\ap    {\alpha}

\renewcommand\inf {\infty}
\newcommand  \ninf{-\inf}

% Combinatorics shortcuts
\newcommand\sumnk     {\sum_{k = 0}^{n}}
\newcommand\sumni     {\sum_{i = 0}^{n}}
\newcommand\sumnko    {\sum_{k = 1}^{n}}
\newcommand\sumnio    {\sum_{i = 1}^{n}}
\newcommand\sumai     {\sum_{i = 1}^{n} A_i}
\newcommand\nsum[2]   {\reflectbox{\displaystyle\sum_{\reflectbox{\scriptsize$#1$}}^{\reflectbox{\scriptsize$#2$}}}}

\newcommand\bink      {\binom{n}{k}}
\newcommand\setn      {\{a_i\}^{2n}_{i = 1}}
\newcommand\setc[1]   {\{a_i\}^{#1}_{i = 1}}

\newcommand\cupain    {\bigcup_{i = 1}^{n} A_i}
\newcommand\cupai[1]  {\bigcup_{i = 1}^{#1} A_i}
\newcommand\cupiiai   {\bigcup_{i \in I} A_i}
\newcommand\capain    {\bigcap_{i = 1}^{n} A_i}
\newcommand\capai[1]  {\bigcap_{i = 1}^{#1} A_i}
\newcommand\capiiai   {\bigcap_{i \in I} A_i}

\newcommand\xot       {x_{1, 2}}
\newcommand\ano       {a_{n - 1}}
\newcommand\ant       {a_{n - 2}}

% Other shortcuts
\newcommand\tl    {\tilde}
\newcommand\op    {^{-1}}

\newcommand\sof[1]    {\left | #1 \right |}
\newcommand\cl [1]    {\left ( #1 \right )}
\newcommand\csb[1]    {\left [ #1 \right ]}

\newcommand\bs    {\blacksquare}

%! ~~~ Document ~~~

\author{שחר פרץ}
\title{היסטוריה 8 – על הציונות}
\begin{document}
	\maketitle
	\section{מבחן/בוחן/יוסי}
	סיימנו לאומיות כללית. המבחן נדחה לכבעוד שבועיים וחצי. המבחן הוא "עבודת כיתה עם ציון" (מבחן, קטן). ב־18.11. יארך שעה (עגולה, 60 דקות). ת.ז. עוד 15 דק'. כולל את ה חומר הנלמד עד שבוע הבא כולל, ובפרט את הלאומיות הכללית והגורמים לצמיחת הציונות.
	
	המבנה: שאלה אחת, עם קטע מקור. בשאלה שני סעיפים. עם ספר פתוח, כמו בבגרות, אך לא עם חומר פתוח. לכן כדאי לסמן בשספר עם דיבקיות. 
	
	בערך 15\%. 
	
	\section{סיכום – הלאומיות באירופה}
	\begin{itemize}
		\item חיזוק מבגורות השתייכות ליצירת מוקדי זהות חדשים לבני האדם. המהכפה התעשייתית, הלילון, ועוד תהליכים שגרמו לאובדן או היחלשות מוקד הזהות הישן. ווקאום שהלאומיות מילאה. 
		\item תופעה חדשה – בעידן הלאומיות. צמחה במאה ה־19. 
		\item שפה ומושגים חדשים – מולדת, שפת לאום המנון לאומי, ועוד. 
		\item אחד הכוחות החזקים בעולם. 
		\item התפשטות הראיון הלאומי. 
	\end{itemize}
	בעבר הזהות לא הייתה קשורה במיקום הגיאוגרפי. 
	\section{הגורמים לצמיחת הציונות – התנועה הלאומית היהודית}
	
	\subsection{הגדרות מחוץ לתוכנית הלימודים}
	ציונות := התנועה הלאומית היהודית. התנועה התפתחה יחסית מאורח, בסוף המאה ה־19. 
	
	בניגוד לזהות האיטלקית, לדוגמה, שם איטלקי יכול להיות מוסלמי, נוצרי או בכל דת אחרת, המדינה הלאומית היהודית היא גם בהקשר הדתי. זה יוצר בלבול (הפרדת דת מהמדינה וכו'). 
	
	\begin{itemize}
		\item ישראלי – בעל שתי הגדרות. הראשונה – בעל אזרחות ישראלית. השנייה – בעל אזרחות ישראלית, וחלק מהתנועה הציונית. בהגדרה הראשונה ייתכן פלסטינאי ישראלי, ובשנייה לא. 
		\item ערבים – מוצא אתני. לא קשור ללאום. 
		\item פלסטיני – לאום בפני עצמו. 
	\end{itemize}
	
	\subsection{רקע}
	לאחר חורבן בית שניה, האומה היהודית איבדה עצמאות. לאורך הגלות, התפיסה היהודית האורתודוקסית הפכה לכך שאסור לדחות את הקץ (בוא המשיח) בצורה התנגדה לתנועה הלאומית. 
	
	בסוף המאה ה־19, רוב היהודים היו במזרח איופה (רוסיה ופולין, בידח עם אוסטרו הונגריה). בתקופה הזו, היהודים במעטים (50K מתוך מיליונים) נמצאו כאן מסיבה דתית. הם לא תמכו בבוא הציונים. הרוב בארץ היו לא יהודים, ובארץ היה שלטון איפריאלי עותמני. כל אחד דיבר שפה אחרת. יש גם שוני תרבותי. יש גם שוני באופן ניהול החיים ההדתיים. 
	
	\subsection{הגורמים עצמם}
	הגורמים העיקריים הרלוונטיים תנועה היהודית בלבד: 
	\begin{itemize}
		\item כשלון האמנסיפציה
		\item שנאת ישראל במזרח אירופה ואנטישמיות מודנית במרכז ומערב אירופה
		\item השפעת התנועות הלאומיות. 
		\item תהליך החילון וכוחות פנימיים בחברה היהודית. 
	\end{itemize}
	
	\subsection{דוגמה מסרט}
	איגוץ זוננשטיין: 
	\begin{multicols}{2}
		\begin{itemize}
			\item ד"ר למשפטים
			\item שינוי שם הוא התנאי לתפקיד (שופט בית המשפט העיליון)
			\item ניתוק מהדת
			\item חגיגת סיום המאה ה־19 ותחילת המאה ה־20 הנוצרית, ולא היהודית. 
			\item אחים שינו שמות גם (רופאים)
			\item האבא דאג משינוי השם, היה חשוב לו לשמור על המסורת. בסוף ויתר בטענה שהשמות לא ניתנו ע"י ה'. 
		\end{itemize}
	\end{multicols}
	מנגד: 
	\begin{multicols}{2}
		\begin{itemize}
			\item קידוש
			\item חתונה יהודית, בבית כנסת. 
			\item יהודי. עדיין שומר על הדת. 
		\end{itemize}
	\end{multicols}
	
	אנשים בגרמניה קראו לעצמם "גרמנים בני דת משה". הדת היהודית נשמרה בבית, אבל למראית עין הם גרמנים רגילים. הזהות הלאומית הגרמנית חזקה יותר. הם רוצים להשתלב בחברה. 
	
	\subsection{אמנסיפציה}
	מקור המושג ברומה העתיקה, ובה ציין המושג שחרור של בן ממרותו של אביו והעמדתו ברשות עצמו. 
	
	משמעותו שוווין זכויות בחוק משפטי ־ הענקת כויות משפטיות בדבר שוויון מעמדם של היהודים והכרה בהם כאזרחים שווי זכויות. 
	
	בעבר היהודים קיבלו זכויות קקהילה. במקרים מסויימים האדון היה מזמין את היהודים לגור אצלו, כי הם היו יכולים להלוות בריבית וכו' ובכיולתם היה לעושת דברים שלנוצרים היה אסור. אם זאת, האמנסיפציה ניתנה באופן אישי ובאופן כללי ההגבלות היו רבות. 
	
	האמנציפניה התחילה בצרפת. בה, הזכויות ניתנו באופן קהילתי ולא באופן אישי. רק במערב אירופה הייתה אמנציפציה – \textbf{לא הייתיה אמנציפציה במזרח אירופה}, ובעקבות זאת האנטישמיות במזרח אירופה הייתה חותה ושונה במקורותיה. 
	\subsubsection{התמורות שחוללה האמנסיפיה}
	\begin{multicols}{2}
		\begin{itemize}
			\item התעוררות היהודים בחברה הסובבת. 
			\item חופש תנועה. 
			\item שינויים דמוכרפים – ריכוז בערבים הגדולות, יציאה מהשוכנות היהודיות (שאז נקראו גטאות), הגירה בין יבשתית. 
			\item תמורה בתעסורה – עיסוק בפוליטיקה, פרלמנט, שרים, קצונה, מקצועות חופשיים ועוד. 
			\item שילוב בהשכלה הגבוה, וחייב לימוד שפה נוספת. 
			\item פתיחות והתנגדות לשינויים שהובילו לשבירת המבנה המסורתי הדתי. 
			\item הקצה והתנגדות להשתלבות – קבוצות שהסתגרו והקשיחו את עמדותיהן כדי לא להתלשב בחברה (חרידים)
		\end{itemize}
	\end{multicols}
	
	מנדלסטון, אוצ'ילד, פרמייה, איינשטין, מאהלר, ועוד רבים אחרים. 
	
	\begin{center}
		\sen
		\begin{forest}
			[\he{אמנסיפציה}
			[\he{שילוב}]
			[\he{דחייה}]
			]
		\end{forest}
		\she
	\end{center}
	
	מבחינה חברתית, הייתה דחייה גדולה לאמנסיפציה. בפרט מסיבות של קנאה – היהודית הצליחו להתקדם באופן לא פורפורציונלי לכמה היו בחברה. לכן, יהודים רבים לא התקבלו לעמדות כאלו ואחרות מעצם היותם יהודים. 
	
	\subsection{אנטישמיות מודנית}
	שנאת יהודים לא כי "הם הרגו את ישו", ולא מסיבות דתיות, לפיה היהודים רוצים לשהתלט על העולם, וכו'. שנאת ישראל והודים – תמיד יש פתרון. בשנאת ישראל יש פתרון – המרת דת. לאנטישמיות מודנית אין פתרון – מוצאך יהודי, בכל האשמה. 
	
	אם מישהו ביוון שונא יהודים כי הם מאמינים באל אחד ולא עובדים אלילים, די בהמרת דת כדי "לפתור את הבעיה". באופן דומה גם בימי הביניים (אם כי שם קראו להם "היהודים החדשים", tag שנשאר לדור אחד או שניים). האנטישמיות המודנית היא גזעית ולכן לא די בכך. 
	
\end{document}