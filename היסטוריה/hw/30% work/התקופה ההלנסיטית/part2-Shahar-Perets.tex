%! ~~~ Packages Setup ~~~ 
\documentclass[]{article}
\usepackage{lipsum}
\usepackage{rotating}


% Math packages
\usepackage[usenames]{color}
\usepackage{forest}
\usepackage{ifxetex,ifluatex,amssymb,amsmath,mathrsfs,amsthm,witharrows,mathtools,mathdots}
\usepackage{amsmath}
\WithArrowsOptions{displaystyle}
\renewcommand{\qedsymbol}{$\blacksquare$} % end proofs with \blacksquare. Overwrites the defualts. 
\usepackage{cancel,bm}
\usepackage[thinc]{esdiff}



% code 
\usepackage{xcolor}

\definecolor{codegreen}{rgb}{0,0.35,0}
\definecolor{codegray}{rgb}{0.5,0.5,0.5}
\definecolor{codenumber}{rgb}{0.1,0.3,0.5}
\definecolor{codeblue}{rgb}{0,0,0.5}
\definecolor{codered}{rgb}{0.5,0.03,0.02}
\definecolor{codegray}{rgb}{0.96,0.96,0.96}


% Design
\usepackage[labelfont=bf]{caption}
\usepackage[margin=0.5in]{geometry}
\usepackage{multicol,multirow}
\usepackage{tabularx,makecell}
\usepackage[skip=4pt, indent=0pt]{parskip}
\usepackage[normalem]{ulem}
\forestset{default}
\renewcommand\labelitemi{$\bullet$}
\usepackage{graphicx}
\graphicspath{ {./} }

\usepackage[colorlinks]{hyperref}
\definecolor{mgreen}{RGB}{25, 160, 50}
\definecolor{mblue}{RGB}{30, 60, 200}
\usepackage{hyperref}
\hypersetup{
    pdftitle={Document by Shahar Perets},
    %	pdfpagemode=FullScreen,
}
\usepackage{titlesec}
\titleformat{\section}[block]
{\fontsize{15}{15}}
{\sen \dotfill (\thesection)\dotfill\she}
{0em}
{\MakeUppercase}



% Hebrew initialzing
\usepackage[bidi=basic]{babel}
\PassOptionsToPackage{no-math}{fontspec}
\babelprovide[main, import, Alph=letters]{hebrew}
\babelprovide[import]{english}
\babelfont[hebrew]{rm}{David CLM}
\babelfont[hebrew]{sf}{David CLM}
%\babelfont[english]{tt}{Monaspace Xenon}
\usepackage[shortlabels]{enumitem}
\newlist{hebenum}{enumerate}{1}

% Language Shortcuts
\newcommand\en[1] {\begin{otherlanguage}{english}#1\end{otherlanguage}}
\newcommand\he[1] {\she#1\sen}
\newcommand\sen   {\begin{otherlanguage}{english}}
    \newcommand\she   {\end{otherlanguage}}
\newcommand\del   {$ \!\! $}

\newcommand\npage {\vfil {\hfil \textbf{\textit{המשך בעמוד הבא}}} \hfil \vfil \pagebreak}
\newcommand\ndoc  {\dotfill \\ \vfil {\begin{center}
            {\textbf{\textit{שחר פרץ, 2025}} \\
                \scriptsize \textit{קומפל ב־}\en{\LaTeX}\,\textit{ ונוצר באמצעות תוכנה חופשית בלבד}}
    \end{center}} \vfil	}

\newcommand{\rn}[1]{
    \textup{\uppercase\expandafter{\romannumeral#1}}
}

\makeatletter
\newcommand{\skipitems}[1]{
    \addtocounter{\@enumctr}{#1}
}
\makeatother

\newcommand\cb[1]   {\color{codeblue}#1\color{black}}


%! ~~~ Document ~~~

\author{שחר פרץ, אנטיוכוס השביעי}
\title{\textit{מטלה מסכמת $\sim$ התקופה ההלניסטית}}
\date{15 ביוני 332 לפני הספירה}
\begin{document}
    \maketitle
    
    \section{}
    בשנת 332 לפני הספירה, מנהיג יוון בשם ``אלכסנדר מוקדון'' כבש חלקים מאירופה, אסיה ואפריקה, ביניהם ארץ ישראל. לאחר מותו נחלקה האימפריה שלו לכדי מספר ממלכות, כך שב־301 לפני הספירה הממלכה היוונית הקרויה ``בית תלמי'' שלטה בארץ. לאחר מכן בשנת 200 לפני הספירה הוחלף בית תלמי בבית סלווקוס. האימפריה היוונית שהגדיל מוקדון הביאה איתה למקומות שכבשה את התרבות היוונית, הקרויה התרבות ההליניסטית, והובילה להפצת ה``הלניזם'' (נגזרת של המילה ``הלאס'' משמעותה יוון) – היא התרבות היוונית באותה התקופה. 
    
    תרבות יוון כללה מגוון מאפיינים, בינהם השפה (יוונית), חינוך (בגימנסיון, בו למדו גאומטריה, מוסיקה, חינוך גופני ועוד), חיי תרבות שהתבטאו בתיאטראות, טקסים פולחניים ואסיפות, וענפי מדעים שחקרו כגון מתמטיקה, פילוסופיה, רטוריקה ושפה ועוד, הערצת הגוף הגברי, בעתות רבות דמוקרטיה ישירה, הפצת מידע בניירות פפירוס, ועוד. יוון חולקה ל''פוליסים'', יחידה הדומה לעיר שספקה את צרכיה שלה בתוך עצמה. בערים שכבש אלכסנדר מוקדון הוקמו מוסדות ובניינים יוונים, בינהם גימנסיון ומקדשים לפולחן אלי יוון. בתרבות יוון, תפקיד הכהן עבר בין האזרחים ולא בשושלת. ההלניזם היא התרבות הנוצרה מההיתוך בין התרבות היוונית, לתרבות שהייתה במזרח, והתרבות ההלנסטית היא השילוב הזה. 
    
    היו בין האנשים שבאיזור יהודה שהתנגדו לתפוצת התרבות ההלסנטית, וכן על גזירות אנטיוכוס הרביעי, מלך בית סלווקוס. ובעקבות כך פרץ מרד המכבים, שתוך 3 שנים הוביל לטיהור בית המקדש וביטול גזירות אנטיכוס, וכן לעליית בית החשמונאים (השושלת שהנהיגה את המרד) שלאורך דורותיהם הרחיבו את השטחים של ממלכתם. עם התהוות ממלכת יהודה, ניתן למנות שני תנועות בולטות שנוצרו בה – הפרושים, שראו את השליט כבעל חובות דתיות בדיוק כמו העם, והצדוקים שבדומה לאמונה ההלניסטית התנגדו לפרושים. בתקופת החשמונאים, הסנהדרין (מקור שמו הלניסטי) היווה את ``השרות השופטת'' של ממלכת יהודה. 
    
    בשאלה זו נסקור את קורות חייהם של שני מנהיגים חשמונאים, יוחנן הורקנוס הראשון, ובנו – אלכסנדר ינאי. 
    \begin{itemize}
        \item \textbf{יוחנן הורקנוס הראשון. }יוחנן הוּרקנוס הראשון היה בנו של שמעון ונכדו של מתתיהו הכהן, נצר ומשפחת החשמונאים. הוא נולד במאה השנייה לפנה''ס ומת ב־104 לפנה''ס. \cb{הוא נכנס לתפקידו כנשיא יהודה ב־134 לפנה''ס וכיהן בתפקיד זה עד מותו}. עלייתו לשלטון נגרמה לאחר שאנטיוכוס השביעי רצח בעורמה את אביו, המלך. 
        השם שצירף לשמו היהודי, היה ``הורקנוס'' – \cb{מדובר בשם הלניסטי}. למרות זאת, הקפיד לשמור גם על שמו היהודי, ``יוחנן'', וכמנהג שושלת החשמונאים כינה עצמו ``חבר היהודים''. על אף שגייר אנשים רבים, ועל אף שראה עצמו ככהן גדול, \cb{רצח את חכמי הפרושים}. אנטיוכוס השביעי שליט בית סלווקוס, \cb{הוביל למצור על יהודה בשנים 132-134}, עם עלייתו של יוחנן לשלטון. בתום המצור אנטיוכוס הגיע להסכם, שכלל ניתוץ חומות ירושלים החלפת בין ערובה בכסף, והחזרת איזורים שנכבשו. כאשר אנטיוכוס השביעי מת בית סלווקוס נכנס לעת של חוסר יציבות, בשנת 112 לפני הספירה החל יוחנן לצאת למסע כיבושים וכבש הרים הלניסטיות רבות, מצפון, דרום, מזרח ומערב לממלכת יהודה. בין הערים הבולטות שכבש, נמנת שומרון, זאת לאחר מצור שארך שנה שלמה. כמו כן סקיתופוליס (בית שאן של היום), עמק יזראל, והגליל התחתון. הוא גייר את העמים בחלק ניכר מהמקומות שכבש. 
        יוחנן גם פעל ביחסי חוץ על־מנת לאחד את מצרים ורומא, אויבי בית סלווקוס, ביחד איתו. 
        \item \textbf{אלכסנדר ינאי.}
        אלכסנדר ינאי היה בנו של יוחנן הורקנוס הראשון, נצר למשפחת החשמונאים, ובימיו השפעת התרבות ההלניסטית על יהודה הגיעה לשיאה. הוא ס\cb{שלט בשנים 103 לפנה''ס עד 76 לפנה''ס}, ונולד ב־127 לפנה''ס. בימיו הגיעה ממלכת יהודה לשיא גודלה. \cb{בהתאם למנהגים ההלניסטים, אלכסנדר ינאי מינה את עצמו להנהגה, וכינה את עצמו ``מלך''} \, (ולא ``נשיא''). ברוב מטבעותיו אף לא כינה את עצמו ``כהן גדול'' ו\cb{השתמש בעיקר בשמו היווני אלכסנדר}. ההתנכלות כלפי הפרושים בתקופתו הגיעה לשיאה. על מטבעותיו הופיעו בעיקר סמלים הלנסטיים. בדומה לראיה ההלניסטית, ראה עצמו כשליט אבסולוטי וביטל את הכיתוב ``חבר היהודים'' על מטבעותיו. הוא העביר את כל סמכויות השפיטה מידי ארגונים דתיים, לידיו. במהלך שנות מלכותו הרחיב את ממלכתו אף יותר, הגדיל משמעותית את אחזיתו בעבר הירדן, וכבש את עזה ושטחים דרומיים רבים נרחבים. בגלל שנאתו לפרושים, ובגלל שראה עצמו כתמצית החוק, הצדק והשליט האבסולוטי של יהודה (בניגוד לתפיסה היהודתי המקובלת עד אז, בה ישנם ארגונים כמו הסנהדרין שאחרים על משפט), במללכתו היו מגוון מרידות גדולות עד מאוד, שהובילו למלחמת אחים בפועל, שינאי הצליח לדכא אך ורק בזכות תשלום לשכירי חרב. מקורות יהודיים מספרים שחמישים אלף הרוגים נפלו באותן מלחמות. גינוניו ומלבושו דמו לאלו של מלכים הלנסטיים אחרים. 
    \end{itemize}
    
    \npage
    
    \section{}
    
    \begin{enumerate}[A.]
        \item נשלים את הטבלה הבאה: 
    \end{enumerate}
    
    \begin{tabularx}{\textwidth}{|X|X|c|}
        \hline
        \centering\textbf{אלכסנדר ינאי} & \centering\textbf{יוחנן הורקנוס הראשון} & \\
        \hline
        כתובת בעברית ``יהונתן המלך'' (בנו של שאול), וכתובת ביוונית של המלך אלכסנדר. &
        כתובת בעברית קדומה ``יהוחנן הכהן הגדול וחבר היהודים''. &
        \textbf{שפת ההטבעה} \\ 
        \hline
        כנה עצמו ``מלך'', ועל אף ששמו היה ``יהונתן'' קיצר את שמו לכדי ``ינאי''. קיימים מטבעות בהם הוא מכונה כוהן גדול. &
        הורקנוס הראשון (הורקנוס כנראה הוא שם של מקום, וזהו שם יווני). מכנה עצמו במטבע ``הכהן הגדול'', ושמר על שמו ``יוחנן'' היהודי. במטבעות כינה עצמו ``חבר היהודים''. 
        &
        \textbf{תואר המנהיג} \\
        \hline
        עוגן המייצג את כיבוש ערי החוף, ובתרבות ההלניסטית מייצג איחול להשגת היעד הנכון + כוכב מוקף בנזר מלכים, כאשר הכוכב מייצג את המלך. נזר המלכים הוא סימן הלניסטי, וגם העוגן.  &
        סמלים הלניסטיים כמו קרנות שפע, ועטרה (סמל הלניסטי המייצג מנהיגות ושלטון). לצידם סמלים יהודים, כמו רימון, ושילוב בין רימון לקרני שפע (סמל חשמונאי עם השפעות הלניסטיות). 
        & \textbf{סמלים יהודים/הלניסטיים} \\
        \hline
        ינאי השתמש בכוכב המוקף בנזר מלכים, כאשר הכוכב מייצג את המלך, בשביל לייצג את המלכות. שני אלו שמלים אלנסטיים. &
        הורקנוס השתמש בעטרה (זר) כדי לייצג את מנהיגות השלטון (זהו סמל הלניסטי)& \textbf{סמלי ההנהגה} \\
        \hline
    \end{tabularx}
    \begin{enumerate}[A.]
        \skipitems{1}
        
        \item בשנת 332 לפני הספירה, מנהיג יוון בשם ``אלכסנדר מוקדון'' כבש חלקים מאירופה, אסיה ואפריקה, ביניהם ארץ ישראל. לאחר מותו נחלקה האימפריה שלו לכדי מספר ממלכות, כך שב־301 לפני הספירה הממלכה היוונית הקרויה ``בית תלמי'' שלטה בארץ. לאחר מכן בשנת 200 לפני הספירה הוחלף בית תלמי בבית סלווקוס. האימפריה היוונית שהגדיל מוקדון הביאה איתה למקומות שכבשה את התרבות היוונית, הקרויה התרבות ההליניסטית, והובילה להפצת ה``הלניזם'' (נגזרת של המילה ``הלאס'' משמעותה יוון) – היא התרבות היוונית באותה התקופה. 
        
        תרבות יוון כללה מגוון מאפיינים, בינהם השפה (יוונית), חינוך (בגימנסיון, בו למדו גאומטריה, מוסיקה, חינוך גופני ועוד), חיי תרבות שהתבטאו בתיאטראות, טקסים פולחניים ואסיפות, וענפי מדעים שחקרו כגון מתמטיקה, פילוסופיה, רטוריקה ושפה ועוד, הערצת הגוף הגברי, בעתות רבות דמוקרטיה ישירה, הפצת מידע בניירות פפירוס, ועוד. יוון חולקה ל''פוליסים'', יחידה הדומה לעיר שספקה את צרכיה שלה בתוך עצמה. בערים שכבש אלכסנדר מוקדון הוקמו מוסדות ובניינים יוונים, בינהם גימנסיון ומקדשים לפולחן אלי יוון. בתרבות יוון, תפקיד הכהן עבר בין האזרחים ולא בשושלת. ההלניזם היא התרבות הנוצרה מההיתוך בין התרבות היוונית, לתרבות שהייתה במזרח, והתרבות ההלנסטית היא השילוב הזה. 
        
        היו בין האנשים שבאיזור יהודה שהתנגדו לתפוצת ההלניזם ביהודה, על אחת וכמה וכמה לאחר גזירות אנטיוכוס הרביעי, מלך בית סלווקוס. בעקבות כך פרץ מרד המכבים, שתוך 3 שנים הוביל לטיהור בית המקדש וביטול גזירות אנטיכוס, וכן לעליית בית החשמונאים (השושלת שהנהיגה את המרד) שלאורך דורותיהם הרחיבו את השטחים של ממלכתם. 
        
        על אף שהחשמונאים התחילו את מרידותיהם על נס גירוש ההלניזם מיהודה, לאורך השנים ההלניזם חלחל הן למלכי החשמונאים והשפיע על תרבותם במגוון דרכים. 
        
        נתבונן במלכים אלכסנדר ינאי ויוחנן הורקנוס הראשון, שניהם נצרים לשוללת החשמונאים ושניהם הנהיגו אלת מדינת יהודה. הראשון הוא בנו של השני. 
        
        נבחין במטבעות שיוצרו בתקופה של יוחנן הורקנוס הראשון – בצד אחד שלהם, היו קרנות שפע (סמל הלניסטי המייצג פריון, שפע ושגשוג) יחדיו עם רימון (סמל יהודי) ששילובם יחד נחשב סממן של שוששלת החשמונאים. בצידם השני, הופיע כיתוב בעברית קדומה ``יהוחנן הכהן הגדול וחבר היהודים'' (``חבר היהודים'' הוא תואר מקובל לנשאי יהודה). השימוש בכתב עברי קדום בולט במיוחד על רקע העובדה שהכתב האדומי היה נפוץ יותר באותה התקופה, וככל הנראה הקדם העברי שימש תזכורת לימי הזוהר של ממלכת יהודה בזמן בית ראשון. נוסף על כך, צד זה של המטבע עוטר בעטרה, סמל הלניטי המייצג מנהיגות ושטון. 
        
        עתה, נתבונן במטבעות מתקופתו של אלכסנדר ינאי. בצידם האחד ישנו עוגן, סמל הלניסטי המייצג איחול להשגת היעד הנכון. נוסף על כך הוא סימן את כיבוש ערי החוף וסיפוחן אליו. כמו כן הופיע כתובת ביוונית עליה היה כתוב ``המלך אלכסנדר'', כאשר התואר ``מלך'' היה מקובל בתרבות ההלניסטית אך לא נפוץ בשושלת החשמונאים. בצדו השני של המטבע מופיע כוכב המוקף בנזר מלכים, כאשר הכוכב אמור לייצג את המלך. כיאה למסורת היהודית, אלכסנדר לא הטביע את דיוקנו (מעשה מקובל בתרבות ההלניסטית), והחליף אותו בכוכב. עם זאת, נזר המלכים הוא סממן הלניסטי מובהק. נוסף על זאת, בצד זה הופיע כתובת בעברית – ``יהונתן המלך''. 
        
        נתבונן בקטע המקור מתוך הספר ``יהדות ויוונות  בעת העתיקה – עימות או מיזוג?'' (תש''ס 2000, עמ 23, מאת ישראל לוין, הוצאת מרכז זלמן שזר לתולדות ישראל). במקור, ישראל לוין מסביר כיצד נוצרה התרבות ההלניסטית, והוא טוען שהלניזם אינו השפעה יוונית בלבד על העולם הלא יווני, אלא מערכת גומלין מורכבת בין מספר רב של גורמים תרבותיים. הוא עוד טוען שהתרבות ההלנסטית היוותה כור היתוך תרבותי ביחס לתרבויות מסביב. 
        
        נבחין שהמטבעות של אלכסנדר ינאי ויוחנן הראשון, תומכים בדבריו של ישראל לוין, במגוון צורות. 
        \begin{itemize}
            \item
             ראשית, במבטע מימיו של אלכסנדר ינאי, מופיעות קרנות שפע לצד רימון. יש כאן מפגש והשתלבות של התרבות היהודית בהלניסטית – \cb{קרני השפע ההלניסטיות, המציינות שגשוג, ביחס עם פרי הרימון המזוהה עם היהדות, יוצרים משמעות חדשה של שפע ושגשוג שצפוי לממלכה החשמונאית. }זאת, בהתאם לדבריו של לוין ``אין לראות את ההלניזם כהשפעה יוונית גרידא [...] אלא כמערכת גומלין בין מספר רב של גורמים תרבותיים''. יש כאן מערכת גומלין שלמה, בין הסמלים היהודים לבין הסמלי ההלניסטיים, שיוצרים דבר חדש. 
            \item
             שנית, נתבונן במטבעותיו של אלכסנדר ינאי. גם כאן, ישנו שילוב בין תרבויות. מצדו האחד של המטבע מופיע בכתב יווני שמו של המלך ביוונית (``המלך אלכסנדר'') ומצדו השני, בכיתוב עברי, שמו בעברית (``יהונתן המלך''). גם כאן, \cb{ישנה מערכת של יחסים בין התרבות ההלניסטית ליהודית, מצד אחד משתמש בשמו העברי בעברית ומצד שני מפרסם את שמו ההלניסטי ביוונית}. עובדה זו מחזקת גם היא את טענתו של ישראל לוין לפיהם ההלניזם הוא ``מערכת גומלין בין מספר רב של גורמים תרבותיים''. 
            \item
             אחרונה, נבחין בשינוי במטבעות בין השנים. בעוד מטבעותיו של יוחנן הורקנוס הראשון שופעים סמלים יהודיים כמו כיתוב יהודי, רימון, קישורים לכתב עברי קדום, והביטוי ``חבר היהודים'', בתקופתו של ינאי הסממן היהודי היחידי על המטבע הוא הכתב היהודי שמופיע באחד מצדדיו. ינאי, בנו של אלכסנדר שבא אחריו, השתלב כבר יותר לעומק בסביבה ההלניסטית שמסביבו. \cb{בכך, נבחין שממלכת יהודה ותרבותה ``הותכה'' לאורך השנים עם הרעיונות והאופנות של העמים ההלניסטיים מסביב}. זאת, בהתאם לדבריו של ישראל לוין – ``עולם ההלנסטי היה לכור היתוך תרבותי, לשוק חופשי של רעיונות ואופנות הפתוח לכל [...]''. 
        \end{itemize}
        
        לסיום, דברים של ישראל לוין, לפיהם התרבות ההלניסטית היוותה כור היתוך תרבותי, ומערכת יחסי גומלין שלמה בינה לבין התרבויות האחרות, מחוזקים ע''י המטבעות של מלכי התקופה, במגוון דרכים שונות. 
        
        \item בשנת 332 לפני הספירה, מנהיג יוון בשם ``אלכסנדר מוקדון'' כבש חלקים מאירופה, אסיה ואפריקה, ביניהם ארץ ישראל. לאחר מותו נחלקה האימפריה שלו לכדי מספר ממלכות, כך שב־301 לפני הספירה הממלכה היוונית הקרויה ``בית תלמי'' שלטה בארץ. לאחר מכן בשנת 200 לפני הספירה הוחלף בית תלמי בבית סלווקוס. האימפריה היוונית שהגדיל מוקדון הביאה איתה למקומות שכבשה את התרבות היוונית, הקרויה התרבות ההליניסטית, והובילה להפצת ה``הלניזם'' (נגזרת של המילה ``הלאס'' משמעותה יוון) – היא התרבות היוונית באותה התקופה. 
        
        תרבות יוון כללה מגוון מאפיינים, בינהם השפה (יוונית), חינוך (בגימנסיון, בו למדו גאומטריה, מוסיקה, חינוך גופני ועוד), חיי תרבות שהתבטאו בתיאטראות, טקסים פולחניים ואסיפות, וענפי מדעים שחקרו כגון מתמטיקה, פילוסופיה, רטוריקה ושפה ועוד, הערצת הגוף הגברי, בעתות רבות דמוקרטיה ישירה, הפצת מידע בניירות פפירוס, ועוד. יוון חולקה ל''פוליסים'', יחידה הדומה לעיר שספקה את צרכיה שלה בתוך עצמה. בערים שכבש אלכסנדר מוקדון הוקמו מוסדות ובניינים יוונים, בינהם גימנסיון ומקדשים לפולחן אלי יוון. בתרבות יוון, תפקיד הכהן עבר בין האזרחים ולא בשושלת. ההלניזם היא התרבות הנוצרה מההיתוך בין התרבות היוונית, לתרבות שהייתה במזרח, והתרבות ההלנסטית היא השילוב הזה. 
        
        היו בין האנשים שבאיזור יהודה שהתנגדו לתפוצת התרבות היוונית (ההלניסטית), וכן על גזירות אנטיוכוס הרביעי, מלך בית סלווקוס. ובעקבות כך פרץ מרד המכבים, שתוך 3 שנים הוביל לטיהור בית המקדש וביטול גזירות אנטיכוס, וכן לעליית בית החשמונאים (השושלת שהנהיגה את המרד) שלאורך דורותיהם הרחיבו את השטחים של ממלכתם. עם התהוות ממלכת יהודה, ניתן למנות שני תנועות בולטות שנוצרו בה – הפרושים, שראו את השליט כבעל חובות דתיות בדיוק כמו העם, והצדוקים שבדומה לאמונה ההלניסטית התנגדו לפרושים. בתקופת החשמונאים, הסנהדרין (מקור שמו הלניסטי) היווה את ``השרות השופטת'' של ממלכת יהודה. 
        
        על אף שהחשמונאים התחילו את מרידותיהם על נס גירוש ההלניזם מיהודה, לאורך השנים ההלניזם חלחל הן למלכי החשמונאים והשפיע על תרבותם במגוון דרכים. 
        
        נתבונן במלכים אלכסנדר ינאי ויוחנן הורקנוס הראשון, שניהם נצרים לשוללת החשמונאים ושניהם הנהיגו אלת מדינת יהודה. הראשון הוא בנו של השני. 
        
        בחלק הראשון של השאלה, הצגנו לעומק את אלכסנדר ינאי והורקנוס הראשון [שרית, אם הייתי צריך להעתיק גם את זה לכאן, תתייחסי לזה כאילו שני הבוּלטים הועתקו לכאן]. נתבונן בתבחינים הם יחסם לדת היהודית, ויחסם לתרבות ההלניסטית, ונשווה בינהם. 
        \begin{itemize}
            \item \textbf{הדת יהודית. }\textit{יוחנן הורקנוס הראשון} העריך מאוד את הדת היהודית, ולכן הקפיד לשמר את שמו היהודי, כנה עצמו כהן גדול, וגייר (העביר מדת אחרת ליהדות) עמים רבים שכבש. כמקובל, הוא קרא לעצמו ``חבר היהודים''. עם זאת, הוא רצח חלק מחכמי הפרושים. במטבעותיו היו מספר סימנים יהודים כגון רימון, כתב עברי קדום, והתואר  ``חבר היהודים'' המסורתי. 
            
            \textit{אלכסנדר ינאי} התעלם כמעט כליל מהיותו יהודית, הוא מנה עצמו למלך (בניגוד למקובל בדת היהודית), לא נהג להשתמש בשמו היהודי ``יהונתן''. במטבעותיו לא הופיע סמלים יהודים (רק מעט כתב בעברית), וצמצם את השפעת של מוסדות יהודים כמו הסינהדרין. הוא רצח רבים מן הפרושים והוביל למלחמת אחים בתוך עם ישראל שדוכאה באמצעות שכירי חרם חיצוניים ששכר. 
            
            \cb{\textbf{באופן דומה, }אלכסנדר ינאי ויוחנן הורקנוס הראשון לא אהבו את הפרושים ורצחו את חכמיהם. \textbf{באופן שונה, }ינאי היה אכזרי הרבה יותר ברציחותיו, ובניגוד לאביו אלכסנדר, הוא התכחש למוסדות יהודיים כמו הסינהדרין, נמנע מלקרוא עצמו ``חבר היהודים'', ולא פעל לקידום הדת היהודית באמצעים כגון גיור. }
            
            \item \textbf{התרבות ההלניסטית. }\textit{יוחנן הורקנוס הראשון }הרבה להשתמש בסממנים הלנסטיים במטבעות שייצר, ואימץ לעצמו שם הלניסטי – הורקנוס. 
            
            \textit{אלכסנדר ינאי} הרבה אף יותר להשתמש בסימנים הלנסטיים במטבעותיו, והשתמש לרוב בשמו ההלניסטי ``אלכסנדר''. כמקובל בתרבות ההלנסטית הוא מנה את עצמו לבד למלך, וראה את עצמו כשליט אבסולוטי ויחיד על פני העם. הוא השתמש בכינוי ההלניסטי ``מלך'' כדי לתאר את עצמו. הוא פעל בדומה למנהיגים הלנסטים אחרים במלבושו והתנהגותו. 
            
            \cb{\textbf{באופן דומה, }שניהם השתמשו בסממנים הלנסטיים במטבעותיהם, ולשניהם היה שם הלניסטי מקביל. \textbf{באופן שונה, }מטבעותיו של הורקנוס  כללו פחות סמלים הלנסטיים, וינאי ראה את עצמו כמלך יחיד ואבסולוטי על פני העם (בניגוד לאלכסנדר שנתן מקום לסינהדרין ולגופים יהודיים נוספים). ינאי גם פעל כמנהיג הלנסטי בבגדיו והתנהגותו. }
        \end{itemize}
        
        \item
        במהלך המאה ה־2 לפנה''ס, שלטה באיזור יהודה השושלת החשמונאית, והנהיגה את היהודים באיזור. את ממלכת יהודה הקיפו מספר ממלכות יווניות, ותרבותן חלחלה לתוך מלכי יהודה והשפעה על פועלם. לשילוב בין תרבות יוון לבין התרבות המזרח התיכון קוראים התרבות ההלנסטית. בין המלכים הללו מצוין יוחנן הורקנוס הראשון, ובנו, אלכסנדר ינאי. שניהם השתמשו בסמלים הלנסטיים במטבעותיהם, כאשר ינאי עשה שימוש אף יותר נרחב באותם הסמלים. בימיו של ינאי התרחשה מלחמת אחים, עקב חוסר הסכמתו וטבח בזרם בתרבות היהודית הקרוי הפרושים. זרם זה דגל בין היתר באמונה כלפי מגבלות דתיות מסויימות על השליט, שינאי לא חיבב (הוא הושפע יותר מצורת השלטון ההלנסטית). המרדים בעקבות כך התגלגלו למלחמת אחים. שניהם הרחיבו משמעותית את שטחי יהודה במהלך כהונתם. 
        
        \textit{\textbf{הערה. }שרית, כאן אמורה להופיע העתקה מדויקת של כל סעיף ג' כולל פתיחתו, ע''מ שאוכל לקשר את התוכן של סעיף זה להשוואה שעשיתי בסעיף ג'. כדי לא להכביד על המסמך אני לא באמת אבצע את ההעתקה, אבל תתייחסי להמשך השאלה כאילו יש כאן העתק מדויק של סעיף ג'}
        
        מההשוואה, ניכר שהנשיא הורקנוס פחות פעל והושפע מהתרבות ההלנסטית, בעוד המלך ינאי פעל כמעט כמו כל מלך הלנסטי אחר. עוד ניכר מההשוואה, שבתקופתו של ינאי היו מלחמות פנים ומצב מאוד לא יציב בתוך ההמלכה, לעומת בתקופתו של הורקנוס שהתאפיינה בשקט יחסי. עוד נבחין שינאי הצליח להגדיל אף יותר את השטחים שכבש אביו. 
        
        מכך, נסיק שלושה דברים. 
        
        \begin{itemize}
            \item \textbf{ככל שההשפעה ההלנסטית גברה, ממלכת יהודה צברה שטחים. }ככל שהתרבות ההלניסטית הלכה וצברה אחיזה ביהודה עם הזמן, הממלכה הלכה והגדילה את שטחיה, ואף בימי ינאי הגיע לשיאה הן ההשפעה ההלנסטית והן גודלה. עם זאת, אין סיבה להאמין שצבירת השטחים \textbf{הייתה בזכות ההלניזם} (``correlation does not imply causation'' / קום הוק). 
            \item \textbf{ככל שההשפעה ההלנסטית גברה, התגברו הבעיות הפנימיות בתוך יהודה. }עקב ההתנגדות לתרבות ההלנסטית בתוך יהודה, בעוד אצל הורקנוס היה שקט פנימי יחסי, בימיו של השליט הכמעט הלנסטי הוא המלך ינאי היו מלחמות פנימיות עקובות מדם, שנבעו מההתנגדות לתרבות ההלנסטית. 
            \item \textbf{השפעת התרבות ההלנסטית הייתה חזקה מאוד. }עוד בימיו של הנשיא הורקנוס התרבות ההלנסטית בלטה בהיבטים רבים בחייו, כמו מטבעותיו ושמו. אצל ינאי, התרבות ההלנסטית השליטה כבר את מרבית פעולותיו, כמו אופן התייחסותו אל עצמו כמלך, בגדים וגינונים, ועוד. עם זאת, עד ימיו של ינאי, התרבות והמוסרת היהודית עוד המשיכו להתקיים באופן ניכר לא רק ברמת העם, אלא בכל המקומות בהררכייה השלטונית, ואף תחת שלטונו של ינאי היו רבים שתנגדו להשפעה ההלנסטית עליו ומרדו בו. כלומר, חירף חוזקת ההשפעה ההלנסטית, היא לא מחקה כליל את התרבות היהודית מהממלכה החשמונאית, והתקיימה והתקשרה איתה ביחד. 
        \end{itemize}
        
        לסיום, התרבות ההלנסטית \cb{לא מחקה את התרבות היהודית אלא חייה לצידה}, למרות השפעה מכרעת על יהודה. התרבות ההלנסטית הובילה לאי־יציבות פנימית בתוך יהודה. 
        
    \end{enumerate}
    
    
    
    \ndoc
\end{document}