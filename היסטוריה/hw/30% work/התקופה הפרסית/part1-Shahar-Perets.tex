%! ~~~ Packages Setup ~~~ 
\documentclass[]{article}
\usepackage{lipsum}
\usepackage{rotating}


% Math packages
\usepackage[usenames]{color}
\usepackage{forest}
\usepackage{ifxetex,ifluatex,amssymb,amsmath,mathrsfs,amsthm,witharrows,mathtools,mathdots}
\usepackage{amsmath}
\WithArrowsOptions{displaystyle}
\renewcommand{\qedsymbol}{$\blacksquare$} % end proofs with \blacksquare. Overwrites the defualts. 
\usepackage{cancel,bm}
\usepackage[thinc]{esdiff}



% code 
\usepackage{xcolor}

\definecolor{codegreen}{rgb}{0,0.35,0}
\definecolor{codegray}{rgb}{0.5,0.5,0.5}
\definecolor{codenumber}{rgb}{0.1,0.3,0.5}
\definecolor{codeblue}{rgb}{0,0,0.5}
\definecolor{codered}{rgb}{0.5,0.03,0.02}
\definecolor{codegray}{rgb}{0.96,0.96,0.96}


% Design
\usepackage[labelfont=bf]{caption}
\usepackage[margin=0.6in]{geometry}
\usepackage{multicol}
\usepackage[skip=4pt, indent=0pt]{parskip}
\usepackage[normalem]{ulem}
\forestset{default}
\renewcommand\labelitemi{$\bullet$}
\usepackage{graphicx}
\graphicspath{ {./} }

\usepackage[colorlinks]{hyperref}
\definecolor{mgreen}{RGB}{25, 160, 50}
\definecolor{mblue}{RGB}{30, 60, 200}
\usepackage{hyperref}
\hypersetup{
    colorlinks=true,
    citecolor=mgreen,
    linkcolor=black,
    urlcolor=mblue,
    pdftitle={Document by Shahar Perets},
    %	pdfpagemode=FullScreen,
}
\usepackage{yfonts}
\def\gothstart#1{\noindent\smash{\lower3ex\hbox{\llap{\Huge\gothfamily#1}}}
    \parshape=3 3.1em \dimexpr\hsize-3.4em 3.4em \dimexpr\hsize-3.4em 0pt \hsize}
\def\frakstart#1{\noindent\smash{\lower3ex\hbox{\llap{\Huge\frakfamily#1}}}
    \parshape=3 1.5em \dimexpr\hsize-1.5em 2em \dimexpr\hsize-2em 0pt \hsize}



% Hebrew initialzing
\usepackage[bidi=basic]{babel}
\PassOptionsToPackage{no-math}{fontspec}
\babelprovide[main, import, Alph=letters]{hebrew}
\babelprovide[import]{english}
\babelfont[hebrew]{rm}{David CLM}
\babelfont[hebrew]{sf}{David CLM}
%\babelfont[english]{tt}{Monaspace Xenon}
\usepackage[shortlabels]{enumitem}
\newlist{hebenum}{enumerate}{1}

% Language Shortcuts
\newcommand\en[1] {\begin{otherlanguage}{english}#1\end{otherlanguage}}
\newcommand\he[1] {\she#1\sen}
\newcommand\sen   {\begin{otherlanguage}{english}}
    \newcommand\she   {\end{otherlanguage}}
\newcommand\del   {$ \!\! $}

\newcommand\npage {\vfil {\hfil \textbf{\textit{המשך בעמוד הבא}}} \hfil \vfil \pagebreak}
\newcommand\ndoc  {\dotfill \\ \vfil {\begin{center}
            {\textbf{\textit{שחר פרץ, 2025}} \\
                \scriptsize \textit{קומפל ב־}\en{\LaTeX}\,\textit{ ונוצר באמצעות תוכנה חופשית בלבד}}
    \end{center}} \vfil	}

\newcommand{\rn}[1]{
    \textup{\uppercase\expandafter{\romannumeral#1}}
}

\makeatletter
\newcommand{\skipitems}[1]{
    \addtocounter{\@enumctr}{#1}
}
\makeatother


%! ~~~ Document ~~~

\author{שחר פרץ}
\title{\textit{מטלה מסכמת $\sim$ התקופה הפרסית}}
\date{15 ביוני 539 לפני הספירה}
\begin{document}
    \maketitle
    \begin{enumerate}[A.]
        \item
        שבי ציון היו יהודים שגלו בבבל עד 538 לפני הספירה, אז התאפשר להם לחזור לארץ ישראל לאחר ש\textit{כורש}, מלך פרס דאז, פרסם הצהרה המתירה לכל העמים תחת חסותו לחזור לפולחן אלוהיהם, ובפרט ליהודים לחזור לא''י ולבנות את בית המקדש בירושלים. שבי ציון התקשו לבנות ולבסס מחדש את ירושלים עקב התנגדות של העמים השכנים, שלא רצו שירושלים תחזור להיות מרכז יהודי חזק באיזור והפריעו לעבודות הבנייה של חומת ירושלים. תחילה, גם חלק מעמי האיזור הפירעו לעבודות הבנייה של בית המקדש, בכך שניסו לכפות על שבי ציון את התערבותם בבנייה בטענה שהם מכירים את אלוהים (על אף ששבי ציון ראו עצמם נפרדים מהם). באותה העת שבי ציון החלו להתבולל (להתחתן ולאמץ מנהגים) עם עמי המקום, בינהם ה\textit{שומרונים}, שכבר נמצאו במקום עוד לפני כן, ואלם היה דומה לאלוהי ישראל. \textit{עזרא}, צאצא של אהרון הכהן הגדול, הגיע לירושלים כ־80 שנה לאחר תחילת העבודות על בית המקדש והתנגד לתהליכי ההתבוללות הללו, ולשם השגת מטרה זו פעל בין היתר לגירוש הנשים הנוכריות מקרב שבי ציון. 
        
        כ־12 שנים לאחר שעזרא הגיע לארץ, בערך ב־445 לפני הספירה, הצטרף אליו נחמיה, יהודי המקורב למלך פרס (היה שר המשקים שלו) באותה התקופה (ארתחששתא הראשון) שמע על המצב הרע בירושלים, ועל כן ביקש וקיבל מהמלך אישור להגיע לאיזור ירושלים במשך שניים־עשר שנים ולשקם את מצב ירושלים והיהודים החיים בה. לשם כך, המלך מינה את נחמיה לתפקיד פחה – תואר של מושל על שטח – וצייד אותו בכוח צבאי. נחמיה הגיע לירושלים ובמשך 12 שנים בנה, ביסס וחיזק את העיר ירושלים ויושביה. במהלך אותן השנים עזר להנהיג את העם היהודי. עם הגעתו שיקם את חומת ירושלים בצורה שאפשרה את בניית בית המקדש בצורה תקינה, והגנת היהודים שחיו בפנים. 
        
        נחמיה הוביל שינויים שחלקם חופפים לגישתו של עזרא, היא \textit{הגישה הבדלנית} – לפיה, יש צורך לשמור על היהודים מפני השפעתם של עמים אחרים בהקשר חברתי ותרבותי, בין היתר באמצעות מניעת נישואים לזרים. 
        
        נחמיה פיתח רפורמות ל\textit{מניעת התבוללות}, \textit{איסור מסחר בשבת}, \textit{שמירה על שנת שמיטה} (אוסף מצוות לשדות החל אחת לשנה שביעית, בהן בין היתר יש להפקיר את השדות בשנה הזו), ו\textit{הגדרת תקנות לקיוה מקדש ומשרתיו}. הוא היה חבר ב''כנסת הגדולה'', המוסד הראשון מתקופת שבי ציון, שכלל בין היתר את החותמים על האמנה בקטע המקור המצורף לשאלה. הכנסת הגדולה דאגה לתקן תקנות שתאמו את צרכי העם, ממנה יצאו מנהיגים רוחניים, והיא יצגה את העם ובעיותיו באופן הולם. נוסף על זאת, הוא העביר רפורמות סוציאליות וארגן מחדש את עבודת בית המקדש. 
        
        משום שכפי שהבטיח למלך שהותו בארץ הייתה מוגבלת (ב־12 שנים), ולאור העובדה שרצה שפועלו לא יהיה לשווא וימשיך להחזיק עוד אחרי עוזבו, נחמיה כתבה את ``אמנת נחמיה'' המנסה לאגד את שבי ציון יחדיו, להוביל להצלחת ויישום הרפורמות החברתיו תשקידם נחמיה, ולהמשיך את כל זה עוד אחרי עוזבו את הארץ. על האמנה חתמו 84 ראשי אבות, כוהנים ולווים שהובילו והנהיגו את העם חלקם גם לאחר עזיבת נחמיה. נתבונן בהתחייבות העם שקיבל על עצמו בטקס חתימת האמנה, מקטע המקור הוא האמנה עצמה המופיעה בתנ''ך. קטע המקור מתאר את ההתחייבויות שחותמי האמנה והעם מקבלים על עצמם, והיא נחתמה במסגרת טקס שערך נחמיה. האמנה מתארת התחייבויות כמו עבודה לפי תורת ה', איסור חתונות עם עמי הארץ, קיום שנת שמיטה ואי־מסחר בירושלים בשבת,  וגם נהלי עבודות בית המקדש. 
        
        \begin{itemize}
            \item
             ראשית כל, נחמיה \textbf{מבסס} באמנתו את\textbf{ קבלת תורת האלוהים שמשה קיבל כדת על־פיה פועלים} (``ובאים באלו ובשבועה ללכת בתורת האלוהים, אשר ניתנה ביחד משה עבד האלוהים [...]''). הוא גם מבסס את ה' כאלוהים האחד והיחיד (``לשמור ולעשות, את כל מצוות יהוה אדונינו, ומשפטיו וחוקיו''). פעולה זו מחזקת את הזהות היהודית. 
            
            \item         
            שנית, \textbf{מניעת התבוללות}, המתבטאת בכך \textbf{שחתונות עם עמי האיזור יאסרו} (``ושאר לא ניתן בנותינו לעמי הארץ, ואת בנותיהם לא ניקח לבנינו'' – כלומר עם בנותינו העמים לא יתחתנו, ואנחנו לא ניתחתן עם בנות העמים האחרים. דהיינו, אי קיום יחסים משפחתיים בין שבי ציון לבין העמים האחרים, בהנחה שנחמיה לא היה מודע לקיום הומוסקסואליות, שעדיין לא היו נפוצים באותה התקופה). זה מקדם את הגישה הבדלנית. פעולה זו גם מחזקת את הזהות היהודית. 
            
            \item 
            נחמיה ממשיך לחזק את המסורת היהודים באמנה, בציוויו על \textbf{אי־מסחר בשבת וקיום שנת שמיטה אחת ל־7 שנים} (``ועמי הארץ המביאים את המקחות וכל שבר ביום השבת, למכור לא ניקח מהם בשבת, ונטוש את השנה השביעית, ומשע כל יד''). לשנת השמיטה \textit{השפעות חברתיות}, כי בה חל איסור לעבוד בשדה וכל אדם יכול לקחת את הפירות, מה שנותן אוכל לעניים ומערער את מעמדם וכספם של בעלי השדות. נוסף על כך השמיטה מאפשרת לשומט להתרכז בעיסוק רוחני. 
            
            \item
            נוסף על זאת, נחמיה גם \textbf{פורש לעם חוקים הנוגעים לעבודת העם את בית המקדש} (``בית אלוהינו'') הכוללים \textbf{הובלת מיסים לבית המקדש} (``והעמדנו עלינו מצות, לתת אלינו שלישית השקל בשנה, ועבודת בית אלהינו'' – לתת את שלישית השקל לעבודת ה') וכן \textbf{עלייה לרגל לבית המקדש} במועדים הרלוונטיים (``ולעולת התמיד השבתות החדשים למועדים''). הוא מגדיר חוקים המתארים \textbf{מי עובד בבית המקדש} (``הכהנים והלוים'') וכן \textbf{הקרבת קורבנות} (``לבער, על מזבח־יהוה אלהינו, ככתוב בתורה''). גם \textbf{מתן תשלום לכהנים} הוא מתאר (``ואת ראשית עריסותנו ותרומתינו ופרי כל עץ תירוש ויצהר, נביא לכהנים אל לשבות בית אלהינו, ומעשר אדמתנו, ללוים''). 
        \end{itemize}
        
        נחמיה מתאר אוסף רחב של חובות על העם ועל מנהגיו, שמחזקים את הזהות היהודית בכך שהם מבססים את תפקיד בית המקדש ונהליו, ומונעים התבוללות עם שאר עמי האיזור. באופן דומה לרצון עזרא הסופר שרצה לגרש את נשים זרות מהעם כדי למנוע התבוללות, נחמיה אוסר על נישואי תערובת. 
        
        מתוך התחייבויות אלו, ניתן ללמוד על המצב החברתי והדתי ביהודה לפני החתימה על האמנה דברים רבים – ההתחייבויות המצויות בהצהרה נמצאות שם משום שנחמיה רוצה לחייב את העם לפעול על פיהם, כי לא נלקחו כמובן מאליו והעם לא פעל לפיהם טרום הגעתו. 
       
       \begin{itemize}
           \item ניתן להסיק מקטע המקור, בין היתר, \textbf{שנישואי תערובת באותה התקופה היו נפוצים} (אחרת לא היה צריך לאסור על כך) והם השרו סכנה ממשית להמשך קיומו של שבי ציון בנפרדים מהעמים השוכנים בקרבם. לכן, נחמיה מצוה עליהם להמנע מנישואי תערובת (``ושאר לא ניתן בנותינו לעמי הארץ, ואת בנותיהם לא ניקח לבנינו''). הזהות היהודית לא הייתה חזקה דיה באותה התקופה. מהמידע שלמדתי, עזרא חשב גם כן שהזהות היהודית הוחלשה, ורצה לגרש את הנשים הנוכריות. 
           
           \item \textit{בתחום הדתי}, ע''פ קטע המקור, \textbf{אפשר ללמוד כי העם לא הקפיד לשמור על מצוות ה'}, ולכן נחמיה פתח את הצהרותו בביסוס אלוהים כאל היחיד ובחיזוק עמדתה של התורה – ``ובאים באלו ובשבועה ללכת בתורת האלוהים, אשר ניתנה ביד משה עבד האלוהים, ולשמור ולעשות את כל מצות יהוה אדונינו, ומשפטיו, וחוקיו. 
           
           \item ע''פ מידע נוסף שלמדתי, \textbf{טרם החתימה על האמנה המצב ביהודה היה רע כלפי היהודים}, שתושבי המקום התנכרו להם והפריעו להם בעבודות הבנייה את בית המקדש. 
       \end{itemize}
       
       נתניה ניסה לשמר את כוחה של האמנה בכך שהכנסת הגדולה, שמייצגת את שבי ציון ורבים המנהיגים הרוחניםי שיוצאים ממנה, חלקו בחלקם הגדול על האמנה. 
       
        לסיום, בטקס חתימת האמנה של נחמיה העם קיבל על עצמו התחייבויות רבות, שמהן ניתן להסיק מה היה מצבו טרם חתימת האמנה. האמנה חיזקה את הזהות היהודית, ביססה את הדת היהודית, והובילה להסדרת דברים רבים בבית המקדש. 
        
        \npage
        \item
                שבי ציון היו יהודים שגלו בבבל עד 538 לפני הספירה, אז התאפשר להם לחזור לארץ ישראל לאחר ש\textit{כורש}, מלך פרס דאז, פרסם הצהרה המתירה לכל העמים תחת חסותו לחזור לפולחן אלוהיהם, ובפרט ליהודים לחזור לא''י ולבנות את בית המקדש בירושלים. שבי ציון התקשו לבנות ולבסס מחדש את ירושלים עקב התנגדות של העמים השכנים, שלא רצו שירושלים תחזור להיות מרכז יהודי חזק באיזור והפריעו לעבודות הבנייה של חומת ירושלים. תחילה, גם חלק מעמי האיזור הפירעו לעבודות הבנייה של בית המקדש, בכך שניסו לכפות על שבי ציון את התערבותם בבנייה בטענה שהם מכירים את אלוהים (על אף ששבי ציון ראו עצמם נפרדים מהם). באותה העת שבי ציון החלו להתבולל (להתחתן ולאמץ מנהגים) עם עמי המקום, בינהם ה\textit{שומרונים}, שכבר נמצאו במקום עוד לפני כן, ואלם היה דומה לאלוהי ישראל. 
        
        בערך ב־445 לפני הספירה, הצטרף אליו נחמיה, יהודי המקורב למלך פרס (היה שר המשקים שלו) באותה התקופה (ארתחששתא הראשון) שמע על המצב הרע בירושלים, ועל כן ביקש וקיבל מהמלך אישור להגיע לאיזור ירושלים במשך שניים־עשר שנים ולשקם את מצב ירושלים והיהודים החיים בה. לשם כך, המלך מינה את נחמיה לתפקיד פחה – תואר של מושל על שטח – וצייד אותו בכוח צבאי. נחמיה הגיע לירושלים ובמשך 12 שנים בנה, ביסס וחיזק את העיר ירושלים ויושביה. במהלך אותן השנים עזר להנהיג את העם היהודי. עם הגעתו שיקם את חומת ירושלים בצורה שאפשרה את בניית בית המקדש בצורה תקינה, והגנת היהודים שחיו בפנים. היו יהודים מהחברה הגבוהה בין שבי ציון שחששו מכוחו הרב של נחמיה שניתן לו ע''י מלך פרס. 
        
        בידי נחמיה היו מגוון סמכויות שנתנו לו ע''י מלך פרס בעת הפיכתו לפחה. בינהן גביית מיסים, סמכות לבנות מחדש את חומת ירושלים, סמכות להרחיב את אוכלוסיית ירושלים באמצעי כפייה, וצו מלכותי שקבע שעשירית מאוכלוסיית יהודה תעבור להתגורר בתוך ירושלים. זאת ניתן לו במסגרת סמכותו כפחה של אזור יהודה. 
        
        לנחמיה היו מגוון מטרות, הן חיזוק הזהות היהודית, הפיכת ירושלים למרכז איזורי ומרכז העם היהודי והגנת עמו שבירושלים מהעמים העוינים סביבו. 
        
        \begin{itemize}
            \item סמכותו \textbf{לבנות מחדש את חומת ירושלים} אפשרה לו לבנות את חומת ירושלים בעזרת שבי ציון תוך 52 ימים, באמצעות חלוקת העבודה ל־41 משפחות, כל אחת עבדה על קטע אחר, והן הגנו אחת על השנייה באמצעות תקיעה בשופרים בעת הצורך. בניית החומה היתה הכרחית לאור עוינות עמי האיזור, אם נחמיה רצה שירושלים תהפוך למרכז העם היהודי – ואכן זו הייתה אחת מטרותיו. 
            
            \item סמכותו \textbf{להעביר עשירית מאוכלוסיית יהודה לירושלים} התבטאה בכך שבאמצעות העברת האוכלוסיה, נחמיה הגדילה את מספר התושבים בעיר לאלפים רבים. ההעברה התבצעה באמצעות חיוב עשרית מתושבי כל יישוב לעלות לירושליים. כך גם השתמש בסמכותו להרחיב את אוכלוסיית ירושלים באמצעי כפייה. הגדלת אוכלוסיית ירושלים אפשרה לו להפוך את ירושלים לעיר חזקה אף יותר. 
            
            \item כמנהיג, ייפוי הכוח שניתן לו כפחה אפשר לו \textbf{להשיג ולשמר את מעמדו בציבור}, במיוחד כאשר חלק מהנהגת שבי ציון חששה ממנו ופעלה נגדו. הצבא הלקח איתו ביחד עם כל הסמכויות שנתן לו המלך אפשרו לו לבסס את מקומו כמנהיג יהודה (הפחה של יהודה) אבל גם כמנהיג יהודי וחבר בכנסת הגדולה. 
            
            \item סמכותו \textbf{לגבות מיסים} אפשרה לו להוביל את הרפורמות הכלכליות והחברתיות שקידם, בכך שיכל לחלק את נתינת המיסים בצורה כזו שגם העשירים העובדים בבית המקדש ישלמו אותם. אותם עשירים התנגדו לחלק מפעולתיו של נחמיה וחששו מערעור מעמדם עקב מינוי סמכותו לכדי פחה, וללא הסמכויות שהעניק לו מלך פרס, לא היה יכול לגבות מהם מיסים. זאת ביחד עם גביית שליש השקל, לעבודת בית המקדש, מהלך הכרחי לקיום בית המקדש כלכלית. 
        \end{itemize}
        
        כמנהיג, הוא הצליח לקדם את מטרותיו בעזרת הסמכויות שניתנו לו, במספר דרכים. ראשית, בניית החומה שהגנה על ירושלים הפכה אותה למקום נחשק וראשי באיזור. הגדלת האוכלוסיה באמצעי כפייה, גם עזרה לו לבסס את מעמדה של ירושלים. הוא גם רצה להלחם בשחיתות בתוך העם כדי לחזק את העם היהודי, וסמכותו לגבות מיסים אפשרה לו לגבות מיסים מהעשירים (שנמנעו מכך) ובכך לייצר מצב שוויני יותר, כרצונו. בניית החומה גם הגנה על היהודים מפני עוינות עמי האיזור, ולחזק ולהפוך את ירושלים למרכז אזורי. 
        
        לסיום, סמכויותיו הרבות של נחמיה שנתנו לו ע''י מלך פרס אפשרו לו לקדם את מטרותיו כמנהיג באמצעות בניית חומה סביב ירושלים, העברת אוכלוסיה לירושלים, גביית מיסים ועוד. 
        
        
        
        
        
        
    \end{enumerate}
    
    \ndoc
\end{document}