%! ~~~ Packages Setup ~~~ 
\documentclass[]{article}


% Math packages
\usepackage[usenames]{color}
\usepackage{forest}
\usepackage{ifxetex,ifluatex,amsmath,amssymb,mathrsfs,amsthm,witharrows,mathtools}
\WithArrowsOptions{displaystyle}
\renewcommand{\qedsymbol}{$\blacksquare$} % end proofs with \blacksquare. Overwrites the defualts. 
\usepackage{cancel,bm}
\usepackage[thinc]{esdiff}


% tikz
\usepackage{tikz}
\newcommand\sqw{1}
\newcommand\squ[4][1]{\fill[#4] (#2*\sqw,#3*\sqw) rectangle +(#1*\sqw,#1*\sqw);}


% code 
\usepackage{listings}
\usepackage{xcolor}

\definecolor{codegreen}{rgb}{0,0.35,0}
\definecolor{codegray}{rgb}{0.5,0.5,0.5}
\definecolor{codenumber}{rgb}{0.1,0.3,0.5}
\definecolor{codeblue}{rgb}{0,0,0.5}
\definecolor{codered}{rgb}{0.5,0.03,0.02}
\definecolor{codegray}{rgb}{0.96,0.96,0.96}

\lstdefinestyle{pythonstylesheet}{
	language=Python,
	emphstyle=\color{deepred},
	backgroundcolor=\color{codegray},
	keywordstyle=\color{deepblue}\bfseries\itshape,
	numberstyle=\scriptsize\color{codenumber},
	basicstyle=\ttfamily\footnotesize,
	commentstyle=\color{codegreen}\itshape,
	breakatwhitespace=false, 
	breaklines=true, 
	captionpos=b, 
	keepspaces=true, 
	numbers=left, 
	numbersep=5pt, 
	showspaces=false,                
	showstringspaces=false,
	showtabs=false, 
	tabsize=4, 
	morekeywords={as,assert,nonlocal,with,yield,self,True,False,None,AssertionError,ValueError,in,else},              % Add keywords here
	keywordstyle=\color{codeblue},
	emph={object,type,isinstance,copy,deepcopy,zip,enumerate,reversed,list,set,len,dict,tuple,print,range,xrange,append,execfile,real,imag,reduce,str,repr,__init__,__add__,__mul__,__div__,__sub__,__call__,__getitem__,__setitem__,__eq__,__ne__,__nonzero__,__rmul__,__radd__,__repr__,__str__,__get__,__truediv__,__pow__,__name__,__future__,__all__,},          % Custom highlighting
	emphstyle=\color{codered},
	stringstyle=\color{codegreen},
	showstringspaces=false,
	abovecaptionskip=0pt,belowcaptionskip =0pt,
	framextopmargin=-\topsep, 
}
\newcommand\pythonstyle{\lstset{pythonstylesheet}}
\newcommand\pyl[1]     {{\lstinline!#1!}}
\lstset{style=pythonstylesheet}

\usepackage[style=1,skipbelow=\topskip,skipabove=\topskip,framemethod=TikZ]{mdframed}
\definecolor{bggray}{rgb}{0.85, 0.85, 0.85}
\mdfsetup{leftmargin=0pt,rightmargin=0pt,innerleftmargin=15pt,backgroundcolor=codegray,middlelinewidth=0.5pt,skipabove=5pt,skipbelow=0pt,middlelinecolor=black,roundcorner=5}
\BeforeBeginEnvironment{lstlisting}{\begin{mdframed}\vspace{-0.4em}}
	\AfterEndEnvironment{lstlisting}{\vspace{-0.8em}\end{mdframed}}


% Deisgn
\usepackage[labelfont=bf]{caption}
\usepackage[margin=0.6in]{geometry}
\usepackage{multicol}
\usepackage[skip=4pt, indent=0pt]{parskip}
\usepackage[normalem]{ulem}
\forestset{default}
\renewcommand\labelitemi{$\bullet$}
\usepackage{titlesec}
\usepackage{graphicx}
\graphicspath{ {./} }
\usepackage{csquotes}


% Hebrew initialzing
\usepackage[bidi=basic]{babel}
\PassOptionsToPackage{no-math}{fontspec}
\babelprovide[main, import, Alph=letters]{hebrew}
\babelprovide[import]{english}
\babelfont[hebrew]{rm}{David CLM}
\babelfont[hebrew]{sf}{David CLM}
\babelfont[english]{tt}{Monaspace Xenon}
\usepackage[shortlabels]{enumitem}
\newlist{hebenum}{enumerate}{1}

% Language Shortcuts
\newcommand\en[1] {\begin{otherlanguage}{english}#1\end{otherlanguage}}
\newcommand\sen   {\begin{otherlanguage}{english}}
	\newcommand\she   {\end{otherlanguage}}
\newcommand\del   {$ \!\! $}
\newcommand\ttt[1]{\en{\footnotesize\texttt{#1}\normalsize}}

\newcommand\npage {\vfil {\hfil \textbf{\textit{המשך בעמוד הבא}}} \hfil \vfil \pagebreak}
\newcommand\ndoc  {\dotfill \\ \vfil {\begin{center} {\textbf{\textit{שחר פרץ, 2024}} \\ \scriptsize \textit{נוצר באמצעות תוכנה חופשית בלבד}} \end{center}} \vfil	}

\newcommand{\rn}[1]{
	\textup{\uppercase\expandafter{\romannumeral#1}}
}

\makeatletter
\newcommand{\skipitems}[1]{
	\addtocounter{\@enumctr}{#1}
}
\makeatother

%! ~~~ Math shortcuts ~~~

% Letters shortcuts
\newcommand\N     {\mathbb{N}}
\newcommand\Z     {\mathbb{Z}}
\newcommand\R     {\mathbb{R}}
\newcommand\Q     {\mathbb{Q}}
\newcommand\C     {\mathbb{C}}

\newcommand\ml    {\ell}
\newcommand\mj    {\jmath}
\newcommand\mi    {\imath}

\newcommand\powerset {\mathcal{P}}
\newcommand\ps    {\mathcal{P}}
\newcommand\pc    {\mathcal{P}}
\newcommand\ac    {\mathcal{A}}
\newcommand\bc    {\mathcal{B}}
\newcommand\cc    {\mathcal{C}}
\newcommand\dc    {\mathcal{D}}
\newcommand\ec    {\mathcal{E}}
\newcommand\fc    {\mathcal{F}}
\newcommand\nc    {\mathcal{N}}
\newcommand\sca   {\mathcal{S}} % \sc is already definded
\newcommand\rca   {\mathcal{R}} % \rc is already definded

\newcommand\Si    {\Sigma}

% Logic & sets shorcuts
\newcommand\siff  {\longleftrightarrow}
\newcommand\ssiff {\leftrightarrow}
\newcommand\so    {\longrightarrow}
\newcommand\sso   {\rightarrow}

\newcommand\epsi  {\epsilon}
\newcommand\vepsi {\varepsilon}
\newcommand\vphi  {\varphi}
\newcommand\Neven {\N_{\mathrm{even}}}
\newcommand\Nodd  {\N_{\mathrm{odd }}}
\newcommand\Zeven {\Z_{\mathrm{even}}}
\newcommand\Zodd  {\Z_{\mathrm{odd }}}
\newcommand\Np    {\N_+}

% Text Shortcuts
\newcommand\open  {\big(}
\newcommand\qopen {\quad\big(}
\newcommand\close {\big)}
\newcommand\also  {\text{, }}
\newcommand\defi  {\text{ definition}}
\newcommand\defis {\text{ definitions}}
\newcommand\given {\text{given }}
\newcommand\case  {\text{if }}
\newcommand\syx   {\text{ syntax}}
\newcommand\rle   {\text{ rule}}
\newcommand\other {\text{else}}
\newcommand\set   {\ell et \text{ }}
\newcommand\ans   {\mathit{Ans.}}

% Set theory shortcuts
\newcommand\ra    {\rangle}
\newcommand\la    {\langle}

\newcommand\oto   {\leftarrow}

\newcommand\QED   {\quad\quad\mathscr{Q.E.D.}\;\;\blacksquare}
\newcommand\QEF   {\quad\quad\mathscr{Q.E.F.}}
\newcommand\eQED  {\mathscr{Q.E.D.}\;\;\blacksquare}
\newcommand\eQEF  {\mathscr{Q.E.F.}}
\newcommand\jQED  {\mathscr{Q.E.D.}}

\newcommand\dom   {\mathrm{dom}}
\newcommand\Img   {\mathrm{Im}}
\newcommand\range {\mathrm{range}}

\newcommand\trio  {\triangle}

\newcommand\rc    {\right\rceil}
\newcommand\lc    {\left\lceil}
\newcommand\rf    {\right\rfloor}
\newcommand\lf    {\left\lfloor}

\newcommand\lex   {<_{lex}}

\newcommand\az    {\aleph_0}
\newcommand\uaz   {^{\aleph_0}}
\newcommand\al    {\aleph}
\newcommand\ual   {^\aleph}
\newcommand\taz   {2^{\aleph_0}}
\newcommand\utaz  { ^{\left (2^{\aleph_0} \right )}}
\newcommand\tal   {2^{\aleph}}
\newcommand\utal  { ^{\left (2^{\aleph} \right )}}
\newcommand\ttaz  {2^{\left (2^{\aleph_0}\right )}}

\newcommand\n     {$n$־יה\ }

% Math A&B shortcuts
\newcommand\logn  {\log n}
\newcommand\logx  {\log x}
\newcommand\lnx   {\ln x}
\newcommand\cosx  {\cos x}
\newcommand\cost  {\cos \theta}
\newcommand\sinx  {\sin x}
\newcommand\sint  {\sin \theta}
\newcommand\tanx  {\tan x}
\newcommand\tant  {\tan \theta}
\newcommand\sex   {\sec x}
\newcommand\sect  {\sec^2}
\newcommand\cotx  {\cot x}
\newcommand\cscx  {\csc x}
\newcommand\sinhx {\sinh x}
\newcommand\coshx {\cosh x}
\newcommand\tanhx {\tanh x}

\newcommand\seq   {\overset{!}{=}}
\newcommand\slh   {\overset{LH}{=}}
\newcommand\sle   {\overset{!}{\le}}
\newcommand\sge   {\overset{!}{\ge}}
\newcommand\sll   {\overset{!}{<}}
\newcommand\sgg   {\overset{!}{>}}

\newcommand\h     {\hat}
\newcommand\ve    {\vec}
\newcommand\lv    {\overrightarrow}
\newcommand\ol    {\overline}

\newcommand\mlcm  {\mathrm{lcm}}

\DeclareMathOperator{\sech}   {sech}
\DeclareMathOperator{\csch}   {csch}
\DeclareMathOperator{\arcsec} {arcsec}
\DeclareMathOperator{\arccot} {arcCot}
\DeclareMathOperator{\arccsc} {arcCsc}
\DeclareMathOperator{\arccosh}{arccosh}
\DeclareMathOperator{\arcsinh}{arcsinh}
\DeclareMathOperator{\arctanh}{arctanh}
\DeclareMathOperator{\arcsech}{arcsech}
\DeclareMathOperator{\arccsch}{arccsch}
\DeclareMathOperator{\arccoth}{arccoth}
\DeclareMathOperator{\atant}  {atan2} 

\newcommand\dx    {\,\mathrm{d}x}
\newcommand\dt    {\,\mathrm{d}t}
\newcommand\dtt   {\,\mathrm{d}\theta}
\newcommand\du    {\,\mathrm{d}u}
\newcommand\dv    {\,\mathrm{d}v}
\newcommand\df    {\mathrm{d}f}
\newcommand\dfdx  {\diff{f}{x}}
\newcommand\dit   {\limhz \frac{f(x + h) - f(x)}{h}}

\newcommand\nt[1] {\frac{#1}{#1}}

\newcommand\limz  {\lim_{x \to 0}}
\newcommand\limxz {\lim_{x \to x_0}}
\newcommand\limi  {\lim_{x \to \infty}}
\newcommand\limh  {\lim_{x \to 0}}
\newcommand\limni {\lim_{x \to - \infty}}
\newcommand\limpmi{\lim_{x \to \pm \infty}}

\newcommand\ta    {\theta}
\newcommand\ap    {\alpha}

\renewcommand\inf {\infty}
\newcommand  \ninf{-\inf}

% Combinatorics shortcuts
\newcommand\sumnk     {\sum_{k = 0}^{n}}
\newcommand\sumni     {\sum_{i = 0}^{n}}
\newcommand\sumnko    {\sum_{k = 1}^{n}}
\newcommand\sumnio    {\sum_{i = 1}^{n}}
\newcommand\sumai     {\sum_{i = 1}^{n} A_i}
\newcommand\nsum[2]   {\reflectbox{\displaystyle\sum_{\reflectbox{\scriptsize$#1$}}^{\reflectbox{\scriptsize$#2$}}}}

\newcommand\bink      {\binom{n}{k}}
\newcommand\setn      {\{a_i\}^{2n}_{i = 1}}
\newcommand\setc[1]   {\{a_i\}^{#1}_{i = 1}}

\newcommand\cupain    {\bigcup_{i = 1}^{n} A_i}
\newcommand\cupai[1]  {\bigcup_{i = 1}^{#1} A_i}
\newcommand\cupiiai   {\bigcup_{i \in I} A_i}
\newcommand\capain    {\bigcap_{i = 1}^{n} A_i}
\newcommand\capai[1]  {\bigcap_{i = 1}^{#1} A_i}
\newcommand\capiiai   {\bigcap_{i \in I} A_i}

\newcommand\xot       {x_{1, 2}}
\newcommand\ano       {a_{n - 1}}
\newcommand\ant       {a_{n - 2}}

% Other shortcuts
\newcommand\tl    {\tilde}
\newcommand\op    {^{-1}}

\newcommand\sof[1]    {\left | #1 \right |}
\newcommand\cl [1]    {\left ( #1 \right )}
\newcommand\csb[1]    {\left [ #1 \right ]}

\newcommand\bs    {\blacksquare}

%! ~~~ Document ~~~

\author{שחר פרץ}
\title{\textit{עבודת תרגול לקראת המבחן בהיסטוריה}}
\begin{document}
	\maketitle
	\section*{מקור}
	
	\begin{quote}
		"היהודים נשארים עם בתוך עם, מדינה בתוך מדינה, שבט נפרד בקרב כזע זר לו. כל המהגרים מתבוללים לבסוף בתוך העם שבקרבו הם יושבים, ואילו היהודים – לא. בניגוד לטבע הגרמני, הם מציגים את אופייים השמי הבלתי מעורער, בניגוד לנצרות – את פולחן המצוות הקפדני או את שנאת הנוצרים. איננו יכולים להאשים אותם בכך. כל עוד יהודים הם, אין הם יכולים לנהוג אחרת...
		
		בעבר טענו שהאמנציפציה תדחוף את היהודים להתעסקויות אחרות. כעת הם קיבלו את האמנציפציה, אולם קרה ההפך מזה. הם מפתחים יותר מאשר בעבר את אותן ההתעסקויות בהן הרווח קל ומרובה. לאחרונה הם נדחקים גם אל שושורות השופטים, שבר שאינו משפיע לטובה על השיפוט. אין להם שמחת העבודה ואין להם כל אהדה לכבוד העמל הגרמני. יש לזקוף בעיקר על חשבונם את הסיסמה ``זול ורע". הם מצויים בכל מקום שאפל לנצל מצוקה ותאוות ספסרות...``
		
		\textit{(מתוך מצע של מפלגת הפועלים הנוצרית־סוציאליסטית שהוקמה ב־1878 ע"י העומק אדולף שטקר)}
	\end{quote}
	
	\section*{תשובות}
	\subsection*{סעיף א'}
		\textbf{שאלה: }הסבירו מהי האמציפציה? הסבר, על־פי הקטע ועל־פי מה שלמדת, מהו הקשר בין האמנציפציה לבין צמיחת האנטישמיות המודרנית? 
		
		\textbf{תשובה: }
		
		במהלך המאה ה־19, המערב אירופה, ליהודים ניתן היתר להשתלב בחברה במדינות בהן הם חיו. היתר זה, הופיע כחלק מרשימת חוקים הקרויים האמנסיפציה. שילובם בחברה הוביל ליצירת האנטישמיות המודרנית, אנטישמיות המופנית כנגד \textit{גזע} היהודים וצאציהם (כלומר, להם וילדיהם, ללא תלות בדת). 
		
		ה\textbf{אמנציפציה} כללה היתרים רבים, שאיפשרו ליהודים להשתלב במשרות בכירות – אם כשופטים, רופאים, מנהלי חשבונות, ותפקידים רבים אחרים. האמנציפציה ניתנה ע"י מדינות מערב אירופה לאוכלוסיה היהודית בכללותה, ואפשרה להם לנהל אוטונומיה עצמאית כיהודים לצד חייהם החברתיים. 
		
		בין היתר, האמנסיפציה גררה את תיאורית \textbf{האנטישמיות המודרנית}. לפיה, גזע היהודים, על כל צאציהם, ובכל דורותיהם, הם מפגע מזיק בחברה. התיאוריה הגזעית הונעה ברגשות מרמור שחשה האוכלוסיה כלפי הצלחת היהודים במשרותיהם, שהייתה לא פורפורציונית ביחס לשיעורם באוכלוסיה. כתוצאה מהאנטישמיות המודרנית קמו מפלגות ותנועות פוליטיות עם מצע שנועד לצמצם את האמנציפציה והזכויות שניתנו ליהודים. 
		
		נתבונן במצעהּ של מפלגת הפועלים הנוצרית־סוציאליסטית ("קטע המקור``), ועליו נבסס את הציטוטים שיובאו בתשובה זו. בטקע המקור, צויין כי \textit{``בבעבר ענו שהאמנציפציה תדחוף את היהודים להתעסקויות אחרות. כעת הם קיבלו את האמנציפציה, אולם קרה ההיפך מזה"}. נבחין, כי הכותב מניח הרצון הקורא הוא שהיהודים ייתעסקו בתעסקויות אחרות, והוא מותח ביקורת על האמנציפציה כמי שאיפשרה להיהודים להמשיך בעיסוקם. ומה הוא עיסוקם? \textit{"[...] ההתעסקויות בהן הרווח קל ומרובה. לאחרונה הם גם נדחקים אל שורות השופטים, שבר שאינו משפיע לטובה על השיפוט``}. כלומר, לפי הכותב, היהודים עוסקים במלאכות שאין בהן ממש (כלומר, הם מפגע המנצל את החברה), ובפרט פוגעים בשיפוט. ולא רק בשיפוט – \textit{"הם מצויים בכל מקום שאפשר לנצל תאוות ספסרות``}. 
		
		נבחין בשני דברים. הראשון – שהכותב מתנה את הכל בקיומה של האמנציפציה, ובעצם הצגת הנובע מאותה אמנציפציה כדבר שלילי, קורא לביטולה. השני, הוא שהוא משתמש בנימוקים המאפיינים את האנטישמיות המודרנית. דוגמה לנימוקים כאלו הוא הטענה שהיהודים הגיעו לכל מקום, (כלומר, הם בעלי משרות כוח רבות משיעורם בחברה) ומשתמע מהטקסט ש\textbf{הם שולטים על הציבור הגרמני} (בשילובם במערכת המשפט). יותר מכך, שההשפעות היהודים על החברה רעות, \textbf{מעצם היותם יהודים} (\textit{"כל עוד יהודים הם, אין הם יכולים לנהוג אחרת.``}).
		
		לפי הנלמד, עצם קיומה של האמנציפציה איפשרה להיהודים להגיע לאותן משרות, שעוררו את האנטישמיות המודרנית. בכך, למעשה, האמנציפציה היוותה בסיס להתהוות האנטישמיות המודרנית. 
		
		לסיום, הראינו שקטע המקור משתמש בנימוקים הפופולאריים בקרב התומכים באנטישמיות המודרנית, על מנת לקשור את האמנציפציה בפגיעה בחוסן הלאום הגרמני. הוא עושה זאת באמצעות חיזוק התחושה שהיהודים שולטים בחברה, בבנקאות ובמשפט, ושהאמנציפציה איפשרה להם לעשות זאת. \hfill \bs
		
	\subsection*{סעיף ב'}
		\textbf{שאלה: }לפניכם גורמים שהשפיעו על צמיחת התנועות הלאומיות באירופה במאה ה־19: 
		\begin{itemize}
			\item מהפכה התעשייתית
			\item מהפכות פוליטיות
			\item תהליך החילון
		\end{itemize}
		בחרו שניים מן הגורמים והסבירו כיצד השפיעו על צמיחת התנועות הלאומיות באירופה. בחרו קטע מספר הלימוד והסבר כיצד מבטא מקור זה את אחד מהגורמים. 
		
		\textbf{תשובה: }
		
		במהלך המאה ה־19 התרחש תהליך צמיחת התנועות הלאומיות. בגינו, עמים התקבצו יחדיו ופעלו להקמת מדינה בעבור אותו העם. ישנם מספר גורמים לצמיחת התנועות הלאומיות. בהם, מהפכות פוליטיות שהתרחשו באותה התקופה, ומהפכות תעשייתיות גם כן. התנועות הלאומיות שגשגו באירופה, גם מחוצה לה – במזרח התיכון, ברוסיה, אמריקה, ובמרבית העולם המודרני, על להתפשטותן בכל העולם. 
		
		\textbf{הלאומיות} היא רעיון לפיו עם בעל מאפיינים משותפים (כמו היסטוריה, מנהגים, דת ומאפיינים נוספים) יתגבש וינסה למצוא מדינה משלו. התנועה לאומית היא תנועה המבקשת להפוך את רעיון הלאומיות למציאות בעבור עם נתון, ובאותה התקופה רבים השתייכו לתנועה לאומית לאף ראו בה כמוקד זהותם. 
		
		\textbf{המהפכה התעשייתית} הייתה מהפכה שהגדילה באופן משמעותי את תהליך הייצור. תוך הקמת מפעלים, חרישת שדות בצורה מסודרת, ואיפשרה את קיום שתי מהפכות נוספות – \textbf{מהפכת הדפוס} (היכולת להדפיס בסדרי גודל עיתונים, ספרים וכו') ו\textbf{המהפכה האגררית} (מעבר האדם מהכפר לעיר ובידול ממוקד הזהות שהיה לו בעבר). כל אחד מבין אלו עודד את התנועות הלאומיות – בכך שהאנשים עבורו מלייצר בשדה, ללעשות עבודה פשוטה במפעל, התנתקו מהמשפחה ומהכפר שלהם, שבעבר היוו את מוקד הזהות שלהם, וחיפשו מוקד זהות חדש – צורך הלאומיות הצליחה למלא. מהפכת הדפוס גם איפשרה הפצת רעיונות כאש בשדה קוצים. אם, בעבר, חדש היו מגיעות מאוזן לאוזן, מפעם לפעם, עתה עיתונים היו נגישים ונקראים ע"י הציבור. העיתונים איפשרו להוות עוד מקוד להפצת הרעיונות של התנועות הלאומיות, וגיוס אנשים לתנועות הלאומיות (הוא שלב הכרחי בקיומה של תנועה לאומית). 
		
		מ\textbf{המהפכות הפוליטיות} שהתרחשו באותה התקופה, נפרט את שתיים בולטות ביותר. הראשונה – \textbf{המהפכה האמריקאית}. במסגרתה, המתיישבים האמריקאים (שהתחילו כמושבות בריטיות לאסירים, והפכו למושבות עצמאיות של ממש) נלחמו בבריטים ששלטותו בשטחם והצליחו לקבל עצמאות מוחלטת. זו הייתה הוכחה ניצחת שתנועה לאומית (במקרה זה, האמריקאית) יכולה להצליח במאבקה ולהשיג שטחים. השנייה – \textbf{כיבושי נפוליאון}. נופליאון הוא שליט שעלה לשלטון בשם הדמוקרטיה, ונותר שם (חולה כוח) בטענה למצב חירום, וכבש חלקים נרחבים מאירופה. כיבושי נפליאון הראו מה תנועה לאומית יכולה להשיג, אך למרות המטרות הנעלות אותן ביקש לקיים – הוא למעשה דיכא את האוטונומיה במדינות אותן הוא כבש (דבר שגרם לאף יותר מחאות ולהתחזקות התנועות הלאומיות). 
		
		נתבונן בקטע המקור הוא זכרונותיו של הנסיך דה־סגיור, שנמצא בספר "הלאומיות בישראל ובעמים``, הוצאת כנרת, עמוד 26. הנסיך היה אציל צרפתי, ומזכרונותיו נראה כיצד המהפכה המרוחקת שהתרחשה באמריקה, השפיעה על הלאומיות בצרפת. הוא אומר ש\textit{"גבורתם של הרפובליקאים החדשים באמריקה רכשה לה הוקרה ברחבי אירופה וזכתה לאהדתם של כל שוחרי הצדק וההומאניות``} – ניזכר שהצדק וההומאניות הם רעיונות שהבאו כחלק מרעיונות הפילוסופים באותה המאה, אותה הקבוצה שפיתחה את הבסיס ללאומיות. כלומר, גבורתם של אותם רפובליקאים באמריקה (האנשים שהובילו את המהפכה) הייתה אהודה ע"י מייסדי הלאומיות. \textit{"האנשים הצעירים... הללו היו מלאי התלהבות לרגל העניין שעוררה בהם התקוממות האמריקאים.``} – אותם הישגים של האמריקאים, והתקוממות שלהם, עוררו עניין והשראה בצעירים שהשתתפו בתנועות הלאומיות. נסיק שהמהפכה האמריקאית עודדה גם שלב נוסף בהתפתחות התנועות הלאומיות, הוא שלב המאבק וההתקוממות. 
		
		לסיום, המהפכה התעשייתית (שטומנת בחובה מהמפכות רבות נוספות) והמהפכות הפוליטיות במאה ה־19 היו בעלי השפעה רבה על התפתחות התנועות הלאומיות. ראינו שמהפכה האמריקאית (אחת מהמהפכות הפוליטיות) השפיעה על הלאומיות באירופה ועודדה אותה גם היא. \hfill \bs 
		
		\ndoc
		
		
		
		
\end{document}
