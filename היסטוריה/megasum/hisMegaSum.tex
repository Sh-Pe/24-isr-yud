%  HISTORY 2024-2025 SUMMARY ~ by Shahar Perets ~ original .TeX
%  file ~ compile with LuaLaTeX
% 
%  Copyright (C) 2025  Shahar Perets
%  
%  This program is free software: you can redistribute it and/or modify
%  it under the terms of the GNU General Public License as published by
%  the Free Software Foundation, either version 3 of the License, or   
%  (at your option) any later version.                                 
%  
%  This program is distributed in the hope that it will be useful,     
%  but WITHOUT ANY WARRANTY; without even the implied warranty of      
%  MERCHANTABILITY or FITNESS FOR A PARTICULAR PURPOSE.  See the       
%  GNU General Public License for more details.                        
%  
%  You should have received a copy of the GNU General Public License     
%  along with this program.  If not, see https://www.gnu.org/licenses/.
%  

%! ~~~ Packages Setup ~~~ 
\documentclass[a4paper]{book}

% Math packages
	\usepackage[usenames]{color}
	\usepackage{ifxetex,ifluatex,amsmath,amssymb,amsthm,witharrows,mathtools}
	\WithArrowsOptions{displaystyle}
	\renewcommand{\qedsymbol}{$\blacksquare$} % end proofs with \blacksquare. Overwrites the defualts. 
	\usepackage{cancel,bm}
	\usepackage{forest}
	\forestset{default}



% Design
	% Layout
	\usepackage[margin=0.85in]{geometry}
	\usepackage{multicol}
	\usepackage[skip=5pt, indent=0pt]{parskip}
	\usepackage[normalem]{ulem}
	% Titles
	\usepackage{titlesec}
	\titleformat{\section}       [block]
		{\fontsize{15}{15}}
		{\sen \huge \bfseries\thesection \she}
		{0em}
		{\Large \,\,$\sim$\,\,}
		[\vspace{-10pt}\normalsize\dotfill]
	\titleformat{\subsection}    [block]
		{\large\itshape}
		{\normalfont\Large\bfseries\en{{\thesubsection}} \,\,$\sim$\,\,}
		{0em}
		{}
	\titleformat{\subsubsection} [block]
		{\normalsize\bfseries}
		{\normalfont\large\bfseries\en{{(\thesubsubsection)}}}
		{0.5em}
		{}
	\setcounter{secnumdepth}{3} % enable subsubsection
	\usepackage{etoolbox}
	\makeatletter
	\patchcmd{\chapter}{\if@openright\cleardoublepage\else\clearpage\fi}{}{}{}
	\makeatother
	\newcommand\npchapter[1] {\npage\chapter{#1}\thispagestyle{empty}\newpage}
	% Defaults
	\renewcommand\le{\leqslant}
	\renewcommand\ge{\geqslant}
	\renewcommand\labelitemi{$\bullet$}
	\newcommand\name{רשימות היסטוריה 2024-2026}
	% Hyprref
	\usepackage{hyperref}
	\hypersetup{
		colorlinks,
		citecolor=black,
		filecolor=black,
		linkcolor=black,
		urlcolor=blue
	}
	% Headers
	\usepackage{calc}
	\usepackage{fancyhdr}
	\pagestyle{fancy}
	\fancyhead[C]{\textit{\textbf{\name}}}
	\fancyhead[L,R]{}
	\fancyfoot[R]{\textit{שחר פרץ, 2025}\, \en{{\Large({\thepage})}}}
	\fancyfoot[L]{\textit{\rightmark}}
	\fancyfoot[C]{}
	\renewcommand{\headrule}{\vspace{-7pt}\hfil\rule{200pt}{1pt}}

% Hebrew initialzing
	\usepackage[bidi=basic]{babel}
	\PassOptionsToPackage{no-math}{fontspec}
	\babelprovide[main, import, Alph=letters]{hebrew}
	\babelprovide[import]{english}
	\babelfont[hebrew]{rm}{David CLM}
	\babelfont[hebrew]{sf}{David CLM}
	\usepackage[shortlabels]{enumitem}
	\newlist{hebenum}{enumerate}{1}

% Language Shortcuts
	\newcommand\en[1] {\begin{otherlanguage}{english}#1\end{otherlanguage}}
	\newcommand\he[1] {\begin{otherlanguage}{hebrew}#1\end{otherlanguage}}
	\newcommand\sen   {\begin{otherlanguage}{english}}
		\newcommand\she   {\end{otherlanguage}}
	\newcommand\del   {$ \!\! $}
	
	\newcommand\npage {\vfil {\hfil \textbf{\textit{המשך בעמוד הבא}}} \hfil \vfil \pagebreak}
	\newcommand\ndoc  {\dotfill \\ \vfil {\begin{center}
				{\textbf{\textit{שחר פרץ, 2025}} \\
					\scriptsize \textit{קומפל ב־}\en{\LaTeX}\,\textit{ ונוצר באמצעות תוכנה חופשית בלבד}}
		\end{center}} \vfil	}
	
	\newcommand       {\rn}[1]{
		\textup{\uppercase\expandafter{\romannumeral#1}}
	}
	
	\newcommand\middleText[1] {\vfil {\hfil {#1}} \hfil \vfil \newpage}
	\newcommand\envendproof{\vspace{-17pt}}
	\newcommand\mathenvendproof{\par\vspace{-24pt}}
	
	\makeatletter
	\newcommand{\skipitems}[1]{
		\addtocounter{\@enumctr}{#1}
	}
	\makeatother

\author{שחר פרץ}
\title{רשימות היסטוריה}
\date{2024-2025}
\begin{document}
	
	\renewcommand{\footrule}{\rule{\linewidth-19pt}{0.25pt}\vspace{-5pt}}
	\thispagestyle{empty}
	\,\! % To stablize the next lines on smth
	
	{\vspace{0.5\textheight-2em} 
		{
			\begin{center} 
				{
					\textbf{{\name}
					} \\ 
					\textit{אורגן ותומלל ע''י שחר פרץ $\sim$ שיעורי שרית רג'יניאנו או משהו כזה}}
			\end{center}
		}
	}
	
	\newpage
	
	\section{פתיחה}
	\subsection{אזהרות}
	
	הסיכום להלן לא עבר הגהה, והוא פחות או יותר copy paste עם כותרות של כל הסיכומים שלי מאז שהתחלנו ללמוד היסטוריה בתיכון. יש טונות של שגיעות חטיב, תעןיות הקלדב, וכו'. הוא גם מכיל רק את החומר של השיעורים עד עכשיו, דהיינו לא את כל החומר במיקוד לבגרות. בסיכום קיימות תתי־תתי קטגוריות \textit{שלא} מופיעות בתוכן העניינים בעמוד הבא. 
	
	\subsection[רישיון]{License $\sim$ רישיון}
	\en{The following summary is provided under the GNU General Public License version 3 (GPLv3). It can be distributed and/or modified under the terms of the license, or any later version of it. Additional information can be found \href{https://www.gnu.org/licenses}{here}. }
	
	הסיכום להלן מסופק תחת רשיון התוכנה החופשית של גנו גרסה 3 (GNU General Public License (GPL) version 3). ניתן להעתיקו ו/או להפיצו תחת GPLv3 או גרסה מאוחרת יותר. מידע נוסף אפשר למצוא \href{https://www.gnu.org/licenses}{כאן}. 
	
	\newpage
	
	%	\setcounter{tocdepth}{3} % show subsubsections
	\thispagestyle{empty}
	\tableofcontents
	\thispagestyle{empty}
	
	\newpage
	
	\npchapter{לאומיות וציונות}
	
	\section{מאפייני הלאומיות הכללית}
	\subsection{שפה}
	\textbf{שאלה: מה מיחד בני לאום}. תשובה: שפה יחודית (לדוגמה קיצורים של צופים). תשובה אחרת: שקדי מרק40, פרסומת בה מישהו מנסה להסביר מה מיוחד בשקדי מרק. מסקנה: שפה מייצגת תרבות. סיפור: ביום הורים עם דוברים אתיופית, המורה שלנו ביקשה מהמתרדם להסביר לילד מה זה בעיית קשב, והוא אמר שאין מילה כזו באמהרית. סיבה: תרבותית, באתיופה, זה לא רלוונטי. 
	
	סה"כ שפה כלי להעברת האידיאולוגיה של קבוצת הלאום, היא מאחדת ומייחדת, ובני אותו הלאום יכולים לתקשר ביניהם בשפה מוקרת, ובכך מרגישים קירבה אחד לשני. חלק מהתנועות הלאומיות ביטלו את הניבים והשלונות המקומיים, וקבעו שפה לאומית משותפת. 
	
	משימה: לכתוב במחברת בנקודות מה יש בתוך ההמנון של הכפר. 
	\begin{itemize}
		\item פריחה, ציפורים, שקיעה, דשא – דברים המסמנים טבבע, דבר קונוטציה חיובית. 
		\item כנל לגבי אהבה. 
		\item "מצאתי את ביתי" – קשר לבית, לבסיס. 
		\item עוד ארחיד ללכת, קשר לרוח – מסמל אידיאולוגיה. 
		\item עוד ישנם מקומות – ייחוד של הכפר מדברים שתושביו אמורים לשייך אליו כדברים טובים. 
	\end{itemize}
	
	\textbf{שאלה: מה יש בהמנון?}
	
	\begin{enumerate}
		\item מדבר על המקום
		\item מציג את הערכים
	\end{enumerate}
	
	המנורה בסמל – משער טיטוס, התמונה האחרונה שידעו איפה המנורה. מאותו התבלית יצרו את השמל. לצידהי עלי זית. 
	
	\subsection{זיקה למולדת}
	מולדת = אינו בהכרח המקום עליו נולדת, אלא מוגדר להיות הטריטוקיה אליה יש אתה מרגיש רגש. המולדת מבארת את ההיסטוריה של הלאום. לדוגמה, יתכן אדם שעלה מרוסיה, ומגדיר את מולדתו כישראל, אם תחושת הרגש שלו – היא כאן. 
	
	סה"כ: המולדת היא השטח הטאיטואיאלי שבו נוצרה הזהו תהלאומית של העם, בו נולדה האומה אליה שייך האדם. 
	
	המולדת מאחדת את האומה. 
	
	\subsection{תרבות}
	הסיפורים, השירים, האומנות והמאגלים המאפיינים עם מסויימים, ועוברים במסורת. הפולרלור מייצג את מורשת העם ואת זהות הלאום. 
	
	דוג': תפוח בדבש, כל היהודים אוכלים. מיתוס הוא גם חלק מהתרבות – ומוגדר להיות סיפור המבוססס על אמונתו של אותו הלאום (זו הדרך להימנע מלהגיד "לא אמיתי"). 
	
	\subsection{הערות נוספות}
	
	\textbf{מטרה משותפת: }לחברי הלאום יש שאיפה להקים מדינה ריבונית עצמאית ששייכת ללאום. לדוגמה, באמצעות חינוך, מאבק, ומלחמות. 
	
	כל האמורים לעיל, יחדיו עם סמלים והיסטוריה, מאחרים של בני הלאום. 
	
	\subsection{שאלות}
	בעמודים 13-14 שתי מפות, על מה מבוססים ההבדלים בין המפות? 
	
	\textbf{תשובה שלי: }המפה המדינית של איופה בשנת 1815, נוצר ע"מ לייצג את המצב הפוליטי, ולחזק את הכוח של השליטים ביחס נפוליאון וללאומיות שרק נוצרה. בה, הוצגו בעיקר אימפריות עם שאיפות שתלטות על שטחים, ואין יחס בינה לבין העמים. דוגמה לכך היא הקיסרות העוצמנית, שהתפרסה על פני שטחים של מדינות ולאומים רבים, רובם, בלי שום קשר לטורקים ששלטו על השטח. לעומתה, מפת העולם ב־1920, שנהגתה וממושה כחלק מחוזה ורסאי, מייצגת את הלאומים השייכחים למדינות – פולנים, בפולין, אוסטרים, באוטריה, גרמנים ברגמניה, וכו' (לפחחות בערך – ישנם יוצאי דופן), ובמיוחד המדינות החדשות שנוצרו על השטח הכבוש לשעבר, שיצר הפרדה בין מערב אירופה לרוסיה. בפרט, האימפריה העותמנית פורקה, ועליה נוצרו מדינות חדשות – גם בשטחי הבלקן – הצורה על העמים שבהם (מתמטית). 
	
	\subsubsection{ניתוח מפה}
	
	\begin{enumerate}
		\item התבוננות – כותרת, נושא, מרחב, זמן, מקום, סטטית בדכ, עיתים דינמית (לא על דף). 
		\item ניתוח והסקת מסקנות – מידע חדש, אירועים אותה היא משקפת, תרומה להבנת אירוע ותהליכים היסטוריים. 
	\end{enumerate}
	
	נחזור למפות שלנו. 
	
	נתונה לנו הזכות להגדרה עצמית של עמים. נבחין, שבמפה בעמוד 14, אין קווים ישרים – כאשר יש קווים ישרים, לרוב המנצחות קובעות אותן, ולרוב חותך לאומים – ובכך יוצר מלחמות עתדיות. המחסור בקווים ישרים יכול להעיד על כך שנוצרה בראי האדמה, ובמקרה כזה, לרוב לפי להאומים שהיו שם. 
	
	סיכום ההבדלים בי המפות – התפרקות האימפריות, הקמתן של מדינות לאום חדשות, ואיחור נסיכויות וממלכות איטליה וגרמניה למגינות לאום. 
	
	\section{איחוד גרמניה}
	לפני כיבושי נפוליאון, איזור גרמניה היה מחולק לכ־300 ריבונים שונים, ולאחריהם, ההסדרים המדיניים ביניהן נשברו ואלו הפכו ל־38 מדינות הקשורות בברית הגרמנית – ברית חלשה ללא אופי לאומי. בפרט, פרוסיה־סקסוניה כיסתה את מרבית צפון גרמניה, ובדרום – אוסטרו־הונגרייה. 
	
	באותה התקופה, גרמניה התפתחה כלכלית באופן מהיל ביותר ונוצרו מסילות ברזל שחיברו בין המדינות בברית הגרמנית. יתרה מכך – הם יכלו לסחור ביניהן ללא מכסים. בכך המעמד הבורגני התחזק והוקל על הפצת רעיונות ברחבי גרמניה. 
	
	יש לציין כי לגרמניה הייתה היסטוריה ענפה, שנחקרה עוד לפני כיבושי נפוליאון. 
	
	\subsection{מטרות המאבק הלאומי הגרמני}
	מטרת המאבק הלאומי הייתה איחוד גרמניה, והמדינות בברית הגרמני, לכדי מדינה לאומית אחת. 
	\subsection{שלבי המאבק}
	\subsubsection{ההיערכות}
	לאחר כיבושי נפוליאון, המשכילים ראו את רעיונות הלאומיות – ולאחר שנים של חקר התרבות הלאומית – התחילו להתעורר רגשות לאומיים בהם המבוססים על אילו. אותם משכילים התאמצו לשמר ולתעד את התרבות הגרמנית, ויצרו בסיס לבניית הזהות והתודעה הלאומית  בגרמניה. 
	
	\subsubsection{ההרחבה}
	המשכילים יכלו להפיץ את רעיונותיהם בקלות בגרמניה. בין עם בסיפורים ושירים, בין עם בניצול הקשרים הכלכליים בין המדינות, ובין עם בצורות אחרות. התנועה הלאומית הגרמנית שהחלה להתבסס לאחר כיבושי נפוליאון, צברה תאוצה וכ־30 שנה לאחר מכן הייתה ידועה לכל ברחבי גרמניה (בשלב הזה, יחידה כלכלית אחת). התנועה התפרשה לכמעט כל גרמניה, וגם לאיזורים הנידחים יותר, כתוצאה מהקשרים  הכלכליים.
	\subsubsection{המאבק}
	\textit{עממי}: ב־1948-49, החל והסתיים אביב העמים הגרמני. ניכר לכל כי התנועה הלאומית צרובה בתרבות הגרמנית, וכוחה הובהר לכל – אך השליטים המקומיים דיכאו כל נסיון למרד. בפרנקפורט הוקם בפרלמנט הגרמני, שם הנציגים דיברו על המאבק ואף נקבעה חוקה. יש לציין כי כוחו של הפרלמנט היה קטן, עקב מחוסר בכוח צבאי, ומשום שרבים מן הדיבורים הללו – לא התרגמו למעשים על פני השטח. 
	
	\textit{מדיני, דיפלומטי, צבאי: }הפרלמנט הבין שאין ערך במאבקים לאומיים מבלי כוח צבאי, וניסה לגייס את פרוסיה למאבק. מלך פרוסיה סירב לפרלמנט הגרמני, בטיעון שכתרו "אינו כתר שיצרה אסיפה שנוצרה ע"י מהפכה, אלא כתר הנושא את חותם האלוהים", וטען שההצעה שהוצעה לו היא ל"כתר שדבק בו ריח הנבלה של מהפכת 1848". כלומר, הוא אינו אהב את התנועה הלאומית, ראה בה כמשהו מכפיש, לעומת האלוהות שהוא מייצג. הקיסר האוסטרי התנגד באופן תקיף לכל רעיון הלאומיות, ככזה המהווה לו סכנה קיומית. 
	
	אך ב־1964 אוטו פון ביסמק מונה לראש ממשלתו של ויליהם הראשון, מלך פרוסיה. בראי אביב העמים, ביסמק הבין את יכולת המאבק הלאומי, ובסדרה של מהלכים גרם לדנמרק, ואז לאוסטריה וצרפת, כל אחת מהן לפתוח במלחמה נגדו (שהוא עורר) – ותוך ניצול הרצון העז של העם לתנועה לאומית, והכנות מפוארות בהרבה מאלו של הצבאות היריבים. אומנם הוא לא כבש חלקים מאוסטריה ובנצחונו עליהם רק ווידא שלא ייתנגדו לאיחוד צפון גרמניה – אך ההישג בוצע, וצפון גרמניה אוחדה. יש לציין שגם ביסמרק לא העריך את הפרלמנט, וקרא ל"נאומים וההחלטות" שלהם, "המשגה של 1948-9" בטענה ש"בדם ובברזל" "ייפתרו הבעיות הגדולות של ימנו". 
	\subsubsection{לאחר הקמת המדינה הריבונית}
	ביסמק רצה לעודד את התנועה הלאומית הגרמנית, כמאחד גרמניה. בפרט, רצה לערער את מעמדה של הכנסייה הקטולית שייצגה נאמנות לאפיפיור, ולא לגרמניה. בכך פתח ב"מלחמת התרבות" – מספר חוקים שעקרו מהכנסיה את היכולת לשלוט במערת החינוך, פיקוח על הגשת כמרים, מניעתם לדבר על נושאים פוליטיים ועוד. נוסף על זאת, קבע ביטוח לאומי, קצבת זקנה, ביטוח תאונות ועוד – מעשים שגררו גיוס המעמדים הנמוכים, שבעבר הזדהו עם הכנסייה הקטולית. 
	
	\subsection{גורמים מסייעים ומעכבים}
	\subsubsection{מסייעים}
	\begin{itemize}
		\item ברית המכס – הורדת המכס בין המדינות בברית הגרמנית, ויצירת בסיס כלכלי לאיחוד. 
		\item תיעוש וצמיחה כלכלית מהירה, שיצרו מעמד בורגני שרצה ויכל לקבל את הרעיונות הלאומיים. 
		\item עבר היסטורי שכבר היה קיים, וכבר נחקר ועד טרם הקמת התנועה. 
		\item שפה משותפת ותרבות מפותחת, המשותפת תכולם. 
		\item כוחה הצבאי של פרוסיה ונכונות ביסמרק לנצל את התנועה הלאומית ולכבוש שטחים נוספים. 
	\end{itemize}
	\subsubsection{מעכבים}
	\begin{itemize}
		\item התנגדות אוסטריה למאבק. 
		\item רצון המעצמות האירופאיות, ובפרט צרפת, לשמור את גרמניה מפולגת, בגלל חשש מכוח גדול במרכז אירופה. 
		\item שליטים מקומיים שרצו לשמר את כוחם ומעמדם, ולכן דיכאו את התנועה הלאומית בשטחיהם. 
		\item למרות התרבות במשותפת, היה פיצול דתי בגרמניה. הקטולים העדיפו את היקסר האוסטרי, וחששו משליט פרוטסטנטי. 
	\end{itemize}
	
	\subsection{מאפייני המנהיגים}
	בספר לא מפורטים מנהיגים ספציפיים של התנועה הלאומית עצמה. תנועה הלאומית היה פרלמנט, שקיבל את ההחלטות הקשורות אליה, ובו אכן היו אנשים מהמעמד מהמעמד הבורגני ובעל כושר ארגון ודיבור רטורי, אך קשה להצביע על מנהיג אחד. התנועה הלאומית רצתה לשים לעצמה כמנהיג את מלך פרוסיה או את הקיסר האוסטרי, כתלות במקום בגרמניה ממנו הגיע ההצעה (פרוטסנטי או קטולי), אך כך או אחרת מנהיגים אילו סירבו להוות חלק מהתנועה, והם אינם עונים על התנאים של מנהיג תנועה לאומית.
	
	אם אפשר לקרוא למישהו מנהיגה התנועה הלאומית, הוא אוטו פון ביסמרק – ראש ממשלתו של המלך ויליהאם הראשון, שניצל את התנועה הלאומית בשביל לאחד את גרמניה ולהפוך את פרוסיה למעצמה. לא היה לו עניין אידיאולוגי בתנועה אלא עניין צבאי בלבד, אך ללא ספק הוא האדם שאפשר את המאבק המלחמתי, והיה בעל עניין בשימור התנועה. 
	
	\section{הגורמים להתעוררות הלאומיות באירופה}
	\subsection{מבוא}
	הפרק להלן מחולק לשניים: לאומיות כללית וציונות. פירוט בתכנית הלימודים. שימו לב להבדלים בין \textbf{מאפייני התנועות הלאומיות} (מנהיג, גורמים מסייעים/מעכבים, שלבי ההתפתחות, מטרות ועוד) ו\textbf{מאפייני הלאומיות} (מרכביי הלאומיות, מה מיחד את בני הלאום, וכו'). הלאומיות היא הרעיון, והתנועה היא תנועה פוליטית. דפוסי הגשמה של תנועה לאומית נותנים דוגמאות, כמו במדינה מדגמת בשיעור הקודם. עתה נלמד את הגורמים. לאחר מכן, נלמד על התנועה הציונית (גורמים, פיעלות של הרצל, פיעולת בארץ ב־81-14, מלחהע"ר, שואה, בלפור)
	
	\sen
	\begin{forest}
		[\text{\he{לאומיות}}
		[\text{\he{לאומיות בישוב היהודי}}
		[\text{\he{גורמים}}]
		[\text{\he{הרצל}}]
		[\text{\he{פיעלות בא"י בשנים 1881-1914}}]
		[\text{\he{הישוג היהודי בארץ}}]
		[\text{\he{הצהרת בלפור}}]
		]
		[\text{\he{לאומיות באופן כללי}}
		[\text{\he{מבוא}}]
		[\text{\he{גורמים}}]
		[\text{\he{מאפיינים, דפוסי פעילות}}]
		[\text{\he{דפוס ההגשמה}}]
		]
		]
	\end{forest}
	\she
	
	נרחיב על שלושת סוגי הגורמים: 
	\begin{enumerate}
		\item רעיוניים
		\item פוליטיים
		\item חברתיים 
	\end{enumerate}
	\subsubsection{הגורמים בקצרה}
	מספר גורמים: 
	\begin{enumerate}
		\item רעיונות ההשכלה והנאורות – רעיוני
		\item תהליך החילון – התהליך חברתי, החילון רעיוני
		\item תנועה רומנטית – רעיונית
		\item המהפכה האמריקאית – פוליטי
		\item נפוליטון – פוליטי
		\item המהפכה התעשייתית – חברתי / כלכלי
	\end{enumerate}
	נפרט בקרוב על כל אחד מהם. 
	
	\textbf{חייבים להסביר איך זה השפיע}. אם שואלים לגבי הגורמים, אכן יש צורך להסביר מה הגורמים, אבל יש גם צורך להגיד מה ההשפעה על התנועות. 
	\textbf{ליהשאר מפוקסים על השאלה}
	
	
	כאשר מדברים על הגורמים, מנחים שהלאומיות תופעה מודרנית השמתמשת בכלים מודרנים. כמו שנכתב כבר, יש פילוסופים הטוענים אחרת. לדוגמה, אנתוני סמית. בנדירט אנדרסון טוען כי אומה היא המצאה מדומיינת, מושג שהתפתח רק בתקופה המודרנית. במובן השטחי, הספר קהילות מדומיינות של אנדרסון מדבר על יציאת קהילה שלא באמת קיימת, ענקית, ומה שמחבר אותה הוא הדבר שנורצ – לא \textit{הקהילה הייתה} אלא \textit{הקהילה נוצרה} ע"י מישהו. בראיתו, גם הציונות היא לקיחת הדת והפיכתה ללאום. 
	
	\renewcommand{\footrule}{\rule{\linewidth-26pt}{0.25pt}\vspace{-5pt}} % for pages 10+
	
	\subsection{רעיונות ההשכלה והנאורות}
	תזכורת – רעיונות ההשכלה. אזור אומץ, השתמש בשכלך. הכנסייה אינה שולטת על מקור הידע. המשכילים רצו לשנות את המבנה החברתי, בצורה שתאשר חירות, שוויון וזכויות (העיקריות: חיים, חירות וקניין). 
	
	בתפיסה הפילוסופית של האמנה החברתית – העם נותן כוח לשלטון, בתמורה לכך שהשלטון ישמור על זכויותיות הטבעיות. 
	
	רעיונות ההשכלה מדברים על כמה דברים מרכזיים: 
	\begin{itemize}
		\item תנועת ההשכלה שהתתפחה במכה ה־18 באירופה כללה סופרים, מדענים ופילוסופים. נאורות. העז לדעת!
		\item האדם בכוח התבונה שלו יכול לעצב את חייו האישיים והציבוריים
		\item טענה כי לכל אדם יש זכויות טבעיות שאיתן הוא נולד – \textit{חירות, קניין, חיים}. חירות נגזרת לזכויות אדם. 
		\item הסדר החברתי – מעוות (המשולש של המלוכה, כמורה, מעמד שלישי). הבעיה – אנשים לא מעיזים לדעת. 
	\end{itemize}
	
	כיצד תרמו: 
	\begin{itemize}
		\item ערעור על הסדר הקיים. 
		\item עוזרים לתהליך החילון. חילון, אינה נטישת הדת, אלא נטישת הזהות הדתית. בפרט, הם מחפשים מוקד זהות חדש. . 
		\item העם הפך להיות הבסיס לשלטון (לפחות ברעיון). העם ריבון $\impliedby$ מדינה.
	\end{itemize}
	תנאי הכרחי: חייב להתקיים כדי שהאירוע יקרה.
	לדוגמה, בשביל להדליק עץ צריך בין היתר חמצן, חיכוך, וחומר שריפה, ואבקת שריפה (לא כימית, רק כרעיון). כל אחד מהם תנאי הכרחי, וכולם יחדיו תנאי מספיק. 
	
	ברמה הפרקטית תנאים מספיקים לרוב יהיו ערימה של תנאים הכרחיים (בהקשר ההיסטורי). [זה השלב שבו מרצים למתמטיקה בדידה מתעלפים ומתים]
	
	ישנם גורמים הכרחיים רבים כדורמים לצמיחת התנועות הלאומיות. גורמים רעיונים, היסטוריים, וחברתיים־כלכליים. אחד מהם לבדקו הכרחי, אך לבדם הם לא מספיקים לצמיחת התנועות הלאומיות. 
	
	
	\subsubsection{כיצד תרם ללאומיות}
	התפיסה הפכה להיות שהם הופך להיות הבסיס לשלטון. העם זכאי לבחור לשטון המבטא את הרצון הכולל. כך, גם עמים צריכים להיות חופשיים מגורמים זרים או יחידים ולכך שאפו להגיע בני הלאום, שביססו את רעיונותיהם על הנאורות. 
	
	כן גם האצה לתהליך החילון, שהתפתח בעקבות רעיונות ההשכלה. התהליך שבר את הבלעדיות של מוסדות הדת, ושחרר את האנשים ממקור זהות של הדת. אותם אנשים חיפשו מוקד זהות אחר – הלאומיות מלאה חלל זה. זכרו – החילוניות אינה חוסר אמונה, אלא "פחות אמונה". 
	
	לא רק הלאומיות צמחה במאה ה־19, אך לבינתיים נתמקד בה. 
	
	\subsection{רעיונות התנועה הרומנטית}
	לאחר רעיונות ההשכלה והנאורות, באה התמונה הרומנטי – שהפוכה ברעיונותיה. הם מדברים על חשיבות של הרגש והדמיון, במקום חיפוש הרווח שבהשכלה אחר האנושי־כללי, ויוצרים איחור ע"י ייחוד. התנועה הרומנטית חיפשה את הייחודי והפרטי. הרומנטיק ההסתכלה אל העבר, בחיפוש אחר מה משותף לנו אל העבר ולימי הקדם. דוגמה לכך נראה בסיפורי גבורה, ואגדות האחים גרים – אלמנטים תרבותיים שקשורים לקבוצה מסוימת. 
	
	דוגמה נוספת – חנוכה מאז ומתמיד היה חג, אך הוא לא היה פופולרי והיו מבוססים על סיפורים שהגיעו מהספרות שלא היו מוכרים ע"י היהדות ונשמרו בכנסייה הקתולית. הציונות חיפשה משהו לאחד באמצעותו את התנועה, והרימה את החגים הללו, כחלק מהשלב הראשון בהתפתחות תנועה לאומית. עוד יותר התאים לתנועה העובדה שלאחר סיפור חנכוה הוקמה כאן ממלכה יהודית עצמאית בארץ ישראל. גם שירי החגים משרתים את הרעיון הציוני. 
	
	\subsection{גורמים היסטוריים־פוליטיים}
	בעיקר – המהפכה הצרפתית והאמריקאית. שתי המהפכות עודדו את רעיונות הנאורות, והייו השראה לעמים אחרים. 
	
	נפוליאון כבש בשם השחרור מדינות וטען שיוריד את השלטון הזר, אך במציאות הוא הפך להיות השלטון הזר במדינות אותן הוא כבש. 
	\subsection{גורמים חברתיים־כלכליים}
	\subsubsection{הרפורמה האגררית (חקלאית) והמעבר מהכפר לעיר}
	שיטות הזריעה והדישון השתכללו, והומצאו מכונות לחקלאות. השוק התעייל והופחת הצורך בידיים עובדות. הרפואה מתפתחת והתמותה יורדת. הגידול באוכלוסיה, יחד עם פחות צורך בידיים עובדות, לא עזרו. השכלולים הגיעו לעיר, שם מוצרו מפעלים שנוצרו כחלק מהצורך בייצור יותר המוני. 
	
	לדוגמה, הבן של יונה מ"יגון" מת, יונה מגיע לעיר בשביל לעבוד. רובם עובדים במפעלים, שם כל אדם תקוע בפס ייצור שעושה פעולה רפטיטיבית פשוטה (נאמר, יום שלם אתה מדביק סוליה, בתנאים קשים, עם עבודת ילדים). איכות החיים הייתה ירודה. אוויר מחניק, לכלוך ובלי טבע. 
	
	יגון $\leftarrow$ סיגל $\leftarrow$ דיכאון (ככה המורה כתבה). 
	
	בהיסטוריה, יש תחושת ניכור ואובדן מוקד זהות. זהו. לא "היו עצובים, התהלכו להם בעיר והיו שכוכים ומדוכאים". זו \textit{לא} תשובה בהיסטוריה. מה שקרה הוא שהאדם כבר לא "אבא של", "הבעלים של" ואין לו ממש שייכות, ואתם הרעיונות באו למלא את החלל הזה – את הצורך להרגיש שייכות למשהו. החברה והאנשים עברו ממבנה חברתי אחד לאחר, ומתוך אותו התהליך הרעיונות הללו צמחו. 
	
	\subsection{הגדרות מחוץ לתוכנית הלימודים}
	ציונות := התנועה הלאומית היהודית. התנועה התפתחה יחסית מאורח, בסוף המאה ה־19. 
	
	בניגוד לזהות האיטלקית, לדוגמה, שם איטלקי יכול להיות מוסלמי, נוצרי או בכל דת אחרת, המדינה הלאומית היהודית היא גם בהקשר הדתי. זה יוצר בלבול (הפרדת דת מהמדינה וכו'). 
	
	\begin{itemize}
		\item ישראלי – בעל שתי הגדרות. הראשונה – בעל אזרחות ישראלית. השנייה – בעל אזרחות ישראלית, וחלק מהתנועה הציונית. בהגדרה הראשונה ייתכן פלסטינאי ישראלי, ובשנייה לא. 
		\item ערבים – מוצא אתני. לא קשור ללאום. 
		\item פלסטיני – לאום בפני עצמו. 
	\end{itemize}
	
	\subsection{רקע}
	לאחר חורבן בית שניה, האומה היהודית איבדה עצמאות. לאורך הגלות, התפיסה היהודית האורתודוקסית הפכה לכך שאסור לדחות את הקץ (בוא המשיח) בצורה התנגדה לתנועה הלאומית. 
	
	בסוף המאה ה־19, רוב היהודים היו במזרח איופה (רוסיה ופולין, בידח עם אוסטרו הונגריה). בתקופה הזו, היהודים במעטים (50K מתוך מיליונים) נמצאו כאן מסיבה דתית. הם לא תמכו בבוא הציונים. הרוב בארץ היו לא יהודים, ובארץ היה שלטון איפריאלי עותמני. כל אחד דיבר שפה אחרת. יש גם שוני תרבותי. יש גם שוני באופן ניהול החיים ההדתיים. 
	
	\section{הגורמים לצמיחת הציונות}
	הגורמים העיקריים הרלוונטיים תנועה היהודית בלבד: 
	\begin{itemize}
		\item כשלון האמנסיפציה
		\item שנאת ישראל במזרח אירופה ואנטישמיות מודנית במרכז ומערב אירופה
		\item השפעת התנועות הלאומיות. 
		\item תהליך החילון וכוחות פנימיים בחברה היהודית. 
	\end{itemize}
	
	נתבונן בדוגמה מסרט, של הדמות איגוץ זוננשטיין. מצד אחד:
	\begin{multicols}{2}
		\begin{itemize}
			\item ד"ר למשפטים
			\item שינוי שם הוא התנאי לתפקיד (שופט בית המשפט העיליון)
			\item ניתוק מהדת
			\item חגיגת סיום המאה ה־19 ותחילת המאה ה־20 הנוצרית, ולא היהודית. 
			\item אחים שינו שמות גם (רופאים)
			\item האבא דאג משינוי השם, היה חשוב לו לשמור על המסורת. בסוף ויתר בטענה שהשמות לא ניתנו ע"י ה'. 
		\end{itemize}
	\end{multicols}
	מנגד: 
	\begin{multicols}{2}
		\begin{itemize}
			\item קידוש
			\item חתונה יהודית, בבית כנסת. 
			\item יהודי. עדיין שומר על הדת. 
		\end{itemize}
	\end{multicols}
	
	אנשים בגרמניה קראו לעצמם "גרמנים בני דת משה". הדת היהודית נשמרה בבית, אבל למראית עין הם גרמנים רגילים. הזהות הלאומית הגרמנית חזקה יותר. הם רוצים להשתלב בחברה. 
	
	\subsection{אנטישמיות מודנית}
	שנאת יהודים לא כי "הם הרגו את ישו", ולא מסיבות דתיות, לפיה היהודים רוצים לשהתלט על העולם, וכו'. שנאת ישראל והודים – תמיד יש פתרון. בשנאת ישראל יש פתרון – המרת דת. לאנטישמיות מודנית אין פתרון – מוצאך יהודי, בכל האשמה. 
	
	אם מישהו ביוון שונא יהודים כי הם מאמינים באל אחד ולא עובדים אלילים, די בהמרת דת כדי "לפתור את הבעיה". באופן דומה גם בימי הביניים (אם כי שם קראו להם "היהודים החדשים", tag שנשאר לדור אחד או שניים). האנטישמיות המודנית היא גזעית ולכן לא די בכך. 
	
	
	\textbf{שנאת ישראל} קיימת מהיום שיש יהודים. שנאה על רקע דתי. \\
	\textit{הפתרון: }להחליף דת. \\
	\textbf{אנטישמיות} היא מושג מודרני. שנאה על רקע גזעי. בעל היבטים כלכליים, פוליטיים וכו'. לפיה, מקור בעיות גרמניה ושל החברה התעשייתית היא היהודים, וכי ההתנגרות ליהודים מיוסדת לא על דת אלא על גזע. היהודים מנסים להשתלט על העולם מבחינה כלכלית,  (בעיקר כלכלית, אבל לא רק כלכלית). \\
	\textit{הפתרון: }אין. 
	
	יש הבדל בין אנטישמיות, אנטי־ציונות, ואנטי־מדיניות־של־מדינת־ישראל. 
	
	קיימות מפלגות בעלות מצע אנטישמי, ברחבי אירופה. מתפתחת ספרות ועיתונות אנטישמית. 
	
	
	
	\subsection{האמנסיפציה}
	מקור המושג ברומה העתיקה, ובה ציין המושג שחרור של בן ממרותו של אביו והעמדתו ברשות עצמו. 
	
	משמעותו שוווין זכויות בחוק משפטי ־ הענקת כויות משפטיות בדבר שוויון מעמדם של היהודים והכרה בהם כאזרחים שווי זכויות. 
	
	בעבר היהודים קיבלו זכויות קקהילה. במקרים מסויימים האדון היה מזמין את היהודים לגור אצלו, כי הם היו יכולים להלוות בריבית וכו' ובכיולתם היה לעושת דברים שלנוצרים היה אסור. אם זאת, האמנסיפציה ניתנה באופן אישי ובאופן כללי ההגבלות היו רבות. 
	
	האמנציפניה התחילה בצרפת. בה, הזכויות ניתנו באופן קהילתי ולא באופן אישי. רק במערב אירופה הייתה אמנציפציה – \textbf{לא הייתיה אמנציפציה במזרח אירופה}, ובעקבות זאת האנטישמיות במזרח אירופה הייתה חותה ושונה במקורותיה. 
	\textbf{התמורות שחוללה האמנסיפיה: }
	\begin{multicols}{2}
		\begin{itemize}
			\item התעוררות היהודים בחברה הסובבת. 
			\item חופש תנועה. 
			\item שינויים דמוכרפים – ריכוז בערבים הגדולות, יציאה מהשוכנות היהודיות (שאז נקראו גטאות), הגירה בין יבשתית. 
			\item תמורה בתעסורה – עיסוק בפוליטיקה, פרלמנט, שרים, קצונה, מקצועות חופשיים ועוד. 
			\item שילוב בהשכלה הגבוה, וחייב לימוד שפה נוספת. 
			\item פתיחות והתנגדות לשינויים שהובילו לשבירת המבנה המסורתי הדתי. 
			\item הקצה והתנגדות להשתלבות – קבוצות שהסתגרו והקשיחו את עמדותיהן כדי לא להתלשב בחברה (חרידים)
		\end{itemize}
	\end{multicols}
	
	מנדלסטון, אוצ'ילד, פרמייה, איינשטין, מאהלר, ועוד רבים אחרים. 
	
	\begin{center}
		\sen
		\begin{forest}
			[\he{אמנסיפציה}
			[\he{שילוב}]
			[\he{דחייה}]
			]
		\end{forest}
		\she
	\end{center}
	
	מבחינה חברתית, הייתה דחייה גדולה לאמנסיפציה. בפרט מסיבות של קנאה – היהודית הצליחו להתקדם באופן לא פורפורציונלי לכמה היו בחברה. לכן, יהודים רבים לא התקבלו לעמדות כאלו ואחרות מעצם היותם יהודים. 
	
	\textbf{אין במזרח אירופה אמנסיפציה}.
	
	
	\subsubsection{שילוב ודחייה}
	פעם הייתה שאלה בבגרות – האמנסיפציה גרמה לשני תהליכים הפוכים, ציינו תהליכים אלו והסבירו. 
	
	זכרו – תהליך החילון והשילוב בקהילה, לא אומרת שהאדם הפסיק להיות יהודי. אלא, שהזהות נשמרת בעיקר בתוך הבית, פחות כלפי חוץ, ומשנים את הסממנים החיצוניים (דיבור, שמות, לבוש) בהתאם. 
	
	כן היו יהודים שהתנצרו ונשאו בנישואי תערובת, אך אין זה היה הרוב. 
	
	האמנסיפציה מובילה גם לדחייה. היא לא הייתה חברתית, אלא חוקית, ולכן ושבים רבים לא היו בשלים לקבל את השינוי במעמד היהודים ולפיכך היהודים קיבלו יחס מפלה. הגויים המשיכו לראות אותם זרים והיהודים הסיקו שהם לא יכולים לסמוך על תושבי אירופה שישפרו את מצבם ועליהם לסמוך רק על עצמם ולפיכך התפתחה הלאומיות. 
	
	חלק מהיהודים החליטו שהשתלבות בחברה גובה מהמם מחיר יקר שכן עליהם לוותר על הלאומיות שלהם ועל העבר, השורשים והמסורת כדי להשתלחב כאזרחים. אלו סברו שהפתרון לכך הוא פיתוח הלאומיות היהודית. 
	
	אחוז היהודים בתפקידים בכירים היה גדול לאין־שיעור מאחוזם האוכלוסיה. "האמנסיפציה היא לא איזה כפתור שלוחצים עליו ואז כולם הופכים להיות חברים שלנו". 
	
	האיום של התבוללות הוביל לרעיון של חייה לאומית. המטרה הייתה להלם בהשפעות השליליות של האמנסיפציה הגורמות לסכנה לקיום הלאום היהודי. מנגד, יש הטוענים שדווקא האמנסיפציה חיזקה את הלאומיות בכך שחיזקה ביהודי ת הכוחות להילחם על ייחודו כיהודי. 
	
	\subsection{פרשת דרייפוס}
	קצין יהודי הואשם, לאחר שהעלילו עליו עלילות. העובדה שבתוך החברה הצרפתית האשימו את היהודים ככל בבגידה, וקיימו אירועים פומביים משפלים, הופכת את זה לאנטישמיות. צרפת נחלקה ולאחר מספר שנים הוא זוכה. 
	
	
	\section{הרצל}
	
	\subsection{חזונו של הרצל}
	\begin{enumerate}
		\item הפתרון המדיני צריך להיות פומבי ומוסכם
		\item על היהודים להשיג צ'טר מן המעצמות הגדולות
		\item כל טריטוריה היא טריטוריה אפשרית (מאוחר יותר שינה את הגישה)
		\item לצורך הקמת המדינה צריך להקים שני ארגונים – אגורת היהודים שתטפך בעיניין המדיני וחברת היהודים שתטפל בעניין הכלכלי
		\item העניים יהיו הראשונים שיהגרו למדינת היהודים ויניחו בה את התשתיות
		\item במדינת היהודים ישררו צדק, שוויון וסובלנות דתית. 
	\end{enumerate}
	כדי להגשים את חזונו פעל הרצל בשלושה תחומים: 
	\begin{enumerate}
		\item כתבים: 
		\begin{itemize}
			\item ספר – "מדינת היהודים" (הבעיה ופתרונה)
			\item עוד ספר – אלטנואילנד – חזון לאופיהמדינה
			\item עיתון – די־וולט, עיתון בגרמנית. 
		\end{itemize}
		\item ארגוני: 
		\begin{itemize}
			\item הקונרס הציוני הראשון בבאזל
		\end{itemize}
		\item דיפלומטי: 
		\begin{itemize}
			\item מדעים עם מנהידי מדינות זרות להשגת צ'רטר
		\end{itemize}
	\end{enumerate}
	
	"אם תרצו, אין זו אגדה". חידה – מה ההמשך של המשפט. לא צריך להגיש את החידה. 
	
	\subsection{קונגרס באזל ותוצאותיו}
	ב־1897 החליט לכנס קונגרס ציוני בינלאומי בהסתתפות נציגים מקהילות יהודיות. זאת, ע"מ לממש את רצונו "להקין" (להקים עם שגיאות הקלדה במצגת) מדינה ציונית שתקיף את המוני היהודים בארצות השונות. 
	
	קונגרס – ערימה גדולה של אנשים שמגיעים בשביל להתכנס ולדבר. 
	
	הרצל לא המציא שום דבר חדש. הקבוצה מבשרי הציונות, והספר "אוטו אמנציפציה", כבר הם הציגו את רעיון הציונות. אך, הרצל היה היחיד שניסה להפוך את הרעיון למשהו יותר מציאותי. הוא התחיל זאת באמצעות הקונגרס הציוני. 
	
	כינוס הקונגרס נמצא בין שלב ההתאגרנות לשלב ההפצה בשלבי הפיתוח של התנועות הלאומיות. ממהקונגרס יצאו שני דברים – ביסוס והגדרת התנועה הציונית (שלב 1) ודרכי ההפצה שלה (שלב 2). הקונגרס שאף להפיץ את הבשורה הזו גם לשאר העולם, ולא רק ליהודים. 
	
	הרצל הוציא הזמנה. היא פירטה מקום, זמן, ואיך לבוא לבושים. מצא אולם גדול בשביל שתצא תמונה יפה. הרצל מנסה לעשות יחצנות לתנועה הציונית. זה לא רק המהות, גם הניראות. כולם באו עם חליפות "פראק" (שחורות כאלו עם עניבות), והקונגרס הופץ ודווח בעיתונות העולמית. 
	
	נראה בעמוד 95 בספר שמרבית האנשים באו מרוסיהואוסטרו־הונגריה. 
	
	הקונגרס אפשר: 
	\begin{itemize}
		\item הביסס הראשוני למדינה
		\item גרם למעצמות לראות בתנועה הציונית תנועה לאומית לגיטימית (הכרה בין־לאומית)
		\item הקמת מוסדות ובראשם "ההסתדרות הציונית" (נוסף עליה, גם את קק"ל ועוד)
	\end{itemize}
	
	נתעמק במשפט "הציונות שואפת להקים בית מולדת לעם היהודי על פי משפט הכלל, לעם היהודי בארץ ישראל„. [מופיע בעמוד 96] "בית מולדת" (בתרגום מסויימים, בית לאומי) – מכאן הצהרת בלפור המציאה את זה. בית מולדת אינו מדינה. אם הרצל היה שואף להקים מדינה, אז העות'מנים היו מתעצבנים (מי אתה שתקים מדינה על שטחה של אימפריה). "משפט הכלל": כלומר, בין לאומי. הרצל היה עורך דין וכתב כמו עורך דין, בניסוח שלא יעצבן אף אחד. 
	
	הרצל תמך בגישה של ציונות מדינית. לפיה, צריך לקבל הכרה בין לאומית טרם ההתיישבות (ע"פ משפט הכלל). קודם צ'רטר, ולאחר מכן נבנה את המדינה. לא נבוא כמו "גנבים בלילה". 
	
	תוכנית באזל פירטה את האמצעים שיש לנקוט לשם השגת המטרה: 
	\begin{itemize}
		\item יישובה של ארץ ישראל ע"י איכרים, בעלי מלאכה ותעשייה, אם היא מכוונת אל המטרה. [התיישבות מעשית]
		\item ארגונה של היהודות כולה ואיחודה במפעלים מקומיים וכלכליים ע"פ חוקי כל ארץ וארץ. [כלומר, לא נמרוד במדינות בהם היהודים נמצאים כרגע]
		\item הגברת הרגש העצמי היהודי וההכרה הלאומית היהודית. [יצירת תודעה לאומית]
		\item עבודות הכנה כדי להביא להסכמת הממשלות הנדרשות למימוש מטרות הציונות. [צ'רטר]
	\end{itemize}
	(1) שייך לציונות מעשית, (4) לציונית מדינית, (2) ציונות כלכלית. 
	
	היהודים ממזרח אירופה ורוסיה היו אלו שתמכו בציונות המעשית. העליה הראשונה התרחשה ב־1881, בעיקר רוסים ורומנים. 
	
	תוכנית באזל – (בעבר היה צריך לדעת בע"פ)
	\textbf{האמצעים} להשגת המטרה: 
	\begin{enumerate}
		\item יישובה של ארץ ישראל ע"י איכרים, בעלי מלאכה ואנשי תעשייה יהודים (ציונות מעשית). 
		\item ארגון היהדות כולה וליכודה
	\end{enumerate}
	
	בקונגרס באזל הוקמה ההסתגרות הציונות ומוסדותיה. חברי ההסתגרות שלמו דמי חבר. הם בחרו את צעיגי הקונגרסים הציוניים, והקונגרס מתכנס בכל שנה. "תשלום השקל" היה הכיוני לדמי החבר. אכן כונס קונגרס כל שנה חוץ מבמלחמת העולם השנייה. 
	
	ההסתגרות הציונית איפשרה: 
	\begin{enumerate}
		\item (1897) הקונגרס הציוני, שמתוכו: 
		\begin{enumerate}
			\item הועד הפועל הציוני (נבחר ע"י חברי הקונגרס ואחראי לביצוע החלטותיו)
			\item הועד הפועל המצומצם (גוף מצומצם שבחר ע"י הועד ומנהל את הפעילות השוטפת של ההסתדרות)
		\end{enumerate}
		\item (1899) אוצר התיישבות היהודים (מושג כלכלי למימון הפעילות). מתוכו נוצר (1902) בנק אנגלו־פלסטינה, שהיה המוסד הכספי לפעילות של ההסתגרות \textit{בארץ ישראל}. 
		\item (1901) קק"ל (קניית קרקעות. הכסף יצא מבק־אנגלו פלסטינה)
		\item (1908) המשרד הארץ הישראלי (ייצוג ההסתגרות בציונית בארץ ישראל). מעבר מציונות מדינית לציונות סינטית (ציונות מדינית לצד ציונות מעשית). 
	\end{enumerate}
	
	<צפייה ביהודים באים> 
	
	הרעיון של הרצל, היה להפוך את הבעית היהודים לבעיה עולמית, כדי שיהיו אינטרסים למעצמות לסייע לתנועה הציונית. בכך, הם ייתגייסו לנו לעזור לקבל את הצ'רטר. על פלסטינה שלטה האימפריה הע'ותמנית. 
	
	
	\subsection{אוגנדה (למה לא בעצם)}
	ההנחה היא שעשינו את שיעורי הבית משיעור שעבר. 
	\subsubsection{תוכנית אל־עריש}
	הרעיון – להקים במשולש באיזור הנגב דבר הדומה למדינת ישראל. כמה עשרות שנים לאחר מכן היו את התשתיות אבל באותה התקופה לא היו תשתיות. גם ממשלת מצרים התנגדה. 
	\subsubsection{מבוא לתוכנית אוגנדה}
	\textit{(1903)}	לאחר שתוכנית אל־עריש נכשלה, הבריטים הציעו להרצל הקמת אוטונומיה יהודית בחסות בריטית באוגנדה. נפריד בין שני מושגים: 
	\begin{itemize}
		\item \textbf{אוטונומיה: }לא בהכרח מדינה. יכולה להיות אוטונומיה דתית, או תרבותית, וכו'. 
		\item \textbf{מדינה: }ישות אוטונומית ריבונית. 
	\end{itemize}
	זה לא היה בדיוק הצ'רטר שהרצל רצה, אבל זה היה משהו. אבל, למעשה בהצעה בריטניה הכירה בעם היהודי כלאום שמגיעה לו מדינה. היא גם הכירה בתנועה הציונית במייצגת את העם היהודי. לכן, גם אם התוכנית לא יצאה לפועל, היה בה קידום היחסים עם בריטניה. ג'וזף צ'נברליין, שר המושבות הבריטית, הציע את התוכנית. בהתחלה הרצל סירב, אבל לאחר פוגרום קישינב שינה את דעתו. הרצל גם הגיע וביקר בקישינב, והקשיב למה שקרה. 
	
	שרית טוענת שהיא ניכבית ב־1 בצוהריים ביום שישי, אפילו אם לא אכלה בכלל. 
	
	\subsubsection{טיעונים בעד ונגד}
	\begin{itemize}
		\item \textbf{בעד: }
		\begin{itemize}
			\item צריך למצוא פתרון מידי, זמני, כי העם היהודי נמצא בסכנה. פורגום קישינב עזר לתפישה הזו. 
			\item "מוטב ציונות בלי ציון מציון ללא ציונות". 
			\item באוגנדה ילמדו היהודים לחיות חיים עצמאיים וחלקאיים ויעידו לא"י מוכנים יותר. 
			\item לא ניתן לוותר על ההצעה בקלות שכן היא הישג חשוב שאין לדעת אם ישוב – הכרה של מעצמה בזכות של היהודים במסודותיה ובתנועה. 
		\end{itemize}
		\item \textbf{נגד: }
		\begin{itemize}
			\item אין ציונות ללא ציון. רבים מהמתנגדים היו דווקא יהודי רוסיה שלא ראו מקום אחר פרט לא"י כמתאים. 
			\item אוגנדה תהפוך במציאות לפתרון קבועה, ולא באמת תישאר פתרון ביניים. 
			\item אין לעם היהודי זיקה לשטח כפי שיש לו לציון. א"י היא המולדת ההיסטורית היחידה של העם היהודי. 
			\item כל משאבי התנועה הציונית יבוזבזו על התיישבות באוגנדה במקום על א"י. 
		\end{itemize}
	\end{itemize}
	
	\subsubsection{ההחלטה}
	הרצל ורוב הקונגרס ראו באופן חיובי את ההצעה. הפולמוס היה קשה וגרם לסערה גדולה בקרב צירי הקונגרס. הקונגרס השישי קיבל את הכיונוי "קונגרס הבועים". הרצל הציעה פשרה – משלחת חקר לאוגנדה. בסוף התקבלה החלטה בקונגרס השביעי שא"י היא היעד היחידי להקמת מדינה יהודית לעם היהודי, ושהצעות אחרות לא רלוונטיות. זאת בגלל ש־: 
	\begin{enumerate}
		\item המשלחת חזרה עם תוצאות שליליות, לפיהן אין באוגנדה את הדרוש להקמת מדינה. 
		\item הרצל מת. 
	\end{enumerate}
	
	\section{אני לא רוטשילד}
	רוטשילד – מסייע להתיישבות שאינה סוציאליסטית (כי הוא היה קפיטליסטי). ללא רוטשילד, ההתיישבות באותה התקופה הייתה קורסת, ואנשים היו חוזרים לאירופה (וזה לא שאנשים לא חזרו לאירופה). ע"פ המורה, אנחנו מלמדים רק את מה שנוח לנו מהציונות ואת הצלחותיה, אך אנשים רבים אכן חזרו לאירופה. נדגיש את ההתיישבות החקלאית, כי זהו האידיאל הציוני – כיבוש האדמה, העבודה, וכו'. מרבית היהודים שהיגוע לארץ התיישבו לעיר, ולא במושבות או בקבוצה (הבהרה: קוטשילד לא היה סוציאליסט ולכן לא תרם לקבוצה). 
	
	"אנחנו מלמדים היסטוריה באג'נדה מסוימת, אך קחו בחשבון שיש תפישה קצת יותר רחבה". 
	
	רוטשילד לא מנהיג ציוני. למעשה, לא היה ציוני כלל. הוא רואה חשיבות בגרעיני התיישבות של היהודםי בארץ, בשביל לפתור מצוקה של יהודי אירופה ובעיקר יהודי רוסיה (פוגרומים) וכדי לשנות את הדימוי היהודי. 
	\begin{itemize}
		\item רודטשילד ראב באיכרים את גרעין ההתיישבות הלאומית. 
		\item סייע בארץ עוד לפני העליה הראשונה
		\item רצה לשנות את דמית היהודים. 
	\end{itemize}
	הוא לא היה ציוני, כי לא תמך ברעיון בלאומי. 
	
	רוטשילד סייע בישטת האפוטרופסות (החסות). הוא רכש את האדמות מן המתיישבים, והוא הפביא פקידים [נציגים שלו] שניהלו את המשובות. אותם פקידים ניהלו את המשבות בתפישה של רווח – לא בתפיסה אידיאולוגית של לאומיות. הפקידים קבעו את שחכם של המתיישבים. זה מנוגד לאידיאל לציוני – לפיו, אנשים עובדים את אדמתם, בשביל לקיים את עצמם. אותם אנשים שקיבלו כסף הפכו לשכירים, שעובדים אדמה של מישהו אחר. 
	
	פקידי הברון הגיעו מצרפת, לא היו ציונים, והגיעו עם תפישה אחרת. היה להם אג'נדה לקדם את היישוב היהודי ממקום כלכלי ולא ממקום ציוני, והם לא הקשיבו למתיישבים ולדעותיהם. 
	
	לסיכום, קוטשילד קידם מאוד את ההתיישבות ואת התעשייה, אך בצורה שמנוגדת לתפישה האידיאולוגית הציונית. הוא גם לא תרם כסף להרצל. 
	\subsection{ייתרונות: }
	\begin{itemize}
		\item בסיסו המושבות הראשונות
		\item שיטות עבודה השפרו, קיבלו נסיון חקלאי, והשתמשו בכלי עבודה מוקנים יותר. 
		\item טיפל בהתנהלות מול העו'תמנים, ורכישת קרקעות (העו'תמנים אסרו לכאורה על רכישת קרקעות ולא אהבו את ההתיישבות היהודית). 
		\item הקמת מערכת שירותים מפותחת: בתי ספר, גני ילדים, מרפאות, בתי כנסת וכו'. 
	\end{itemize}
	
	לדוגמה, ענף היין הצליח במיוחד. תעשיות הזכוכית והבשמים נכשלו. קפיצת מדרגה – לא רק חקלאות, אלא גם ייזמות. 
	
	בכך פתר בעיות רבות – המתיישבים ניסו לגדל חיטה בהר (לא אפשרי) משום שלא היה להם שום נסיון. רוטשילד יק להם הכשרה חקלאית מתקדמת ובכך אפשר את קיום המושבות. 
	
	המשפחה של רודטישד הייתה עשירה במיוחד, אימפריות רבות לקחו הללבאות מאותם הבנקים. 
	
	\subsection{חסרונות: }
	\begin{itemize}
		\item ניכור בני הפקידים למתיישבים ברמה האידיאולוגית והתרבותית (מזרח אירופה לעומת צרפת). הצרפתים ראו את המזרח־אירופאים כנחותים מהם. התרבות 
		\item המתיישבים שרצו ליצור את דמות היהודי החדש נעשו תלויים בברון. 
		\item התשבות הצרפתית שהגיעה עם הפקידים חדרה חיי המתיישבים והם חששו מכך. 
	\end{itemize}
	
	בשנת 1900 הברון הפסיק לתמוך במושבות והעביר את המושבות לחברה בשם יק"א, ולמעשה "יצא מהמשחק". 
	
	הייתרונות והחסרונות הללו \textbf{יכולים להופיע במבחן}. 
	
	\subsection{המשרד הארץ־ישראלי} 
	הוקם ב־1907 ע"י ההסתדרות הציונית. 
	פחות סייע להתיישבות בארץ באופן אקטיבי (כן היה את קקל, אוצר התיישבות היהודים וכו') ולא הייתה בארץ נציגות של ההסתדרות הציונית. לאחר מותו של הרצל התקבלה ההחלטה לנקוט בגישה של \textbf{ציונות סינטטית} – שילוב בין ציונות מדינית לציונות מעשית. 
	
	ציונית מדינית – לא להיות גנבים בלילה, נקבל צ'רטר. ציונות מעשית – קודם נתיישב, ואז נתעסק עם דברים אחרים. 
	
	דה־פקטו, בשנותיה הראשונות, ההסתדרות הציונית פעלה בשונותיה הראשונה פעלה במישור המדיני. המשרד הארץ־ישראלי הוקם ברעיון של ציונות סינטית. 
	
	הארגון סייע סהקמת דגניה ות"א. גם סייע בהכשרת עולים בחקלאות, מציאת עבודה, ומעודד עלייה מתימן (פחות רלוונטי לבגרות). שמואל יבניאלי היה נציג שעודד רדים מיהודי תחימן לעלות לארץ (מסיבות ציוניות). 
	
	\section{התבטאות מאפייני התנועות הלאומיות בתנועה הציונית}
	\subsection{על השפה}
	שפה היא אחד הגורמים המשמעותיים ביותר לליכוד את בני הלאום. היא מאחדת ע"י ייחוד. לתנועה הציונית אין שפה. דוגמה: גיצד תתאר סלט יירקות, כאשר אין מילה לעבניה ולמלפפון בתנ"ך? גם כל אדם דיבר שפה אחרת. יידיש, לדינו, גרמנית, וכו'. אף אחד לא מדבר עברית, שכן זוהי שפת הקודש, וכמודגם, אין מילים מתאימות. העברית נתפשה כשפת הקודש ועל כן לא התקדמה ביחס לעולם. "הצורך הוא אבי ההמצאה" = לא היה צורך במילה "עגניה" בתנ"ך, ועל כן היא לא קיימת בו. באופן דומה, כן היה שמן זית. אליעזר בן יהודה מגיע לארץ ב־1881 וקידם רבות את השפה העברית ופיתח חלק ניכר, הן מן השפה והן את התפישה של כיבוש השפה – שפה יחודית ללאום היהודי. 
	
	אחד הדברים הראשונים שבן־יהודה עשה הוא לדבר בביתו עברית בלבד. אשתו, ובנו (איתמר בן־אבי) דיברו בעברית בלבד ונאסר עליהם לדבר בשפות אחרות. בנו זכה לכינוי "הילד העברי הראשון". להבהרה: אף אחד מהדברים האלו לא נחשב פעילות לקידום השפה (השפה שנו מדבר לא משפיעה על קידום השפה). 
	
	שאלה: למה עברית ולא יידיש? אומנם עברית היא שפה מתה, אך היא זו המייחדת את העם היהודי. לתפישת אליעזר בן יהודה, ניתן לקחת את אותה השפה המתה – העברית – ולהחיות אותה. יהודים בכל העולם הכירו את העברית התנ"כית. כמו כן, האמין כי יהודים  אינם יכולים להיות עם החי רק בעזרת השיבה למולדת. החייאת השפה ביטאה את התחייה הלאומית והדביר בעברית את הלכידות להאומית. הוא אומר כי שלושה דברים חרותים על דגל הלאומיות היהודית: ארץ, שפה והשכלה [במקורות אחרים: ארץ, שפה ודגל]. הוא מתייחס למאפיינים המלגדים את בני הלאום. 
	
	הוא פועל במגוון צורות: 
	\begin{itemize}
		\item כתיבת והוצאת "מילון עברי־עברי"
		\item הקמת חברת "תחיית ישראל" שאחת ממטרותיה הייתה החייאת השפה העברית כשפת דיבור בחיי היום יום (זה לא קיים היום). 
		\item חידוש והמצאת מילים רבות כגון עיתון, רכבת, תזמורת, טקס, אופניים, משקפיים, גלידה, צלחת וכרביים. הוא הראשון שעשה זאת אך לא היחיד. 
		\item הוציא את עיתון "הצבי" שם הפיץ הפיץ את דעותיו והמילים החדשות אותם המציא. 
		\item הקים את חברת "שפה ברורה" – המוסד העליון לקביעת חידושי הלשון. ממנה, התפתח את "ועד הלשון העברית" שמטרתו הייתה הפצת השפה. זהו אינו מוסד אקדמי, אך מקביל לאקדמיה ללשון העברית, שהיא נגזת של ועד הלשון העברי. 
	\end{itemize}
	
	שפה היא דבר חיי שמתחדש כל הזמן ובהתאם לשינויים לחברה ובתרבות. דוגמה: המילה "מרשתת". גם מושגים רלוונטים לחידוד והעברת אידיאולוגיות מסויימות. 
	
	\subsection{על החינוך}
	דוגמה לארגונים: בראשון לציון הוקם בית הספר העברי הראשון. הגימנסיה העברית לישורליים. בצלאל, הקונסברטוריון, שמינר לוינסקי להגשת מורים, הטכניון, האונ' העברית ועוד. 
	
	מלחמת השפות: ראה שנה שעברה. בחיפה החליטה חברת "עשרה" להקים את הטכניקום (הטכניון של היום). בטכניון מלמדים מקוצעות מדעיים, ו"עזרה" החליטה שהטכניון יילמד בגרמנית, שכן אז נחשבה לשפה הכי גבוהה (בניגוד לאנגלית היום) ומרבית ספרי הלימוד וחומרי הלימוד היו בגרמנית. גם לא היו מושגים בהנגסה בעברית. בעקבות ההחלטה הזו, קבוצה לא קטנה של אנשים (בינהם בטודנטים ובן־יהודה) פתחו במחאה ששמה "מלחמת השפות" (כנד הלימוד בטכניקום בגרמנית). לטענתם, אם אין מילים, נמציא מילים, אך בישראל מלמדים בעברית בלבד. המאבק הצליח – בטכניון מלמדים בעברית. 
	
	
	\npchapter{נאציזם, טוטאליטריות ושואה}
	
	\section{המשטר הטוטאליטרי}
	\subsection{מבוא}
	אחד מבחירי המסטרהמשטר הנאצי, אמר במשפטו "אפשר תמיד לגרום לאנשים ללכת אחרי המנהיגים. זה קל. כל מה שאתה צריך להגיד להם זה שהם מותקפים, ולהאשים את הפצפיסטים בחוסר פטריותיות ובחפישת המדינה לסכנה". 
	
	המשטר הטוטאליטרי עובד על הרגש. מהמילה total. שליטה מוחלטת (טוטאלית) בכל תחומי החיים. בפרט, המחשבות. 
	למשטר הטוטאליטרי לא הייתה אפשרות להתקיים לפני המאה ה־20. אלו משטרים מודרנים. אומנם, היו משטרים בעבר כמו מלוכה ונסיכויות, אך המשטרים הטוטאליטריים משתמשים בכלים מודרנים (כמו תקשורת מונים) כדי להגייש את העם. גיוס העם נעשה בכל מני דרכים: 
	\begin{multicols}{2}
		\begin{itemize}
			\item תעמולה
			\item חינוך
			\item פחד
		\end{itemize}
	\end{multicols}
	תמיכה מפחד היא תמיכה, וגם אם לא מרצון. אם נחוקק חוקים שמנטרלים גורמים פנימיים, נגייס גם כך את תמיכת האנשים. \textbf{המשטר הטוטאליטרי מגייש תמיכה בכל תחומי החיים, בכל הדרכים האפשרויות}. 
	
	בספר "1984" מתואר עולם עם כרזות המתנוססות בכל, ופנים של אדם בעל שפם שחור הבאות מכל מקום. תיאור של משטרה חשאית וכו'. "רק משטרת המחשבות יש לה חשיבות". במשטרים טוטאליטריים הכל היה כל כך חזק שאנשים פחדו לחשוב כנגד המשטר; שמא לא פילט מתוך שינה, לדוגמה. הן במשטר הקומוניסטי והן בנאצי (וחלקית באיטליה) ילדים הלשינו על הוריהם על מנת להכנס לתנועות נוער ולקבל הכרה. אנשים הלשינו על בני משפחותיהם. 
	
	\subsection{הגדרת המשטר הטוטאליטרי}
	
	קיימת הגדרה עם ביקורת מרובה, לפיה אלו מאפייני המשטר הטוטאליטרי: "קיומה של \textbf{אידיאולוגיה אחת}, המתייחסת לכל היבטי החיים ומחייבת את כל בני החברה להזדהות איתה. קיומה של \textbf{מפלגת המונים אחת שבראשה מנהיג}. ראשיה מהווים עלית סגורה המצייתת למנהיג והיא עליונה על מוסגות השלטון או משולבת בהם. קימו של \textbf{פיקוח על האזקחים} באמצעות \textbf{משטרה חשאית} המפילה אימתה על הכל. למנהיג יש שלירה בלתי מוגבלת על \textbf{אמצעי התקשורת}  ועל \textbf{אמצעי הלוחמה}. השלטון \textbf{מכוון את הכלכלה לצרכיו}". 
	
	שרית שמה בכוכבית על שליטה באמצעי הלוחמה – במרבית המדינות הדמוקרטיות היום השלטון שולט על אמצעי הלוחמה. ההגדרה הזו אומנם נכונה, אך נכתבה על סמך משטר מסויים – ברית המועצות. לדוגמה, הנושא של הכלכלה לא קיצוני כמו בברית המועצות. להפך, בגרמניה הנאצית היה שימוש באנשי מפתח בכלכלה. 
	
	
	\section{האידיאולוגיה הנאצית}
	\subsection{אודות אידיאולוגיות}
	איגיאולוגיה – משמעותו איך העולם צריך להיראות ואיך צריך להגיע לשם; מה האידיאל. נתייחס לאידיאולוגיה כמו אל משקפיים, או כמו מסננת. נאמר, ואני אנטישמי (זו אכן אידיאולוגיה) אז כל בעיה אייחס ליהודים – הפסדנו במלחמה, זה היהודים. נפל לי משהו על הרגל, זה היהודי הקרוב. כל ההסתכלות על העולם היא לפי האידיאולוגיה. עיתים, הם מאוד טובים, ועיתים, להפך. אידיאולוגיה לא מספרת רק על מה רע – אלא מה האידיאל ואיך להגיע אליו. 
	
	\textit{נגדיר: }\textbf{מערכת מורכזת של אמונות, דעות, ערכים על האדם ושביבתתו שמנסה להסביר את המציאות שאנו חיים בה ומציעה דרכים לתיקון המציאות הקיימת. }
	
	אנחנו חיים היום בעולם פחות אידיאולוגי. 
	
	דוגמאות: 
	\begin{multicols}{2}
		\begin{itemize}
			\item נאציזם
			\item קומוניזם
			\item פאשיזם
			\item סוציאליזם
			\item לאומיות
			\item אנטישמיות
			\item טבעונות
		\end{itemize}
	\end{multicols}
	
	
	\subsection{עקרונות האידיאולוגיה הנאצית}
	\begin{multicols}{2}
		\begin{itemize}
			\item תורת הגזע
			\item מרחב המחייה
			\item אנטישמיות מודרנית
			\item המנהיג "פיהרר פרנציפ"
			\item הלאום/האומה כערך עליון
			\item שלילת הדמוקרטיה והקומוניזם
		\end{itemize}
	\end{multicols}
	
	בניגוד למה שכתוב בספר הלימוד, הסדר החדש הוא לא עקרון – אלא רעיון – ומרחב המחייה נכלל בו. 
	
	\subsection{תורת הגזע}
	צפינו בדוגמה מהארי פוטר שעקרון "תורת הגזע" נוכח בו. הנאצים לא המציאו את תורת הגזע, אך הייתה להם גישה משלהם. הגזע התחלק לשלושה סוגים: 
	\begin{itemize}
		\item הגזע הארי – הגזע העליון, יוצרי התקבות. גם בו קיימת חלוקה פנימית, לגרמנים ואוסטרים, לעומת ארים ממדינות אחרות (נורדים). 
		\item נושאי התרבות, הסלאבים – גזע שמסוגל לקיים את התרבות אבל לא לייצר אותה בעצמו. 
		\item בתחתית הפרדמיה, הגזע הנחות שסופו להיכחד – צוענים, אנטי־סוציאליסטים, נכים וכו'. 
	\end{itemize}
	היהודים לא נמצאים הפירמידה הזו – הם אנטי־גזע, לא בני אדם. 
	
	מה שמשותף לסלאביים הוא השפה. השפות הסלאביות דומות; רוסית, פולנית, אוקראינית וכו'. רוב דוברי השפות יוכלו להבין אחד את השני. גם כן השפות השמיות (ערבית ועברית לדוגמה) דומות במיוחד. 
	
	ע"פ האידיאולוגיה הנאצית, כתוצאה מתורתו של דארווין, בני בני האדם מתקיימת מלחמת קיום כמו בטבע. בני האדם מחולקים לגזעים שונים אשר אינם שווים בכוחם וביניהם מתנהל מאבק שמטרתו לשרוד המלחמת הקיום. על כן, יש לשמור על הגזע הארי הטוב יותר ולשמור על טוהר גזעו. הגזע הארי בעל שלמות פיזית ורוחנית, הוא יצר את התרבות, וחובה עליו להישאר טהור ולשלוט. הסלאביים נועדו לשרת את האריים. התפקיד של הצוענים וכו' – לא להיות. גם יהודים, צריך להשמיד. 
	
	הנאצים הפכו את תורת הגזע למדע. עד היום יש אנשים שמאמינים בזבל הזה. לכן, הנאצים המציאו קריטריונים ומכשריים לקביעת הגזע – מכשירים למדידת רוחב גשר האף, מכשיר למדידת רוחב המצח, ועוד. למדו אותם מגיל צעיר. התפישה שמדברת על היהודים, טענה שהיהודים חדרו לחברה, מנסים להשתלב ולהשתלט עליה, ולטמא את הדם הארי. לכן, צריך אמצעיים מדעיים בשביל לזהות את היהודים שדומים לגזע הארי ונראים אותו הדבר. על כן מגיל צעיר במיוחד למדו לזהות את הגזעים השונים. 
	
		
	\subsubsection{דוגמאות למימוש תורת הגזע}
	
	ישנו מושג שנקרא T4 (הבניין שהתרחב בו) – מבצע לרצח אנשים "אסצויאליים" – אריים, גרמנים, עם מוגבלויות נפשיות ופיזיות. האנשים האלו הושמדו בתאי גזים. זו הייתה אחת הפעמים היחידות שבה מחאה מלמטה השיגה משהו. אלפי ילדים ומבוגרים אריים נשרפו ונהגרו בתאי גזים. אותם האנשים שתכננו את המבצע, לימים, ממשו את הפתרון הסופי. לרעיון הזה קוראים אוטנזיה. 
	
	מצע לבנסבורן היא תוכנית של הרבאה של הגזע הנאצי. בחורות צעירות בעלות מראה ארי (לעיתים לא באמת טהורות) הורבו עם חיילי SS וילדיהם נתנו למשפחות אריות טהורות. גם ילדים מחטפו לשם המטרה הזו. באופן דומה, היום מבוצעות הרבאות בין כלבים ליצירת גזע טהור של כלבים. 
	
	הארים לא רק בני אדם. הם צריכים להיות מושלמים. האסוציאלים פוגעים בטיהור. ראשית כל, טיהרו את הגזע הארי – האותונזיה קדמה להשמדת היהודים. 
	
	\subsection{אנטישמיות}
	האנטישמיות הנאצית היא אנטישמיות גזענית, אך מהווה עקרון בפני עצמו. תורת הגזע והנאטישמיות הינם שני דברים שונים – טוהר האדם לאו דווקא קשור ליהודים. ליהודים יש דיומיים של חיות – עבר, טפיל, עכברוש וכו'. האידיאולוגיה אנצית עושה להיהודים דה־הומניזציה. ילד בגיל 10 ב־1933 כשהנאצים עלו לשלטון, יהיה בן 16 עם פתיחת המלחמה ב־1939. לדידו, יהודים אינם בני אדם. לא כל הגרמנים האמינו בלב שלם באידיאולוגיה הנאצית; ילדים יותר, אך גם זה תלוי בבית. אף קיימים גרמנים מעטים שהיו לחסידי אומות עולם. היהודים המציאו את הליברליזם והקומוניזם לטענת הנאצים (מארקס היה יהודי לדוגמה) ומטרתם להחריב את האנושות. 
	
	להלן בעית מתמטיקה מספק ילדים משנת 1935: 
	"מטוס סטוקה שעומד להמריא, נושא 12 תריסרים של פצצות, שכל אחת שוקלת 10 ק"ג. המטוס ממריא לוורשה ,המרכז את היהדות הבין לאומית. הוא מפציץ את העיר. בשעת ההמראה, כשמיכל הדלק שלו מכיל 1000 ק"ג דלק, שוקל המטוס 8 טונות. עם שובו ממסע הצלב נותאו עדיין במיכלו 230 ק"ג דלק. מה משקלו של המטוס כאשר הוא ריק?". 
	
	בעיה שכזו היא תעמולה, שטיפת מוח, שמנרמלת את הרעיון של להפציץ את מרכז היהודות הבין לאומית – זה מסע צלב, אין בעיה בכך. זה היה שיעור מתמטיקה – לא שיעור על תורת הגזע – ובו גם החדירו את הרעיון של קנוניה יהודית בין לאומית. 
	
	מיין־קאמפף הופך להיות תורה בגרמניה הנאצים. דוגמה לאשר נכתב בו: "שעות ארוכות אורב היהודי בעל השיעור השחור, לנערה שאינה חודשת לכל רע, והוא מבצע בה מעשה מכונה, כשפניו קורנות תאווה שטנית והי, הנעשה, נשארת תמאה לכל ימי חייאה, ובכך הצליח היהודי לגזה מחיק עימה". הוא ממשיך ומתאר כיצד היהודי מנסה לנוון את היסודות הגזעיים של העמים הטאורים, ולהביא לעולם ייצואים שהם "תערובת בין אדם וקוף". 
	
	\subsubsection{דוגמאות למימוש האנטישמיות}
	נביא דוגמאות מספר בשם "הפטריה המורעלת" – סיפור ילדים שכתב יוליס שטרייכר, עורך העיתון האנטישמי "דר שטאימר". בתרבות הגרמנית יוצאים ליער וקוטפים פטריות, ולכן לומדים לזהות פטריה מורעלת כחלק מתרבותם. בספר "הפטריה המורעלת" מוצגות תמונות של היהודים כסוחר רמאי, בעל אף בולט, וכו'. הסיפור סופר "לפני השינה". נתבונן בתמונה  אחרת שמתאתר את משפחות אוטשילד, עם כתר זהב על הראש, חובקת את העולם בטפרים. תמונה אחרת של יהודי שיושב על כסף, עכביש עם מגן דוד שבכוריו נתלים גופות, ועוד. הקריקטורות האלו מציירות תחושה של גועל. שבוע הבא רואים "הדרדסים". 
	
	
	\section{גרמניה בשנים 1933-1939}
	
	עת סיפור: איך יכול להיות ש...
	
	נקרא חלק מסיפור בשם "מען לא ידוע" (הספר עצמו 60 עמודים). הספר פורסם ב־1938, ומספר על זוג חברים עם גלריית אומנות בארה"ב. שניהם גרמנים, אחד יהודי והשני אינו. לקראת סוף 1932 הבחור הלא יהודי חוזר לגרמניה עם משפחתו, והם מנהלים חלופת מכתבים ביניהם לעניינים עסקיים וכן דברים אישיים (הם היו חברים). לבחור הגרמני קראו מרטין וליהודי מקס. 
	
	הספר מתאר תאריך של שנה וחודשיים שבו מרטין הליברל הופך לפשיסט נאצי. המשטר הנאצי הופך מדמוקרטיה למשטר טוטאליטרי ב"מהירות האור". כבר חודשיים מעליית הנאצים לשלטון גרמניה לא דמוקרטית יותר. 
	
	משבוע הבא ננסה לראות איך ה קורה במקביל על שלושה צירים: שני צירים במיקוד לבחינה, וציר אחר שאומנם לא נבחן עליו אך בלעדיו נתקשה להבין מה קורה באמת. הצירים יתחילו ב־31.1.1933 ויסתיימו באוגוסט 39. הצירים יסוכמו את המחברת (אני לא עומד לעשות ציר זמן בלאטך) אז אם תרצו משהו מהם פנו אלי בפרטי. נעבוד עם צירים שנקבל. אז אתם יודעים מה אולי אם אנחנו מקבלים אותם לא באמת צריך שעבור למחברת. שלושת הצירים יהיו: 
	\begin{enumerate}
		\item מדיניות פנים
		\item היחס ליהודים
		\item מדיניות חוץ
	\end{enumerate}
	חשוב להדגיש שהיחס ליהודים הוא מדיניות פנים בפני עצמה, אך נעשה הפרדה מלאכותית ע"מ לראות את הקשרים בין הדברים. ציר טוב למדי (אך לא מופרד) מופיע בעמוד התוכן הראשון של הספר ("נאציזם, מלחמה ושואה"). 
	
	שרית תלמד כרונולוגית, והספר מחולק לפי נושאים (חקיקה, תעמולה וטרור). בספר, הנושא הזה בעמודים 91-110. יש להכיר את הספר ולדעת שהוא בנוי אחרת. 
	
	נתבונן בקטע "לא הרמתי את קולי", שכתב מרטין נימלר, פרוטסטנטי גרמני. ניתן למצואו באינטרנט. נבחין מהטקסט בין היתר שהצעדים בוצעו בצורה הדרגתית (אומנם מהירה, אך לא בבת אחת). זוהי אחת הסיבות שב־36 עדיין חצי מיהודי גרמניה עדו היו שם. בפרק זה נדבר על המעבר של גרמניה ממדינה דמוקרטית לטוטאליטרית. 
	נקבל את הצירים (המתוארים ב־subsections הבאים) על דף מודפס. 
	
	\subsection{מדיניות פנים}
	ב־30.1.1933, היטלר מונה (לא נבחר) לקנצלר. מאוד מהר, באוגוסט 34, גרמניה הפכה לטוטליטרית מוחלטת. המפלגה בנתה מצב בו אין דבר כזה מנגנוני המדינה – הכל שייך וכפוך למפלגה הנאצית. התפיסה: \textbf{עם אחד, מדינה אחת, מנהיג אחד. }לתהליך הזה קורים נאציפיקציה, ולהלן מאפייניה: 
	\begin{itemize}
		\item ריכוז סמכויות לביצוע חקיקה ושיפוט בידי הפיהרר. 
		\item ביטול המסגרות האוטונומיות במדינה כגון איגודים, תנועות נוער, ארגוני תרבות, ספורט וכו'. 
		\item השתלטות על מערכת החינוך. 
	\end{itemize}
	
	לשם כך היטלר נוקט בשורה של צעדים, ונתבונן בהם כרונולוגית. נבדיל בין חקיקה, טרור וחינוך. 
	
	\subsubsection{27.2.1933 – שריפת הרייכסטאג}
	שבוע לפני הבחירות הוצת הרייכסטאג, בניין הפרלמנט. (בגרמניה הדמוקרטית לנשיא ולראש הממשלה שניהם היו סמכויות, אחרות). בבחירות לפני כן לא היה רוב למפלגה הנאצית והילטר רצה רוב. בדיעבד אנחנו יודעים שאלו היו הנאצים, אך באותה התקופה הואשם בחור קומוניסט, והואשמו כל הקומוניסטים בנסיון הפיכה שלטונית. הילטר משכנע/לוחץ/מאלץ את הידרבורג הנשיא לחתום על צווים לשעת חירום. יוצא צו חירום שנקרא "הצו למען הגנת העם והמידנה" (המדינה = רפובליקת ויימאר). תחת אותו הצו נערכו חיפושים בבתים, הוחרם רכוש, נשלחו מתנגדים של היטלר ללא מאסר, הוטלה צנזורה על העיתונים ועוד. 
	
	בכך היטלר הצליח ביום הבחיריות להרחיק מברלין את היריבים הפוליטיים שלו ולהחליש את המפלגה הקומוניסטית. למפלגה הקומוניסטית באותה התקופה הייתה אחיזה לא מבוטלת בגרמניה. המפלגה הנאצית לא קיבלה את הרוב שהיטלר רצה. 
	
	מחנות הריכוז הראשונים הוקמו באותו הזמן. מטרתם המוצהרת הייתה חינוך מחדש. הם נועדו לשן בהם מתנגדי שלטון, בוגדים ומורדים. בגרמניה לא היו מחנות השמדה, לאורך כל ההיסטוריה (כן היו עבודות כפויות, רעב וכו' שגרמו למוות רב). היו רק 6 מחנות השמדה, בפולין בלבד, פירוטים בהמשך. המחנות כרגע עליהם מדברים היו חנות חינוך ועבודה. אנשים נכנסו למחנות הללו בלי משפט. מאוחר יותר, נכנסו לשם גם להט"בים (היו שם הרבה מאלו). 
	
	\subsubsection{23.3.1933 – חוק ההסמכה}
	החוק אישר לממשלה לחוקק חוקים למשך 4 שנים בלי אישור הרייכסטאג הגרמני (הרשות המחוקקת). היטלר יכל לחוקק חוקים ללא הפרלמנט, גם אם היו מנוגדים לחוקה. אישור חוזים עם מדינות ושינויים בחוקה הועברו לממשלת הרייך. כבר כאן אין יותר ביטול רשויות. התהליכים הארוכים של חקיקה נמחקו, ועכשיו יש לנו חוקי בזק. 
	
	החוק שימש כבסיס חוקי לשלון הטוטאליטרי. עקרון שלטון (ע"פ העקרון כולם כפופים לחוק באותה מידה, והחוק הוא עליון) החוק פהסיק להתקיים. המנהיג בלבד קבע את החור והחוק שיקף את רצונו. אין הפרדת רשויות. 
	
	היה צורך במעטף הזה של החוק בשביל להראות פנימה והחוצה שיש לך חוק. 
	
	\subsubsection{2.5.1933 -- ביטול האיגודים המקצועיים}
	האיגודים המקצועיים בוטלו ובמקומם קמה "חזית העבודה הגרמנית". הסתדרות המורים, העובדים וכו' הם איגודים מקצועיים. אופן פעולתם: שביתות, הפגנות וכו'. איגוד מקצועי הוא כוח. 
	
	
	
	\section{מדיניות היהודים}
	נתבונן בסרטון "המדיניות הנאצית כלפי יהודי גרמניה 1933-1939" (ביוטיוב). נכתוב מה מטרותיהם, מהם האירועים, ולאיזה תחום הם משתייכים. 
	\subsection{המטרות}
	\begin{itemize}
		\item בידוד חברתי
		\item נישול כלכלי
		\item עידוד הגירה
	\end{itemize}
	לזכור בנ"ה. לא כתוב בצורה מפורשת בספר, צריך לזכור. הדרגים להשיג מטרות אלו – באמצעות חקיקה, תעמולה וטרור. 
	\subsection{הפעולות}
	\begin{itemize}
		\item חרם כלכלי (\textit{טרור} ו\textit{תעמולה}). נועד "לצמצם את השפעתם המזיקה של היהודים". \textbf{יום החרם}, מניעת הקונים להכנס לעסקי היהודים במשך יום אחר. \textit{המטרה: בידוד חברתי ונישול כלכלי}. 
		\item \textit{חקיקה} – הפרדת היהודים מהשירות הציבורי. איינשטין סולק. רכושו הוחרם. \textbf{חוק הפקידות}. \textit{המטרה: נישול כלכלי ובידוד חברתי}. 
		\item שריפת שפרים, בפרט ספרים יהודיים – \textit{תעמולה}. "במקום שבו שורפים ספרים שם ישרפו גם בני אדם" – ההימלר כתב 100 שנה לפי כן. 
		\item חוקי נירנברג (1935) – הגדרת יהודיים, הגדרת והפרדתם מהחברה. \textit{חקיקה}. \textit{המטרה: בידוד חברתי
		}
		\item \textit{חקיקה} שמכריחה תעודות זיהוי (טאג). 
		\item אריזציה – העברת נכסי יהודים לארים, וסילוקם מהמשכר (\textit{חקירה}). 
		\item ליל הבדולח – \textit{טרור} – כונה כך בשם ניפוץ זגוגית מוסדי המוגרים, העבודה, וכו'. 
	\end{itemize}
	600K יהודי שלפי החוקה גרמנים לכל דבר. הייתה אמנציפציה, והם ראו עצמם כגרמנים, אך לא ויתרו על זהותם היהודית. כ־30K מהידוי גרמנהי נשלחו למחנות ריכוז, הטיפול במוסדות היהודיים הועבר לידי ה־SS. ב־1 בספטמבר מלחמת העולם השנייה פרצה, והעל נהיה יותר גרוע. 
	
	חשוב להבין באופן כללי מה קורה, ולהיות מסוגל יותר להבין מלקרוא את הספר. אף אחד לא ישראל שאלה שאפשר להעתיק מהספר בשביל לענות עליה. 
	
	\subsection{פירוט האירועים נגד היהודים}
	\subsubsection{יום החרם – 1 באפריל 1933}
	נועד להיות חרם כולל, אך בעקבות מחאה פנימית ובין לאומית שנבע מההבנה שהצעד יכול לפגוע כלכלית בגרמניה (שכלכלתה הייתה ממילא חלשה) היטלר נוכח את מגבלות כוחו והפך זאת ליום אחד. בנושא בידוד, החרם גרם להבחנה בין עסקים יהודים לכאלו שאינם. חלק מהעסקים נפגעו כלכלית. עדיין לא הגיעו למצב שבו הגרמנים לא נכנסו לעסקים יהודים כלל, וזאת לא בוסס בחקיקה. ה־SS נועדו להזהיר את האוכלוסייה באמצעות העמדת משמרות, ובאמצעות כרוזים בעיתונים. האירוע היה פתאומי וללא הזהרה מראש. 
	\subsubsection{חקיקה אנטי יהודית}
	מאוד מאסיבית. היו עשרות ואף מאות חוקים, לא כולם רק ליהודים. כמעט כל יום יצא חוק. 
	\begin{itemize}
		\item "החוק לשירום הפקידות על כנה" (\textit{חוק הפקידות}) – מונע מפקידים לא ארים (ובפרט, יהודים, אך לא רק) אפשר להיטלר להעסקים גרמנים במקום. 
		\item איסור שחיטה כשרה (שחיטה שמנסה להמנע מסבל של חיה, לא בעיני הגויים) – מטרתו לא היה עידוד הגירה, אלא נועד למטרות תעמולה. נועד להציג את היהודים ככאלו שמתאכזרים לבעלי חיים (על אף ששחיטה כשרה היא ההפך הגמור). 
		\item הבלת יהודים במוסדות חינוך ופיטורים של המורים היהודי םוהאנטי־נאציים – קורה בהתחלה, על אף שהילדים היהודים יכלו (במקרים רבים לא רצו שכן לא היו רצויים) ללמוד בבית הספר
	\end{itemize}
	
	\subsubsection{שריפת הספרים}
	שריפה של כל ספר שנורצ ע"י יהודים או שמציג אידיאולוגיה שמנוגדת לאידיאולוגיה הנאצית. נושרים גם ספרים של איינשטין. המטרה: לקבוע מה חלק מהתרבות ומה לא חלק מהתרבות. מה חלק מהמדע ומה לא. לא רק בהקשר היהודי – ג'אז לא שמעו, מוזיקה של שחורים (אפריקאים). הייתה מחתרת של אנשי סווינג שרקדו במחתרת, שבה מי שנעצק נשלח למחנות ריכוז. 
	\subsubsection{חוקי נירנברג}
	ב־1935. סדרה של חוקים, מהם שניים מעניינים אותנו: 
	\begin{itemize}
		\item \textbf{חוק אזרחות הרייך} – ארזחות ניתנה בלבד לארים הנאנים לרייך, ובפרט לא ליהודים (רק לא לבד). 
		\item \textbf{החוק להגנת הדם והכבוד הגרמני}
	\end{itemize}
	
	\section{מלחמת העולם השנייה}
	
	
	למדנו על גרמניה ב־1933-1938. מלהע''ר התחילה ב־ לספטמבר 1939 (מה שהיה ב־39 עד אז לא בחומר הלימוד) ואת הקטע הזה, עד 41 ולפני הפתרון הסופי – נלמד עתה. את  הקטע שכולל את הפתרון הסופי כנראה לא נספיק ללמוד השנה. ביוני 41 גרמניה פלשה לבריה''מ ואירועי המלחמה השתנו באופן משמעותי. 
	
	\subsection{הסכם ריבנטרופ־מולוטוב}
	ב־23.8.1939 חתמו שני צרכי הרחוץ, ריבנטרופ הגרמני ומולוטוב הרוסי על הסכם בעל שני צדדים: 
	\begin{itemize}
		\item \textbf{החלק הגלוי} – הסכם אי התקפה, שיתופי פעולה מודיעינים ולכלכליים. 
		\item \textbf{חלק סמוי} – כלל את חלוקת פולין כאשר גרמניה תפלוש לפולין. לכל אחד מהצדדים היו אינטרסים בפולין, וההסכם חילק באופן כללי את מרכז/מערב פולין לגרמניה, וחלקה המזרחי ומדינות נוספות – לרוסים. מי שפלש לפולין במציאות  – זה גרמניה, אך ברית המועצות לקחה חלק בלחימה כחלק מההסכם הסודי. 
	\end{itemize}
	\textbf{הבעיות בהסכם: }לא יכול להיות שבתפישה הנאצית, האנטי־קומוניסטית, זה נוגד את (1) האנטי־קומוניזם באידיאולוגיה הנאצית ו־(2) את מרחב המחייה של הגזע הארי בברית המועצות. 
	
	כל הסכם שהיטלר חתם – נועד לאינטרסים של גרמניה, בידיעה ברורה של היטלר שההסכם יופר. במקרה הזה היה ברור לשני הצדדים שההסכם יורפ, כי אידיאולוגית גם הקומוניזם מנוגד לנאציזם הפשיסטי. ``פועלי כל העולם, התאחדו'' – ללא תלות בלאום, אנטי־פשיזם. היטלר לא מעוניין להלחם בשתי חזיתות, הצבא הרוסי חזק ולחימה מול הצבא האדום לא התאימה להיטלר. 
	
	\subsubsection{אינטרסים של היטלר: }
	\begin{itemize}
		\item לא רצה להלחם בשתי חזיתות
		\item הפרדה בין בריב''מ למערב – באותה התקופה הקומוניזם נתפש כאיו םיותר גדול מאשר הנציזם
		\item באמצעות ההסכם הוא יכול לממש את עקרון ``מרחב מהחייה''. איך? הוא יפר אותו. בהתחלה הוא יפלוש לפולין ובהמשך הוא יפר אותו. 
	\end{itemize}
	
	\subsubsection{האינטרסים של סטלין: }
	\begin{itemize}
		\item באותה התקופה טיהורים בראש הצבא, כלומר מחסל את מתנגדיו. בכך סטלין החליש את הצבא שלו, וזה לא היה טיימינג טוב מבחינתו להלחחם מול גרמניה.
		\item סטלין יודע שהיטלר יצא מחבר הלאומים ויכול להגדיל את צבאו כרצונו. 
		\item סטלין לא מחבב את מדיניות הפיוס של מדינות המערב – מדינות המערב חשבו שבמאצעות הסכים וויתורים, הם יוכלו למנוע פריצת מלחמה. 
		\item ברית המועצות מרוויחה חלק מפולין
	\end{itemize}
	
	ראה הסכם מינכן (שלא בחומר), הסכם בין מדינות המערב (בריטניה, צרפת ואיטליה) מול גרמניה שעל פיו מדינות המערב נותנות להיטלר את הסודתים – איזור בצ'כוסלובקיה שאוכלוסיתו גרמנית ברובה. הבטחתו של היטלר שזו תהיה בקשתו האחרונה, ורא שממשלת בריטניה עמד על הבמה ואומר ``זה שלום בדורנו, השגנו את השלום המיוחל''. אחרי כמה חודשים היטלר כבש את כל צ'כוסלובקיה, המדינה האחרונה שנשארה דמוקרטית מהסכמי וורסאי. זו היתה מדינה חזקה עם צבא חזק שהיה יכול להתמודד מול גרמניה, אבל מדינות המערב מכרו אותה להיטלר. סטלין היה מאוכז מאוד מההתנהגות הזו. 
	
	23 באוגוסט – שבוא אחרי, ה־1 בספטמבר וגרמניה פולשת לפולין. 
	
	\subsection{הפלישה לפולין}
	אז איך גרמניה פולשת לפולין? הרי היא חתומה איתה על הסכם אי־התקפה. איך עשו זאת? ביימו תקרית על הגבול שבעקבותיה נטען שפולין תקפה את גרמניה, ועל כן גרמניה יכולה לפלוש לפולין כדי להגן על עצמה. בכך גרמניה פולשת לפולין. הצבא הפולני היה צבא שחלקיו היו רובים על חניתות (לא כולם, אך חלק ניכר). זה היה צבא חלש ולא מפתוח שבטח שלא יכל להתמודד עם המטוסים הגרמניים. 
	
	עתה נתבונן בסרטון, ונבחין: מי הצדדים הלוחמים, מה זה בליצקריג, ואיך זה בא לידי ביטוי בהמלע''ר. 
	\begin{itemize}
		\item הגרמנים חילקו את פולין לשניים – מערב פולין, מזרח פולין ומרכז פולין. את מרכז פולין ניהל ממשל אזרחי גרמני, הגנרלגובנמן. מערב המדינה סופח לרייך. חלקה המזרחי סופח לברה''מ. 
		\item פולין מפסיקה להתקיים שבועיים אחרי פרוץ המלחמה. בניגוד למדינות אחרות שממשיכות להתקיים, בין אם כמדינות ואפילו חלקן נשארות בשלטון עם מדיניות פנים. \textbf{פולין לא קיימת יותר}. ממשלתה גולה ונמצאת בבריטניה. 
		\item על אוכלוסיית פולין הוטל משטר קשה, ועבור היהודים אלימות חזקה יותר. 
		\item עם פלישת גרמניה לפולין הוכרז מלחמה ע''י צרפת ובריטניה, אך בהתחלה הן לא פעלו ומלחמו כמעט. 
		\item תוך חודש נכבשה פולין במה שכונה ``מלחמת הבזק'' – הבליצקריג, לפיה התנהלו השנתיים הראשונות של הלחימה. צבא פולין עמד לבדו מול הגרמנים. מה שאפשר להם להתחיל את המלחמה הוא הסכם רינטרוב־מולוטוב. 
	\end{itemize}
	המדינות שלחמו בשנים הללו: \textbf{מדינות הציר: }גרמניה, איטליה ויפן. מדינות כמו הונגריה, רומניה וכו' שיתפו פעולה אך לא באמת היוו חלק ממדינות הציר. \textbf{בנות הברית: }ברה''מ, ארה''ב ובריטניה. ארה''ב נכנסה בפרהל הרבור, צרפת לא חלק מבנות הברית כי היא נכבשה במהרה. 
	
	\textbf{המשלך רוב המלחמה, עד 1941, היחידה שנלחמה בגרמניה היא בריטניה. }היא המדינה היחידה שנלמחמה מתחילתה ועד סופה היא בריטניה. חלקן הצטרפו יותר מאוחר, חלק יצאו וחלק נכנסו, בריטניה היחידה שעברה את כל המלחמה. יש לכך משמעויות כלכליות על גריטניה במהשך. 
	
	חלוקת פולין תלווה אותנו עד סיום מלחמה. יש לחלוקה הזו חשיבות. דנציג – אזור המריבה/ויכוח של גרמניה, יש בו מוצא לים והרבה דברים שנאלצו לוותר עליהם בהסכמי ורסאי. 
	
	אין שום גנרל שקוראים לו גובנמאן. קוראים לזה הגנראל־גוברנאמן, מהמילה general, איזור הממשל הכללי. כל מה שנדבר על פולין ־ מחנות השמדה, כטאות וכו' – בגנראלגוברנאמן. אין מחנות השמדה בגרמניה, יש מחנות ריכוז ועבודה אך לא השמדה. המחנות בגנראלגובנאמן  –טרנוב, קרקוב, ובלין, ורשה ועוד. יש כטאות גם בחלק של ברית המועצמות, אך אלו גטאות מסוג אחר. יש גטו אחד חריג, גטו לודז', הקשה ביותר עם התנאים הקשים ביותר, שלא נמצא ובגנרלגובננאמן. 
	
	היטלר היה בטוח שהצרפתי םוהאנגלים לא ינקטו כל פעולה נגד פולין, אך הן הציבו לא אולטימטום. לפי היטלר, אוביו תולעים קטנות, ומי ירצה להסתבך מלחמת עולם? במהרה היתה הצהרת מלחמה. לפי עדים, היטלר נראה כאילו התאבד. הוא חשש מאוד ממלחמה בשתי חזיתות, אך הפור נפל והכוחות המזויינים של גרמניה הנאצית נשלחו לתוך פולין. 
	
	
	
	\subsection{כיבושי גרמניה הנוספים}
	
	גרמניה כבשה של המדינות הבלטיות ופחות או יותר כל מדינה באירופה, מה שאפשר לה בסוף לכבוש את צרפת וזהו זה. 
	
	תחילת הכיבוש, מאביב 1940: המדינות האריות. בפרט, דנמרק, נרווגיה, בלגיה והולנד. \textit{היחס לא כמו בפולין}. ביוני 1940 נכבשה צרפת. באוגוסט 1940 החל \textbf{הקרב על בריטניה} – המדינה הראשונה שגרמניה לא מצליחה להביס. הכשון הצבאי הראשון של גרמניה במלחמה. 
	
	בחורף 1940-1941 – היטרל אילץ את רומניה, הונגריה, בולגריה וסלובקיה להפוך למדינות חסות. המשמעות: המדינה נשארת קיימת, אך השלטון הוא שלטון נאצי פשיסטי. 
	
	יש הבדל בין מה שקרה בין בולגריה לשאר המדינות ליהודים. זו המדינה היחידה מבין המדינות לעיל שהיהודים בה לא נרצחו. הממשל בה \textit{הגן} על היהודים. זה לא משהו ידוע, מדברים בעיקר על דנמרק כעל מדינה שהצילו את היהודים שלה. לעומת הולנד, שיש לה תדמית של חסידי אומות עולם וכו' – רובם נרצחו בשואה. ``זוכרים את אנה פרנק בעליית הגג אבל שוכחים שיש מי שהלשין עליהם''. 
	
	בפריז, החתנתה הרכבת, אוגוסט 1914 – צעדו אל המלחמה בפרחים, אך לא כן במלחמת העולם השנייה. אף אחד לא רוצה להלחם. 4 מיליון גברים, ספרט כוכבי קולנוע, מתכייסים לצבא, מרביתם חלקאים. היה מחסור גדול בציוד, רובה אחד לכל שני אנשים. ``הייתה קופסא אחת עם 10 כדורים והיה אסור לנו לפתוחאותה''. בצבא הצרפתי התבססס על סוסים. כינויי גנאי רבים היו לגרמניים. 
	
	\subsubsection{סיכום שלי מהסרטים}
	
	חצי מיליון צרפתים מול 200 אלף גרמנים. 4 ימים אחרי הצהרת המלחמה של צרפת. ההתקפה שהיתה אמורה להראות שפולין לא הופקרה התקדמה 9 ק''מ. הצרפתים הציגו בגאווה את השלל המלחמתי, אופניים (סרטון של חיילים על אופניים בסרטון תעמולה צרפתי). ``החייל הצרפתי הראשון המעלה'' הפך למשתף פעולה הדוק על הנאצים והוצא הלורג בסו ףהמלחמה. הצרפתים חשבו שבגלל יתרונם המספרי ינצחו את הגרמנים בקלות. אף אחד לא רצה להלחם את המלחמה של 1914, וההנחה של צרפת הייתה כי יש להתבצר מאחורי קו מאג'ינו, באורך מליון וחצי מטרים מועקבים של בטון ו־16 שנים של בנייה, עם מבוך של מנהרות, ו־200 אלף איש איכלסו אותו. ההנחה – טנקים גרמנים לא יצליחו לעבור באיזור הצפוני לים, כי בלגיה התנגדה לכך (טרם נכבשה). קו ציקפריד – שרשרת הביצורים שהיטרל הקים מול מאג'ינו. הגרמנים לא תוקפים ומנסים להמנע מחזית שנייה. הצרפתים פינו את אלזס לורן כאמצעי זהירות. 
	
	[הבהרה לא קשורה ביחס לתגובה שקראתה בכיתה: סרט דוקומנטרי $\neq$ אמת. הוא ערוך, ואף אם הוא מורכז מסרטים ועדויות מאותה התקופה, יש לו עריכה וקריינות, ויש בחירה מה נכנס אליו ומה לא. יותר חשוב מזה, מישהו החליט מה לא להכניס לסרט, כי יש לו אג'נדה. תיאורטית, אם בסרט היה נערך ע''י האנצים, זה היה מוצג אחרת, כמו סרטי תדמית לגטאות]
	
	צנחים גרמים צנחו על אדמת הולנד וצעדו לתוך בלגיה. תרחיש ההטעיה של היטלר: כאילו הם עומדים לכבוש את צרפת דרך בלגיה, כמו ב־1914. גמלה המפקד העביר את עילית הצבא הצרפתי לתוך בלגיה. הם בלמו את התקדמות הגרמנים בבלגיה, אך היטלר שמח על כך – הוא הצליח להטעות את צרפת. ``יכולתי לבכות מרוב שמחה''. הוא נתן הוראה לא תקוף את הכוחות בבלגיה מהאוויר, אלא לתקוף מאוחר. היטלר תקף ממקומות עליהם לפי צבא צרפת ``אף טנק לא יוכל לעבור את ההרים הללו היוערים בצפיפות'', אך הדבר היחיד שעצר אותם הם פקקי התנועה שהם עצמם יצרו. הטנקים הגרמנים עברו מהר את העריות של צפון צרפת. הגרמנים מחצו כל התנגדות. מפקדי הצבא הצרפתי לא ידעו איפה האויב נמצא. תוך 8 ימים המפקדים כבר היו מוטשים כליל מבחינה פסיכולוגית. הגמנים הגיעו לתעלת למנש. גם חיילות אנגלים כותרו באיזור. ראש ממשלת צרפת מפטר את גמלה. הוא ממנה שני גיבורים ממלחמת העולם הראשונה.
	
	היטרל הניח לצבא צרפת לסגת בשקט, בבשיל לא לעצבן את בריטניה שכוחותיה נמצאו שם. צ'רצ'יל מצווה לקחת כל שלי שייט שמסוגל לשוט לחלץ את הלוחמים הנצורים, בינהם הגנרלים הבריטיים. מטוסי שטוקה הפילו ספינות רבות. אף על פי כן מאות אלפי חיילים חלצו, והצבא הבריטי ניצל. הבריטים נשלחים למרכזי הצטיידות מחדש, ובריטניה משבחת את המבצע. צ'רצ'יל טען ש''אין מנצחים מלחמות באמצעות נסיגה''. המבצע לעיל התרחש בדנקרק. ב־4 ביוני צ'רצ'יל נאם נאום לפיו לעולם לא יכנעו. אכזריותם של הנאצים הייתה כבר ברורה לבריטניה. הלוחמים הבריטים התירו מאחוריהם בדנקרק את כל ציודם, שגרמניה באותו היום (4 ביוני) לקחו. 
	
	\subsubsection{סיכום של שרית}
	המלחמה המדומה – בריטניה וצרפת מכריזות מלחמה בגלל פולין, אך בגלל תחושת העליונות והמחשבה שקו מאג'ינו (כמו הגדר של ביבי סביב עזה, או כמו ב־73' ביום כיפור עם קו מוצבים הקרוי קו בר־לב שחשבנו שהטנקים המצרים לא יצליחו לעבור אותו). טעות. 
	
	שיטת הלחימה: בליצקריג. הכנסת כל הכוחות, אוויר ים ויבשה, באש מאסיבית וכיבוש מהיר של אזורים. ``ברגע שזה מתחיל זה מאוד מאוד מהיר. לכבוש את פולין בשבועיים זה כלום''. אופן הלחימה הזה חדש, והתאפשר באמצעות התחזקותה לפני מלחמת העולם השנייה. גריטניה כורתת בריטות בין גרמניה לבין איטליה (ברית רומא־ברלין) ואח''כ רומא־ברלי־טקויו, שסייעו לגרמניה לנצח במלחמה בשנתיים הראשונות. 
	
	הפעם הראשונה שגרמניה לא מצליחה לנצח היה במבצע הרי הים, נגד בריטניה. 
	
	\subsection{רעיון ``הסדר החדש''}
	``מאוד מאוד חשוב ויש נטייה לתלמידים לא להבין את זה''. הסדר החדש: כבוש, שלוט, ונצל. זהו פרק נפרד בספר שלנו. זהו רעיון, ולא עקרון. רעיון הסדר החדש מדבר על סדר כלכלי, חברתי ופוליטי חדש בעולם. הוא מבוסס על שני עקרונות: מרחב מחייה ותורת הגזע. זכרו: התפישה הנאצית למרחב מחייה לא היא לא תפישה של כיבוש, אלא כאילו זה היה שלהם מעולם. פירוט המצגת הקודמת. העולם מחולק ע''פ תורת הגזע, ולכן ברגע שכובשים מדינה מתייחסים למדינה בהתאם לגזעם. לדוגמה, עם נכבשת הולנד, מתייחסים אליהם כמו לארים. עם נכבשת פולין – ``פחות''. \textbf{היחס לעמים הנכבשים הוא בהתאם למדרג הגזעי שלהם}. יש לזה משמעות בניהול המדינה. מדיניות אריות יכלו לנהל את מדיניות הפנים בצורה ``חלקית+'' (אך לא מדיניות חוץ כי ``הם לא מדינה הם חלק מהרייך''). לעומתם ``אין פולין'' – היא נכבשה ולכן השלטון ישיר ונאצי. היחס לפולנים הוא ``קצת, לא הרבה'' יותר טוב מליחס ליהודים. מיליוני פולנים נרצחו (היו שם גם משתפים וכו'). מחנות ההשמדה נמצאים בפולין בין היתר בגלל שזהו האיזור הסלאבי. 
	
	עיקרי רעיון סדר החדש: 
	\begin{itemize}
		\item הקמת סדר חברתי־כלכלי־מדיני במסוסס על האידיאולוגיה הנאצית – סדר חדש שיחליף את הדר הקיים, במבנה חדש ובחוקים חדשים אשר יקבעו אלו עמים זכאים לחיות בעולם, מה מקומו של כל עם ועם, מהם הגבולות המדיניות וכיצד יתפתחו העמים, ומה יהיה אופי היחסים ביניהם. 
		\item מיון ודירוג העמים האירופים לפי תורת הגזע: 
		\begin{itemize}
			\item העמים ממוצע ארי – הולנגים, דנים, ונרווגים יהיו בעלי מעמד שווה זכויות בבאימפריה הגרמנית. 
			\item העמית הסלאבים – רוסים ופולינים, ייעודם הוא לשרת עת הגזע האי. 
			\item הגזעים הנחותים – יהודים וצוענים, (בהגדרות מסויימות, אנטי־גזע). הם נועדו להשמדה ואין להם מקום בסדר החדש
		\end{itemize}
	\end{itemize}
	מוטיב חוזר של ויכוח בין מורה כלשהו לעמירם, הוא יוצא. 
	
	\textit{הערה: }צרפתים ואנגלים מחשבו למשהו ביניים – מעל הסלאבים ומתחת לארים. ערבים בתחתית, הם גזע אך מתחת לסלביים. המדרג הגזעי של היפנים וה''צהובים'' הוא גם היה נמוך. 
	
	הפתרון הסופי הוא אחד מהתוצאות של הסדר החדש – בסדר החדש אין יהודים, כי מקומם הוא ``אין''. 
	
	\subsubsection{מימוש הסדר החדש}
	\begin{itemize}
		\item עבודות כפייה – רוסים, פולנים, רומנים – לא רק יהודים – נשלחו לעבוודת כפייה בתנאים קשים, ורבים מהם מתים מהתנאים (אף כי לא נרחצים במחנות השמדה). כשבעה מיליון עובדי כפייה הועברו לגרמניה. זאת כדי לשרת את כלכלת הרייך. 
		\item העברת אוכלוסייה בכפייה – העברת אוכולוסייה ממקום מגוריהם, על מנת לאפשר יישוב גרמנים במקומם. עם חלוקת פולין יישבו הנאצים אוכלוסייה ארית שם. 
		\item המתות ``חסד'' ולבנסבורן – כדי ``לטהר'' את הגזע הארי, אישר היטלר ב־39 את מבצע ``המתת החסד'' (האוטנזיה) התחילה בתוך גרמניה המתתות חולי רוח, נכים, מפגרים, חולים במחלות סופניות ותורשתיות, ועוד. עשות אלפי גרמנים ``ארים'' נרצחו במבצע זה. המבצע הופסק לבסוף בלחץ הכנסייה הקתולית ובשל לחץ דעת קהל. 
		
		לונסבורן – התחיל בגרמניה, מרכזי הרבייה. התחיל עם נשים אריות גרמניות והמשיך שמדינות כבושות עם נשים אריות לא גרמניות. מיועד להשבחת הגזע הארי. 
		\item ניצול כלכלי – מזון, מוצרי תעשייה וחומרי גלם הועברו מהארצות הכבושות לגרמנייה. הנאצים שדוו סחורות שטרי כסף, זה, אוצרות אומנות ועוד. ברוסיה שרפו הכל בשיטת ``אדמה חרוכה'' כדי שהנאצים לא ישתמשו בזה. ``הצבע צועד על קיבתו'', וכן על ציודו – ביגוד, לחימה וכו'. קווי אספקה ארוכים יותר אממ יותר קשה. והחורף הרוסי הוביל אח''כ להפסדם באוסייה. מזון לדוגמה, לא נשלח לגרמנייה, אך יצירות אומנות, בגדי ילידים וכו' כן. דברים שמשמשים את הצבא כמובן שלא נשלחו לגרמנייה. 
		\item הפתרון הסופי – אתם יודעים מה זה. חוץ מזה הרחבה בשנה הבאה. רצח שיטתי של היהודים, שמתחיל מאמצע 41, ובעיקר ב־42 ו־43. במהלכו, הנאצים הקימו מפעלים של רצח. זה \textit{לא} אנשים שמתים מהתנאים במחנות עבודה, או מהתנאים בגטאות. אלא ברצח שיטתי שלטובתו מוקמים מרכזים, AKA ``מחנות השמדה''. התחיל בבורות ולא במחנות השמדה. 
	\end{itemize}
	
	פירוט הקרבות והכיבושים – תראו את ``אפוקליפסה'', סדרה ממש טובה. לא נפרט על זה. 
	
	\subsection{יהודי פולין עם הכיבוש}
	בשיעורים הקרובים נדבר על יהודי פולין ועל הגטאות. 
	\subsubsection{הקדמה}
	אנחנו חיים במדינה ריבונית המדינה על אזרחיה. זה לא המצב בפולין. לכן, צריך להיזהר בשיפוטיות כלפי דברים שמדברים עליהם – קל יחסית לשאול ``איך הם לא מרדו/ברחו/עזרו''. זה לא אומר שא־אפשר לשאול שאלות, אבל, אין לנו זכות לשפוטם כי לא היינו במצב הזה. אנחנו לומדים יהום יותר מ־80 שנה אחרי ויודיעם את סוף הסיפור (מחנות השמדה, פתורן סופי וכו') – הם לא. לפני כן, תמיד היו פוגרומים ואנטישמיות, אך היהודים הורידו את הראש, אמרו אוקי זה יעבוד וזה היה נגמר. \textit{אין תקדים לאירועי השואה}. עד שלב מאוד מאוחר, הם עלו על הרכבות ולא ידעו שהם נוסעים למחנה השמדה. 
	
	\subsubsection{חזרה על החומר}
	במשך כשלושה חודשים לאחר פרוץ המלחמה היה הגבול הסובייטי במזרח יחסית פתוח ו־300K יהודים נמלטו אל השטח הרוסי. כבר ידעו מה קורה ליהודים בגרמנייה ולכן הם ברחו. חלקם הגדול, היו צעירים ללא משפחות. משתי סיבות: ה־state of mind של צעירים יותר גמיש, וכן הם לא צריכים לגרור משפחה אחריהם. מנהיגי תנועות הנוער היו חלק ממנהיגות הצעירה. חלקם, נכנסו לגטאות, פעלו וסייעו שם. מאוחר יותר הם היו אלו להוביל את המרידות בכטאות. 
	
	כאשר מדברים על גטאות ומחנות השמדה – מדברים על מרכז פולין. 
	\subsubsection{השואה בפולין עד תחילת הפתרון הסופי – ``בחירה בעולם ללא בחירה''}
	יש בחירה. הבחירה היא בעולם שלא בחרו להיות שם, בתנאים קשים. ובתוך כך צריך לבצע בחירות – במרה האנושית, איך להתנהג ולדאוג לילדם, איך והאם לשמור על צלם אנוש, וכו'. בהינתן העולם שאתה לא יכול לבחור (הגטו), יש לך בחירה. ``אבל אנחנו לא שופטים את הבחירה''. 
	
	$\rangle$ שיר – מה הייתה המילה שואה $\langle$
	
	ב־21 בספטמבר 1939, נשלחה איגרת הבזק של היידריך, ראש משטרת הבטחון הגרמנית, אל מפקדי האיינזצגרופן – תפקידן לטפל ולנהל את הדברים במקום, בדגש על מתנגדי משטר ויהודים. 
	
	היא הייתה אגרת סודית, וממנה אפשר ללמומד על תוכניותה העתידיות של גרמניה כלפי היהודים. \textit{הערה: }היא לא מופיעה בספר, אך חשוב לדעת ולהבין אותה. 
	
	\begin{itemize}
		\item פינוי כל היהודים מאיזור מערב פולין. איכוזם של יהודים ברבעי מגורים מיוחדים בערים שגודלן מעל 500 נפש ונמצאות בסמוך לתחנות ומסילות רכבת. 
		
		\textit{הערה: }הם לא ידעו למה. הם רצו לרכז עד שיעבירו. 
		\item אריזציה של מפעלים יהודים. לא כל הרכוש והמפעלים. הם השאירו את מה ששירת את האינטרסים שלהם. 
		\item הקמת מועצת זקנים = יוגנראטים שיואכו מעד 24 אנשים סמכותיים בעלי השפעה, או מהרבנים שנותרו בקהילה. הם אמורים להיות \textit{כפופים} לצבע הרייך ולהשמע להוראותיו. 
	\end{itemize}
	
	\section{הגטו}
	
	\subsection{לפני הגטאות (חזרה)}
	כל הדברים שנעשו בגרמניה עד 39 הגיעו בבום אחד לפולין, ואף מעבר לכך (כלומר, היחס ליהודי פולין היה אף יותר גרוע מזה של יהודי גרמניה, או דנמרק לצורך הנקודה). 
	\begin{enumerate}
		\item \textbf{פיגעה במוסדות דת ותרבות יהודיים: }זקני היהודים נגחו בהתרסה ברחובות, ספרי קודש חוללו, בתי כנ סובתי מדרש נשרפו, לעיתים עם המתפללים בפנים. כמו כן נאסרה שחיטה כשרה, לימודי תורה, תפילה בציבור, וחובת עבודה בימי שבת וחג. תשומת הלב העיקרית הפונתה להשפלתם את היהודים
		\item \textbf{צעדים להפרדה: }טלאי צהוב, איסור כניסה וכוק. 
		\item \textbf{הגבלת חופש התנועה}
		\item \textbf{הרס קיום כלכלי: }אריזציה של כל בית מלאכה גדול (עד לכדי חנויות מכולת קטנות), הוחרמו דירות יהוהידם אמידים, נאסר להחזיק כסף בסכומים גדולים וכו'. 
		\item \textbf{חטיפות יהודים לעבודות כפייה: }לא קרה בגרמניה. חדש לפולין. חטפו מהרוב לעבודות כפייה (לפני הקמת הגטאות). דברים כמו סחיבת משאיות, עבודות שירות במחנות צבאיים, גרייה, בניית תשתית, ועוד. חלק מהעבודות מועילות, וחלקן חסרי פואנטה ונוצרו אך ורק בשביל להשפיל ולנצל את היהודים – לדוגמה ללכת ולהחזיר אבנים ממקום למקום. מיום הקמת הגנרל גוברנמן חויבו יהודים בגיל עבודה לצאת ולעבוד. 
		\item ישנה תמונה מפורסמת של גרמנים הבוזזים יהודים זקנים פולניים. במקרים אחרים הכריחו אותם לרקוד ברחוב, ועמדו וצחקו. המון דברים שמטרתם הייתה השפלה בלבד. 
	\end{enumerate}
	
	ההפרדה הוא לא בידוד בין היהודים לגרמנים (הגרמנים עוד לא גרים בפולין), אלא גם מול האוכלוסיה המקומית. 
	
	זוהי לא מילה שאנשים נבהלו ממנה לפני השואה. זו לא הייתה מילה עם קונוטציה של צפיפות, רעב, מחלות ומוות. לפני כן, זו פשוט הייתה שכונה יהודית סגורה שבה חיו יהודים. היהדוים באירופה חיו באיזור סגורים. אפילו במדינות ערביות – שם קראו לזה ``מלאח'' (דומה מאוד לגטו), כי ע''פ הדת המוסלמי יש מעמד מיוחד ליהודים שנועד להשפלי אותם. 
	
	בערים באירופה, היהודים לא יכלו לגור מחוץ לגטו, אבל זה היה גם אינטרס של היהודים – זה שמר עליהם. זו הייתה שכונה סגורה, עם שער, שבה הם יכלו לשמור על המנהגים שלהם, ולא להתבולל. בחלק גטאות באירופה, בחגים וימי ארשון היו עיתות שבהם לא הייתה אופציה ליציאה וכניסה מהגטו. לא היה שם רעב, זה לא גרם למגיפות, ולא הייתה לכך גונוטציה שלילית. כמו הרובע היהודי והמוסלמי בירושלים. הגטו הראשון היה הגטו נובו, בונציה. התחיל בימי הביניים. 
	
	\subsection{הגטאות הנאציים}
	בשונה מבעבר, בגטו הנאצים שיסודותיו הוקמו בעקבות אגרת הבזק של היידריך, התפיסה השתנתה – רוצים לבודד אותם, בשביל ``משהו'' (לא ידעו עוד מה). 
	
	\begin{enumerate}
		\item הגטו הראשון – פיוטקרוב, הוקם ב־39. 
		\item הגטאות ברוסיה הוקמו בידיעה שהולכים להשמדה, בניגוד לגטאות ברוסיה. 
		\item גטו לודג' – השני מבינהם, מאוד מיוחד ועליו נדבר בהמשך
		\item גטו ורשה – הגדול והצפוץ מבינהם. בסוף היו שם 60K היהודים והוא התחיל עם 400K שנדחפו לשם בהדרגה (כשליש מאוכלוסיית ורשה). הקמתו נמשכה כשנה. תמדי היה צפוף כי גם כאשר לא היו שם הרבה אנשים צמצמו את השטח. 
	\end{enumerate}
	
	גטאות הוקמו בעיקר המרכזית/הגדולה ביותר באיזור, ועוד בהתחלה הגיעו לשם יהודים מכל האיזור. לגטאות מסויימות כמו גטו ורשה הגיעו בשלבים מסויימים אף ממידנות אחרות כמו גרמניה. 
	
	המעבר לגטאות היה מהיר וחד. מותר להם לקחת כמות מסויימת של דברים, ורוב רכושם נשאר בבתים. הם לא ידעו לאן הם הוכלים והגטאות האירופיים היו הדבר שחשבו עליהם. נאלצו למסור לגרמנים תגישטים, כסף וזהב. גם בתוך הגטו וגם מחוצה לו נגרשו לשאת את הטלאי הסימן (כי בפנים התובבו אנשים שלא היו יהודים). הרכבת החשמלית לדוגמה, עוברת בתוך הגטו (על אף שאין תחנה שם). בפולין לא היה טלאי צהוב כמעט, אלא סרט לבן עם מגן דוד כחול. 
	
	\subsection{הקמת הגטאות}
	לא קורה ביום אחד. לודג' היה בין הראשונים, ובגלל מיקומו (צמוד לגנרל גוברנמן, אבל בתוך גרמניה והאוכלוסיה סביב גרמנית). לכן היה בין הסגורים והקשים ביותר. הגטאות הוקמו בשכונות הכי עניות בעיר – ליד איפה שעוברים פסי רכבת. מקומות רועשים ובהם התנאים פחות טובים, האוויר פחות נקי וכו'. כמובן שלנושא פסי הרכבת היו מטרות נוספות. 
	
	ישנה תרומה מפורסמת של גרמנים הבוזזים יהודים זקנים פולניים. במקרים אחרים הכריחו אותם לרקוד ברחוב, ועמדו וצחקו. המון דברים שמטרתם הייתה השפלה בלבד. 
	
	\textit{הבהרה: }כל הגטאות שונים. ננסה לדבר על הדברים שדומים בינהם. אך כמות המגיפות, אופן הניהול, כמות הצפיפות, רמת הסגירות וכו' – מאוד שונים בין גטאות. 
	
	השטח היה מאוד מצומצם. בגטו וילנא היו בין 5-8 נפשות בחדר אחד. בגטו ורשה הייתה הצפיפות הגדולה ביותר, בין 15-30 אנשים בחדר. לכל דירה (קבוצה של חדרים) היה מקלחת אחת – בערך 30-40 אנשים על מקלחת
	
	בגטו אין שוויון כלכלי־חברתי. לא כולם עניים. לא כולם רעבים. יש שם אנשים מאוד עניים ויש בה אנשים עשירים. גטו זו שכונה. יש שם נשים, וגברים, וילדים, ופושעים, ומנהלים והכל. וחיים שם כמשפחות. בניגוד למחנות (לא בהכרח מחנות השמדה), גטו היה שכונה לכל דבר. במחנות כמעט ואין ילדים, יש הפרדה מגדרית, לא חיים כמשפחה, ומטרתם אחרת. 
	
	אומנם רוב האוכלוסייה ענייה ורעבה, אך לא כולם. עדה מורשה במסע הקודם לפולין, ספירה שבגלל שחיו עוד קודם לכן בורשה והנאצים היו צריכים את המפעל של המשפחה העמידה שלהם, היא לא כל־כך סבלה מרעב. 
	
	\subsubsection{מטרות הקמת הגטאות}
	כאשר נדבר על הסיבות באופן כללי – נדבר על הסיבות האמיתיות. לא על הסיבות שהנאצים פרסמו תאוכלוסייה וליהודים. 
	
	הסיבות המוצהרות: 
	\begin{itemize}
		\item נועד לשמור על היהודים בגלל האנטישמיות והכל
		\item למנוע שמועות שליליות פוליטיות, חתרניות ותבוסתניות מצד היהודים
		\item למנוע התפשטות של מגיפות מדבקות שמקורן ביהודים, ובכך לשמור על מצב סניטרי תקין. 
		\item למנוע שוק שחור (שהיהודים הואשמו בניהולו)
	\end{itemize}
	
	המטרות המוצהרות נועדו גם להרחיק את הפולנים לגטאות. רצו שהפולנים לא ירצו להיות קרובים ולעשות עסקים עם היהודים (אם כי התקיימו כאלו במקומות מסויימים). להסברים האלו לא היה בסיס. להלן הסיבות האמיתיות: 
	\begin{itemize}
		\item \textbf{ריכוז ובידוד: }הפחדת האוכלוסיה הפולנית במאצעות הפצת שמועות בדבר מחלות וכו'. 
		\item \textbf{ניצול כלכלי:} כאשר נכנסים לגטו הנאצים לוקחים את הרכוש שלהם. חלק מהרכוש נשלח לגמרניה, בחלק השתמשו. הניצול הכלכלי בא גם בנושא שם עובודת כפייה, כי הפכו לכוח עבודה זול/חינם. בתמורה היהודים קיבלו עוד מנה של אוכל. עבודות כפייה היו משהו שבשלב מסויים יהודים ``רצו'' לצאת אליו, כי מקבלים אוכל (שלא היה בגטו). חלקם הצליחו להבריח את האוכל לתוך הגטו. 
		\item \textbf{אמצעי הכחדה עקיף: }תמותה כתוצאה מרעב, צפיפות, מחסור ומחלות. הנאצים דאגו לפיפות בטו, הן לשם נוחות לתפוס את היהודים אבל גם לשם תמותה. הנאצים קראו לזה מוות ``טבעי''. אומנם הרעבה זה מוות מאוד יחסית ישיר ו''עקיף'' זה קצת ``מכבשת מילים'', אך אין זה מחנה השמדה שיטתי = מטרת ההדעה היא לא לרצוח אותם. בין 41-42 מתו 112K יהודים בלודג' ובורשה. ביולי אוגוסט חום אימים, הדירות צפופות ולא מאווררות. אין הגיינה ותרופות (רופאים לא חסרים). 
		
		עד שלב מסויים אפשרו לארגונים להיכנס לגטו ולסייע, מה שנגמר אחרי שארה''ב נכנסה למלחהמה. 
		\item \textbf{לשבור את רוחם של היהודים: }כשנדבר על התנגדות, זה חלק מהעניין. 
	\end{itemize}
	
	
	בשלב הזה באירופה כבר לא היו גטאות. 
	
	\subsection{ניהול הגטו והיודנראטים}
	הגטאות שונים אחד מהשני. אבל באופן כללי: 
	\begin{itemize}
		\item הקפת הגטו (חומה/טיל). יש מקרי קצה זניחים של גטאות ``פתוחים'', כלומר אנשים יכלו לצאת בחוץ ולחזור (שהנאצים אישרו). 
		\item חובת נשיאת סימן זיהוי ותעודות מזהות. 
		\item משטרה גרמנית השומרת מחוץ לגטו (כדי שהאינטראקציה עם היהודים תהיה מינימלית). השומרים פולניים. 
		\item יודנראטים – אמצעי פיקוח מאוד משמעותי שאף הקים את המשטרה היהודית (תזכורת: זו שכונה לכל דבר וצריך למנוע שם פשע). תפקיד היוגנראטים היה לשרת את הנאצים ולסייע להם בפיקוח הגטו. נדבר עליהם בנפרד ובגדול הם קיבלו יחס יותר טוב (מה שלא מנע לרצוח אותם אחכ). 
		\item גאסטפו (המשטרה החשאית הגרמנית). הפעילה הגטאות רשתות של סוכנים ומרגלים (שלא היו גרמנים). 
	\end{itemize}
	
	שוב, יש הבדלים בין הגטאות. נבתונן בחומות של הגטאות – החומה הגבוהה של קרקוב זה לא החומות עץ של הגטאות האחרים. לדוגמה, קרקוב היה מוקף חומת אבן אולם והיה פחות סגור ומנותק מגטו ורשה, והציאה ממנו לצורכי עבודה הייתה קלה יותר. גטו לודג' היה שונה משאר הגטאות והוא היה מבודד לחלוטין, וסגור יותר. הוא נשמר מבחוץ ע''י שוטרים גרמניים ובתוכו המשטרה היהודית פיקחה על הסדר. בגטאות מסויימים היו מצליחים להבריח אוכל לתוך הגטו. לא רק פוחי אדמה, אלא כמויות של אוכל. בגטאות כמו לודג' לא היה אפשר להכניס אוכל בכלל ולכן היה הרעב ביותר. 
	
	היו גם גטאות קטנים יותר, עם מידת ניתוק קטנה יותר. 
	
	באגרת הבזק של היידריך, רא משטרת ה־SS, נאמר להקים מועצה שתוקם מהזקנים והרבנים. הם מנהלים את הגטו והם אנשי הקשר עם הנמאצים. הם גם אחראים על מה שקורה בתוך הגטו. 
	למה צריך אותם? 
	\begin{itemize}
		\item כדי שיספגו את האשמה מצד היהודים. לפהנות את הזעם והמרירות כנגדם. 
		\item חסיכת כוח אםד גרמני. 
		\item השוואה והטעיה, כאילו חייהם של היהודים מתנהלים ע''י היודנראטים
	\end{itemize}
	
	היודנראט כיחיד היה מוכר ומוסמך, בעל אחריות אישית לביצוע המדיניות הנאצים, כלומר אם היא לא בוצעה הוא לא ימשיך בתפקידו וכנראה ישלח למחנה. הם שימשו כמתווכים ומשתפי פעולה עם הנאצים. 
	
	כאשר הצלב האדום הגיע לבקר, הנאצים הכינו גטו לדוגמה. לשם כך שלחו המון אנשים לגטאות אחרים מטרזין, צבעו את טרזין, והאיכלו אותם. היהודים בגטו היו צריכים לשחק את המשחק. ``היה נראה בונבון''. הצלב האדום האמין להם (``לא ארגון וואו, ארגון בלי שיניים שיכולתו לסייע היא מול מדינות ריבוניות החתומות על אמנות (לא רק חמאס)''. הוא מצליח רק בזכות אמנות בין לאומיות [מכאן ואילך רנאט מוצדק על אונר''א והאו''ם שמום]). 
	
	תפקדי היודנראט מול הנאצים: 
	\begin{enumerate}
		\item ריכוז והעברת היהודים ומציאת מדומות מגורים ליהודי הכפרים והייערות בגטו. לא פשוט – בורשה בשלב מסויים אין מקום בבתים, ואנשים חיים בחדרי מדרגות. 
		\item ארגון מפקדים לפי מין וקבוצת גיל – כדי לדעת כמה אוכל להכניס, כמה אפשר לקחת לעבוודת כפייה וכו'. 
		\item סיוע בהחרמת רכוש, תשלומי ביצוע קנסות ויישום גיזרות שונות. 
		\item דאגה על הגיינה בגטו, כדי שמגיפות לא יצאו מחוץ לגטו. 
		\item ארגון ואיסוף היהודים לגירושים עם תחילת ``הפתרון הסופי''. היודנראטים קלטו לאן הם אוספים את היהודים מאוד מהר. 
	\end{enumerate}
	בכל פעולה הם תיווכו בין הנאצים לבין היהודים. 
	
	לאחר השואה, מעטי היודנראים ששרדו הואשמו ע''י ניצולי שואה כמשתפי פעולה עם הנאצים. חוק עשיית עוזרי הדין בנאצים ובעוזריהם נולד מהצורך, בין היתר, לפעול כנגדם. בדיעבד, היום מבינים שזה תפקיד יותר מורכב ממה שנתפש בעבר. ישנה הבנה כי למרות שחלקם רצו להיות יודנראטים כדי לקבל תנאים יותר טובים, זהו תפקיד נוראי בעיקר בזמן הפתרון הסופי. 
	
	היו מעט דברים שהיודראטים יכלו לעשות (ועשו) כדי לעזור. לדוגמה, בגטו לודג' גידלו בחצרות הבתים אוכל (חצרות קטנות, משהו בגודל של הכיתה פעמים לבניין עם 700-800 איש). הם גם הקימו מטבחים ציבוריים – מקומות שבהם בישלו אוכל והעניין יותר היגעו לשם פעם ביום כדי לקבל ארוחה אחת. זאת עשו באמצעות ההקצבה שהייתה להם. הם גם אומנם לא יכלו לעזור לאנשים בהברחות (יהרגו אותם) אך במקרים רבים העלימו עין מהברחות לתוך הגטו. אומנם היו אסורים בתי ספר, הקמת מניין, בתי חולים וכו' – אך הם אפשרו ואף הקימו מוסדות כאלו. תפקידים אזרחיים נפלו לידם, כמו דת, קבורה, חינוך ועוד – והם טיפלו בהם. 
	
	``אין פה האם לבצע – יש פה האיך לבצע'' – שרית על היודנראט. במקרים אחדים הם גם שאלו עצמם ``האם'', אל המשמעות היא כנראה מוות ובמקרה הטוב מחנה עבודה. 
	
	\subsubsection{מפעלי תעשייה}
	הייתה תפיסה שהיהודים לא פרודקטיבים ויצרנים. לכן יודנראטים רבים עסקו בהקמת מפעלי תעשייה, הן בשביל לספק תעסוקה והן בשביל להועיל לנאצים ולקוות שלא יהרגו אותם. ``עבודה כהצלה''. בשלב מסויים הם יבינו שבכל מקרה ירצחו את כולם, וניסו לדחות כמו שאפשר את הגירוש בתקווה שבעלות הברית ישחררו את האיזור לפני. 
	
	\subsection{דילמות היודנראטים ביחס לניהול הגטו}
	
	כאשר שואלים מה הדילמות של היודנראט – \textbf{הן שתיים}: 
	\begin{itemize}
		\item מציאת איזון בין חובת הציות לגרמנים לבין טובת הגטו: הדילמה אינה לבצע את הפקודות, אלא איך לבצע אותן. יש איזון דק בין איך לסייע ליהודים, לבין איך לגרום לך להשאר יודנראט ובחיים. עצם העובדה שהעלימו עין מאותם הילדים שהיבאו אוכל, זה גם משהו, במיוחד כאשר לדיד הנאצים הברחת אוכל דינה מוות. כמובן שהיהודים בגטו, שאינם היו יודנראטים, לא ראו את זה כטובה אלא כמובן מאליו (``ואנחנו לא שופטים אותם לגבי זה''). 
		\item כיצד להתמודד עם יחס הציבור היהודי ששנה ובז להם: זכרו, אחת ממטרות היודנראט זה להפנות את הזעם כלפיהם ולא כלפי הנאצים. לכן, ``הקהילה הידוית במקרה הטוב שונאת אותם''. 
	\end{itemize}
	הבהרה: לא הכריחו כמעט אף אחד להיות יודנראט. 
	
	קורים כל מני דברים לא הוגנים: יוהודים ספציפיים שניתנת להם האופציה ללכת לעבודות כפייה (ולקבל יותר כסף), שניתנתה פעמים רבות למקרים שלהם (רצו להציל קודם כל את המשפחה שלהם). נוסף על כן, לעיתים גם חלוקת האוכל הייתה לא שווה. 
	
	``ויטמין P זה קיצור של פרוטקציה'', וגטאות בהרבה מקרים שנאו אותם על זה. היה מדובר בעניין של חיים ומוות, לא על כסף או בגדים. 
	
	
	\subsection{דילמת היודנראטים בזמן ביצוע הפתרון הסופי}
	\textit{מומלץ לקרוא לאחר קריאת הפרק על הפתרון הסופי!}
	
	נכיר שלושה ראשי יודנראטים – חיים רומקובסקי מלודג', אדם צ'רניקוב מורשה ויעקב גנץ מגטו וילנה. נחזור בקצרה על הנושא: 
	\begin{itemize}
		\item מציאת איזון בין חובת הציות לגרמנים לבין טובת הגטו. הקושי העיקרי לא היה האם למלא את הפקודות, אלא כיצד למלא את הפקודות. לא הייתה אופציה לא לציית (כי ירצחו אותם ומשפחתם ואז יחליפו אותם) אך בתוך הסמכויות שלהם ניסו לסייע ככל יכולתם (לדוגמה מטבחים ציבוריים, כיתות לימוד, הקמת מרפאות ובתי חולים, ולקבל בשתיקה דברים שלא הקימו). 
		\item כיצד להתמודד עם יחס הציבור היהודי ששנא ובז להם – היה להם אורח חיים ראוותני, זכויות יתר, יותר אוכל ושיתוף פעולה עם הנאצים. 
	\end{itemize}
	תזכורת – כשקמה המדינה נתפסו היודנראטים כמשתפי פעולה, ואותם הניצולים שהיגעו ושרדו נתפסו כך. החוק לעשיית דין בנאצים ובעוזריהם כתוצאה מכך. אנשים הגיעו למשטרה ודיווחו על כך שראו משת''פים יודנראטים, ולא יכלה לעשות עם זה כלום משום שלא היה חוק לטיפול בנושא. עשרות יהודים שורדי שואה שהיו חברי יודנראט נשפטו במדינת ישראל בשנות ה־50, וחלק ניכר מהם נידונו למאסר. אחד מהם נידון למוות (לבסוף הוא לא הוצא להורג כי היה חולה וקיבל חנינה). 
	
	בתקופת הפתרון הסופי (1941-1945) נדרשו ליודנראטים לספק עשרות אלפי אנשים לגרמנים ל''התיישבות מחדש''. האם להמשיך לבצע את ההוראות? האם לנסות לקיים מדיניות שבה מצילים לפחות חלק מתושבי הגטו? את מי שולחים להשמדה? באופן כללי הגירושים החלו בינואר 1942, בהתחלה בעיקר לחלמנו. 
	
	גירוש גטו ורשה החל ב־22 ביולי 1942. בהתחלה נאלצו לספק 6K ביום ולאחר כן 10K. 
	יעקב גנץ החליט לענות על השאלה באמצעות התאבדות. לאחר התאבדותו תוך 58 יום נשלחו לטרבלינקה כ־300K יהודים. 
	
	ליודנראטים במקרים רבים הייתה את הגישה של ``עבודה כהצלה'' שבה ניסו להועיל לנאצים בבניית מפעלים בתקווה שזה יחזיק מספיר זמן (בתקופה הזו בעלות הברית החלו לנצח במזרח ותהקדם מערבה). 
	
	אז מה קרה בגטאות ששרדו הרבה זמן? נדבר על גטו לודג', שאותו ניהל חיים רומוקובסקי. האדם שנוי במחלוקת, כנראה היה כוחני. שורדי לודג' מספרים התנהגות ``מאוד לא סימפטית'' להתנהגות של אותו האיש ביחס לנשים יהודיות, בעיקר צעירות (ביחד עם סיפורים לא טובים אחרים). עם זאת, בעקבות האידיאולוגיה של ``עבודה כהצלה'' לודג' שרד הכי מאוחר וניצלו ממנו הכי הרבה אנשים. יש שתי סיבות. 
	\begin{itemize}
		\item מרבית יהודי לודג' הושמדו בחלמנו, וכאשר הוא נסגר הם הועברו לאשוויץ. 
		\item באושוויץ הייתה סלקציה, וככל שהגעת יותר מאוחר (ומלודג' הגיעו מאוחר) היה להם יותר סיכוי לשרוד את התנאים. 
	\end{itemize}
	רומוקובסקי נאם נאום הידוע בתור ``נאום הילדים''. בנאום רומוקובסקי בא ואומר – אני צריך לשלוח 20 אלף יהודים להשמדה. הוא מבהיר שאלו ימותו. הוא ביקש מהאימהות להביא את ילדיהן למות, וכן את הזקנים והחולים, ולהשאיר בגטו את מי שאפשר להציל. ``אני מוכרח לבצע את הניתוח הקשה השותת דם, אני מוכרח לקטוע איברים, בכדי להציל את הגוף! אני מוכרח ליטול ילדים ואם לא, עלולים להילקח, חס ולשום, גם אחרים''. ``אך כיוון שלא היינו מודרגים על ידי המחשב ``כמה יאבדו'' אלא ``כמה ניתן להציל?'' הגענו אנחנו [...] למסקנה שיהיה הדבר קשה ככל שיהיה, אנו מוכרחים לקבל את ביצוע הגזירה לידינו.''. 
	
	לא כל היודנראטים אמרו וסיפרו ליהודים שהם הולכים להשמדה, בגלל הסתכנות במרד ובלגנים. עם זאת זה אפשר ליהודים לברוח. 
	
	
	
	
	
	\subsection{תנאי החיים בגטו}
	יש לציין, שאינם זהים לכל אדם, ולא בכל גטו. יש אנשים עשירים יותר וכו'. 
	
	חלק מהגטאות שרדו 3 שנים, חלק חצי שנה. אחד הגטאות גם שרד 4 שנים. משמע, אנשים חיו שם. הגטאות הגדולים בפולין (בילאיסטו, ורשה, לודז' וכו') פעלו לאורך תקופה יותר ארוכה (בערך 3 שנים). בתחילת 42' התחיל שילוח היהודים למחנות השמדה, ואז חלק ניכר מהגטאות נסגרו (כי רצחו את כולם). 
	\begin{itemize}
		\item \textbf{צפיפות: }בורשה הגטו היה 2.4\% מהעיר. חיו בו עד 450K יהודים בשיאו. בלודז' אפילו יותר גרוע. בגדול – הצפיפות גדולה מאור, ואנשים חיו בתוך חדר. על כן אנשים יצאו מהגטו, וזה הגיע למצבים שבהם בעיתות מסויימות המדרכה לא גדולה דיה בשביל להכיל את כל האנשים. הצפיפות גוררת מגיפות. הצפיפות נשארה קבועה כאשר התחילו לרצוח אנשים במחנות השמדה, כי פשוט הקטינו את שטח הגטו. 
		\item \textbf{אוכל: }הנאצים העבירו אוכל בסדר גודל של 184 קלוריות ליום בעבור אדם מבוגר. (``משהו כמו שני פרוסות לחם עם גבינה רק בלי הגבינה''). בחלק מהגטאות הבריחו אוכל. בגטו ורשה, ־80\% מצריכת המזון הגיעו בהברחה. בלודז' לא היה אפשר להבריח ולכן היה הגטו הרעב ביותר. בשאר הגטאות, נאמר, ורשה  –היו גם הברחות של ילדים שיצאו דרך תעלות ביוב וחורים בגדר (בסיכון שאם ימצאו אותם יהרדו אותם). בורשה אף היו הברחאות סיטנאיות (ממש שוק שחו). 
		\item \textbf{תברואה: }נוסף על הצפיפות, ביוב ברחובות, מים לא תמיד זורמים, אין מערכת בריאות. היודנראטים הקימו בחלק מהמקומות בתי חולים, רק תרופות לא היו בהם. עד פרל הארבור, הנאצים הסיכו לארגונים להכניס דברים (בין היתר בשביל הטעיה של העולם). אלפי יהודים מתו מדי חודש בורשה. בין רופאים היו דילמות למי לתת תרופות – למי שמצבו יותר גרוע ובלי התרופה ימות, או למי שמצבו פחות גרוע וישרוד יות רזמן? 
		\item \textbf{עבודה: }אין ממש עבודה. יש עבודה למי שמקבל עבודה מהיודנראט, או במפעלים בגטו או בעבודות כפייה. אנשים רצו לעבוד כי עבודה=אוכל. 
		
		בגטו לודז', רמוקובסטי ראש היודנראט שכנע את הציבור שהדרך היחידה לשרוד היא הפיכת הגטו לאמצעי יצרני, אך השכר לא הספיק כדי למנוע רעב. 
		\item \textbf{תרבות וחינוך: }אסור על מעל 10 אנשים להתכנס. אסור להפעיל בתי ספר. היו בתי ספר מחתרתייםם שפועלים בצורה מסודרת, חלקם ביוזמה/עידוד/העלמת עין של היודנראטים. בבתי הספר האלו גם למדו עברית ועל הארץ. נתנו תקווה שתגמר המלחמה, ועוד. רוב המחנכים היו ציונים. 
		
		הייתה תזמורת, מקהלה, תיאטרון וכו'. ישנו מחזה הקרוי ``הגטו'', המדבר על תיאטרון שהוקם באחד הגטאות. אנשים ציירו, כתבו, היו ספריות, וכל דבר בחלק השני אחרי הבסיס בפירדמית הצרכים של מאסלו. בתחתיתה, צרכים בסיסיים (מים, אוכל וכו') ולמעלה בהגשמה עצמאית. אלו כל אותם הדברים שהם מעבר לצרכי החיים הבסיסיים. ביד ושם יש ציורים רבים של החיים בגטאות. יש ספר בשם ``טומי'' שכתב אבא לבנו בטרזין (יש כאלו שמתמשים בו ביום השואה בגנים, שם לא רוצים לתאר הכל). עיתונים, תנועות נוער וכו'. 
		
		ישנו פרויקט חשוב בשם ``עונג שבת''. יזם אותו עמנואל בינגלבלום, שהיה היסטוריון והבין שחשוב לתעד. הוא ביקש מאנשים רבים, רגילים (לא יודנראטים, גופים רשמיים וכו') שתיעדו את חיי היום־יום שלהם. חלק ציור, סיפור. שירה וכל אמצעי אחר. את התיעוד הזה הכניסו לכמה כדים שמוטמנים באדמה בגטו. אחרי השואה מצאו את הכדים הללו (לא את כולם, חלק נמצאו יותר מאוחר) והיום הם ביד ושם. חלק גדול מהדברים שאנחנו יודעים על גטו ורשה מגיעים מהארכיון הזה. גם קשור לקידוש החיים. 
	\end{itemize}
	
	מעט מאוד בניינים שרדו בוארשה אחרי WWII. בחלק מהבניינים בוארשה גרים עד היום. 
	
	\subsection{קידוש השם וקידוש החיים}
	\begin{enumerate}
		\item קידוש השם – (``נמות ולא נגייס'', ``אללא אכבר'', ``אני אתעלם''). הג'יאהד באיסלם הוא לא בדיוק מה שמדברים עליו (האסלם האמיתי והדבר הזה שנמצא בעזה אלו שני דברים בלי קשר). בכל הדתות המונוטאיסטיות קיים דבר דומה. 
		\item הורשה היה אדם העונה לשם ``הרב ניסנבוים''. לפי הרעיונות שלו, בכל היסטורית העם היהודים הניסיון היה להשמיד את העם היהודי. פה המטרה היא אחרת – להרוג אותנו. אם בעבר אחת ונהיית נוצרי הכל טוב, אצל הנאצים זה לא המצב. אף בכל הגטאות הגדולים היו אנשים בגטו שנכנסו לשם בשל יהודתם – לפי האידיאולוגיה הנאצית, והם היו נוצרים והלכו לכנייסה (היו בורשה אף שתי כנסיות גדולות). 
		
		נחזור לניסנבוים. אם פעם רצו להמיר את דתנו, ,אז קידשנו את השם בשביל לא להמיר את הדת (למות ובלבד לא להמיר את הדת). עתה זה לא המצב – הם רוצים להרוג אותנו, ולא יעזור שנקדש את השם. על כן, נקדש את החיים. 
	\end{enumerate}
	התפישה הזו הייתה קיימת בכל הגטאות. היא הייתה רלוונטית עד תקופת הפתרון הסופי – שם אין קידוש החיים. זה משהו אחר. קידוש החיים בא לידי ביטוי בשני תחומים: פיזי, אך גם רוחני. 
	
	
	הגדרת קידוש החיים: להמשיך לדבוק בחיים בהיבטים הפיזיים והרוחניים (לזכור – לא רק פיזי, לא רק ההשרדות). 
	
	\section{הפתרון הסופי}
	
	\textbf{תזכורת: }בשנים 1939-1941 מדובר על השואה \textit{עד} הפתרון הסופי. החל מיוני 41 מדובר על השואה \textit{בתקופת} הפתרון הסופי. מה היווה את המפנה? מבצע ברברוסה ב־1941. תחילה, ננסה להבין מה הקשר בין מבצע ברברוסה לבין הפתרון הסופי. 
	
	\textbf{תזכורת נוספת: }''מבצע ברברוסה'' הוא שם של הפלישה הגרמנית לברה''מ. 
	
	במה הפתרון הסופי שונה ממה שהיה קודם? זהו רצח שיטתי. מדובר על מבצע רצח מתוכנן. כאשר נדבר על שלבים בפתרון הסופי, נבחין שזה מתכונן, אך לא בדיוק. לקח זמן עד שזה הגיע לאושוויץ־בירקנהו. בגטו אחת המטרות הייתה תמותה עקיפה. הרוב לא מתו כי רצחו אותם באופן ישיר, אלא מתו מהתנאים. 
	
	\textbf{הגדרה: }{\textit{הפתרון הסופי} הוא שם קוד למבצע להשמדת היהודים בעולם באופן שיטתי, המוני וישיר. }
	
	\subsection{הסיבות לפתרון הסופי}
	\begin{enumerate}
		\item \textbf{כמות היהודים} – הנאצים העריכו את כמות היהודים בברימה''ש בכחמישה מיליון. זה מספר לא נכון, אבל זה לא משנה – יש מיליוני יהודים בברית המועצות. 
		\item \textbf{יהודים \textit{קומוניסטים}} – יש פה מלחמה אידיאולוגית כפולה. חלק גדול ממנהיגי הקומוניזם בברית המועצות היו יהודים. 
		\item \textbf{אובדן בלמים מוסריים} – נבע מהריחוק מהבית, מרצף הנצחונות בשנתיים הראשונות, ומדעת הקהל הגרמנית. 
		\item \textbf{לכוון לרצונו של הפיהרר} – תשתדלו פחות לדבר על זה. הרעיון, הוא שאנשים בשטח רצו למממש את האידיאולוגיה של היטלר. 
	\end{enumerate}
	
	\subsection{שלבי הפתרון הסופי}
	\begin{multicols}{2}
		\begin{enumerate}
			\item בורות ירי
			\item משאיות גז
			\item חלמנו
			\item ועידת ואנזה (יפורט בפרק נפרד \textbf{למרות היותה חלק משלבי הפתרון הסופי})
			\item מבצע ריינהרד
			\item בירקנאו (אושוויץ)
			\item צעדות המוות
		\end{enumerate}
	\end{multicols}
	יש כאן מדרג. ברור שהמדרג מתאר את הממדים של הרצח, אך לא רק זה. נבחין שככל שהמדרג מתקדם, באופן כללי, האינטראקציה עם הנאצים פוחתת – ``הם לא רואים אותם בלבן של העיניים''. נסכם: שני דברים משמעותיים קורים. שינוי בכמות (משתכלל אממ capacity יותר גדול) והגדלת הריחוק. 
	
	
	\subsubsection{בורות ירי}
	בורות הירי מתחילים ממש בהתחלה של כיבוש ברית המועצות. זמן קצר מבצע ברברוסה, תוך כדי ההתנהלות שלו, באיזורים שנכבשו. 
	
	כדי להבין את הנושא לאומק, נדבר את אייזנצגרופן (בתרגום מילולי: ``עוצבות המבצע''). אלו יחידות SS מובחרות שנלוות לכוחות של הורמכט. כאשר השטחים נכבשים, תפקידם לנהל אותו. הם קיבלו פקודה בשם \textit{פקודת הקומיסרים}, שבה היה כתוב ``להשמיד את אויבי המשטר'' (לא בהכרח יהודים), וכתוצאה מכך נבנו בורות הירי. 
	
	\textbf{תזכורת: }הורמכט הוא הצבע, ה־SS וה־SA אינם. 
	
	בורות ירי הוא דבר ``בסיסי''. לוקחים אנשים מהבית ליער הקרוב, נותנים להם לחפור בור / משתמשים במקומות בהם היו משוחות לחימה. אומרים להם להתפשט ויורים בהם לתוך הבור. בבורות ירי, במקרים רבים, אנשי ה־SS היו מיעוט ובפועל המקומיים השתתפו ברצח. כשמדובר בקהילות קטנות, הקהילה כולה נלקחת ליער. כמות הניצולים מבורות הירי היא אפסית. אלו שניצלו נכלאו בתוך כיסי אוייר ולא מתו מהירי, ובלילה יצאו וברחו. 
	
	הדבר החשוב: הוא הקרבה בין הרוצח לנרצח. דבר שני, הרצח מתקיים ליד הבית, ואין יכולת להסתיר את זה מהסביבה. אחד הדברים המרכזיים אצל הנאצים הוא \textit{הסוואה והטעייה}. זה לא מאפשר לא את זה ולא את זה. 
	
	גם ככל שנקדם בתהליך, \textbf{בורות ירי ממשיכים להתקיים} במלחמה. בכל פולין יש בורות ירי. זה המשיך לקרות לאורך כל המלחמה, בכל מיני מקומות. אבל כשיטה, זה התקדם. 
	
	\subsubsection{משאיות גז}
	``ואז החליטו לנסות משהו אחר''. מכניסים את האגזוז של המשאיות פנימה, ``נוסעים נוסעים נוסעים עד שהאנשים מתים''. גם זה התרחש במקומות המגורים של האנשים. התסובבו בערך 20 משאיות כאלו. 
	
	\subsubsection{חלמו}
	זהו מחנה ההשמדה הראשון. הוא לא ממש מוכר. המחנה נבצע באיזור לודג'. במחנה הזה רצחו בהתחלה באמצעות משאיות גז. אז מהוא ההבדל בין חלמו לבין משאיות הגז? זוהי הפעם הראשונה שישנו מקום יהודי שאליו מביאים אנשים כדי לרצוח אותם, והוא נבנה כדי לרצוח אנשים. זהו עדיין מחנה השמדה. 
	
	\textbf{תזכורת: }לודג' היה גטו צמוד לגבולות גרמניה. 
	
	יש כמה סיבות שלא המשיכו עם המתודות הללו: 
	\begin{itemize}
		\item \textbf{עלות כלכלית} – רצח בבורות ירי הוא רצח יקר. זה מבזבז כדורים. היו נסיונות להעמיד אנשים צמודים ולירות בהם עם כדור אחד כדי לא לבזבז תחמושת. 
		\item \textbf{איטי}
		\item \textbf{לא עמדו בזה מבחינה נפשית} – לא כל הנאצים הם פסיכופטים. הם אנשים רגילים. הם היו שיכורים כל הזמן כדי להיות מסוגלים לקחת חלק בזה, והם ``שתו את נפשם לדעת''. מדובר הן על האייזנצבורפן והן על משתפי הפעולה. 
	\end{itemize}
	
	\subsubsection{מבצע ריינהרד}
	במהלך המבצע הקימו שלושה מחנות השמדה: 
	\begin{itemize}
		\item בליז'ץ
		\item סוביבור
		\item טרבלינקה
	\end{itemize}
	המחנות הללו הוקמו בחלק המזרחי, ונועדו לרצח יהודי פולין (אם כי בהמשך הביאו אליהם אנשים ממקומות אחרים). אין קרמטוריומים (משרפות) במחנות האלו. הרגו את האנשים בתאי גזים ואז כיסו אותם בסיד כדי למנוע התפשטות מחלות. באיזור 95\% מהאנשים נרצחו עוד באותו היום, ולא הייתה כמעט סלקציה. בסוביבור למשל, המחנה היה מחולק לשניים, האיזור של הרצח והאיזור השני. 
	
	\textbf{המלצה של שרית לסרט. }הבריחה מסוביבור. 
	
	
	מובן שכאשר העלו אותם על רכבות למחנות השמדה, עשו הצגה שלמה של הסוואה והטעייה. אמרו להם לקחת מזוודות, כאשר הגיעו ביקשו מהם לכתוב מכתב למשפחה, להביא כמה שיותר דברים, ושיעבירו אותם להתיישבות במזרח. ביקשו מהם להתפשט בתואנה שמכניסים אותם למקלחת, ואז זרקו פחית ציקלון B לתוך החדר. זהו גז שבמגע עם האוויר הוא מתפשט, ואז אנשים נחנקים ומתים בתאי הגזים. 
	
	\subsubsection{אושוויץ־בירקנאו}
	אשוויץ הוא קונפקס של מחנות. הוא איננו מחנה אחד. ``מצד החיים'' שרואים בטלווזיה, זה בין אושוויץ אחד לאושוויץ בירקנאו. יש לנו 3 אושוויצים – אושוויץ 1 שהיה מחנה צבאי פולני עוד לפני הנאצים (עם השלט ``העבודה משחררת'', ויש בו בניינים מאבנים), אושוויץ 2 (בירקנאו, שם אין בניני אבנים, והוא הוקם כמחנה השמדה), ואשוויץ 3 (כרגע לא בחומר). בירקנאו מחנה ענק, והוא מהווה את פסגת הטכנולוגיה של הנאצים. במחנה הזה הנאצים מגייסים את הטכנולוגיה הכי מתקדמת של התקופה, כדי לבנות מפעל לרצח. רופאים ביצעו שם ניסויים רפואיים (לא רק מנגלה). משום שבירקנאו תוכן בצורה ``מושלמת'', הוא היה הרבה יותר יעיל. יש בו 4 קומפקסים של תאי גזים וקרמטורים ביחד. כלומר, אדם נכנס למלתחות להתקלח, ובאותו הקומפלקס מעבירים אותו למשרפות. אף אחד מהם לא נישאר שלם, ומי שהפציץ אותם היו הנאצים בבריחתם. זהו מפעל רצח, ``לא לצטט אותי אבל נכנס בן־אדם יוצא אפר''. סוביבור וטרבלינקה לא כל כך יעילים. בבריקנאו נרצחו 800K יהודים, רובם מפולין. 
	
	למה מיידנק לא כתוב כאן? כי הוא לא הוקם כמחנה השמדה. הוא הוקם כמחנה שבויים. מיידנק עבר הסבה למחנה השמדה ואז הקימו בו תאי גזים וקרמטורים. זה המקום היחיד במיידנק שבו הכל נשאר. המחנה הוקדם בתוך העיר לובלין (יכול להיות שטעיתי באיות), בניגוד למחנות אחרים שהוקמו בתוך יער. הבית של האנשים בעיר הוא 100 מטר מהמחנה, ואנשי לובלין ``לא יכולים להגיד שהם לא הבינו מה קרה''. שמעו את היריות, הריחו את שריפת הגופות. 
	
	
	כל מחנות ההשמדה הוקמו בתוך פולין, פרט לבירקנאו שהיה קצת על הגבול. הסיבה היא שהוא היה במיקום נוח ביחס לפסי רכבת, עם תשתית שהייתה קיימת עוד לפני הנאצים. המחנה הזה פעל עד סוף 44'. פסי הרכבת שנכנסים לתוך הגטו זהו ``שידרוג'' שהוכנס לגטו בעת השמדת יהודי הונגריה (כי היו המון יהודים). 
	
	באושוויץ־בירקנאו ישנה סלקציה. הסלקציה נעשית בבירקנאו וחלקם הולכים לאושויוץ 1, ומי שנשאר שם, נשאר כדי לעבוד. 
	
	\subsubsection{צעדות המוות}
	אף אחד לא באמת יודע למה הנאצים ביצעו את צעדות המוות. הרעיון: ברגע שבורחים מאשוויץ, הנאצים לקחו איתם את מי שמסוגל ללכת ואיינו על סף מוות, ומצעידים אותם לכיוון גרמניה (מערבה). הם הולכים בחורף (באותה השנה, היה נוראי) עשרות ואף מאות קילומטרים, ועוברים בישובים וכפרים. כל מי שנעצר, נורה. מי שנפל, ווידאו הריגה. יש כאלו שטוענים שהמטרה הייתה להעביר אותם לגרמניה לוהעביד אותם בכפייה. חלק לא קטן מבית שורדי אושוויץ נרצחו בצעדות המוות. במקומות מסויימים היו כמרים (בעיקר) שאספו את האנשים וקברו אותם קבורה ראויה בקבר אחים. על הקבר כתוב את המספר שעל היד (שכן הם לא ידעו את השם של האנשים). בשנים האחרונות זיהו את האנשים (אושוויץ היה מקום מסודר וידעו למי שייך כל מספר) וכתבו את השמות על הקברים. 
	
	
	אז, אם שואלים שאלה, האם הפתרון הסופי תוכנן מראש, התשובה היא לא. מכל שיטה למדנו והשתכללו כדי לבנות את אמצעי ההשמדה ההמונית הבא. הרעיון היה עוד במיין־קאמפף, אך הביצוע לא תוכנן וארך זמן. 
	
	
	
	\subsection{הלמידה מהשלבים השונים בתהליך ההשמדה}
	צעדות המוות עקרונית נכללות בפתרון הסופי אבל הן די יוצאות דופן מסיבות שנציין בהמשך. 
	\begin{itemize}
		\item \textbf{תיעוש וייעול של הרצח}
		\item \textbf{ריחוק של רוצח מהנרצח: }התהליך הגיע לשיא באשוויץ (בירקנאו). הרוצחים לא רואים את הרוצח בלבד. יש לו 4 מתחמי קרמטוריום, ``יש שם תהליך שבו בן אדם נכנס חיי, ויוצא אפר''. לרוצח יש כיסא שעליו הוא יושב, פותח חלון, זורק פחית ומסיים. הריחוק הזה איפשר את הדה־הומניזציה של היהודים. ``מגייסים את כל הכוחות האינטלקטואלים כדי לגרום לזה לקרות'' – הכל היה תוחכם מבחינה טכנולוגית. 
		\item \textbf{הרחבה גיאוגרפית: }בהתחלה התחיל משטחי הכיבוש שנכבשו במבצע ברברוסה, באזורים המערביים, ואז התקדם מערבה (חלמו נמצא בלודז', במערב פולין). לאושוויץ מגיעים מכל מדינה באירופה, כולל יוון. 
	\end{itemize}
	
	ליהודי יוון (שידעו לדינו וקצת יוונית) התקשו מאוד להבין את הפקודות הגמרניות, ולכן הם סבלו קשות מהתעללות/רצח – הם לא הבינו את הפקודות הגרמניות. 
	
	\subsection{ועידת ואנזה}
	הועידה התקיימה ב־20 בינואר 1942 בברלין, כ־7 חודשים \textit{אחרי} שהחל הפתרון הסופי, וכמליון יהודים כבר נרצחו בשטחי ברימה''ש. הועידה \textit{לא} נועדה להחליט על הפתרון הסופי, אלא ליעל, לארגן, לתכנן ולהחיש את הפתרון הסופי. \textit{אין החלטה רשמית מתועדת על הפתרון הסופי}, וכנראה הדברים נאמרו באופן לא ישיר. היטלר לא נכח בוועידה כדי שלא יווצר קשר בינו לבין תהליך ההשמדה. 
	
	הזמינו אנשים לוילה בה אכלו ושתו. בוועידה נכחו ראש הגאסטפו, ראש שירות הבטחון, הימלר (ראש ה־SS) ואייכמן. אייכמן היה האדם היחיד שאיננו היה משכיל. גם ראשי המשטרה מחבלי הכביוש השונים במזרח, וראשי מנהל נוספים (\textit{מבצעים}). זוהי וועדת ביצוע, שרוצים להסיק ממנה דברים ברמה הביצועית. מטרותיה: 
	\begin{itemize}
		\item לתת אישור רשמי להשמדת היהודים, דהיינו לרתום את כל הרשויות והמערכת האזרחית לתיאום ולמימוש המטרה. 
		\item יידע את אנשי המפתח על המבצע המיוחד שבראשו יעמוד היידריך היינרג, ולהבהיר להם שהם נדרשים להרתם ולפעול ע''פ ההנחיות. נוסף על כך קבעו מי הפוסק במקרה של חילוקי דעות בין הרשויות. 
		\item להדגיש שהפתרון הסופי מתייחס לכל יהודי אירופה ללא הגבלות גיאוגרפיות (שטחים כבודים, מדינות גרורות, ניטרליות, בעלי ברית בעתיד ובהווה וכו'). 
	\end{itemize}
	
	\subsubsection{פרוטוקולי ועידת ואנזה}
	מה המשמעות והחשיבות של הנתונים בטבלה בעמודים 236-238? 
	
	תשובה שלי – מטרת הנתונים היא לארגן את השמדת היהודים, כך שכל מנהל איזורי יוכל לדעת כמה יהודים צריכים לצאת משטחו ולהגיע אל מחנות ההשמדה. חשיבותם היא שקיום הטבלה הופך את התהליך למסודר ורשמי, תוך הגדרת מטרות מספריות ברורות. הן מבטאות את הטוטאליות של הפתרון הסופי, בכך שהיעד להשמדת יהודים מכל מדינה, היא כמות היהודים באותה המדינה. יש תיעוד אף של המדינות עם מיעוט היהודים, כגון אלבניה וכו'. איפה שיש מידע מדויק (אוקראינה לדוגמה), סופרים במדויק כדי לדעת כמה להשמיד. 
	
	דוגמה לנתונים: הרייך הישן, 131K, ברה''מ 5000K, אנגליה 330K וכו', אגטוניה חופשית מיהודים, אלבניה 200, צרפת בלתי כבוש 700K, צרפת כבוש 165K, נרווגיה 1.3K, אוקראינה 2,994,685 וכו'. בלראוס 4K, שוויץ 8K וכו'. הסכום הכולל הוא 11M, כולם מיועדים להשמדה. כנראה המספר של חמשיה מליון בברית המועצות לא היה מדויק, כי הייתה תזוזה בין ברימה''ש לפולין וכנראה ספרו כפלים מאלו שברחו מזרחה. 
	
	חזרה למה שאומרים בכיתה. ממספרים כמו ``אסטוניה חופשית מיהודים, אלבניה 200'' – עד היהודי האחרון. כולם מיועדים להשמדה. התפישה היא שילכו עד ההר האקראי באלבניה ובלבד שישמידו אותם. באוקראינה נבחין שהמספרים מדוייקים להפליא – זה לא בערך, סופרים עד היהודי האחרון, וייוודאו שכולם ילכו להשמדה. היו מקומות שבהם היה אפשר לדעת יותר בקלות כמה יהודים יש, באמצעות פנקסי קהילות – בכל קהילה היה פנקס שתיאר אנשים שנולדו ומתו. הדבר המיוחד בשוודיה, שוויץ וכו' – אלו מדינות ניטרליות, והם סופרים יהודים גם שם. הם ספרו יהודים גם באנגליה, מקום שהם בכלל לא כבשו, וכנ''ל על השטחים הלא כבושים בצרפת (זה היה משטר וישי שהיה משטר בובות, הוא היה כבוש בפועל). 
	
	צרפת לאורך ההיסטוריה שלטה על מדינות כמו טוניס ואלז'יר, אתיופה ולוב ע''י איטליה, סוריה ולבנון גם צרפת, וכו'. הם ספרו גם את יהודי צפון אפריקה. מדובר על משהו טוטאלי. ביטויי הטוטאליות הם: 
	\begin{itemize}
		\item לצד 5 מיליון יהודים, יש 200 מאלבניה. רצו להשמיד את כולם עד האחרון
		\item נכללו יהודים ממדינות ניטרליות ומדינות שתרם נכבשו. 
		\item הטבלה כללה את יהודי צפון אפריקה בקולוניות. 
	\end{itemize}
	
	לא הצליחו להגיע להסכמות בדבר בני תערובת. ההחלטה הזו לא הוחלטה, ומה שקרה הוא שגורלם של בני התערובת נקבע ע''י המפקדים המקומיים באותו האיזור. 
	
	\subsubsection{סיכום ועידת ואנזה}
	הועדה הייתה מנהלית ומתאמת, שמנסה להבין מי יחליט במקרה של חילוקי דעות (לאייכמן היה תפקיד מרכזי בכך, למרות שטען שהיה רק פקיד שקיבל פקודות). היא נחשבת שלב בפתרון הסופי, והיא מעידה על כך ש־: 
	\begin{itemize}
		\item זוה רצח עולמי טוטאלי ולא מקומי
		\item כל המערכות רתומות לרצח
		\item הרצח יהיה מאורגן ע''י רשות מרכזית אחת
		\item ההשדמה תהיה שיטתית ותעשייתית. 
	\end{itemize}
	
	פרוטוקול הוועידה הוא המסמך הכי קרוב למסמך רשמי המאשר את השמדת העם היהודי. 
	
	\section{תנועות הנוער במלחמה}
	\textbf{הגדרה. }מקום חברתי, חינוכי, ופוליטי. היום לתנועות הנוער יש פחות ערך פוליטי, אך בעבר, העולם היה עולם אידיאולוגי, והשתייכותך לתנועת נוער מסויימת – אומר שההורים שלך היו מזדהים למפלגה פוליטית מסויימת. באופן כללי, אם היית שייך למפלגה מסויימת, למפלגה היה עיתון, וקופת חולים, ומערכת חינוך, וכו'. זה היה דבר הרבה יותר כולל והכל סבב סביב לאיזו אידיאולוגיה אתה משתייך אליה. המאה ה־20 קרויה ``המאה של האידיאולוגיות הגדולות''. 
	
	
	\subsection{הבונד (התנועה הסוציאליסטית)}
	\textbf{לא במיקוד}. 
	אנחנו נדבר בעיקר על תנועות הנוער הציוניות. הסיבה: כי אנחנו חיים במדינה ציונית. אז לצורך הנקודה לא נדבר על ``הבונד''. הבונד לא קיים אחרי מלחמת העולם השנייה כי רובם נרצחו בשואה. הסיבה: הדתיים והחרידים, אם לא היו שייכים למפלגות דתיות, היו שייכים לבונד. זו תנועה קומוניסטית ולכן לא ציונית, ולא מרחיב עליה. 
	
	קונטקסט: מנהיג תנועת הבונד בורשה מת לפני 16 שנים, והגיעו לשם מכל העולם. חוץ מישראל. הציונות והבונד זרים. 
	
	רוב האוכלוסיה היהודית בפולין הייתה שייכת לבונד, והציונים היו מיעוט, אך מי שלרוב הנהיג את המרידות היו הציונים והבונד לקח חלק (במקומות אחדים אף כנראה הנהיג). בגטאות הגדולים, בכולן, הציונים הנהיגו את המרדים. 
	
	(ידע כללי שלי: מקור הבונד בגרעין של יהודים בברית המועצות, והייתה לו השפעה מכרעת גם על המהפכות בברית המועצות עצמה)
	
	\subsection{תנועות הנוער הציוניות}
	תנועות הנוער ידעו לפני כולם מה קורה במקומות אחרים. דיברנו על הקשריות (לא היו קשרים, כי גברים אפשר לזהות יהדות אם הוא נימול). הן היו עוברות בין הגטאות ומעבירות מידע. ככה הגיע המידע על ההשמדה לגטאות. תנועות הנוער הובילו בשלבים מאוחרים יותר את המרידות בגטאות. 
	
	עם עיבוש פולין ברחבה רוב הנהגת תנועת הנוער מזרחה. לאחר תקופה, חלק מהמדריכים חזרו לגטאות. בינהם – מרגכי אנילביץ' מ''השומר הצעיר'' שינהיב בעתיד את המרג בגטאות. תנועות הנוער באותה התקופה ניהלו מטבחים וכו', כארגון הררכי שיכול לגייס אנשים רבים. \textbf{למה הן היו רלוונטיות: }
	\begin{itemize}
		\item ערעור המסגרות החברתיות הקיימות
		\item ההנגה היהודית המבוגרת וההורים התקשו להתסגל למצב החדש. 
		\item חברי תנועות הנוער הצליחו להסתכל אל המצב החדש
		\item הקשר הקבוע של הצעירים עם תנועת התנוער הפכו את תנועת הנוער למקום שמקנה בטחון לצעירים מעורערים. 
		\item חברי תנועות הנוער לא היו אחריים על משפחה ולכן לקחו יותר סיכונים. 
	\end{itemize}
	
	מה הם עשו: 
	\begin{itemize}
		\item לרוב, מתחו חברי תנועות הנוער ביקורת על מדיניות היודנראט. בעיקר – בפתרון הסופי. 
		
		חברי תנועות הנוער הבינו יותר מהר מהאחרים, ש''להוריד את הראש'' ו''נתעלם וזה יעלם'' לא יעובד. בין היתר בגלל שיש להם מידע מוקדם יותר לאזורים האזרחיים, וידע שהתחילה ההשמדה. המבוגרים לא האמינו ברובם, ואלו שהאמינו, קיוו להפטר ממנה בשיטות הישנות. 
		\item אם לפני המלחמה תנועות הנוער הציוניות חינכו לעליה לא''י, כרגע מכחנים לטובת כלל הציבור. 
		\item הקמת מרכזי עזרה, חינוך ותרבות לתושבי הגטו (לא רק הילדים, אך בעיקר הילדים)
		\item יצירת קשרים עם גטאות אחרים – גטו ורשה שימש מרכז ממנו יצאו שליחים לגטאות אחרים כדי לפקח על קיום ופעילות תנועות הנוער ברים אחרות, ולהעביר מידע. הקשרים (ובעיקר קשריות) היו בעלי חזות ארית ולרוב זהות נוצרית שאולה. רבים מצאו את מותם. 
		\item פרסום עיתונות מחתרתית – חרף האיסורים. [שרית מסבירה: ``עיתון זה מן משהו כזה, שהיו מדפיסים על נייר, והיו בו כתבות, וטורים, ומגזינים'']. 
		
		למה העיתונות נעשתה ע''י תנועות הנוער? כי להן הייתה את המידע. העיתונים לרוב היו על נייר דק, או עמוד אחד. 
	\end{itemize}
	
	``חשוב שתבינו את זה – ילד שנכנס לגטו הוא לא ילד''. ילדים בגטו במקרים רבי םדאגו למשפחתם, כלומר הביאו אוכל (נאמר דרך פרצות בגדרות ותעלות), דאגו לילדים בבית וכו'. לפעמים בידי ילד בגיל 8-9 הייתה האחריות בנוגע לאיזה אוכל יהיה למשפחה שלו. בגטו הורים לא יכלו לשמור על ילדים, ואף לפעמים המצב היה הפוך. זאת, בין היתר, כי רבים מההורים היו נשלחים לעבודות כפייה. את תפקיד ראש המשפחה תפסה פעמים רבות האישה. לא הייתה כמעט שום מסגרת לילדים. 
	
	\subsubsection{שתיקת הארכיון}
	
	ביום רביעי נראה את ``שתיקת הארכיון''. זהו סרט שצולם בגטו ורשה. קחו בחשבון ב־95\% ממה שצולם בגטאות, היו צילומים של הנאצים. הסרט מעולם לא הופק. שליטת הארכיון מנסה להסביר את הצילומים, ואיך הם רצו להציג את הגטו, היהודים ואת עצמם. 
	
	\subsection{המרידות בתנועות הנוער}
	\subsubsection{אבא קובנר ודבריו}
	מנהיג התנועה הציונית, אבא קובנר. וילנה, אמר שמשונים אלף היהודים ביורשלים־דלתיא (וינה) שרדו רק עשרים אלף. ``איה מאות הגברים שגורשו?'' ``מי שהוצא משער הגטו – לא חזר עוד. כל דרכי הגאסטפו מובילות לפונאר, ופונאר זה מוות!''. פונאר אילו בורות ירי ענקיים ליד וינה, שנרצחו בהם עשרות אלפי מיהודי וינה. ``הפרו את האשליה, המיואשים: ילדכם, בעליכם, נשיכם – אינם עוד! אתכולם רצחו שם. היטלר חושב להשמיד את כל יהודי אירופה. יהודי ליטא נעמדו בתור הראשונים. אל נלך כצאן לטבח``. 
	
	באמרתו, ``אחים, טוב לפול כלוחמים בני־חורין מלחיות בחסד מרצחים'', מופיעה בחירה – בחירה איך למות. בציטוט הזה מובא כחצי שנה אחרי מבצע ברברוסה. 
	
	הם נלחמו במדירות, על החופש למות בכבוד. המרידות תוכננו בצורה כזו שאין כוונה להמשיך לחיות אח''כ. ברור היה לכולם שהם נלחמים בכוח גדול מהם וחזק מהם בסדר גודל. 
	
	\subsubsection{על המורדים}
	מאפייני מנהייג תנועות הנועד והמורדים: 
	\begin{itemize}
		\item צעירים בגיל 17-25 בתחילת המלחמה
		\item רווקים וללא ילדים
		\item בעלי השקפת עולם ערכית מוצקה
		\item הבינו את העולם של האידיאולוגיות הגדולות בו הם חיו
		\item חשו מחויבים באופן מוחלט לתנועה
	\end{itemize}
	
	לעומתם, היו היודנראטים, שהחזיקו בתקופה הישנה – ננסה לשרוד כמה שיותר, עד שבעלות הברית ישחררו. חברי תנועות הנוער (הציוניות) לא האמינו בכך, והיו בטוחים שהנאצים ירצחו אותם עוד לפני שבעלות הברית יגיעו אליהם. 
	\subsection{מטרות הלחימה בגטאות}
	\begin{itemize}
		\item הרצון לנקום בגרמנים. 
		\item למות בכבוד ולא ללכת ``כצאן לטבח''. 
		\item ``למען שלוש שורות בהיסטוריה'' – תפישה שהייתה לכל חברי תנועות הנוער, כדי שהדורות הבאים יידעו שהייתה התנגדות יהודית לנאצים, והיהודים לחמו ומתו כבני חורין. 
		\item רצון לשמש דוגמה ומופת לנוער בא''י. 
	\end{itemize}
	
	אין פה נסיון להנצל. עם זאת, הם לא התאבדו, הם היו צריכים לבחור בין למות בדרך אחת לבין למות בדרך אחרת. בסופו של דבר בסיום המרידות, ניצלו מעטים דרך תעלות הביוב וכו'. אבל זו לא מטרה מוגדרת. 
	
	
	\subsubsection{קשיי המרד}
	המרידות לא החלו ברגע שהתחיל הפתרון הסופי. הפתרון הסופי התחיל בברה''מ ב־1941 והמרידות ב־1943. יש לזה כמה סיבות. 
	\begin{itemize}
		\item הטעייה והסוואה מצד הנאצים. 
		\item זה לא התחיל בגטאות. זה התחיל במשאיות. 
		\item גם כאשר כבר היו מחנות, הם התנהגו כאילו הם הולכים ל''התיישבות מחדש במזרח'', הם הביאו מזוודות, וכאשר הגיעו למחנה הכניסו אותם למקלחות עם סבון וזרקו אז. 
		\item ה''יהודי הישן'' באג'נדה ובתרבות שלה לא חברה מורדת, אלא חברה שמורידה את הראש עד שיעבור זעם. זו הייתה ההתנהגות ביחס לפוגרומים. 
	\end{itemize}
	הצעירים שהתחילו לקבל מידע מגטאות אחרים באמצעות קשריות (ומאוחר יותר, אפילו ממחנאות ההשמדה עצמה, כמו טרבלינקה וסובידור, מאושוויץ ברחו ממש מעטים). אך גם אז אנשים לא האמינו להם. פרוטוקולים של אוושויץ הגיעו אף לשליטי מדינות אחרות, ולא האמינו להם. זה לא נתפס באותה התקופה, ולא היה תקדים למחנות שנועדו לרצח עם. גם אחרי השואה, אנשים בארץ התקשו להאמין שהשואה אכן התרחשה בממדים הללו. התחילו להקשיב להם במשפט אייכמן. 
	
	גם כאשר הידיעות התחילו, היו עוד כמה בעיות. 
	\begin{itemize}
		\item \textbf{מחסור בנשק} – הגטו היה סגור ולכן היה קשה להשיג נשק. היה ייצור מקומי של נשק ובקבוקי תבערה (בקבוקי מולוטוב), והפולנים הבריחו קצת. 
		\item \textbf{חוסר ידע קרבי} – באופן מסורתי היהודים לא התגייסו לצבא. רק מעטים שירתו בצבא הפולני, והיה קשה להתאמן בגטו. 
		\item \textbf{התנגדות פנימית} – הרבה מהאנשים לא היו מעוניינים במרד, ואף יהודים הסגירו את חברי המחתרת. הסיבה? המשמעות של מרד היא מוות לכל הגטו. היה חשש שבמידה והנאצים יגלו את אשר מתרחש הם יסגירו וישמידו את כל הגטו. 
	\end{itemize}
	
	\subsubsection{מיקום המרד}
	\begin{itemize}
		\item עם במרד היה מתרחש בגטו – אין סיכוי להצלחה, והגטו על תושביו יחוסל. עם זזאת, נישאר עם המשפחה (עד שישמידו אותה בענישה קולקטיבית). 
		\item לצאת להלחם עם הפרטיזנים ביערות (עם כי היו גטאות בלי קרבה ליער שם זה לא התאפשר). כאן יש סיכוי שהגטו לא יחוסל, ואף סיכוי להישאר בחיים. אבל לא כולם יכולים לצאת והמשפחה נשארת מאחורה. 
		
		הלחימה עם הפרטיזנים לא פשוטה. חלק מהפרטיזנים היו אנטישמיים והרגו את היהודים שבאו להלחם. אין מה לנסות להגיע לפרטיזנים אם אין לך נשק. אבל האופציה השנייה לא ממש עדיפה. 
	\end{itemize}
	
	\subsubsection{עיתוי המרידות}
	משום שהמרד הוא מוות וודאי, לרוב עיתוי המרד היה כאשר הנאצים החליטו לחסל את הגטו (ואי לך תלוי בנאצים ולא בנו) דהיינו ''לפנות אותו``. 
	
	עם זאת, יש יוצא דופן, הוא מרד גטו ורשה. 
	
	\subsubsection{מרד גטו ורשה}
	גטו ורשה הוקם על 3\% משטח ורשה, שהיו 30\% מתושבי העיר. רובם הושמדו בטרבלינקה, והחלו להגיע ידיעות על הרצח בטרבלינקה. זה לקח זמן כי היהודים נרצחו ברגע שהגיעו לטרבלינקה. הייחוד של מרד גט וורשה: 
	\begin{itemize}
		\item במרד לקחו חלק כל תושבי הגטו, לא רק 700 לוחמים, אלא מרי עממי של 60K יהודים. הוא היה המרד העירוני הראשון באירופה הכבושה ע''י הנאצים. זה לא אומר שכל הגטו נלחם, אלא שיתף פעולה. הם עזרו לבנות את התשתית והסתתרו מתחת לאדמה בבונקרים, והיו כוח עורף שסייע בדברים אחרים. 
		\item הדרך בה הנאצים הצליחו להכניע את הבונקרים, היה באמצעות להביורים להוכנסו לבונקרים. לקח מה־19 באפריל ועד ה־16 במאי, כחודש ימים (לפורפורציות פולין נפלה תוך שלושה שבועות). 
		\item הלוחמים בגטו לא הכינו דרך נסיגה. יש ארגון מטורף של המרד עם תכנונים ארוכים מראש, והיה ברור שנסיגה היא איננה מטרה, אנשים הצליחו בסוף לברוח משם, בסיום הלחימה, אבל זו לא הייתה מטרה של המרד ולרוב הם ברחו דרך תעלות ביוב. 
		\item באופן יחסי הרבה גרמנים נהרגו. הם אפילו הצליחו להשמיד טנק שנכנס. למעלה ממאה גרמנים מתו במרד. 
	\end{itemize}
	הנאצים לא ציפו לזה, והייתה הפתעה מצדם. 
	
	מרד גטו ורשה הוא לא אירוע שנשאר בין הגטו לבין הנאצים, אלא הגיע לעיתונים ועודד מרידות נוספות בגטאות נוספים. 
	
	מרדכי אנילביץ' פיקד על הגטו, ונהרג בו. במכתבו האחרון הוא כותב שהוא ששאיפותיו התמלאו ושהצליח להיות מ''ראשוני הלוחמים היהודים בגטו''. 
	
	הגטו הורכב מכל מני תנועות נוער, לרוב ציוניות, אך גם הבון נלחמו בגטו ורשה. גם בגטו ורשה הנרחב היו פערים אידיאולוגיים, והוא היה מורכב בהרבה מאיך שהוא מוצג בד''כ. ההחלטות שבוצעו במרד שיקפו במקרים רבים את זהות היהודי החדש, בהתאם לאידיאולוגיה של התנועות הציוניות. 
	
	בגטו ורשה לחמו תנועות (גם ציוניות) נוספות פרט ל''אייל'' הציונית. אבל הם די נמחקו מההיסטוריה. גם בין התנועות הציוניות היו פערים אידיאולוגים, נאמר בין ההגנה, הפלמ''ח האצ''ל והלח''י. במרד גטו ורשה גם לחם ארגון ובשם האצ''י (הארגון הצבאי היהודי). אפילו בסיטואציה הקיצונית הזו האצ''י ואייל לא התאחדו ונלחמו לבד. כאשר הסתיים המרד, מי שסיפר את הסיפור במדיתנו הקדושה – היו שורדי אייל, ההנהגה (מפא''י) בארץ, והאצ''י נמחקו מההיסטוריה. בשנות ה־90 בערך התחילו לדייק את הסיפור, ויצא הסרט ``הסיפור של סופר'' (לפני כמה שנים). האצ''י היה לארגון החמוש ביותר בגטו בזכות המנהרות שבנה. לפי הסרט, בן גוריון לא אישר להכניס ספרי לימוד עם המילים אצ''ל, בית''ר וכיו''ב. 
	
	שרית מדגישה כאן דבר – כל מורה אשר נכנס לכיתה מגיע עם מטען ערכי/אידיאולוגי, לכל ספר לימוד יש אג'נדה, לכל תוכנית לימודים יש אג'נדה. לא להאמין למה שמספרים לכם כסיפור הנכון ולא כסיפור המלא. ''הסיפור הוא תמיד מורכב``. 
	
	\section{ההתנגדות יהודית}
	ההתנגדות היהודית מחולקת לארבע שיטות שונות: 
	\begin{enumerate}
		\item מרד בגטאות
		\item יהודים חיילים בצבאות בעלות הברית
		\item פרטיזנים
		\item מרד במחנות השמדה
	\end{enumerate}
	אנחנו נלמד רק את השניים הראשונים, מתוכם את (1) למדו בפרק הקודם. שרית תבהיר לקראת המבחן האם (2) נכלל במבחן הקרוב או לא. 
	
	\subsection{חיילים יהודים בצבאות בעלות הברית}
	למעוניינים – מוזיאון בלטרון. שמציג את הלחימה הזו בדיוק. הוא אינטראקטיבי עם סרטונים ומסכי מגע. צריך תיאום מראש. 
	
	רובם – פשוט אזרחים (המקרה של הבריגדה בארץ והצנחנים בה מקרה מיוחד). מה יש לציינם בצורה מיוחדת? הרי מרבית אזרחי המדינות התגייסו. בפועל יש דברים מיוחדים שכדאי לעבור עליהם, בפרט הסיבות להצטרפות שלהם: 
	\begin{itemize}
		\item פטריוטיות למדינה שלהם
		\item חובת גיוס (תלוי בצבא)
		\item התנדברות מתוך רצון לנקום בנאצים שהשמידו את בני עמם
		\item יהודי א''י שהתנדבו לצבע הבריחי במסגרת ``הבריגדה היהודית'', שקיוו שהבריגדה תהפוך לגרעין של הצהבע העברי כאשר תוקם מדינה יהודית בא''י. בריטניה לא התלהבה מלהקים את הבריגדה, היא הוקמה הנהגת הישוב היהודי. נוסף על כך היו הצנחנים. 
		
		
	\end{itemize}
	
	בפועל, באמת ראינו גיוס גבוהה יותר (ביחס לאוכלוסיה) של יהודים למלחמה. 
	
	כמליון וחצי יהודים נלחמו בכל החזיתותת, כרבע מיליון לוחמים יהודית נהרגו. נפרט קצת כל צבא: 
	\begin{itemize}
		\item בצבא האדום: לחמו כחצי מיליון יהודים. מתוכם 150 זכו בעיטור הגובהה הגבוהה ביותר שניתן שם, ``גיבור בריה''מ''. חלקם נמנו על משחררי מחנות ההשמדה ומהלוחמים שכבשו את ברלין. היהודים היו 10\% מהלוחמים בעור שיעורם באוכלוסיה היה 2\%. מתוכם 200 אלף (כ־40\%) נפגעו בקרב. אחת הלוחמות המפורסמות היא סרן פולימה גלמן, נווטת, שביצעה 869 משימות ב־1300 שעות טיסה והטילה 113 טון פצצות על ריכוז אוייב. זכתה לתואר ``גיבורת ברית המועצות''. בצבא האדום היו לא מעט חיילות ומפקדות באופן כללי. 
		
		לצבא האדום הייתה חובת גיוס יחסית משמעותית, אך גם יהודים מחוץ לטווח הגילאים התגייסו. 
		
		\item בצבא ארה''ב היו כחצי מליון לוחמים יהודים, כ־40 מתוכם הגיעו לפקידים בכירים. אחד מהם (קולנל מרקוס) אחרי המלחמה הגיעה לארץ כדי לסייע במלחמת העצמאות, ונהרג בטעות לאחר שלא ידע את סיסמת הכניסה לבסיס (לא דבר עברית). בצבא ארה''ב לא ממש הייתה חובת גיוס. 
		\item בצבא הבריטי, כ־30 מיהודי א''י התנגבו כדי להלחם בנאצים. הבריגדה היהודית מנתה כ־5000 מבני הישוב בארץ ישראלי שהוקמה בסוף המלחמה ולחמה בחזית הצרפתית. נוסף על כך הייתה את קבוצת המתנדבים הצנחנים, כולם נכנסו בצורה לא חוקית למדינות באמצעות תעודות מזוייפות. 
	\end{itemize}
	
	תצטרכו להכיר סיפור של חייל יהודי בצבאות בעלות הברית. יש דוגמה בספר. 
	
	\section{יחס האוכלוסייה ליהודים בתקופת הפתרון הסופי}
	צפינו בשני סרטונים על ניסיון סטנפורד ומילגרד. ננסה ממנו להבין איך קרה שאנשים רגילים השתתפו ברצח של היהודים בשואה (הייתה להם אפשרות לא להשתתף ברצח). איך קרה, שאנשים רגילים לכאורה, שאתמול לא היו מעלים על דעתם מעשים כאלו, היו רוצחים? בנסיון מילרד היו אנשים ש(חשבו) שהם הרגו אדם אחר, נטו מתוקף סמכות. 
	
	ישנם שלושה דפוסי פעילות של יחס לאוכלוסיה הכללית ביחס ליהודים: 
	\begin{itemize}
		\item משתפי פעולה: מסייעים לנאצים בהשמדה, והסגרת יהודים לנאצים. ברוב המקרים הנאצים פשוט פקרחו על הדברים. 
		\item הרוב הדומם: הגיבו באדישות לנעשה. לא סייעו לנעצים, אך לא עזרו ליהודים. אם ידפוק בדלתו יהודי בלילה, הוא לא יסגיר אותו, אך לא יעזור לו. 
		\item מצילים: אנשים בהצילו יהודים בתקופת השואה, והם מתפצלים לשניים: 
		\begin{itemize}
			\item יהודים שהצילו יהודים. יש כאן קושי רב, והסיכון כאן כפול. רובם היו במחנות פרטיזנים. 
			\item לא יהודים שהצילו יהודים בתקופת השואה, תוך סיכון חיהם ומשפחתם. הכרה בהם כחסידי אומות עולם אם יש תיעוד. היום יש בערך 25 אלף אנשים שהוכרו כחסידי אומות עולם, מתוך כנראה מאות אלפים שסייעו. 
		\end{itemize}
	\end{itemize}
	
	יש דיון על מי מוכר כחסיד אומות עולם. מי שקיבל תשלום ממשי עבור תעודות מזויפות, בבירור לא חסיד אומות עולם, כי למרות הסיכון זה נעשה למטרות רווח. אך אנשים שהצילו יהודים, וניתן להם כסף לדוגמה בשביל לקנות אוכל למשפחה, לרוב לא מוכרים כחסידי אומות עולם. לפי שרית, היא מכירה שני אנשים שאמרו לה מפורשות שהם לא סיפרו על כך שנתנו תמורה (או סתם כסף כדי שההצלה תתאפשר) למי שהציל אותם כדי שייקבלו את התואר. 
	
	אנשים שהוכרו כחסידי אומות עולם זוכים להכרכה ביד ושם ועזרה במידה והם (שבשלב הזה כבר מתו) או בני משפחותיהם (לדוגמה במלחמה באוקראינה) צריכים עזרה. 
	
	להלן המניעים של דפוסי הפעולה: 
	\begin{itemize}
		\item משת''פים: מסורת אנטישמית, הזדהות על המדיניות הנאצים, תגמול כספי, וזיהוי עם הקומוניזם. \textbf{פחד מהנאצים הוא לא גורם לשיתוף פעולה}. כולם פחדו מהנאצים, ומי שפחד באמת בד''כ לא עשה כלום. 
		\item הרוב הדומם היה מונע מאנטישמיות, פחד מהשלטון, פחד משכנים אנטישמים/הלשנה, מצוקה בשל המלחמה, ורצון לזכות ברכוש היהודי. בגלל המלחמה, רבים היו עסוקים בהשרדות של עצמם, ובמקרים רבים אין להם את האמצעיים לעזור לעוד מישהו. ``הפולנים היו רק קצת פחות מסכנים מהיהודים''. מיליוני פולנים נרצחו וחלקם הגדול עבדו בעבודות כפייה. 
		\item מניעי חסידי אומות עולם לרוב היו מונעים מרשות, רצון לסייע למכרים, מניעים דעתיים (לדוגמה אנשי דת וגמרים, ולעיתים גם סתם כתוליים), דרך לבטא התנגבות לנאציזם (מחתרות הוקמו בנידון), ועוד. 
	\end{itemize}
	
	למרות השנים הרבים שעברו, עדיין מתרחשת הכרה בחסידי אומות עולם, כי מתגלים סיפורים חדשים. לרוב ילדהם מקבלים את התעודה ומשתתפים בטקס. 
	
	
	
	
	
	
	
	
	
	
	
	
	
	
	
	
	
	
	
	
	
	
	\npchapter{נספחים}
	\section{התייחסויות למבחנים ספציפיים}
	\subsection{25A}
	\subsubsection{לקראת המבחן}
	סיימנו לאומיות כללית. המבחן נדחה לכבעוד שבועיים וחצי. המבחן הוא "עבודת כיתה עם ציון" (מבחן, קטן). ב־18.11. יארך שעה (עגולה, 60 דקות). ת.ז. עוד 15 דק'. כולל את ה חומר הנלמד עד שבוע הבא כולל, ובפרט את הלאומיות הכללית והגורמים לצמיחת הציונות.
	
	המבנה: שאלה אחת, עם קטע מקור. בשאלה שני סעיפים. עם ספר פתוח, כמו בבגרות, אך לא עם חומר פתוח. לכן כדאי לסמן בשספר עם דיבקיות. 
	
	בערך 15\%. 
	
	\subsubsection{בעקבות המבחן}
	השאלה מהמבחן חולקה בכיתה בשביל קונטקסט. 
	\begin{enumerate}[A.]
		\item \textbf{הסבירו} ע"פ \textit{הקטע} מהו \textit{הגורם לצמיחת התנועה הצונית}. \textbf{הסבירו} על פי מה שלמדתם \textit{גודם נוסף ייחודי} שהשפיע על צמיחת התנועה הציונית. 
		
		חילון, לדוגמה, הוא לא ייחודי (אלא אם מדברים על גך שבגלל הריחוק מהדת היו קבוצות שחששו מהיעלמות היהדות ולכן ניסו ליצור זהות על בסיס לאומי). 
		
		ההרחקה עליה הטקסט מדבר, היא לא הרחקה המתבססת על אנטישמיות, או על כשלון האמנסיפציה. ההרחקה הזו מתבצעת ע"פ רעיון הלאומית – קיימת בעם קבוצה שאינה חלק מהתנועה הלאומית. זהו "מוקש" – אפשר להוציא את המשפט המדבר על ההרחקה מהקשר, ולאבד נקודות. 
		
		חלק לא מובטל לא למד מהסיכום, אלא מהספר, ולכן ההגדרות היו לא מלאות. 
		
		\item כאשר ההצג לא מלא, אז הקישור חזרה לקטע לא מלא. זה שובר את החיבור בין הציטוט להגדרה. 
		
		באופן חד־פעמי – לא ירדו נקודות על זה שלא תיארו את הקטע. בגלל זה, לאחדים יהיה ציון שכתוב נמוך יותר ולידו גבוהה יותר. 
		
	\end{enumerate}
	
	"שרית את יודעת שהסינים מאוד חכמים" $\sim$ עמירם. 
	
	\subsection{25B}
	\subsubsection{לקראת המבחן}
	
	\textbf{לא משננים את הסיכום. }לוקחים נושא, עוברים עליו, מסכמים אותו בנקודות (המלצה: לעשות תרשים). לאחר כל נושא, בה"כ, הלאומיות, ונניח שהוא מחולק לפאפיינים, גורמים ומאפייני תנועות לאומיות, ונניח שלמנו היום מאפיינים וגורמים – לעשות שאלה בנושא הזה. לקחת את רשימת הנושאים ביחס לזמן, ולתכנן מראש. לתכנן אילו נושאים ללמוד בכל יום, תוך התחשבות בכמה אתם חזקים בכל נושא. 
	
	בשאלות – תמיד יהיה סעיף אחד על לאומיות כללית וסעיף אחד על ציונות. לא ייתכנו שאלות שהן רק על אד מבין השניים. את יום ראשון ושני מומלץ להקדיש לשאלות כלליות, אחרי שלמדנו כל נושא בנפרד. יש שעה לכל שאלה – הרבה זמן. עם זאת, בהתחלה זה יכול לקחת זמן, וע"כ מומלץ להשתמש בשעון. 
	
	זכרו – 5 נק' מציון המבחן הן על פתיחה ומרקור. כתבו תשובות מארגנות מלאות. ככל שתתחילו יותר מוקדם, כן ייטב בהטמעת אופן הכתיבה. 
	
	מי שרוצה לשלוח שאלה, ששרית תעבור עליה ותחזיר משו"ב, מוזמן לשלוח בצ'אט בטימס לא יאוחר מיום ראשון לפני 12:00 בערב ("עד הצוהריים ככה"). כך או אחרת ניתן לשאול שאלות הן בקבוצה של הכיתה עם שרית (בשעות "רלוונטיות") ובצ'אט בטימס. ככל הנראה יישלחו עוד שאלות תרגול בטימס. 
	
	תזכורת – מבחן עוד שבוע. ביום רביעי נתחיל ללמוד חומר שאינו למבחן. ביום של המבחן יש גם שיעור וגם מבחן. ראשית כל, תחולק תוכנית לימודים, נושאי הלימוד ומבנה הבחינה. סיכום כבר עלה. ניעזר גם בטבלה שעשינו. אפשר לשלוח שאלות לשרית והיא תנסה לענות, למרות שאבא שלה בביה''ח. עדיף וואטסאפ בפרטי. השאלות במבחן מערבות הכל מהכל – שאלה של ציונות יכולה להיעזר בלאומיות כללית. אין דבר כזה שאלה ``רק על בניית המשטר הנאצי'' – צריך ללמוד הכל, למרות שיש בחירה, כי שאלות מערבות חומרים. 
	
	חומרי עזר: הספרים. לעבור על הטבלה של קטעי המקור – לעיתים חלקכם נותנים דוגמה לקטע מקור למשהו שהוא אינו קטע מקור. תמיד מצוין מאיפה קטע מקור לקוח. אם לא בטוחים בזמן הלימודים, אפשר לשאול את שרית והיא תגיד לנו. הן ספר טוטאליטריות ושואה, ניתן לנו איפה לשים דיבקיות. מאוד יעזור לדעת איפה קטעי המקור בזמן המבחן, כמחפשים קטע מקור. יש לעבור על כל אחד מהנושאים, ע''פ הרשימה של ``יודע לא יודע'' ש(אולי) תעלה לטימס. בדכ מומלץ לעבור קודם על החומר שלא נבחנו עליו. שמתאמנים בבית – להתאמן עם סטופר. בערך 45 דקות לשאלה (כולל בדיקה והכל). מאוד מומלץ לענות על תשובה כמו ששרית למדה אותנו. במבחן שאלות ברמה של בגרות, עם דרישות של בגרות. 
	
	
	`''כל שנה יש פה צרצר, אני מלמדת את הכיהת הזו מאז שבנו אותה. יש כיתות שיש יונים ויש כיתות שיש צרצרים. עדיף צרצרים. יונים זו חיה ממש מסוכנת. שתדעו לכם שזו חיה ממש מסוכנת. לא בגלל אלימות, בגלל מחלות''
	
	``מישהו מכם הגיש כבר משהו (בנוגע ל־30\%)? כי אני לא נכנסתי לזה. `` (אף אחד לא התחיל). ``אתם יכולים לעשות ב־AI, אבל 100 הוא לא קיבל אצל אף אחד שהגיש, שכבתי''. 
	
	\subsubsection{לאחר המבחן}
	\subsubsection*{כללי}
	\begin{itemize}
		\item הפתיחות שלכם במקרים רבים היו לא רלוונטיות או לא טובות, יש הערה על כך אבל לא ירדו נקודות. 
		\item אם מרקרתם רע, גם לא ירדו נקודות אך יכול להיות שקיבלתם הערות (נאמר, מרקר אבל לא מספיק, או מרקר מילות קישור קישור ``שזה בכלל לא קשור לכלום'')
		\item ניתוח קטעי מקור: 
	\end{itemize}
	\subsubsection*{הפרק הראשון – לאומיות וציונות}
	``הייתם אמורים לקבל 100, לא קיבלתם 100''. לשאלה הראשונה: 
	\begin{itemize}
		\item בשאלה הראשונה, סעיף א' – היה צריך להציג 2 גורמים להתפתחות התנועה הלאומית הציונית, אחד משותף ואחר לא. ממש לא מומלץ לעשות על חילון, כי הרבה פעמים נופלים בהסבר של זה. גורם יחודי אין בעיה. לגבי משותף יש בעיות נוספות. 
		\item דיבורים על הרצל בשלושה תחומים – לכתוב ``הקונגרס הציוני=דיפלומטי'' זה לא דיפלומטי. זה חוסר הבנה. ביקשו תחומים. 
		\item בהמשך, ביקשו לכתוב משהו ``שלא בתהייחס לעובדה ההיסטורית שכתבתם''. משמעותי. 
	\end{itemize}
	לשאלה השנייה: 
	\begin{itemize}
		\item לדבר עברית עם הבן שלו זה לא פעילות לקידום השפה העברית (מי שכתב בתור הרביעי לא ירדו נקודות כי היה צריך רק 3, ``אבל זה בגלל שאני נחמדה''). 
		\item ``הסבירו ע''פ מה שכתבתם מה החשיבות של השפה לבני הלאום'' – לבני הלאום באופן כללי, לא השפה העברית לציונות. 
	\end{itemize}
	``על שאלה 3 ענה תלמיד אחד'' [יפתח מנופף, יותם שואל ``למה בחרת'']. התייחסות לשאלה באופן כללי כדי לא להתייחס לתשובה ספציפית. 
	\begin{itemize}
		\item כמבקשים משהו ``ע''פ הקטע וע''פ מה שלמדתם תנו בלה בלה בלה'' הכוונה אחד לפי הקטע ואחד לפי מה שלמדתם. בדכ יקבלו אם תעשו שניים לפי הקטע, אבל לא בוודאות יקבלו את זה תמיד. לכן כדאי גם להביא משהו אחד מחוץ לקטע. 
		\item בטאו קושי \textbf{אחר} של היישוב. ``\textbf{אחר}. \textbf{אחר}. הקושי האחר לא קשור'' (יפתח בלחץ). (``כתבתי עמוד שלם על קושי אחר ואז מחרתי את כל התשובה שלי'' – יפתח). 
	\end{itemize}
	
	\subsubsection*{הפרק השני (נאציזם וכו')}
	התייחסות כללית למקור חזותי: לא ייתכן שמדברים על שאלה שיש בה מקור חזותי, ואם מתארים לא יכול להיות שכותבים ``רואים את הרצל, רואים את הרצל – אהההה [התבלבלה, זה היה היטלר] – היטלר – [צחוק] וכתוב ``עם אחד, רייך אחד, מנהיג אחד''. ואח''כ צריך להסביר מה זה עקרון המנהיג. מהתיאור הזה אי אפשר לענות. אז מה עם רואים את היטלר? איך זה מבטא את עקרון המנהיג? שוב: ציין, הצג, הסבר. לא תגדירו בצורה מלאה את עקרון המנהיג, ולא תתארו בצורה טובה את המקור, הקישור שלכם לא יהיה מלא. כי אתם צריכים לקשר בין שני חלקים לא מלאים בתשובה שלכם. 
	
	נקודות נוספות על השאלה עם המקור של היטלר: 
	\begin{itemize}
		\item חוק ההסמכה זה נכון לביסוס עקרון המנהיג. אבל זה קלאסי לשבירת הדמוקרטיה. אפשר להסביר שזה נותן כוח להיטלר, אבל ביקשו עקרון המנהיג, וצריך לתת הסבר תואם. עדיף לקחת ליל הסכינים הארוכות/איחוד המשרות, וגם כאן חייבים להסביר איך זה מתקשר לעקרון המנהיג. ``אז יופי שהידנבורג מת''. 
		\item 
		רובכם נתתם עובדה היסטורית וקטע מקור שמדברים על אותו הדבר. קילו את זה כי במחוון ככה היה כתוב. אך, בבגרות, זה לא יקרה. אם יש ספק, אין ספק. אם אתם לא יודעים אם זה יכול להיות אותו הדבר, זה לא יכול להיות אותו הדבר. 
		\item כאשר מדברים על ההחמרה כלפי היהודים, ההחמרה היא בהיות החוקים כלפי כל יהודי גרמניה ולא כלפי יהודים ספציפיים (היו רק שניים שכתבו עליהם בכל מקרה). צריך גם להציג את כל החוק (יש שני חוקים). 
		\item לגבי ליל הבדולח – צריך לתאר את האירוע במלאו. חייבים לכתוב זמן. כדי להציג את ליל הבדולח, יש לתאר את האירועים שקדמו לו (כדי לקשר). רובכם קישרתם נכון, אך עם אי־דיוקים קטנים, על שני הדברים הבאים: 
		\begin{itemize}
			\item המאסר של יהודים בשל היותם יהודים בפעם הראשונה, לא סתם מאסר של יהודים. 
			\item זו פעם ראשונה שיש אלימות שלטונית מאורגנת ומתוכננת מלמעלה כלפי היהודים, אך זו לא פעם ראשונה שיש אלימות כלפי היהודים. אלימות יש עוד מלפני שנאצים עולים לשלטון, אך אי כי הנאצים לא מנסים למנוע אותה, הם לא מהנלים אותה. זה ההבדל. 
		\end{itemize}
	\end{itemize}
	
	``לטעות בחיים, זה חשוב''. 
	
	``היה איזה מישהו אחד או שניים לדעתי שענה'' (לגבי השאלה האחרונה). 
	
	\begin{itemize}
		\item ``אז למה להמציא תחומים?'' (אוהד בפאניקה, היחיד שעשה את השאלה הזו בכיתה). תחומים כתובים בשאלה. 
		\item יש תמונה, וביקשו לקשר לאיזה \textbf{תחום} היא מתאימה. לא עקרון, לא דבר אחר. איזה תחומים יש? כתוב שורה מעל. ראשית כל צריך להסביר מה זה התחוםפ, ואז צריך לקשר את זה. יכולה להיות תשובה של חינוך, תרבות ותעמולה – יש כאן כמה אפשרויות. אבל צריך קישור והסבר מתאים. ``אם אתם מציגים תעמולה ומסביירם חינוך, אז יש בעיה'' (``וואי וואי'' – אוהד). 
		\item פעולה או אירוע בתחום \textbf{אחר} כדי לבסס את שלטונם. \textit{לא נגד היהודים}. ביקשו מתחום אחר. יש המון חוקים, פשוט לכו על חקיקה. זה הכי פשוט. 
		\item בסעיף הבא, הייתה שאלת עמדה. כל תשובה שכתבתם שמבססים כמו שצריך היא תקינה ויכולה להתקבל, בהינתן שיש הסבר טוב. ``דעתכם לא מעניינת אותי, מה שמעניינן אותי אם הצלחתם לבסס את העמדה או לא''. 
	\end{itemize}
	
	
	אם לא מספיקים סעיף כי לא היה זמן, תכתבו. שרית תעשה עם זה מה שבא לה. אולי זה יעזור לכם. 
	
	
	\subsection{26A1}
	\begin{itemize}
		\item \textbf{פרק ראשון – }טוטאליטריות ושואה. 
		\item \textbf{פרק שני – }לאומיות וציונות. 
	\end{itemize}
	בכל פרק יש 3 שאלות, ובוחרים ממנו שאלה 1. בכל שאלה 2 סעיפים. משך הבחינה שעתיים (שעון, כלומר 120 דק') + חצי שעה תוספת זמן. 
	
	
	\section{מידע לגבי כתיבת תשובות}
	
	\subsection{מיון מקורות היסטוריים}
	סוגי מקורות: 
	\begin{enumerate}
		\item מקור מילולי
		\item מקור חזותי
		\item מקור משולב (לדוגמה מטבע, גרף או כרזה)
	\end{enumerate}
	מבין בין מקורות ראשוניים למשניים; עדות אישית, היא מקור ראשוני, בעוד מקור משני מכתב לאחד האירוע. מקור ראשוני נכתב בזמן האירועים. 
	
	מקורות כמו יומנים וכו', הם לא בהכרח הכי אמינים, למרות היותם ראשוניים, והם סובייקטיבים. במבחן לא תממיד אפשר להצליב אבל צריך לזכור שלא הכל אמין לחלוטין. 
	
	הספר, לדוגמה, הוא מקור משני. אבל אם נחליט לחקור על מקורות הלימוד בספרי ההיסטוריה בשנות ה־2000, הספר יכול להוות מקור ראשוני. לכן, שאלת הקשר יכולה לקבוע אם המקור ראשוני או לא, בהקשר הזה. 
	
	\subsection{הסבר ראשון לתשובה בהיסטוריה}
	\textbf{שלב ראשון: הצגת מקור. }\textit{דוגמא: }
	
	"מיהו איטלקי"?
	
	שם הקטע, הכותב, שם הספר, שם הכותב (ספר), הוצאה לאור, עמוד בספר. 
	
	הקטע שבחתי מיהו אילטקי של דה אמיציס, נמצא בספר לאומיות בישראל ובעמים, הוצאת כינרת, עמוד 18. אחרת, יירדו נקודות. 
	
	\textbf{מבנה התשובה: }
	
	דבר ראשון, לקורא את השאלה. חוסך זמן, ומאפשר לאחר מכן קריאה ממוקדת בקטע. \textbf{דבר ראשון, קוראים את השאלה}. תוך כדי קריאת השאלה, כדי לסמן מילות הוראות (תארו, הציגו , הסבירו). דוגמה: 
	
	\textit{תארו} את \textit{\textbf{כל} מרכיבי הלאומיות} המוזכרים בטקסט. 
	
	\textit{הציגו} את \textit{רעיונות התנועה הרומנטית}. 
	
	אם ביקשו לתאר, לדוגמה, כדאי לסמן \textit{מה} לתאר, וכמה לתאר. 
	
	מסמנים מילות הוראה ומשימות. 
	
	רצוי למספר את המשימות. מומלץ לכתוב בצד את התשובה, בלי קשר לשאלה. (אם ביקשו את כל מרכיבי הלאומיות, כדאי לכתוב בצד את השמות שלהם. עוזר לזכור וחוסך קריאות חוזרות). לא חובה, יש כאלו שרואים את זה כמסורבל. 
	
	\subsection{הסבר שני באמצעות דוגמה לכתיבת תשובה בהיסטוריה}
	
	\textbf{\textit{נפרט את סעיפי השאלה: }}
	\begin{enumerate}[A.]
		\item הסבירו כיצד \textit{אחד} מהעקרונות באו לידי ביטוי במקור. 
		
		בחרו מספר הלימוד \textit{מקור אחד} המבטא את העקרון \textit{שכתבת עליו}, יהסבר כיצד עקרון זה בא לידי ביטוי במקור שבחרת. תנו \textit{2} דוגמאות. 
		
		
		(כלל אבצע – אם אתם לא יודעים, פשוט הסבר. לא יודעים את ההבדל בין הצג להסבר? תעשו הסבר. תזכורת – הסבר, הצג, ציין (רק בבחינות מלפני 20 שנה) הוא המדרג. ראשון במדרג – כולל את השאר. כלומר, אם הסברנו והיה צריך להציג, נקבל את כל הנקודות. בכיוון ההעפוך, נקבל רק 60\%. ציין זה פשוט רשימת מכולת. לא עושים את זה, זה 40\% - 50\%. 
		
		גם כאן, להקפיד מאוד. אם בחרתם מקור שלא מבטא את מה שכתבת עליו – לא תקבל על זה ניקוד. 
		
		לגבי ``2 דוגמאות'' – ברוב המקרים, כאשר קטע המקור הוא תמונה, אין בו 2 דוגמאות – ואם יש לא תמצאו אותן. בשאלה להלן, רבים התלמידים הנותנים דוגמה לקטע המקור מעמוד 69. זוהי קריקטורה אנטישמית שלא עונה להסברים של תורת הגזע, אבל אפילו אם היה כאן ציור של ארי ויהודי – אומנם רואים את ההבדלים הפיזיים, אבל זהו! אין את הדבר השני. ככל אצבע, כדי להבדיל בין תורת הגזע לאנטישמיות תנסו לחפש משהו שמדבר על טוהר האדם. דברים היכולים להוות הסבר על תורת הגזע: חוקי נירנברג (זהו טוהר הגזע – רק ארים יכולים לקבל אזרחות), בדיקות רפואיות וכו'. בעיקר תלוי בהסבר. אפשר להיעזר בטבלה כדי ללמוד את זה. 
		
		\item הצג עקרון \textit{אחר} באיג' האנצית, הסבר כיצד \textit{שניים} מן הצעדים שנקטו הנאצים \textit{ב־33-38} מימשו עקרון זה. 
		
		תזהרו מהשנים – אם כתבתם תשובה מעולה על שנים שלא רלוונטיות, תקבלו 0, אף אם התשובה מדהימה, כי השאלה לא נענתה. 
	\end{enumerate}
	
	\textbf{\textit{נבנה שלד לתשובה: }}
	
	בודקים בבגרות ע''פ מכוון, לא ע''פ מה שלמדו. 
	
	לסעיף א': 
	\begin{enumerate}
		\item פתיחה $\leftarrow$ הצגה כללית של האיגיאולוגיה. איך מציגים? חמשת המ''מים – מי כתב
		\item העקרון $\leftarrow$ [ציין] תורת הגזע. [הסבר] טוהר הדם (הדם הוא האדם), שיוך לגזעים, תכונות פנימיות וחיצוניות. [על הכל צריך לכתוב – אלו הדגשים שלנו]. מי שיודע ומבין – אפשר להתייחס בקצרה לדרווין. לא צריך אבל לדבר על ציפורים, וספינות, ודרווניזם חברתי. 
		\item איך זה בא לידי ביטוי $\leftarrow$ [הסבר] דרישה שמתייחסת למוצא, לדם. הקפדה על טוהר הגזע. [תזכורת: מחולק לביסוס וקישור]
		
		הערה: הצג \textbf{לא שייך לקטע. }תציגו את כל ההגדרה וכל מה שקשור. הגדרה ע''פ מה שנלמד, ללא קשר לקטע המקור. אחרי כן תשתמשו במה שצריך מזה בקישור. נימוק: אם לא היה קטע מקור הייתם צריכים להסביר את הכל, וההוראה לא השתנתה רק בגלל שהיה קטע מקור אחכ. יתרה מכך, הגדרה חלקית כנראה לגרום לקישור לא מלא. 
		
		\item מקור: 
		\begin{itemize}
			\item בחירת מקור
			\item פרטי מקור (שם המקור, כותב, ספר, כותב הספר, הוצאה לאור, ע''מ). (יש ספרים עם אותו השם, אך הוצאה וכותב שונים). הספר של מלחמה ושואה– היי־סכול
			\item תיאור/הצגת המקור. מה כתוב, מה מצויר בו, וכו'. אם זה ציור או רקירטוקה – לתאר לפרטי פרטים. 
			\item ביסוס + קישור. אין צורך לחזור על הגדרות. תמיד 2 דוגמאות מהמקור לפחות, אף אם לא כתוב. 
		\end{itemize}
	\end{enumerate}
	
	
	
	\subsection{טיעון והבעת עמדה בהיסטריה}
	בכיתה ניצור שלוש קבוצות, כל אחת תצטרך לבסס את הטיעון לפיו התחום שלה הוא התחום המשמעותי ביותר, באמצעות קטעי מקור מהספר. 
	עכשי וצופים באינטרנט בקרוס יהל"ום או משהו כזה של משרג החינוך. 
	
	בשאלת מדה נתבקס להביע את דעתנו בנושא מסויים, ולבסס באמצעות נימוקים שמתבססים על עובדות היסטוריות. הדבר הכי לא חשוב שאלת עמדה הוא מה העמדה שלכם. "העמדה שלכם לא מעניינת" – הדבר חשוב הוא כמה טוב אנחנ נבסס את זה. לדוגמה, בשנה הבאה נשווה את המאבק הצבאי של האצ"ל והלח"י לעומת העפלה והתיישבות. בעיננו, מאבק צבאי הוא לא אתי, אך יותר קל לביסוס. אזי, עדיף לכתוב על מדוע מאסק צבאי עדיף. אף אחד לא בודק אידיאולוגי, אלא רק עד כמה הצלחנו לבסס את הטענה הכתבים היסטוריים. 
	
	שאלה לדוגמה: [...] הבע את עמדתך בנוגע למי היה גורם מרכזי לצמיחת התנועות הלאומיות – תנועת ההשכלה או הנאורות. 
	
	\textit{צריך} לכתוב "אני חושב שתנועת ההשכלה הייתה גורם מרכזי לפיתוח התנועות הלאומיות" (או הנאורות). לאחר מכן, צריך לבסס בעובדות היסטוריות. לדוגמה, אפשר לתת כדוגמה את המהפכה הצרפתית והאמריקאית שלא התבססו בכלל על רעיונות הרומנטיקה. 
	
	בשאלת תטיעון לוקחים את מה שלמדנו ובונים אחרת את התשובה. 
	
	איך נדע מה יותר קל לבסס? נעשה רשימה של הטיעונים (כלומר העובדות ההיסטוריות) של כל אחד מהם, ונבחר את האחד אם יותר נימוקים. 
	
	לא צריך להגדיר מושגים מההתחלה. שאלת עמדה היא יחסית מנוקדת, ואינה חפירה של ארבעה עמודים. 
	
	\section{חרא כזה או אחר שלא קשור לשום כותרת אחרת והוא בהחלט לא בחומר}
	
	\subsection{ניהול זמנים}
	
	דוגמה בעבור איתמר: 
	\begin{center}
		\begin{tabular}{|c|c|c|c|c|c|c|}
			\hline ראשון & שני & שלישי & רביעי & חמישי & שישי & שבת \\
			\hline לימודים עד 16:20 &14:00&14:10&16:20&16:20&פרויקט&\\
			\textbf{1} + 2 חור& \textbf{3} & \textbf{1} & \textbf{1} &\textbf{1}&\textbf{5}&\textbf{4}\\
			דר כושר 19:00-21:00 && צהריים 16:00-20:30. & חדר כושר 19:00-21:30& ישיבת צוות 17:00-21:30 & פעולה 15:30-17:30  & \\
			\hline
			הגשה + כושר & הגשה & מבחן בהיסטוריה &&&&\\
			\textbf{1} & \textbf{4} &&&&&\\\hline
		\end{tabular}
	\end{center}
	להמשיך חודשי. 
	\subsubsection{חשיבות שיבוץ דברים במערכת: }
	\begin{multicols}{2}
		\begin{enumerate}
			\item נשבץ "זמן מת" – זמן שלא תכננו בו שום דבר, ויורד מסך השעות. לא זמן אוכל/מקלחת; זמן מת לחלוטין לבהייה בקיר, או להשלמת בלת"מים. 
			\item ארוחת צוהריים
			\item ארוחת ערב
			\item למידה שוטפת (ש.ב.)
			\item למידה למבחנים
		\end{enumerate}
	\end{multicols}
	ראשית כל, נוריד מכל יום שעה. 
	נסמן בבולד שעות לימודים. 
	
	נסכום את הזמן עד המבחן בהיסטוריה: 
	שבוע ראשון: 16, שבוע שני: 5 [עד היסטוריה]. סה"כ 21 שעות עד היסטוריה. 
	\begin{multicols}{2}
		\begin{itemize}
			\item 1 סקירה
			\item 5 הנדסת תוכנה
			\item 7 פרויקט
			\item 8 פרויקט
			\item 5 מבחן בהיסטוריה (לפי שרית, לוקח פי 2 יותר)
		\end{itemize}
	\end{multicols}
	
	\subsubsection{עוד הערות}
	יומן שבועי: דבר מאוד נחמד, אך לא מאפשר לראות את המשך החודש כראוי, ולתכנן בהתאם. גם עם עכשיו ניהול הזמן הנוכחי עובד, בי"א זה לא יעבוד, כי יש יותר מבחנים ומטלות. 
	
	לא ללמוד להיסטוריה, ספרות, תנ"ך, אזרחות וכו' אחד אחרי השני – עדיף שיהיה לשון והיסטוריה, מתמטיקה ותנ"ך או משהו כזה, כי זה "יושב על אותו המקום במוח ועושה סמתוכה". אם אין ברירה, כדאי לעשות הפסקה. 
	
	
	
	
	\subsection{טרור כיתתי $\#1$}
	
	זכרו, שטרור הוא חלק ממאפייני המשטר הטוטאליטרי. 
	
	"היות והכללים שלי לא מתאימים שלכם [...] כנראה זה יותר מדי נזיל בשבילם": 
	\begin{itemize}
		\item אף אחד לא יוצא לשום מקום. לא יוצאים. 
		\item בהגעה – כולם יהיו עם הציוד ליד השולחן. מי שלא יהיה, לא יהיה. 
		\item הטלפונים יהיו פה (מצביעה על השולחן). 
		\item לא אוכלים. 
		\item שותים מים בלבד מבקבוק מים שיש בתוכו מים. 
		\item (לא יהיו וויכוחים לגבי זמנים, ואם המרוה מגיעה חצי דקה לפני ומתווכחים איתה לגבי זה שלא מאחרים תבוטל הארוחה).
	\end{itemize}
	
	(צועקת על גלי – "את כדאי שלא תדברי על זה. כי בשיעור הקודם את לקחת את הרגליים שלך והלכת למלא מים. אני במקומך לא הייתי מנהלת עכשיו ויכוח") (ליבי לוחשת – ומה אם אני מדממת על כל המכנסיים)
	
	התלוננויות – אצל הדר. 
	
	\subsection{טרור כיתתי $\#2$}
	ישן 4 סיבות מרכזיות לנצחון גרמניה: 
	\begin{itemize}
		\item יש צלצול. 
	\end{itemize}
	
	``יום שני ב־7:45, טלפונים מקדימה, כלי כתיבה, ספרים דפים לא אני פשוט מחכה שאני אוכל לדבר אז מבחינתי שזה יקרה עוד שעה תביאו מרקרים אתם מסמנים אתם זה''. זה שעתיים לפי השעון. 
	
	    משהו משהו אסור לאכול יותר והיא לא חייבת לנו לאפשר לאכול היום. 
	
	
	\subsection{תקופת בית שני}
	תקופת הבית השני היא תקופה ארוכה (מ־586 לפנה"ס עד 132-135 לספירה, ספציפית בהקשר שלנו). היא מחולקת לתתי־תקופות שמאופיינות ע"י השלטונות בארץ. ארץ ישראל, משום שמחברת בין יבשות וכללה דרכים רבות, שינתה שלטונה פעמים רבות במהלך השנים. עד לאחרונה עוד אימפריות החליפו את מקומן כאן. בעבודה העצמאית – תקופת השלטון ההלניסטי ועוד משהו שלא שמעתי. 
	
	תרבות התקופה תלוי בטיב השלטון, וביכולת היהודים להתאים עצמם לתרבות במשתנה. בתוך התקופה ישנם מנהיגים יהודים רבים עלינם נלמד בתקופה הזו. 
	
	אז מהו היסוד המארגן של היחידה? מנהיגות – איך פועלים מנהיגים בסיטואציות שונות תחת שלטונות זרים. לא נדבר כמעט על החשמונאים (שכן זהו שלטון יהודי, וכן הפעם האחרונה עד הקמת המדינה בה היה כאן שלטון יהודי עצמאי). 
	
	\subsubsection{יחידה עצמאית}
	נוצרה כיתה חדשה (מה שזה לא אומר במונחים של טימס) שכולל בתוכו מספר ערוצים – תקופה פרסית, הלניסטית ורומית. בכל אחת מהן ישנה יחידה לימוד עצמאית (לא חייבים לבצעה דרך הטימס, ישנו קישור לדפדפן). לכל אחד מהנושאים מטלה ומחוון בעבור המטלה (שמתאר את הניקוד וכו'). העבודות לא מאוד ארוכות, אך יש לענות עליהם במבנה של תשובה בגרותית. בכל נושא יש "מיומנות" חדשה שמשתקפת במטלה. העבודה מצועצעת. "מצד אחד זה מאוד פשוט ומצד שני אל תתייחסו לזה בזלזול כי זה 30\% מציון הבגרות שלכם". סיו םלימוד הפרקים מגישים 2 מטלות מתוך ה־3. מועד אחרון 15.6.2025. גם אם העולם נופל לא יתקבלו מטלות לאחר מכן. לא מובטח שיינתנו תזכורות. מי שלא יגיש יעשה חיצוני (יש בו יותר חומר). 
	
	\subsubsection{עוד מנהלות}
	
	(מתחיל משיעור היסטוריה אודות ההיסטוריה של הבגרות) אמ;לק החומר שלומדים בשיעור הזה (תקופת הבית השני וכו') נכנס ל־30\% הפנימיים. זה בתוכנית הלימודים של משרד החינוך. יש כאן גם יחידה ללמידה עצמאית. 
	
	הבגרות בהיסטוריה מחולקת ל־30\% (שזו היחידה לעיל) שציונו הסופי נשלח למשרד החינוך בשנה הבאה, ו־70\% נוספים. העבודות וכו' לא חוזרות אליכם. את הציון תגלו "בשנה הבאה מתישהו". מה שאנחנו לומדים בכיתה הוא החלק של 70\% האחוזים שמתחלק ל־70\% בחינה חיצונית ו־30\% ציון הגשה. כלומר, הבגרות היא 49\% מסך הציון הכולל וציון ההגשה 21\%. משהו כמו 15\% מציון ההגשה הוא מכיתה י' אבל זה יכול להשתנות בהתאם לתלמיד כדי שיצא יותר גבוה. 
	
	\subsection{קסתנר}
	(ערב יום השואה)
	
	נבדיל בין שתי מילים: 
	\begin{itemize}
		\item \textbf{השואה} – שואת היהודים באירופה ובעולם, בין השנים 1933 (יש כאלו שסופרים מ־41, תלוי הגדרה) עד 1945. 
		\item \textbf{שואה} – אסון גדול. מילה שמופיעה עוד בתנ"ך, וטרם השואה שימשה עוד לדברים אחרים. 
	\end{itemize}
	
	ההבדל – השואה היא אירוע ספציפי שתחום במקום וזמן. שואה – אסון. 
	
	עתה נצפה בסרטון "מרגע זה אתה לבד": סיפורו של ניצול השואה מאיר ברנד. 
	
	קרקוב, העיר המקורית של מאיר, היוותה עיר הבירה של השליש הזה של פולין (פולין חולקה ל־3 חלקים) ועל כן לא חרבה (בניגוד לוורשה, בה אין בניינים סובייטים סטייל חארצ'ושנוב). יחס ליהודים יחס אלים, היה רצח ברחובות עוד לפני הגטאות (אך לא שיטתי). בלג'ט היה מחנה השמדה במזרח פולין, ושלחו הרבה יהודים ממערב פולין לשם. אלו מחנות השמדה מלאים (בניגוד לשאוויץ' ומיידנק) שאליהם מגיעים ונרצחים באותו היום. 
	
	שואת יהודי הונגריה לא מדוברת ישיורת לרוב, אך היא האירוע הכי טרגי בשואה. הסיבה: היא הייתה יכולה להמנע. הם נרצחו ביולי־אוגוסט 1944, בשלב שבו חצי אירופה הייתה משוחררת. תוך חודש וחצי הושמדו מרבית יהודי הונגריה. הדיבורים על כך שאם היו מפציצים את פסי הרכבת הם היו ניצלים, מדברים על כך. 
	
	הסיפור של רכבת קסטנר – יש שמאשימים אותו בבגידה ושיתוף פעולה עם הנאצים (סיפור מורכב בו הוא טבע דיבה נגד מישהו אך המשפט התהפך כנגדו), והוא נושא רגיש. אייכמן פיזית היה בהונגריה (כדי לנהל את ההשמדה) והרכבת נוצרה המטרה להציל את היהודים. פרשת קסטנר מסובכת ויש קורס שלם באונ' עליה (הרעיון היה שקטסנר עשה את עצמו נאצי, במטרה לקחת רכבת ולהציל יהודים). זה לכאורה היה סיכום עם הנאצים ולא מתחת לאף שלהם, וחיילים גרמנים היו ברכבת כחלק מההסכם. הרכבת הגיעה לברגן בלזן (שם נרצחה אנה פרנק), שהוא בניגוד ליעוד של רוב יהודי הונגריה – אינו מחנה השמדה. 
	
	ישנם כמה דברים מרכזיים שנרצה להתייחס אליםה מהעדות: 
	,\begin{itemize}
		\item \textit{לזכר משפחת ברנד – לזכרון לדורות הבאים}. עבור מרבית העדים, זהו הציווי. זכור: ציווי מאוד משמעותי ביהדות בהקשרים רבים. ספציפית מאיר נשלח ע"י הוריו בשביל מטרה זו. אצל חלק מהעדים, לא לשכוח את היותך יהודי היה חלק חשוב מהעדות. לפי שרית, זה גם תפקידנו, ואם העדויות לא יועברו האירוע יהפוך להיות חורבן בית שני לדיד הדורות הבאים. עוד כמה דורות לא תהיה נוכחות פיזית של האנשים האלו. 
		\item \textit{יהודים שהצילו יהודים}. גויים שהצילו יהודים קרויים חסידי אומות עולם (בהמשך נלמד את הקריטריונים הספציפיים). בעיקר נוצרים, אך יש גם מעט מוסלמים. יהודים שהצילו יהודים לרוב לא זוכרים להכרה, מיד ושם או מהמדינה. רק בשנים האחרונות לאט לאט מתחילים לדבר עלזה. יש לא מעט כאלו, היו הרבה במחתרות בצרפת, במחתרת העולמית וכו'. 
	\end{itemize}
	
	ישנם חסידי אומות עולם שעלו לארץ, חלקם התגיירו וחלק נשארו בדתם. בהתחלה לא היה לכך זכר במרחב הציבורי. ראה פרויקט "לא נצר אחרון אני תיכף אזכר ואגיד לכם". תפקידו: לספר את סיפורם, ולערוך טקס בקברם. יש כמה עשרות כאלו בארץ. שרית הכירה את הפרויקט דרך תלמיד שסיים לימודים ב־2014 ונסע ב־2013 לפולין, וציין שסבתא שלו חסידת אומות עולם, על אף שסבתו יהודיה. לקח לשרית הרבה זמן לבדוק מה קורה איתה (כי היא שינתה של שמה וכו') שהצילה שלוש משפחות, שהתחתנה עם אדם יהודי והתגיירה. כששאלה הלמה לא דיבר על כך כל השנים, ענה שבבית הספר היסודי כאשר ביקשו להביא סיפורים משפחתיים המורה השיבה שזה לא סיפור מעניין ומשם הוא הניח שזה לא סיפור מעניין. אביה גם היה חסיד אומות עולם. שם העמותה (עכשיו בדקתי בגוגל ושרית אישה) – חסד אחרון. 
	
	
\end{document}
