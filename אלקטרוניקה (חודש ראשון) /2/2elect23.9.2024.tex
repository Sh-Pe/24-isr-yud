%! ~~~ Packages Setup ~~~ 
\documentclass[]{article}


% Math packages
\usepackage[usenames]{color}
\usepackage{forest}
\usepackage{ifxetex,ifluatex,amsmath,amssymb,mathrsfs,amsthm,witharrows,mathtools}
\WithArrowsOptions{displaystyle}
\renewcommand{\qedsymbol}{$\blacksquare$} % end proofs with \blacksquare. Overwrites the defualts. 
\usepackage{cancel,bm}
\usepackage[thinc]{esdiff}


% tikz
\usepackage{tikz}
\newcommand\sqw{1}
\newcommand\squ[4][1]{\fill[#4] (#2*\sqw,#3*\sqw) rectangle +(#1*\sqw,#1*\sqw);}


% code 
\usepackage{listings}
\usepackage{xcolor}

\definecolor{codegreen}{rgb}{0,0.35,0}
\definecolor{codegray}{rgb}{0.5,0.5,0.5}
\definecolor{codenumber}{rgb}{0.1,0.3,0.5}
\definecolor{codeblue}{rgb}{0,0,0.5}
\definecolor{codered}{rgb}{0.5,0.03,0.02}
\definecolor{codegray}{rgb}{0.96,0.96,0.96}

\lstdefinestyle{pythonstylesheet}{
	language=Python,
	emphstyle=\color{deepred},
	backgroundcolor=\color{codegray},
	keywordstyle=\color{deepblue}\bfseries\itshape,
	numberstyle=\scriptsize\color{codenumber},
	basicstyle=\ttfamily\footnotesize,
	commentstyle=\color{codegreen}\itshape,
	breakatwhitespace=false, 
	breaklines=true, 
	captionpos=b, 
	keepspaces=true, 
	numbers=left, 
	numbersep=5pt, 
	showspaces=false,                
	showstringspaces=false,
	showtabs=false, 
	tabsize=4, 
	morekeywords={as,assert,nonlocal,with,yield,self,True,False,None,AssertionError,ValueError,in,else},              % Add keywords here
	keywordstyle=\color{codeblue},
	emph={object,type,isinstance,copy,deepcopy,zip,enumerate,reversed,list,set,len,dict,tuple,print,range,xrange,append,execfile,real,imag,reduce,str,repr,__init__,__add__,__mul__,__div__,__sub__,__call__,__getitem__,__setitem__,__eq__,__ne__,__nonzero__,__rmul__,__radd__,__repr__,__str__,__get__,__truediv__,__pow__,__name__,__future__,__all__,},          % Custom highlighting
	emphstyle=\color{codered},
	stringstyle=\color{codegreen},
	showstringspaces=false,
	abovecaptionskip=0pt,belowcaptionskip =0pt,
	framextopmargin=-\topsep, 
}
\newcommand\pythonstyle{\lstset{pythonstylesheet}}
\newcommand\pyl[1]     {{\lstinline!#1!}}
\lstset{style=pythonstylesheet}

\usepackage[style=1,skipbelow=\topskip,skipabove=\topskip,framemethod=TikZ]{mdframed}
\definecolor{bggray}{rgb}{0.85, 0.85, 0.85}
\mdfsetup{leftmargin=0pt,rightmargin=0pt,innerleftmargin=15pt,backgroundcolor=codegray,middlelinewidth=0.5pt,skipabove=5pt,skipbelow=0pt,middlelinecolor=black,roundcorner=5}
\BeforeBeginEnvironment{lstlisting}{\begin{mdframed}\vspace{-0.4em}}
	\AfterEndEnvironment{lstlisting}{\vspace{-0.8em}\end{mdframed}}


% Deisgn
\usepackage[labelfont=bf]{caption}
\usepackage[margin=0.6in]{geometry}
\usepackage{multicol}
\usepackage[skip=4pt, indent=0pt]{parskip}
\usepackage[normalem]{ulem}
\forestset{default}
\renewcommand\labelitemi{$\bullet$}
\usepackage{titlesec}
\graphicspath{ {./} }


% Hebrew initialzing
\usepackage[bidi=basic]{babel}
\PassOptionsToPackage{no-math}{fontspec}
\babelprovide[main, import, Alph=letters]{hebrew}
\babelprovide[import]{english}
\babelfont[hebrew]{rm}{David CLM}
\babelfont[hebrew]{sf}{David CLM}
\babelfont[english]{tt}{Monaspace Xenon}
\usepackage[shortlabels]{enumitem}
\newlist{hebenum}{enumerate}{1}

% Language Shortcuts
\newcommand\en[1] {\begin{otherlanguage}{english}#1\end{otherlanguage}}
\newcommand\sen   {\begin{otherlanguage}{english}}
	\newcommand\she   {\end{otherlanguage}}
\newcommand\del   {$ \!\! $}
\newcommand\ttt[1]{\en{\footnotesize\texttt{#1}\normalsize}}

\newcommand\npage {\vfil {\hfil \textbf{\textit{המשך בעמוד הבא}}} \hfil \vfil \pagebreak}
\newcommand\ndoc  {\dotfill \\ \vfil {\begin{center} {\textbf{\textit{שחר פרץ, 2024}} \\ \scriptsize \textit{נוצר באמצעות תוכנה חופשית בלבד}} \end{center}} \vfil	}

\newcommand{\rn}[1]{
	\textup{\uppercase\expandafter{\romannumeral#1}}
}

\makeatletter
\newcommand{\skipitems}[1]{
	\addtocounter{\@enumctr}{#1}
}
\makeatother

%! ~~~ Math shortcuts ~~~

% Letters shortcuts
\newcommand\N     {\mathbb{N}}
\newcommand\Z     {\mathbb{Z}}
\newcommand\R     {\mathbb{R}}
\newcommand\Q     {\mathbb{Q}}
\newcommand\C     {\mathbb{C}}

\newcommand\ml    {\ell}
\newcommand\mj    {\jmath}
\newcommand\mi    {\imath}

\newcommand\powerset {\mathcal{P}}
\newcommand\ps    {\mathcal{P}}
\newcommand\pc    {\mathcal{P}}
\newcommand\ac    {\mathcal{A}}
\newcommand\bc    {\mathcal{B}}
\newcommand\cc    {\mathcal{C}}
\newcommand\dc    {\mathcal{D}}
\newcommand\ec    {\mathcal{E}}
\newcommand\fc    {\mathcal{F}}
\newcommand\nc    {\mathcal{N}}
\newcommand\sca   {\mathcal{S}} % \sc is already definded
\newcommand\rca   {\mathcal{R}} % \rc is already definded

\newcommand\Si    {\Sigma}

% Logic & sets shorcuts
\newcommand\siff  {\longleftrightarrow}
\newcommand\ssiff {\leftrightarrow}
\newcommand\so    {\longrightarrow}
\newcommand\sso   {\rightarrow}

\newcommand\epsi  {\epsilon}
\newcommand\vepsi {\varepsilon}
\newcommand\vphi  {\varphi}
\newcommand\Neven {\N_{\mathrm{even}}}
\newcommand\Nodd  {\N_{\mathrm{odd }}}
\newcommand\Zeven {\Z_{\mathrm{even}}}
\newcommand\Zodd  {\Z_{\mathrm{odd }}}
\newcommand\Np    {\N_+}

% Text Shortcuts
\newcommand\open  {\big(}
\newcommand\qopen {\quad\big(}
\newcommand\close {\big)}
\newcommand\also  {\text{, }}
\newcommand\defi  {\text{ definition}}
\newcommand\defis {\text{ definitions}}
\newcommand\given {\text{given }}
\newcommand\case  {\text{if }}
\newcommand\syx   {\text{ syntax}}
\newcommand\rle   {\text{ rule}}
\newcommand\other {\text{else}}
\newcommand\set   {\ell et \text{ }}
\newcommand\ans   {\mathit{Ans.}}

% Set theory shortcuts
\newcommand\ra    {\rangle}
\newcommand\la    {\langle}

\newcommand\oto   {\leftarrow}

\newcommand\QED   {\quad\quad\mathscr{Q.E.D.}\;\;\blacksquare}
\newcommand\QEF   {\quad\quad\mathscr{Q.E.F.}}
\newcommand\eQED  {\mathscr{Q.E.D.}\;\;\blacksquare}
\newcommand\eQEF  {\mathscr{Q.E.F.}}
\newcommand\jQED  {\mathscr{Q.E.D.}}

\newcommand\dom   {\mathrm{dom}}
\newcommand\Img   {\mathrm{Im}}
\newcommand\range {\mathrm{range}}

\newcommand\trio  {\triangle}

\newcommand\rc    {\right\rceil}
\newcommand\lc    {\left\lceil}
\newcommand\rf    {\right\rfloor}
\newcommand\lf    {\left\lfloor}

\newcommand\lex   {<_{lex}}

\newcommand\az    {\aleph_0}
\newcommand\uaz   {^{\aleph_0}}
\newcommand\al    {\aleph}
\newcommand\ual   {^\aleph}
\newcommand\taz   {2^{\aleph_0}}
\newcommand\utaz  { ^{\left (2^{\aleph_0} \right )}}
\newcommand\tal   {2^{\aleph}}
\newcommand\utal  { ^{\left (2^{\aleph} \right )}}
\newcommand\ttaz  {2^{\left (2^{\aleph_0}\right )}}

\newcommand\n     {$n$־יה\ }

% Math A&B shortcuts
\newcommand\logn  {\log n}
\newcommand\logx  {\log x}
\newcommand\lnx   {\ln x}
\newcommand\cosx  {\cos x}
\newcommand\cost  {\cos \theta}
\newcommand\sinx  {\sin x}
\newcommand\sint  {\sin \theta}
\newcommand\tanx  {\tan x}
\newcommand\tant  {\tan \theta}
\newcommand\sex   {\sec x}
\newcommand\sect  {\sec^2}
\newcommand\cotx  {\cot x}
\newcommand\cscx  {\csc x}
\newcommand\sinhx {\sinh x}
\newcommand\coshx {\cosh x}
\newcommand\tanhx {\tanh x}

\newcommand\seq   {\overset{!}{=}}
\newcommand\slh   {\overset{LH}{=}}
\newcommand\sle   {\overset{!}{\le}}
\newcommand\sge   {\overset{!}{\ge}}
\newcommand\sll   {\overset{!}{<}}
\newcommand\sgg   {\overset{!}{>}}

\newcommand\h     {\hat}
\newcommand\ve    {\vec}
\newcommand\lv    {\overrightarrow}
\newcommand\ol    {\overline}

\newcommand\mlcm  {\mathrm{lcm}}

\DeclareMathOperator{\sech}   {sech}
\DeclareMathOperator{\csch}   {csch}
\DeclareMathOperator{\arcsec} {arcsec}
\DeclareMathOperator{\arccot} {arcCot}
\DeclareMathOperator{\arccsc} {arcCsc}
\DeclareMathOperator{\arccosh}{arccosh}
\DeclareMathOperator{\arcsinh}{arcsinh}
\DeclareMathOperator{\arctanh}{arctanh}
\DeclareMathOperator{\arcsech}{arcsech}
\DeclareMathOperator{\arccsch}{arccsch}
\DeclareMathOperator{\arccoth}{arccoth} 

\newcommand\dx    {\,\mathrm{d}x}
\newcommand\dt    {\,\mathrm{d}t}
\newcommand\dtt   {\,\mathrm{d}\theta}
\newcommand\du    {\,\mathrm{d}u}
\newcommand\dv    {\,\mathrm{d}v}
\newcommand\df    {\mathrm{d}f}
\newcommand\dfdx  {\diff{f}{x}}
\newcommand\dit   {\limhz \frac{f(x + h) - f(x)}{h}}

\newcommand\nt[1] {\frac{#1}{#1}}

\newcommand\limz  {\lim_{x \to 0}}
\newcommand\limxz {\lim_{x \to x_0}}
\newcommand\limi  {\lim_{x \to \infty}}
\newcommand\limh  {\lim_{x \to 0}}
\newcommand\limni {\lim_{x \to - \infty}}
\newcommand\limpmi{\lim_{x \to \pm \infty}}

\newcommand\ta    {\theta}
\newcommand\ap    {\alpha}

\renewcommand\inf {\infty}
\newcommand  \ninf{-\inf}

% Combinatorics shortcuts
\newcommand\sumnk     {\sum_{k = 0}^{n}}
\newcommand\sumni     {\sum_{i = 0}^{n}}
\newcommand\sumnko    {\sum_{k = 1}^{n}}
\newcommand\sumnio    {\sum_{i = 1}^{n}}
\newcommand\sumai     {\sum_{i = 1}^{n} A_i}
\newcommand\nsum[2]   {\reflectbox{\displaystyle\sum_{\reflectbox{\scriptsize$#1$}}^{\reflectbox{\scriptsize$#2$}}}}

\newcommand\bink      {\binom{n}{k}}
\newcommand\setn      {\{a_i\}^{2n}_{i = 1}}
\newcommand\setc[1]   {\{a_i\}^{#1}_{i = 1}}

\newcommand\cupain    {\bigcup_{i = 1}^{n} A_i}
\newcommand\cupai[1]  {\bigcup_{i = 1}^{#1} A_i}
\newcommand\cupiiai   {\bigcup_{i \in I} A_i}
\newcommand\capain    {\bigcap_{i = 1}^{n} A_i}
\newcommand\capai[1]  {\bigcap_{i = 1}^{#1} A_i}
\newcommand\capiiai   {\bigcap_{i \in I} A_i}

\newcommand\xot       {x_{1, 2}}
\newcommand\ano       {a_{n - 1}}
\newcommand\ant       {a_{n - 2}}

% Other shortcuts
\newcommand\tl    {\tilde}
\newcommand\op    {^{-1}}

\newcommand\sof[1]    {\left | #1 \right |}
\newcommand\cl [1]    {\left ( #1 \right )}
\newcommand\csb[1]    {\left [ #1 \right ]}

\newcommand\bs    {\blacksquare}

%! ~~~ Document ~~~

\author{שחר פרץ}
\title{אלקטרוניקה 2}
\begin{document}
	\maketitle
	\section{תא חשמלי}
	תא המכיל תמיסה נוזלית ובו שתי אלקטרודות מחומרים שונים (לכל אחד מטען שונה) [דוגמה: שקל וחצי שקל (הם מחומרים שונים) בתוך לימון, ואז לשים את הלשון באמצע בשביל לסגור מעגל] 
	\section{הזרם החשמלי}
	בתא יש הפרדת מטענים (חיובי ושלילי). אם נחבר תיל ביניהם, התיל עשוי ממתכת עשירה באלקטרונים, האלקטרונים ינועו מהפוטנציאל הנמוך לגבוה. 
	
	זרם := תנועה מכוונת ומסוגרת של אלקטרונים חופשיים. מסמנים ב־$I$ ומודדים ב־$\frac{c}{s}$ (כאשר c = קולון) או $[A]$ (אמפר). עוצמת הזרם החשמלי היא המטען החשמלי העובד ביחידת זמן דרך חתך של מוליך. 
	
	כיוון הזרם האמיתי הוא מהפוטנציאל הנמוך לגבוה (- ל־+) אך כיווון הזרם המוסכם הוא מה־+ ל־-. [המלצה של ורד: לענות בשאלות גם אמיתי וגם מוסכם, במבחנים]. 
	
	\section{מוליכים ומבודדים}
	חומרים המאפרשרים תנועה מכוונת שם מטענים חשמליים נקראים חומרים מוליכים. חומרים שאינם מפאשרים תנועה של מטענים נקראים חומרים מבודדים. 
	
	\section{התנגדות חשמלית}
	באמצעות מקור מתח אפשר ליצור זרם חשמלי לאורך מוליך. האלקטרונים החופשיים במוליך נועים ומתנדשים ביונים של המוליך ואז תנועתם איטית יותר. משמע, התנועה מכוונת של האלקטרונים החופשיים מערכת בגלל ההתנגשויות ובמוליך נוצרת התנגדות לזרם. את ההתנגדות, באמצעות זרם ומתח, ניתן להגדיר ביחס בין המתח לזרם. כלומר: 
	\[ R = \frac{U}{I} \]
	(חוק אוהם). התנגדות מסמנים באות $R$ ומודדים ביחידות $[\Omega]$ (אוהם) או $[\frac{V}{A}]$ (מתוך חוק אוהם). 
	\subsection{גודל התנגדות של חומר}
	הגורמים הקובעים את גודל התנגדותו של תייל מוליך (או כל חומר מוליך אחר): 
	\begin{itemize}
		\item אורך – ככל שהתיל ארוך יותר, התנגדותו גדולה יותר. נסמן באות $\ell$ ונמדוד ב־$[m]$. 
		\item שטח חתך [עובי] – ככל ששטח החתך גדול יותר, ההתנגדות קטנה יותר. שטח נסמן באות $A$ (\textit{לא} אמפר) ומדדים ביחידות $[mm^2]$ (ממ"ר). 
		\item התנגדות סגולית של החומר – התנגדות המוליך נמצאת ביחס ישר להתנגדות הסגולית שך חומר ממנו עשוי המוליך. התנגדות סגולית מסמנים מסמנים באות $\rho$ והיא נמדדת ביחידות $[\frac{\Omega mm^2}{m}]$. 
	\end{itemize}
	
	סה"כ: 
	\[ R = \frac{\rho \ell}{A} \]
	ואכן מתקיים: 
	\[ \Omega = \frac{\frac{\Omega mm^2}{m} \cdot m}{mm^2} = \frac{\rho \, mm^2}{m} \]
	
	\section{תרגול}
	
	\subsection*{3-11}
	(תרגיל ביחד עם ורד, זו הדרך בה פותרים תרגילים)
	
	\textbf{Q }המתח בין קצותיו של מוליך הוא $6V$, והזרם בו הוא $1.5A$. מהי התנגדות המוליך? 
	\textbf{A }נדע: 
	\begin{align*}
		U&=6V \\
		I&=1\frac{1}{2}A \\
		R&=\frac{U}{I}  = \frac{6}{1.5}\Omega
	\end{align*}
	
	\subsection*{3-12}
	
	נתון נחושת, אורך 10 מטר, שטח 4 ממ"ר: 
	\[ \rho = 0.018, \ \ell = 10m, \ A = 4mm^2 \]
	אזי: 
	\[ R = \frac{\rho \ell}{A} = \frac{0.018 \cdot 10}{4} = 0.045 \Omega \]
	
	מה יהיה הזרם אם נחבר ל-$1V$? 
	\[ I = \frac{U}{R} = \frac{1}{0.045} = 22\frac{2}{9}A \]
	
	\subsection*{3-13}
	
	\textbf{Q }מהו שטח החתך של תיל כסף שאורכו 5 מטר, הזרם $0.5A$, והוא מחובר למתח $12V$? 
	
	\textbf{A }
	נתון: 
	\[ I = 0.5A, \ U = 12V, \ \ell = 5m, \ \rho = 0.016 \]
	נמצא התנגדות: 
	\[ R = \frac{U}{I} = \frac{12}{0.5} = 24\Omega \]
	
	נציב: 
	\[ R = \frac{\rho \ell}{A} \implies 24 = \frac{0.016 \cdot 5}{A} \implies A = 0.016 \cdot 5 \cdot 24\op = \bm{\frac{1}{300}mm^2} \]
	
	\subsection*{3-14}
	
	\textbf{Q} מוליך טונגסטן (1) ומוליך נחושת (2) בלי שטחי חתך שווים. כל אחר מהם מחובר למתח $10V$. הזם בשני המוליכים שווה. אורך הטוגסטן $20$ מטר. מצאו את אורך הנחושת. 
	
	\textbf{A}
	נסמן את הטוגסטן ב־1 והנחושת ב־2. 
	\[ \rho_1 = 0.055, \ \rho_2 = 0.018, \ A_1 = A_2 =: A, \ U = 10V, \ I_1 = I_2 =: I, \ \ell_1 = 20m \]
	
	מחוק אוהם: 
	\[ R := R_1 = R_2 = \frac{U}{I} = \frac{10}{I}\Omega \]
	\[ R_1 = \frac{\rho_1\ell_1}{A} = \frac{10}{I} = R_2 = \frac{\rho_2\ell_2}{A} \]
	נכפול ב־$A$: 
	\[ \rho_1\ell_1 = \rho_2\ell_2 \implies 0.055 \cdot 20 = 0.018 \cdot \ell_2 \implies \ell_2 = \bm{61\frac{1}{9}m} \]
	
	\subsection*{3-7}
	\textbf{A} ב. במוליך הארוך יותר זורם זרם קטן יותר. 
	\subsection*{3-8}
	\textbf{A} ב. 2U
	\subsection*{3-9}
	\textbf{A} א. תגדל פי שניים
	\subsection*{3-10}
	\textbf{A} אף אחד לא נכון. אורכו של הטוגסטגןטגןסטן יותר קטן. הברזל גדול פי $4\frac{1}{6}$
	\subsection*{3-15}
	נסתמך על כך ש־: 
	\[ R = \frac{\rho\ml}{A} \]
	נחושת: 
	\[ R = \frac{0.018 \cdot 20}{2} = 0.18\Omega \]
	כסף:
	\[ R = \frac{0.016 \cdot 10}{8} = 0.02\Omega \]
	זהב: 
	\[ R = \frac{22 \cdot 10^{7} \cdot 50}{5} = 10^{-6}\Omega \]
	סה"כ תיל הכסף בעל ההתנגדות הקטנה ביותר, וחוט הנחושת בעל הגדולה ביותר. 
	
	
	
	
\end{document}