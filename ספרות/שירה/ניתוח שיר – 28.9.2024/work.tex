%! ~~~ Packages Setup ~~~ 
\documentclass[]{article}


% Math packages

% Deisgn
\usepackage[labelfont=bf]{caption}
\usepackage[margin=0.6in]{geometry}
\usepackage{multicol}
\usepackage[skip=4pt, indent=0pt]{parskip}
\usepackage[normalem]{ulem}
\renewcommand\labelitemi{$\bullet$}
\usepackage{hyperref}
\hypersetup{
	colorlinks=true,
	linkcolor=cyan,
	filecolor=magenta,      
	urlcolor=blue,
	pdftitle={עבודת הגשה – תרגול שירה},
	pdfpagemode=FullScreen,
}
\urlstyle{same}


% Hebrew initialzing
\usepackage[bidi=basic]{babel}
\PassOptionsToPackage{no-math}{fontspec}
\babelprovide[main, import, Alph=letters]{hebrew}
\babelprovide[import]{english}
\babelfont[hebrew]{rm}{David CLM}
\babelfont[hebrew]{sf}{David CLM}
\babelfont[english]{tt}{Monaspace Xenon}
\usepackage[shortlabels]{enumitem}
\newlist{hebenum}{enumerate}{1}

% Language Shortcuts
\newcommand\en[1] {\begin{otherlanguage}{english}#1\end{otherlanguage}}
\newcommand\sen   {\begin{otherlanguage}{english}}
	\newcommand\she   {\end{otherlanguage}}
\newcommand\del   {$ \!\! $}
\newcommand\ttt[1]{\en{\footnotesize\texttt{#1}\normalsize}}

\newcommand\npage {\vfil {\hfil \textbf{\textit{המשך בעמוד הבא}}} \hfil \vfil \pagebreak}
\newcommand\ndoc  {\dotfill \\ \vfil {\begin{center} {\textbf{\textit{שחר פרץ, 2024}} \\ \scriptsize \textit{נוצר באמצעות תוכנה חופשית בלבד}} \end{center}} \vfil	}

\newcommand{\rn}[1]{
	\textup{\uppercase\expandafter{\romannumeral#1}}
}

\makeatletter
\newcommand{\skipitems}[1]{
	\addtocounter{\@enumctr}{#1}
}
\makeatother

%! ~~~ Document ~~~
\usepackage{csquotes}
\author{שחר פרץ $\sim$ י'5}
\title{עבודת הגשה – תרגול שירה}
\begin{document}
	\maketitle
	\section*{השיר – "בשל תפוח"}
	\textbf{מילים: }חיים נחמן ביאליק
	
	\textbf{מקור: }פרויקט בן־יהודה $\leftarrow$ תל אביב: דביר, תשל"ג 
	
	\textit{הערה: }הגרסה המושרת של שלמה ארצי מקוצרת במקצת
	\begin{multicols}{3}
		\begin{displayquote}
			הֲתֹאבוּ דַעַת בְּשֶׁלְּמָה אָהָבְתִּי? \\
			אֲנִי אָהַבְתִּי בְּשֶׁל־תַּפּוּחַ. \\
			כִּי אֵין יוֹדֵעַ נְתִיב הָרוּחַ – \\
			זֶה כְּבָר יָדוּעַ; עַתָּה בָאתִי \\
			לְהָבִיא רְאָיָה מֵאַהֲבָתִי. \\
			וַאֲנִי, הֲתֵדְעוּ בְּשֶׁלְּמָה אָהָבְתִּי? \\
			אֲנִי אָהַבְתִּי בְּשֶׁל־תַּפּוּחַ!
			
			לְדוֹדִי הָיָה פַּרְדֵּס יָפֶה, \\
			וּלְדוֹדִי בַת יְפַת עֵינָיִם. \\
			וּבְתוֹךְ הַפַּרְדֵּס בְּרֵכַת מַיִם \\
			מְפַכִּים חֶרֶשׁ, שֹׁקְטִים, בָּרִים, \\
			כְּמַיִם גְּנוּבִים, כְּיֵין סְתָרִים. \\
			לְדוֹדִי הָיָה פַּרְדֵּס יָפֶה, \\
			וּלְדוֹדִי בַת יְפַת עֵינָיִם.
			
			בַּפַּרְדֵּס יֵשׁ מַחֲבֹאֵי שֶׁקֶט, \\
			עֵץ פְּרִי וּנְטִיעִים מְגֻדָּלִים; \\
			וּבְעֵת הַקַּיִץ הַמַּזְהִירָה \\
			כָּל־זֶה מִתְמַלֵּא צִיץ וְשִׁירָה \\
			וְשִׁפְעַת אוֹר וְשִׁפְעַת צְלָלִים. \\
			בַּפַּרְדֵּס יֵשׁ מַחֲבֹאֵי שֶׁקֶט, \\
			עֵץ פְּרִי וּנְטִיעִים מְגֻדָּלִים.
			
			וַיְהִי הַיּוֹם בַּעֲלוֹת הַקַּיִץ, \\
			כְּחֹם הַיּוֹם – הַיּוֹם הַבָּהִיר – \\
			וַאֲנִי עִם־פְּנִנָּה שַׁאֲרָתִי \\
			לָשׂוּחַ אֶל־הַגַּן יָצָאתִי. \\
			הִיא עַלְמָה רַכָּה וַאֲנִי אִישׁ צָעִיר; \\
			וַנְּצַחֵק יַחְדָּו לְרוּחַ קָיִץ, \\
			כְּחֹם הַיּוֹם – הַיּוֹם הַבָּהִיר. 
			
			\columnbreak
			הַפַּרְדֵּס כְּלִיל יְרַקְרַק חָרוּץ, \\
			זִיז אוֹר וְגִיל עַל־כֹּל זָרוּחַ. \\
			מְסֻבָּל עוֹמֵד הַתַּפּוּחַ, \\
			מַזְהִירִים עִנְּבֵי דֻבְדְּבָנִים, \\
			מִתְאַדְּמִים גַּרְגְּרֵי הַשָּׁנִים \\
			בְּצֵל הַשִּׂיחִים בִּירַקְרַק חָרוּץ; \\		
			זִיז אוֹר וָגִיל עַל־כֹּל זָרוּחַ. 
			
			וּבְקוֹל עֲנוֹת “פִּי־פִּי, צִיף־צִיף” \\			
			מִבֵּין עֳפָאיִם כַּנְפֵי רְנָנִים \\			
			בְּשָׂפָה בְלוּלָה יְשׁוֹרֵרוּ. \\
			גַּם־מֵעַי הָמוּ זְמִיר עֲדָנִים \\			
			וּפִזְמוֹן חָדָשׁ יְעוֹרֵרוּ – \\
			וּבְקוֹל עֲנוֹת גַּם “פִּי” גַּם “צִיף” \\
			תִּפְצַחְנָה סָבִיב כַּנְפֵי רְנָנִים. 
			
			וַנְּצַחֵק יַחְדָּו וַנִּתְעַלָּס, \\
			אֲנִי וּפְנִנָּה שַׁאֲרָתִי. \\
			פִּזַּזְנוּ, שַׁרְנוּ וַנְּהִי כִילָדִים, \\
			וּכְמוֹ הִתְעָרְבוּ בְשִׂמְחָתִי \\
			גַּם בְּנֵי הַשִּׁיר וּפְרִי הַמְּגָדִים. \\
			וַנְּצַחֵק יַחְדָּו וַנִּתְעַלָּס, \\
			אֲנִי וּפְנִנָּה שַׁאֲרָתִי. 
			
			הִיא קָרְאָה לִי אֲזַי בְּכוֹר שָׂטָן, \\
			אֲנִי קְרָאתִיהָ צִפּוֹר קְטַנָּה,\\
			אוֹ דִמִּיתִיהָ לְשׁוֹשַׁנָּה – \\
			אֶת־זֹאת לֹא אֶזְכֹּר – לִבִּי תָעָה – \\
			מַה־יָּפְתָה פְנִנָּה בְּאוֹתָהּ שָׁעָה! \\
			הִיא קָרְאָה לִי אֲזַי בְּכוֹר שָׂטָן, \\ 
			וַאֲנִי קְרָאתִיהָ צִפּוֹר קְטַנָּה. 
			\columnbreak
			
			וּפִתְאֹם בָּאנוּ עַד־תַּפּוּחַ \\
			וַיַּעַל רֵיחוֹ הַנִּיחוֹחַ – \\
			חִישׁ קַל מִיָּדִי כָּאֶפְרֹחַ \\
			יוֹנָתִי פְנִנָּה הִתְמַלָּטָה, \\
			וַתָּרֶם יָד וַתָּכָף מַטָּה \\
			אֶת־אַחַד עַנְפֵי הַתַּפּוּחַ \\ 
			הַנּוֹתֵן רֵיחוֹ הַנִּיחוֹחַ. 
			
			וּכְרֶגַע שָׁבָה וּבִימִינָהּ \\
			תַּפּוּחַ גָּדוֹל צַח וְאָדֹם; \\
			וַתִּטְעַם מֶנְהוּ, וַתַּבְרֵנִי \\
			מִיָּדָהּ מִן־הַחֲצִי הַשֵּׁנִי. \\
			אֲנִי פָּעַרְתִּי פֶה וָאָדֹם, \\
			וְהִיא עֹמֶדֶת, וּבִימִינָהּ \\
			פֶּלַח תַּפּוּחַ צַח וְאָדֹם. 
			
			וּבְמַחֲשֹׂף לִבְנַת הַתַּפּוּחַ \\
			הִכִּירָה שִׁנִּי עִקְּבוֹת שִׁנָּהּ, \\
			וַיָּרַח אַפִּי רֵיחַ פְּנִנָּה; \\ 
			אַךְ־רֵיחָהּ – רֵיחָהּ הִשְׁכִּירָנִי \\
			וּכְיַיִן מָתוֹק עֲבָרָנִי – \\
			לְבָבִי פָג, לֹא־קָם בִּי רוּחַ, \\
			מֵאָז – מֵאָז לִי הָיְתָה פְנִנָּה. 
			
		\end{displayquote}
	\end{multicols}
	
	\section*{אמצעי עיצוב}
	לפני הניתוח האומנותי, ארצה להעביר כי על־אף שהשיר מתורגם בחלקו – ביאליק הוסיף בתים, שינה שורות והוסיף משמעויות העוברות דרך עברית בלבד (או לפחות, כך אני מעריך \href{https://www.zivashamir.com/post/%D7%A9%D7%99%D7%A8%D7%99-%D7%90%D7%94%D7%91%D7%94-%D7%91%D7%A9%D7%A4%D7%94-%D7%91%D7%9C%D7%95%D7%9C%D7%94}{ממה שקראתי}, שכן אני לא דובר צרפתית) ולכן אתייחס לשיר כיצירה העומדת בפני עצמה. 
	\begin{enumerate}
		\item \textbf{אימאז'}: כל בית מצייר תמונה חדשה, פרט לבית הראשון, המהווה מעין פתיחה. בגלל הריבוי, לא אפרט את כולם. 
		\begin{enumerate}[A. ]
			\item \textit{הבית השני} – תיאור של פרדס, ושל בת־דודו. מפרט לגבי הבריכה. מתואר בצורה רומנטית, כמו הבתים הבאים. 
			\item \textit{הבית השביעי} – תיאור אקט מיני, השמת המילה "נתעלס" ו"שארתי" בשתי שורות צמודות, כדי להגעיל את הקורא. 
			\item \textit{הבית התשיעי והעשירי} – למעשה, שני אימאז'ים בבית יחיד. הראשון; תיאור מפורש מאוד של אכילת תפוח טעים אליו "פתאום באו" בתוך שדה דודו, ו"הבראה" (באחת המשמעויות של המילה – לשבוע, להשמין). השני, התיאור ביחס למתאפורות המתוארות בהמשך, שם "הבראה" מתארת הרגשת נוחות (עתה  נבחין בכך היא הבריאה אותו) פה אדום...באופן דומה, אפשר להקביל הרמת יד והטיית ענף התפוח הריחני, לתזוזה פיזית של הגוף. 
		\end{enumerate}
		ציור תמונה בכל בית, יחדיו עם כפל־המשמעות, עוזר לקורא לדמיין את אשר קורה, לחוש את הרומנטיות, בניגוד מוחלט מהעובדה ש"שארתי" מתייחס לשארי המשפחה (כלומר, בת משפחה), ואהובתו מתוארת כ"בת דודי". הכל מצויר בהדר וביופי, לצד סלידה גמורה של הקורא מאשר מתואר. 
		
		\item \textbf{חזרה: }בכל בית, המשורר פורס בשתי השורות הראשונות על אשר הוא ידבר, בארבע שלאחריהן מפרט, ואז חוזר על הראשונות מחדש, לעיתים עם שינויים קלים אך מובחנים – לרוב, ליצירת עניין בלבד, ולעיתים לדעתי נושיאם משמעות נוספת. לא אפרט את החזרה בכל אחד מהבתים, אך כן אציין כמה מהשינויים אותם עושה בין החזרות: 
		\begin{enumerate}[A. ]
			\item \textit{"ויהי היום בעלות הקיץ"} לעומת \textit{"ונצחק ביחד לרוח הקיץ"} – הפעם הראשונה בה ביאליק שובר מהותית את החזרה על שתי השורות הראשונות. אני מעריך שההשוואה בין עלות קיץ לרוח קיץ מרמזת על הזמן הרב שעבר (השמש העולה, לעומת הרוח המגיע בסוף היום, או בסוף העונה). 
			\item \textit{"מבין עפאים כנפי רננים"} לעומת \textit{"תפצחנה סביב ענפי רננים"} – עפא הינו ענף, וכנראה השורה הראשונה מהווה שני ארמזים; ארמז ראשון לתהילים, "עֲלֵיהֶם עוֹף-הַשָּׁמַיִם יִשְׁכּוֹן מִבֵּין עֳפָאיִם יִתְּנוּ-קוֹל", גם שם ישנו סוג של עוף המזמר מתוך אותם עפאים, והשני לאיוב, " כְּנַף רְנָנִים נֶעֱלָסָה אִם אֶבְרָה חֲסִידָה וְנֹצָה". נבחין כי בהקשר המתאים, הקשר בתהילים בין עפאים לשירה מתאים לאופיו הארס־פואטי של השיר (באותו הבית ביאליק כותב ש"פזמון חדש יעוררו") יחדיו אופיו הרומנטי (התעלסות, שוב). החלפת צמד המילים "מבין עפאים" ל"תפצחנה סביב", לדעתי מתבצעת אך ורק כדי לעורר עניין, במקום לחזור שוב על אותן השורות הכבדות. 
			
		\end{enumerate}
		
		שימוש נוסף לחזרה הוא בבית הרביעי, שם השורות "ונצחק יחדיו ונתעלס / אני ופנינה שארתי" נושאות משמעות רבה, ומאמתות בצורה חד־משמעית את חשדותיו של הקורא עד אותה הפסקה, בנודע לטיב יחסיהם. לפעמים, המשורר חוזר גם על מילים או ביטויים בודדים: 
		\begin{itemize}
			\item "אך ריחה ריחה השכרני", "מאז מאז היתה לי פנינה" – הדגשה, שמקלה גם על המשורר להגיע למספר ההברות הרצוי בשורה (איזון). 
			\item "אני פערתי פה ואדום / ... / פלח תפוח צח ואדום" – לא ברור לי אם ה"אדום" הראשון הוא מתייחס לעצירה פיזית, או לצבע, ועל כן לא אדע לנתח חזרה זו. 
		\end{itemize}
		\item \textbf{סמל: }
		\begin{enumerate}[A. ]
			\item \textit{תפוח} – מסמל ליופי חיצוני. הכרחי ליצירת האימאז' בבית השביעי, ומשחק גם על העובדה שלדודו פרדס. סמל המספק אופי לירי ואלגנטי, לשיר שיכול להחשב כהפך הגמור ממנו. ובפרט: "אני אהבנתי בשל תפוח", "תפוח גדול צח־אדום" ועוד. 
			
			ייתכן והשימוש בתפוח מהווה גם מעין ארמז, לספר בראשית. דבר ש"נוגסים" ממנו פעם אחת, ומצטערים לאחר מכן. 
			\item \textit{שכרות} – בבית האחרון, מופיעות השורות "אך איחה איחה השירני / וכיין מתוק עברני". מובן שריח לבדו אין משכר, לבטח לא כיין, אך שכרות מסמלת רצון, עשיית מעשים, ללא מחשבה, וללא שליטה על עצמך. לכך המשורר התכוון, והשימוש בסמל מאפשר לכותב לכלול משמעויות רבות בשתי שורות, שאילולאהו היו אורכות מספר רב של שורות. 
			
			נשים לב שביאליק מקשר לשדה הסמנטי הזה, עוד בבית השני ("כמים גנובים, כיין סתרים") וממשיך לעבוד איתו עד סוף השיר. נסיק כי חשוב למשורר להדגיש את אותן תכונות, של פזיזות וקלות לדעת, ולהשאיר אותן במוחנו כחלק מאופיו של הדובר עד תום השיר. 
		\end{enumerate}
		\item \textbf{ארמז: }אומנם נדרש מאיתנו לכתוב על שלושה אמצעי עיצוב בלבד, אך אציין כי בכל מקרה דיברתי בהרחבה על ארמזים ב־2(ב) וב־3(א) תחת כותרת זו. 
		\item \textbf{אמצעים נוספים שמן הראוי לציין: }"ציף ציף" היא \textit{אנומטופיאה}, גם \textit{פסיחה} מתבצעת רבות בשיר (לדוגמה, "כי אין יודע נתיב הרוח – / זה כבר ידוע"), ישנן מספר \textit{מתאפורות} ("יונתי פנינה" – פנינה היא בהחלט לא יונה), \textit{האנשה} ("מזהירים ענבי דובדבנים"), \textit{חריזה} (בכל בית, לעיתים חריזה רגילה ולעיתים מסורגת, תמיד בין השורה השנית לחמישית), ו\textit{דימוי} (בגוף ראשון – "אני קראתיה ציפור קטנה / או דימיתיה לשושנה", וכחלק מהשיר – "כמים גנובים").  יתכן ויש עוד. 
	\end{enumerate}
	\section*{מה הופך את השיר לאהוב}
	לדעתי, האירוניה (העצמית – השיר מסופר בגוף ראשון) המצויה בשיר, בשילוב עם היושר (ביאליק לא פוחד לגלוש לתיאורים מזוויעים, ולא חוסך ברמיזות גסות, לא מצנזר דבר) הופכים את השיר ליחודי – ולא רק בשירה העברית. לעיתים רחוקות נתקלים בשיר שמצד אחד כתוב ברמה כל־כך גבוהה, ומצד שני אינו נוטה לשמרנות וסגירות מצד תוכנו. 
	
	לדעתי היא, כי אירוניה עצמית הכרחית היא, בעבור כל אדם; כולנו איננו מושלמים, וגישה של להיות רע עם עצמך על כל מעשך היא חשובה, אך לא יכולה לעבוד בעבור כל הזוטות בהן חטיאת (הרי, גם מן העבר יש צורך להתקדם הלאה). פרט לכך, אירוניה עצמאית מציגה את עצמך חשוף ביחס לאחרים, וכוללת בתוכה שיפוט עצמי. אני מעריך שגם אם ביאליק לא סיפר על עצמו באופן אישי – הוא אכן צחק על רגשותיו שלו־הוא. 
	
	\ndoc
\end{document}