%! ~~~ Packages Setup ~~~ 
\documentclass[]{article}
\usepackage{lipsum}
\usepackage{rotating}


% Math packages
\usepackage[usenames]{color}
\usepackage{forest}
\usepackage{ifxetex,ifluatex,amssymb,amsmath,mathrsfs,amsthm,witharrows,mathtools,mathdots}
\usepackage{amsmath}
\WithArrowsOptions{displaystyle}
\renewcommand{\qedsymbol}{$\blacksquare$} % end proofs with \blacksquare. Overwrites the defualts. 
\usepackage{cancel,bm}
\usepackage[thinc]{esdiff}


% tikz
\usepackage{tikz}
\usetikzlibrary{graphs}
\newcommand\sqw{1}
\newcommand\squ[4][1]{\fill[#4] (#2*\sqw,#3*\sqw) rectangle +(#1*\sqw,#1*\sqw);}


% code 
\usepackage{algorithm2e}
\usepackage{listings}
\usepackage{xcolor}

\definecolor{codegreen}{rgb}{0,0.35,0}
\definecolor{codegray}{rgb}{0.5,0.5,0.5}
\definecolor{codenumber}{rgb}{0.1,0.3,0.5}
\definecolor{codeblue}{rgb}{0,0,0.5}
\definecolor{codered}{rgb}{0.5,0.03,0.02}
\definecolor{codegray}{rgb}{0.96,0.96,0.96}

\lstdefinestyle{pythonstylesheet}{
	language=Java,
	emphstyle=\color{deepred},
	backgroundcolor=\color{codegray},
	keywordstyle=\color{deepblue}\bfseries\itshape,
	numberstyle=\scriptsize\color{codenumber},
	basicstyle=\ttfamily\footnotesize,
	commentstyle=\color{codegreen}\itshape,
	breakatwhitespace=false, 
	breaklines=true, 
	captionpos=b, 
	keepspaces=true, 
	numbers=left, 
	numbersep=5pt, 
	showspaces=false,                
	showstringspaces=false,
	showtabs=false, 
	tabsize=4, 
	morekeywords={as,assert,nonlocal,with,yield,self,True,False,None,AssertionError,ValueError,in,else},              % Add keywords here
	keywordstyle=\color{codeblue},
	emph={var, List, Iterable, Iterator},          % Custom highlighting
	emphstyle=\color{codered},
	stringstyle=\color{codegreen},
	showstringspaces=false,
	abovecaptionskip=0pt,belowcaptionskip =0pt,
	framextopmargin=-\topsep, 
}
\newcommand\pythonstyle{\lstset{pythonstylesheet}}
\newcommand\pyl[1]     {{\lstinline!#1!}}
\lstset{style=pythonstylesheet}

\usepackage[style=1,skipbelow=\topskip,skipabove=\topskip,framemethod=TikZ]{mdframed}
\definecolor{bggray}{rgb}{0.85, 0.85, 0.85}
\mdfsetup{leftmargin=0pt,rightmargin=0pt,innerleftmargin=15pt,backgroundcolor=codegray,middlelinewidth=0.5pt,skipabove=5pt,skipbelow=0pt,middlelinecolor=black,roundcorner=5}
\BeforeBeginEnvironment{lstlisting}{\begin{mdframed}\vspace{-0.4em}}
	\AfterEndEnvironment{lstlisting}{\vspace{-0.8em}\end{mdframed}}


% Design
\usepackage[labelfont=bf]{caption}
\usepackage[margin=0.6in]{geometry}
\usepackage{multicol}
\usepackage[skip=4pt, indent=0pt]{parskip}
\usepackage[normalem]{ulem}
\forestset{default}
\renewcommand\labelitemi{$\bullet$}
\usepackage{titlesec}
\usepackage{graphicx}
\graphicspath{ {./} }

\usepackage[colorlinks]{hyperref}
\definecolor{mgreen}{RGB}{25, 160, 50}
\definecolor{mblue}{RGB}{30, 60, 200}
\usepackage{hyperref}
\hypersetup{
	colorlinks=true,
	citecolor=mgreen,
	linkcolor=black,
	urlcolor=mblue,
	pdftitle={Document by Shahar Perets},
	%	pdfpagemode=FullScreen,
}
\usepackage{yfonts}
\def\gothstart#1{\noindent\smash{\lower3ex\hbox{\llap{\Huge\gothfamily#1}}}
	\parshape=3 3.1em \dimexpr\hsize-3.4em 3.4em \dimexpr\hsize-3.4em 0pt \hsize}
\def\frakstart#1{\noindent\smash{\lower3ex\hbox{\llap{\Huge\frakfamily#1}}}
	\parshape=3 1.5em \dimexpr\hsize-1.5em 2em \dimexpr\hsize-2em 0pt \hsize}



% Hebrew initialzing
\usepackage[bidi=basic]{babel}
\PassOptionsToPackage{no-math}{fontspec}
\babelprovide[main, import, Alph=letters]{hebrew}
\babelprovide[import]{english}
\babelfont[hebrew]{rm}{David CLM}
\babelfont[hebrew]{sf}{David CLM}
%\babelfont[english]{tt}{Monaspace Xenon}
\usepackage[shortlabels]{enumitem}
\newlist{hebenum}{enumerate}{1}

% Language Shortcuts
\newcommand\en[1] {\begin{otherlanguage}{english}#1\end{otherlanguage}}
\newcommand\he[1] {\she#1\sen}
\newcommand\sen   {\begin{otherlanguage}{english}}
	\newcommand\she   {\end{otherlanguage}}
\newcommand\del   {$ \!\! $}

\newcommand\npage {\vfil {\hfil \textbf{\textit{המשך בעמוד הבא}}} \hfil \vfil \pagebreak}
\newcommand\ndoc  {\dotfill \\ \vfil {\begin{center}
			{\textbf{\textit{שחר פרץ, 2025}} \\
				\scriptsize \textit{קומפל ב־}\en{\LaTeX}\,\textit{ ונוצר באמצעות תוכנה חופשית בלבד}}
	\end{center}} \vfil	}

\newcommand{\rn}[1]{
	\textup{\uppercase\expandafter{\romannumeral#1}}
}

\makeatletter
\newcommand{\skipitems}[1]{
	\addtocounter{\@enumctr}{#1}
}
\makeatother


%! ~~~ Document ~~~

\author{שחר פרץ}
\title{\textit{רייכמן 1} – טרור, ביטחון ומזרח־תיכון}
\begin{document}
	\maketitle
	\textbf{מרצה: }לורנה איטאס־לבובסקי מנדל, חוקרת בכירה במכון למידניות נגד טרור (ICT)
	
	על ICT – 
	\begin{itemize}
		\item הוקם ב־96' בבינתחומי
		\item ב־2001 הופך להיות מרכז ידע עבור גורמי ממשל ובטחון בעולם (9/11)
		\item כיום, מורכב מ־30 חוקרים וכ־100 עמיתי מחקר, וכן 200 מתמחים. 
	\end{itemize}
	הבהרה: המכון איננו גוף מודיעני. זהו גוף מחקר אקדמי עצמאי. עבודת המכון משתדלת לעסוק בפרקטיקה, ומייעץ למוסדות רשמיים בארץ ובחו''ל. 
	
	דוגמה לשאלה בה עוסקים: \textit{מהי ההגדרה של טרור?} גם בשביל שפת דיון משותפת (לדוגמה, האיחוד האירופאי לא הגדיר את חיזבאללה כארגון טרור) וגם בשביל הפן המשפטי (לדוגמה: בארץ ''חוק הלוחמה בטרור`` נרחב מדי). 
	
	\section{ניטור פעילות של טרורסטים ברשת}
	במכון יש דוברי ערבית שנמצאים ברשתות מקצועיות (בינהן רשתות יותר נישתיות). מצאו ברשתות, לפני 14 שנה, הלבשת חומר נפץ על מצנח מתעופף (בדומה לצורה שבה מחבלים נכנסו לארץ).  
	
	נעשה פרויקט מחקר במארה לבדוק את היכולת של מחבלים לעסוק בפיגועי טרור בשדות תעופה ובכניסה לארץ באמצעות מטוסים. לדוגמה, בצרפת התברר שמועסקים 80 עובדים עם עבר פלילי בשדה התעופה הגדול במדינה. ``מנכל משרד התחבורה נשאר עם הלסת פתוחה, כי באמת הצלחנו למצוא פרצות שאחרי כן [...]  שנסגרו''. 
	
	דוגמה אחרת, היא בלוגר שיוצר חומרי נפץ באמצעות חומרים שעברו בידוק ביטחוני. 
	
	\section{תופעת ``המפגע העצמאי'', 2015-2017}
	בד''כ חושבים שהמניעה העיקרי לבצע טרור הוא דת. כמובן, שהסגויה הדתית נמצאת ברקע, אך אין זה מה השניע את האנשים במחקר שנעשה ב־2015-17 במחקר שהמכון עשה בנוגע לפיגועים בארץ. במחקר, נעשו שיחות ארוכים עם הטרורסטים, ומבחנים פסיכולוגיים. התראיינו גם ילדים, הצעיר מבינהם בגיל 11. הסתבר שלמעלה משליש מהמחבלים, סובלים מהפרעות נפשיות. המחבלים האלו, ברצו לקדם בעיקר מטרה אישית. אחד הילדים שראיינו סיפר שאחד הילדים סיפר שהמורה שאל משהו בכיתה וכאשר הוא לא ידע לענות כל הילדים צחקו עליו, וזה היה התריגר מבחינתו – כדי להחזיר את כבודו, הוא יוצא להרוג יהודים. בין הנשים שראיינו, אחת בעלה סרסר בה, ועל אחרת בעלה איים עליה שיקח לה את ילדיה. רוב הנשים רצו למות במהלך הפעולה – נטייתן האובדנית הייתה גבוהה יותר מזו של גברים, ולכן, רובן ניסו לפגוע בחיילים (בידיעה שהסיכוי שהן ימותו גבוה יותר). 
	
	במחקר אחר ניסו להבין מחבלים שמבצעים פיגועי התאבדות. 
	\begin{itemize}
		\item פיגועי דריסה ודקירה, בניגוד לפיגועי התאבדות, מתבצעים בצורה הרבה יותר ספונטנית, ובעוד מחבלים מתאבדים עוברים אימונים והכנה מראש. 
		\item מחבלים מתאבדים כמעט בוודאות יכולים לבצע את הפיגוע – גם אם תופסים אותם, הם יתפוצצו על השוטרים. 
	\end{itemize}
	
	בפיגועי התאבדות יש יותר גורמים שמעוברים בשרשרת (הם אינם \textit{מחבלים עצמאיים}, יש ארגונים מאחוריהם) ויותר קל להשיג מודיעין שם. עם זאת, היום כבר יותר קל לגלות על מפגעים עצמאיים – כשני שליש מהם פרסמו משהו ברשתות שהעידו על כוונה לבצע פיגוע. 
	
	\section{כמה דברים נוספים}
	\begin{itemize}
		\item נשווה בין מנהרות של חמאס לבין מנהרות של חיזבאללה. המנהרות של חמאס דחוסות וקטנות, בעוד המנהרות של חזיבאללה ``יותר מרווחות מהדירה בת''א''. 
		\item בניוזילנד, בוצע פיגוע טרור שהוסטרם בפייסבוק באורך של 52 דקות, בו מחבל טבח במוסלמים בשני מסגדים. לפני הפיגוע הוא פרסם מסמך באורך 72 עמודים שמסביר מדוע הוא מבצע את הפיגוע. 
	\end{itemize}
	
	\section{רציונליות הטרורסטים}
	באופן כללי, שחקני טרור הם שחקנים רציונלים (לא בהבט של תורת המשחקים) – הן הארגונים, והן האנשים. הסיבה? הם עושים שיקולי עלות־תועלת. גם האדם האישי, הפעיל שיקולים והחליט שכדי להחזיר את כבודם, הם יוצאים ועושים פיגוע נגד יהודים. רציונליות השחקנים מאפשרת לנו להבין יותר טוב את החשביה של השחקנים. צריך להגדיר את מערך השיקולים, וקשה להבין את אותו – קשה בעבור ארגונים, ואף קשה יותר בעבור אנשים פרטיים. לדוגמה, לפני השבעה באוקטובר שיקולי חמאס היו להשמר את השלטון, לפמפם את האידיאולוגיה, ולהשאיר מתנגדים בחוץ. היום, מטרתו לשרוד. 
	
	לעיני המרצה, היום מדינת ישראל לא מייצרת אלטרנטיבה אחרת לאידיאולוגיית חמאס. חמאס פמפם את האידיאולוגיה שלו בעזה במשך 15 שנה, ואין אידיאולוגיה אחרת שתחליף את זו שלו. לא משמידים אידיאולוגיה באמצעות טנקים. 
	
	\ndoc
\end{document}